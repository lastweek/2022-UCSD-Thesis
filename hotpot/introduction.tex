\section{Introduction}
\label{sec:hotpot:introduction}

Next-generation non-volatile memories ({\em NVMs}), 
such as 3DXpoint~\cite{Intel3DXpoint}, 
phase change memory ({\em PCM}),
spin-transfer torque magnetic memories ({\em STTMs}),  and the memristor
will provide byte addressability, persistence, high density, and DRAM-like performance~\cite{NVMDB}.
These developments are poised to radically alter the landscape of memory and storage technologies
and have already inspired a host of research 
projects~\cite{Bailey10-OSImpl,Coburn11-ASPLOS, sosp09:bpfs, Dulloor14-EuroSys, hotos09:mogul, Volos11-ASPLOS, Xiaojian11-SC,Zhang15-Mojim,Octopus}.
However, most previous research on NVMs has focused on using them in a single machine environment.
Even though NVMs have the potential to greatly improve the performance and reliability of large-scale applications,
it is still unclear how to best utilize them in distributed, datacenter environments. 

This paper takes a significant step towards the goal of using NVMs in distributed datacenter environments.
We propose {\em Distributed Shared Persistent Memory (\dsnvm)},
a framework that provides a global, shared, and persistent memory space 
using a pool of machines with NVMs attached at the main memory bus.
Applications can perform native memory load and store instructions to access both local and remote data in this global memory space 
and can at the same time make their data persistent and reliable.
\dsnvm\ can benefit both single-node persistent-data applications that want to scale out efficiently
and shared-memory applications that want to add durability to their data.

Unlike traditional systems with separate memory and storage layers~\cite{Larchant,Perdis00,Larchant94,PerDis},
we propose to use just one layer that incorporates both distributed memory and 
distributed storage in \dsnvm.
\dsnvm's one-layer approach eliminates the performance overhead of data marshaling and unmarshaling,
and the space overhead of storing data twice. 
With this one-layer approach, \dsnvm\ can potentially provide the low-latency performance, 
vast persistent memory space, data reliability, and high availability
that many modern datacenter applications demand. 


Building a \dsnvm\ system presents its unique challenges.
Adding ``Persistence'' to Distributed Shared Memory (DSM)
is not as simple as just making in-memory data durable.
Apart from data durability, \dsnvm\ needs to provide two key features that DSM does not have:
persistent naming and data reliability.
In addition to accessing data in \nvm\ via native memory loads and stores,
applications should be able to easily
name, close, and re-open their in-memory data structures.
User data should also be reliably stored in NVM and sustain various types of failures; %(\eg, to have $N$ copies of persistent data).
they need to be consistent both within a node and across distributed nodes after crashes.
To make it more challenging, 
\dsnvm\ has to deliver these guarantees without sacrificing application performance
in order to preserve the low-latency performance of NVMs.

We built {\em \hotpot}, a \dsnvm\ system in the Linux kernel.
\hotpot\ offers low-latency, direct memory access, data persistence, reliability, and
high availability to datacenter applications.
It exposes a global virtual memory address space to each user application
and provides a new persistent naming mechanism that is both easy-to-use and efficient.
Internally, \hotpot\ organizes and manages data in a flexible way
and uses a set of adaptive resource management techniques to improve performance and scalability.

\hotpot\ builds on two main ideas to efficiently provide data reliability with distributed shared memory access.
Our first idea is to integrate distributed memory caching and data replication 
by imposing {\em morphable} states on persistent memory ({\em \nvm}) pages.

In DSM systems, when an application on a node accesses shared data in remote memory {\em on demand},
DSM caches these data copies in its local memory for fast accesses
and later evicts them when reclaiming local memory space.
Like DSM, \hotpot\ caches application-accessed data in local \nvm\
and ensures the coherence of multiple cached copies on different nodes.
But \hotpot\ also uses these cached data as {\em persistent data replicas}
and ensures their reliability and crash consistency.

On the other hand, unlike distributed storage systems, which {\em creates} extra data replicas 
to meet user-specified reliability requirements, 
\hotpot\ makes use of data copies that {\em already exist} in the system when
they were fetched to a local node due to application memory accesses.
 
In essence, every local copy of data serves two simultaneous purposes.
First, applications can access it locally without any network delay.
Second, by placing the fetched copy in \nvm, it can be treated as a persistent replica 
for data reliability.

This seemingly-straightforward integration is not simple. 
Maintaining wrong or outdated versions of data can result in inconsistent data.
To make it worse, these inconsistent data will be persistent in \nvm.
We carefully designed a set of protocols to deliver data reliability and crash consistency guarantees 
while integrating memory caching and data replication.

Our second idea is to exploit application behaviors and intentions in the \dsnvm\ setting. 
Unlike traditional memory-based applications, persistent-data-based applications,
\dsnvm's targeted type of application, have well-defined data {\em commit points}
where they specify what data they want to make persistent.
When a process in such an application makes data persistent,
it usually implies that the data can be {\em visible} outside the process (\eg, to other processes or other nodes). 
\hotpot\ utilizes these data commit points to also push updates to cached copies on distributed nodes
to avoid maintaining coherence on every \nvm\ write. %~\cite{XXX},
Doing so greatly improves the performance of \hotpot, 
while still ensuring correct memory sharing and data reliability.

To demonstrate the benefits of \hotpot, we ported the MongoDB~\cite{MongoDB} NoSQL database to \hotpot\
and built a distributed graph engine based on PowerGraph~\cite{Gonzalez12-OSDI} on \hotpot. 
Our MongoDB evaluation results show that \hotpot\ outperforms a \nvm-based replication system~\cite{Zhang15-Mojim} by up to 3.1\x{}, 
a recent \nvm-based distributed file systems~\cite{Octopus} by up to 3.0\x{}, and a DRAM-based file system by up to 53\x{}. 
\hotpot\ outperforms PowerGraph by 2.3\x{} to 5\x{}, a recent DSM system~\cite{Nelson15-ATC} by 1.3\x{} to 3.2\x{}.
Moreover, \hotpot\ delivers stronger data reliability and availability guarantees than these alternative systems.

Overall, this paper makes the following key contributions:

\begin{itemize}
\item We are the first to introduce the Distributed Shared Persistent Memory (DSPM) model
and among the first to build distributed \nvm-based systems.
The DSPM model provides direct and shared memory accesses to a distributed set of \nvm{}s 
and is an easy and efficient way for datacenter applications to use \nvm.

\item We propose a one-layer approach to build \dsnvm\ by 
integrating memory coherence and data replication.
The one-layer approach avoids the performance cost of two or more indirection layers.

\item We designed two distributed data commit protocols with different consistency levels 
and corresponding recovery protocols to 
ensure data durability, reliability, and availability.

\item We built the first \dsnvm\ system, \hotpot, in the Linux kernel, 
and two traditional kernel-level DSM systems (as comparison to \hotpot). 
\hotpot\ and the two DSM systems are both open sourced.

\item We demonstrated \hotpot's performance benefits and ease of use with two real datacenter applications
and extensive microbenchmark evaluation. 
We compared \hotpot\ with five existing file systems and distributed memory systems, 
and two in-house DSM systems.

\end{itemize}

The rest of the paper is organized as follows.
Section 2 presents and analyzes several recent datacenter trends that motivated our design of DSPM.
We discuss the benefits and challenges of DSPM in Section 3.
Section 4 presents the architecture and abstraction of Hotpot.
We then discuss Hotpot's data management in Section 5.
We present our protocols and mechanisms to ensure data durability, consistency, reliability, and availability in Section 6.
Section 7 briefly discusses the network layer we built underlying \hotpot,
and Section 8 presents detailed evaluation of Hotpot.
We cover related work in Section 9 and conclude in Section 10.
