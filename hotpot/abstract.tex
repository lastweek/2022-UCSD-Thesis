\begin{abstract}


Next-generation non-volatile memories ({\em NVMs}) 
will provide byte addressability, persistence, high density, and DRAM-like performance.
They have the potential to benefit many datacenter applications.
However, most previous research on NVMs has focused on using them in a single machine environment.
It is still unclear how to best utilize them in distributed, datacenter environments.

We introduce {\em Distributed Shared Persistent Memory (\dsnvm)}, a new framework for
using persistent memories in distributed datacenter environments. 
\dsnvm\ provides a new abstraction that allows applications to both perform traditional
memory load and store instructions  and to name, share, and persist their data.

We built {\em \hotpot}, a kernel-level \dsnvm\ system that provides low-latency,
transparent memory accesses, data persistence,  data reliability, and high availability.
The key ideas of Hotpot are to integrate distributed memory caching and data replication
techniques and to exploit application hints. We implemented \hotpot\ in the Linux kernel
and demonstrated its benefits by building a distributed graph engine on \hotpot\ and porting a NoSQL database to \hotpot.
Our evaluation shows that \hotpot\ outperforms a recent distributed shared memory system
by 1.3\x{} to 3.2\x{} and a recent distributed \nvm-based file system by 1.5\x{} to 3.0\x{}.

\end{abstract}
