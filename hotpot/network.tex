\section{Network Layer}
\label{sec:hotpot:network}

The networking delay in \dsnvm\ systems is crucial to their overall performance.
We implement \hotpot's network communication using RDMA. %(Remote Direct Memory Access), 
RDMA provides low-latency, high-bandwidth direct remote memory accesses with low CPU utilization.
\hotpot's network layer is based on LITE~\cite{Tsai17-SOSP}, an efficient RDMA software stack we
built in the Linux kernel on top of the RDMA native APIs, {\em Verbs}~\cite{ibverbs}.

Most of \hotpot's network communication is in the form of RPC. % uses this RPC interface to implement most of its functionality. 
We implemented a customized RPC-like interface in our RDMA layer based on the two-sided RDMA send and receive semantics.
We further built a multicast RPC interface where one node can send a request to multiple nodes in parallel and let them each 
perform their processing functions and reply with the return values to the sending node.
Similar to the findings from recent works~\cite{FaSST}, two-sided RDMA works better and is more flexible 
for these RPC-like interfaces than one-sided RDMA. 

To increase network bandwidth, our RDMA layer enables multiple connections between each pair of nodes. 
It uses only one busy polling thread per node to poll a shared ring buffer for all connections, 
which delivers low-latency performance while keeping CPU utilization low.
Our customized RDMA layer achieves an average latency of 7.9\us\ to perform a \hotpot\ remote page read. %an RPC of 8B outgoing and 4KB incoming message size (equivalent to a remote page read).
In comparison, IPoIB, a standard IP layer on top of Verbs, requires 77\us\ for a round trip with the same size.




\if 0 
%Our \ib\ layer provides a richer and more efficient abstraction of \ib\ operations than 
%existing kernel-level \ib\ layers such as IPoIB and RDS.
To provide low-latency network performance for \hotpot, 
we made several unique design and implementation decisions that are different from previous 
\ib- and RDMA-based network implementations~\cite{Dragojevic14-NSDI,Nelson15-ATC,Kalia14-SIGCOMM}.
%IB-Verbs requires the application to post send (receive) requests to send (receive) queues.
%It uses completion messages in the completion queue to indicate the completion of requests 
%and supports both polling and interrupts to detect completions.
%IB-Verbs offers native IB performance and outperforms alternative IB protocols such as IPoIB and RDS.

%uses a thin protocol based on the reliable transportation mode of IB-Verbs. 

The basic primitive of our \ib\ layer is a pair of send and reply messages
between two nodes. 
With this primitive, one node sends a message to a remote node,
usually a request for an operation, and then waits for a reply.
After the remote node handles this request, it sends a reply back to the calling node,
finishing a send-reply round. 

Interestingly, although \ib's one-sided RDMA operations allow direct read and write to remote memory 
without involving remote side's CPU~\cite{Dragojevic14-NSDI}, 
using them to implement \hotpot\ incurs higher latency than RDMA send and receive~\cite{Kalia14-SIGCOMM}.
This is because one-sided RDMA is mainly useful when accessing remote memory without any states managed by remote side.
\hotpot\ maintains various states and metadata at each node. 
%These metadata need to be consistently updated with their data.
%For example, when a \dn\ accesses a remote page, 
%\hotpot\ uses the send-reply primitive to send a request to the \on. 
%The \on\ updates its list of \dn{}s with committed data copies
%and then sends the page back to the \dn.
Implementing complex \hotpot\ operations using one-sided RDMA requires the combination of several RDMA commands.
Even though the latency of one send or reply is slightly higher than a single one-sided RDMA command,
using send-reply to implement \hotpot\ operations achieves better performance than using one-sided RDMA.

On top of the send-reply primitive, our \ib\ layer further provides an efficient implementation of 
two new interfaces:
atomically sending a group of messages and waiting for reply, 
and multicasting send messages to a set of nodes and waiting for all replies from them.
Both these operations are useful in implementing \hotpot's transaction system.

Our \ib\ layer implements a persistent, append-only log in \nvm\ for receiving messages.
with the help of the Linux kernel slab allocator.
The slab allocator performs object-based allocation efficiently 
by maintaining lists of free objects 
where it allocates new objects from and frees objects into.
%Thus, the slab allocator has v
%The slab allocator is more efficient than 
%Instead of preallocating and maintaining a circular log,
The \ib\ layer allocates new receiving buffers in \nvm\ using the slab allocator
and provides an interface to free these buffers back to the slab lists.
%The \ib\ layer supports atomic operations 
%in \nvm\ and provides a free let .
\hotpot\ uses this mechanism to maintain a persistent redo log for transactions.
\hotpot\ only frees buffers in this log after a transaction has been committed.
\fi
