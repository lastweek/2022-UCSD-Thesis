\section{Related Work}
\label{sec:hotpot:related}

There have been a host of distributed shared memory systems and
distributed storage
systems~\cite{AdyaEtAl-Farsite,calder11-azure,DeCandia+07-Dynamo,Ghemawat03-GoogleFS,KubiEtAl00-Ocean,Petersen97-Bayou,Terry13-Pileus,Chun06-NSDI,Gibbons91-SPAA,Krieger90-HICSS,Zhang15-SOSP,Zhou92-IEEE,Stumm90-IEEE,Stumm90-IPDPS,HLRC,Shasta}
over the past few decades.
While some of \hotpot's coherence protocols may resemble existing DSM systems, none of them manages persistent data.
There are also many single-node \nvm\ systems~\cite{MemoryPersistency,pmxact-asplos16,Delegated-persist,sosp09:bpfs,Dragojevic14-NSDI,Dulloor14-EuroSys,Xiaojian11-SC,HiNFS-Eurosys16,Kamino-EuroSys17,Coburn11-ASPLOS,Volos11-ASPLOS},
but they do not support distributed environments.

\Octopus~\cite{Octopus} is a user-level RDMA-based distributed \nvm\ file system developed in parallel with \hotpot.
\Octopus\ manages file system metadata and data efficiently in a pool of PM-equipped machines. 
\Octopus\ provides a set of customized file APIs including read and write
but not any memory-mapped interfaces.
\Octopus\ does not provide data reliability and high availability either.
\hotpot's abstraction is memory based rather than file based,
and it offers data reliability, availability, and different consistency levels.

Grappa~\cite{Nelson15-ATC} is a DSM system that supports modern data-parallel applications.
Instead of fetching remote memory to a local cached copy, Grappa executes functions at the remote side.
\hotpot\ is a \dsnvm\ system and lets applications store persistent data.
It fetches remote data for both fast local access and data replication.

FaRM~\cite{Kalia14-SIGCOMM,Dragojevic14-NSDI} is an RDMA-based
distributed system on battery-backed DRAM.
RAMCloud is a low-latency distributed key-value store system that keeps a single copy of all data in DRAM~\cite{Ongaro11-RamCloud}
%While \hotpot\ and RAMCloud both provide reliable memory-based storage systems,
%RAMCloud provides a key-value interface to applications
and replicates data on massive slower storages for fast recovery.
%rather than a memory-like interface to applications.
The major difference between \hotpot\ and FaRM or RAMCloud is that
FaRM and RAMCloud both adds a software indirection layer for key-value stores
which can cause significant latency overhead over native load/store operations
and obscures much of the performance of the underlying \nvm.
\hotpot\ uses a memory-like abstraction and directly stores persistent data in \nvm.
\hotpot\ also performs data persistence and replication differently
and uses a different network layer based on two-sided RDMA.

Crail~\cite{crail} is an RDMA-based high-performance multi-tiered distributed storage system that integrates with the Apache Spark ecosystem~\cite{Zaharia12-NSDI}.
Crail mainly consists of a file system that manages tiered storage resources (\eg, DRAM, flash, disk) 
with flexible allocation policies across tiers.
\hotpot\ is a pure \nvm-based system that exposes a memory-like interface. 

PerDis~\cite{PerDis} and Larchant~\cite{Larchant,Larchant94} use a distributed file system below a DSM layer.
%, data management,
%data replication, and cache coherence, a set of challenges more demanding than these layered systems.
Unlike these systems, \hotpot\ is a single-layer system that provides shared memory access, data persistence, and reliability.
%Unlike Crail, \hotpot\ leverages PM to combine DRAM and persistent storage into one layer, and
%fulfills application's requirement at the same time.


Our own previous work, Mojim~\cite{Zhang15-Mojim}, provides an efficient mechanism to replicate \nvm\
over IB using a primary-backup protocol.
\hotpot\ is a \dsnvm\ system that provides a shared-memory abstraction
and integrates cache coherence and data replication.
