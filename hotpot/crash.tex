\subsection{Crash Recovery}
\label{sec:recovery}

\hotpot\ can safely recover from different crash scenarios
without losing applications' data.
% as long as the number of concurrent
%node failures is less than the application-specified replication degree.
\hotpot\ detects node failures
%by detecting an unresponsive node during a network operation and 
by request timeout 
and by periodically sending heartbeat messages from 
the \cd\ to all \hotpot\ nodes.
We now explain \hotpot's crash recovery mechanism
in the following four crash scenarios.
Table~\ref{tbl-crash} summarizes various crash scenarios and \hotpot's recovery mechanisms.

%\subsubsubsection{Crash During Transaction Commit}

{
\begin{table}[t]\small
\begin{center}
\begin{center}
\begin{tabular}{ p{0.18in} | p{0.5in} | p{0.25in} | p{0.5in} | p{4in} }
 & \small Node & \small \nvm & \small Time & \small Action \\
\hline
\hline
& any & Y & any & resume normal operation after reboot \\
\hline
& \cd\ & N & any & reconstruct using mirrored copy \\
\hline
& \on\ & N & NC & promote an existing \dn\ to \on\ \\
& \dn\ & N & NC & reconstruct data to meet replication degree \\
\hline
\multirow{7}{*}{\rotatebox{90}{\mrmw\ Commit}} & \xn/\on\ & N & p1 & undo commit, \on{}s delete redo logs \\
%& any & Y & any & continue commit \\
%& \on\ & N & p1 & undo commit, \on{}s delete redo logs \\
& \xn\ & N & p2 & redo commit, \on{}s send new data to \dn{}s \\
& \on\ & N & p2 & redo commit, \xn\ sends new data to new \on\ \\
& \dn\ & N & p2 & continue commit, \on\ sends data to new \dn\ \\
& \xn\ & N & p3 & complete commit, \on{}s delete redo logs \\
& \on/\dn\ & N & p3 & complete commit, new chunk reconstructed using \committed\ data \\
\hline
\multirow{5}{*}{\rotatebox{90}{MRSW}} & \xn\ & N & commit & undo commit, \on{}s send old data to \dn{}s \\
& \on\ & N & commit & \xn\ redo commit from scratch\\
& \xn\ & N & release & complete commit, release data \\
& \on/\dn\ & N & release & complete commit, new chunk reconstructed using \committed\ data \\
%& \xn\ & N &  & \\
%& \xn\ & N &  & \\
%\hline
\end{tabular}
\end{center}
\caption[Crash and Recovery Scenarios.]
{Crash and Recovery Scenarios.
Columns represent crashing node, if \nvm\ is accessible after crash, time of crash, and actions taken at recovery.
NC represents non-commit time.
}
\label{tbl-crash}
\end{center}
\end{table}
}


\noindent{\bf Recovering \cd\ and \master.}
\cd\ maintains node membership and dataset name mappings.
\hotpot\ currently uses one \cd\ but can be easily extended to 
include a hot stand-by \cd\ (\eg, using Mojim~\cite{Zhang15-Mojim}).
%When the primary \cd\ fails, the stand-by \cd\ can resume immediately.

\master\ tracks which node has acquired write access to a page under the \mrsw\ mode.
\hotpot\ does not make this information persistent
and simply reconstructs it by contacting all other nodes during recovery.
%If the \cd\ fails, a new node will be selected as \cd.
%Since all the metadata that the \cd\ maintains can be reconstructed, 
%\hotpot\ simply rebuild them by contacting all \hotpot\ nodes.

\noindent{\bf Non-commit time crashes.}
Recovering from node crashes during non-commit time is fairly straightforward.
If the \nvm\ in the crashed node is accessible after the crash (we call it {\em with-\nvm\ failure}),
\hotpot\ directly restarts the node and
lets applications access data in \nvm.
% provides the data in \nvm\ to applications,
%which can choose to resume from where they left over~\cite{Narayanan12-ASPLOS}.
As described in Section~\ref{sec:singleconsistency}, \hotpot\ ensures crash consistency of a single node.
Thus, \hotpot\ can always recover to a consistent state 
when \nvm\ survives a crash.
\hotpot\ can sustain arbitrary number of with-\nvm\ failures concurrently. 

When a crash results in corrupted or inaccessible \nvm\ (we call it {\em no-\nvm\ failure}),
\hotpot\ will reconstruct the lost data using redundant copies.
\hotpot\ can sustain $N-1$ concurrent no-\nvm\ failures, where $N$ is the user-defined degree of replication.

If a \dn\ chunk is lost, 
the \on\ of this chunk will check what data pages in the chunk
have dropped below user-defined replication degree
and replicating them on the new node that replaces the failed node. 
%If a node fails and loses all its \dn{} data, and \hotpot\ 
%no longer meets the degree of replication requested by an application,
%the \on\ will send a committed copy of the failed node's data to the new node.
There is no need to reconstruct the rest of the \dn\ data;
\hotpot\ simply lets the new node access them on demand.

%\noindent{\bf Reconstructing \on{} chunks.}
When an \on\ chunk is lost, it is critical to reconstruct it quickly,
since an \on\ serves both remote data read and 
commit operations.
%If a node fails and loses its \on\ chunks,
Instead of reconstructing a failed \on\ chunk from scratch,
\hotpot\ promotes an existing \dn\ chunk to an \on\ chunk
and creates a new \dn\ chunk.
%To promote a \dn\ chunk to an \on\ chunk,
%the newly promoted \dn\ will communicate with all the other \dn{}s of this chunk.
The new \on\ will fetch locally-missing committed data from other nodes
and reconstruct \on\ metadata for the chunk.
Our evaluation results show that it takes at most 2.3 seconds to promote
a 1GB \dn\ chunks to \on.

%\subsubsection{}
\noindent{\bf Crash during commit.}
If a with-\nvm\ failure happens during a \commitxact\ call,
\hotpot\ will just continue its commit process after restart.
When a no-\nvm\ failure happens during commit,
\hotpot\ takes different actions to recover depending on when the failure happens.

For \mrmw\ commit, if no-\nvm\ failure happens before all the \on{}s have created the persistent redo logs (\ie, before starting phase 2),
\hotpot\ will undo the commit and revert to the old committed state
by deleting the redo logs at \on{}s.
If a no-\nvm\ failure happens after all \on{}s have written the committing data to their persistent redo logs (\ie, after commit phase 1),
\hotpot\ will redo the commit by replaying redo logs.

For \mrsw, since we combine \mrmw's phase 1 and phase 2 into one commit phase,
%do not allow each \on\ to proceed with pushing updates to \dn{}s right after it has written its own persistent redo log,
we will not be able to tell whether or not an \on\ has pushed the committing data to \dn{}s 
when this \on\ experience a no-\nvm\ failure.
In this case, \hotpot\ will let \xn\ redo the commit from scratch. 
Even if the crashed \on\ has pushed updates to some \dn{}s,
the system is still correct after \xn\ redo the commit;
it will just have more \redundant\ copies.
When the \xn\ fails during \mrsw\ commit, \hotpot\ will undo the commit
by letting all \on{}s delete their redo logs and send old data to \dn{}s to overwrite \dn{}s' updated data.
%During this recovery process, \hotpot\ needs to know what are the \on{}s that are involved in the commit.
%Instead of contacting all nodes to discover which ones are the \on{}s,
%\hotpot\ uses the \master\ %to reduce the overhead of contacting all nodes
%to only contact the nodes that owns pages that have been acquired by the failed \xn.

During commit, \hotpot\ only supports either \xn\ no-\nvm\ failure or \on\ no-\nvm\ failure.
%Although concurrent \xn\ and \on\ no-\nvm\ failures during commit can lead to inconsistent state,
We choose not to support concurrent \xn\ and \on\ no-\nvm\ failures during commit,
because doing so largely simplifies \hotpot's commit protocol and improves its performance.
\hotpot's commit process is fast (under 250\mus\ with up to 16 nodes, see Section~\ref{sec:results}).
Thus, the chance of \xn\ and \on\ both fail and lose their \nvm\ during commit is very small.
\hotpot\ always supports \dn\ no-\nvm\ failures during commit regardless of whether there are concurrent \xn\ or \on\ failure.

