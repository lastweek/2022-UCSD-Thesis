%
% The top tex file
%

\documentclass[10pt,twocolumn]{article}
%\documentclass[sigconf]{acmart}
%\documentclass[pageno]{jpaper}

\usepackage[small,compact]{titlesec}

\usepackage[normalem]{ulem}

\usepackage{hyperref}
\hypersetup{hidelinks,colorlinks=true,linkcolor=red,citecolor=purple}
%\urlstyle{rm}
%\usepackage[usenames]{color}
%\usepackage{xcolor}
%\usepackage{oneinchmargins}
%\usepackage{tightenum}
%\usepackage{enumitem}
\usepackage[numbers,sort,compress]{natbib}

\usepackage{soul}

%\usepackage{authblk}
\usepackage{setspace}
\usepackage{epsfig}
\usepackage{graphicx}
\usepackage{times}
\usepackage{amsmath}

\usepackage{bbding}
\usepackage{pifont}
\usepackage{amssymb}
\usepackage{wasysym}
\usepackage{latexsym}

% This setting can break long URL into multiple lines.
\usepackage{url}
\def\UrlBreaks{\do\/\do-}
%\usepackage[hidelinks,breaklinks]{hyperref}
%\usepackage{breakurl}

\usepackage{nonfloat}
\urlstyle{rm}
%\usepackage{algorithmic}
% \titlespacing*{\section}{0em}{-0ex}{-2ex}
% \titlespacing*{\subsection}{0em}{-0ex}{-2ex}
% \titlespacing*{\subsubsection}{0em}{-0ex}{-2ex}
%\setstretch{0.98}

\usepackage{multirow}

\let\oldthebibliography=\thebibliography
\let\endoldthebibliography=\endthebibliography
\renewenvironment{thebibliography}[1]{%
\begin{oldthebibliography}{#1}%
  \setlength{\parskip}{0ex}%
  \setlength{\itemsep}{0ex}%
}%
{%
\end{oldthebibliography}%
}

\usepackage{geometry}
\geometry{reset, letterpaper, height=9in, width=7in, hmarginratio=1:1, vmarginratio=1:1, marginparsep=0pt, marginparwidth=0pt, headheight=15pt}
\setlength{\columnsep}{0.33in}
%\setlength{\rightmargin}{0.0in}
%\setlength{\topmargin}{-0.5in}
\setlength{\textheight}{9.25in}
\setlength{\textwidth}{7.0in}
%\setlength{\oddsidemargin}{-0.275in}
%\setlength{\evensidemargin}{-0.275in}

\usepackage{listings}
  \usepackage{courier}
 \lstset{
         basicstyle=\footnotesize\ttfamily, % Standardschrift
         %numbers=left,               % Ort der Zeilennummern
         numberstyle=\tiny,          % Stil der Zeilennummern
         %stepnumber=2,               % Abstand zwischen den Zeilennummern
         numbersep=5pt,              % Abstand der Nummern zum Text
         tabsize=2,                  % Groesse von Tabs
         extendedchars=true,         %
         breaklines=true,            % Zeilen werden Umgebrochen
         keywordstyle=\color{red},
    		frame=b,         
 %        keywordstyle=[1]\textbf,    % Stil der Keywords
 %        keywordstyle=[2]\textbf,    %
 %        keywordstyle=[3]\textbf,    %
 %        keywordstyle=[4]\textbf,   \sqrt{\sqrt{}} %
         stringstyle=\color{white}\ttfamily, % Farbe der String
         showspaces=false,           % Leerzeichen anzeigen ?
         showtabs=false,             % Tabs anzeigen ?
         xleftmargin=17pt,
         framexleftmargin=17pt,
         framexrightmargin=5pt,
         framexbottommargin=4pt,
         %backgroundcolor=\color{lightgray},
         showstringspaces=false      % Leerzeichen in Strings anzeigen ?        
 }
 \lstloadlanguages{% Check Dokumentation for further languages ...
         %[Visual]Basic
         %Pascal
         C
         %C++
         %XML
         %HTML
         %Java
 }

\sloppy

%% Leave this on, so we can see them!!!
%\remarktrue
\newcommand{\shortenum}{\vspace*{-0.1in}}
%\newcommand{\shortsec}{\vspace*{-0.2in}}
%\newcommand{\sparagraph}[1]{\vspace*{-0.2in}\paragraph{#1}}
%\newcommand{\sparagraph}[1]{\vspace*{-0.15in}\paragraph{#1}}
\newcommand{\sparagraph}[1]{\vspace*{0.0in}\paragraph{#1}}

%\setlength{\marginparwidth}{0.6in}
\setlength{\marginparwidth}{2cm}
\usepackage[textsize=tiny,textwidth=0.6in]{todonotes}
\newcommand{\allnotes}[1]{}
\renewcommand{\allnotes}[1]{#1} % Comment to turn off notes
%\newcommand{\fixme}[1]{\allnotes{{\bf\textcolor{red}{[#1]}}}}
\newcommand{\notearvind}[1]{\allnotes{\todo[color=yellow!50]{AK: #1}}}

\newcommand{\noteyiying}[1]{\allnotes{\todo[color=yellow!50]{YZ: #1}}}

\newcommand{\noteryan}[1]{\allnotes{\todo[color=yellow!50]{RK: #1}}}

\newcommand{\noteys}[1]{\allnotes{\todo[color=yellow!50]{YS: #1}}}






%\documentclass[10pt,times,twocolumn]{z2-article}

%
% OSDI
%
%\documentclass[letterpaper,twocolumn,10pt]{article}
%\usepackage{usenix,epsfig,endnotes}


%\documentclass[11pt,times,twocolumn]{article} %\documentclass[11pt]{article} 

%\usepackage{hyperref,url}%,bibunits}
%\urlstyle{rm}
\usepackage[usenames]{color}
\usepackage{times}
\usepackage{oneinchmargins}
\usepackage{tightenum}
\usepackage{enumitem}
\usepackage{ulem}
\usepackage[numbers,sort]{natbib}

\usepackage[hidelinks]{hyperref}

%

\usepackage{setspace}
%\usepackage{epsf}
\usepackage{epsfig}
\usepackage{graphicx}
\usepackage{times}
%\usepackage{algorithm}
%\usepackage{algorithmic}
\usepackage{amsmath}
\usepackage[compact]{titlesec}
\usepackage{url}
\usepackage{nonfloat}
\usepackage{subcaption}
\urlstyle{rm}
%\usepackage{algorithmic}
% \titlespacing*{\section}{0em}{-0ex}{-2ex}
% \titlespacing*{\subsection}{0em}{-0ex}{-2ex}
% \titlespacing*{\subsubsection}{0em}{-0ex}{-2ex}
%\setstretch{0.98}

\usepackage{multirow}

\usepackage{tikz}
\usetikzlibrary{calc}
\newcommand*\circled[1]{\tikz[baseline=-3pt]{
            \node[shape=circle,draw,inner sep=1pt,minimum size=10pt] (char) {\small #1};}}


\usepackage{amssymb}% http://ctan.org/pkg/amssymb
\usepackage{pifont}% http://ctan.org/pkg/pifont
\newcommand{\cmark}{\ding{51}}%
\newcommand{\xmark}{\ding{55}}%


%\setstretch{2.2}

\setlength{\rightmargin}{0.0in}
\setlength{\topmargin}{-0.5in}
\setlength{\textheight}{9in}
\setlength{\textwidth}{6.4in}
\setlength{\oddsidemargin}{0in}
\setlength{\evensidemargin}{0in}


  \let\oldthebibliography=\thebibliography
  \let\endoldthebibliography=\endthebibliography
  \renewenvironment{thebibliography}[1]{%
    \begin{oldthebibliography}{#1}%
      \setlength{\parskip}{0ex}%
      \setlength{\itemsep}{0ex}%
  }%
  {%
    \end{oldthebibliography}%
  }

%\setlength{\rightmargin}{0.0in}
%\setlength{\topmargin}{-0.5in}
%\setlength{\textheight}{9.6in}
%\setlength{\textwidth}{7.0in}
%\setlength{\oddsidemargin}{-0.275in}
%\setlength{\evensidemargin}{-0.275in}

%\newcommand{\ignore}[1]{}
%\setlength{\rightmargin}{-0.3in}
%\setlength{\topmargin}{-0.6in}
%\setlength{\textheight}{9.8in}
%\setlength{\textwidth}{7.5in}
%\setlength{\oddsidemargin}{-0.55in}
%\setlength{\evensidemargin}{-0.55in}

%\setlength{\topmargin}{-0.5in}
%\setlength{\textheight}{9.2in}
%\setlength{\textwidth}{6.9in}
%\setlength{\oddsidemargin}{-0.25in}
%\setlength{\evensidemargin}{-0.25in}


\usepackage{listings}
  \usepackage{courier}
 \lstset{
         basicstyle=\footnotesize\ttfamily, % Standardschrift
         %numbers=left,               % Ort der Zeilennummern
         numberstyle=\tiny,          % Stil der Zeilennummern
         %stepnumber=2,               % Abstand zwischen den Zeilennummern
         numbersep=5pt,              % Abstand der Nummern zum Text
         tabsize=2,                  % Groesse von Tabs
         extendedchars=true,         %
         breaklines=true,            % Zeilen werden Umgebrochen
         keywordstyle=\color{red},
    		frame=b,         
 %        keywordstyle=[1]\textbf,    % Stil der Keywords
 %        keywordstyle=[2]\textbf,    %
 %        keywordstyle=[3]\textbf,    %
 %        keywordstyle=[4]\textbf,   \sqrt{\sqrt{}} %
         stringstyle=\color{white}\ttfamily, % Farbe der String
         showspaces=false,           % Leerzeichen anzeigen ?
         showtabs=false,             % Tabs anzeigen ?
         xleftmargin=17pt,
         framexleftmargin=17pt,
         framexrightmargin=5pt,
         framexbottommargin=4pt,
         %backgroundcolor=\color{lightgray},
         showstringspaces=false      % Leerzeichen in Strings anzeigen ?        
 }
 \lstloadlanguages{% Check Dokumentation for further languages ...
         %[Visual]Basic
         %Pascal
         C
         %C++
         %XML
         %HTML
         %Java
 }

\sloppy
\newcommand{\horizbar}{\rule{\linewidth}{.5mm}}
\newcommand{\app}[1]{{\sc #1}}
 
\renewcommand{\em}{\it}

  
\newcommand{\BigO}[1]{${\cal O}(#1)$}
\newcommand{\BigOmega}[1]{$\Omega(#1)$}
\newcommand{\BigTheta}[1]{$\Theta(#1)$}
 
\newcommand{\ceiling}[1]{\left\lceil #1 \right\rceil}
\newcommand{\faM}{\lfloor \alpha M \rfloor}
%\newcommand{\C}[2]{{#1 \choose #2}}

\newcommand{\x}{$\times$}
 
%\newcommand{\comment}[1]{}
\newcommand{\ignore}[1]{}


%\newcommand{\boldparagraph}[1]{\vspace*{-0ex}\paragraph{#1}}
\newcommand{\boldparagraph}[1]{\vspace*{1ex}\noindent\textit{#1}\hspace{1em}}

%%%%% SINGLE FIGURE
\def\cfigure[#1,#2,#3]{
\begin{figure}
\vspace*{0mm}
\begin{center}

\includegraphics[width=3in]{#1} 
 
\vspace*{-3mm}\caption[]{#2
} \label{#3}
 
\vspace*{-5mm}
\end{center}
%\horizbar
%\vspace*{-2mm}
\end{figure}}

%%%%% SINGLE FIGURE 4in wide
\def\cfigurefour[#1,#2,#3]{
\begin{figure}
\vspace*{0mm}
\begin{center}

\includegraphics[width=4in]{#1} 
 
\vspace*{-3mm}\caption[]{#2
} \label{#3}
 
\vspace*{-5mm}
\end{center}
%\horizbar
%\vspace*{-2mm}
\end{figure}}

%%%%% SINGLE FIGURE
\def\cfiguretemp[#1,#2,#3]{
\begin{figure}
\vspace*{0mm}
\begin{center}

\includegraphics[width=3.5in]{#1} 
 
\vspace*{-3mm}\caption[]{#2
} \label{#3}
 
\vspace*{-5mm}
\end{center}
%\horizbar
\vspace*{-2mm}
\end{figure}}

%%%%% SINGLE WIDE FIGURE
\def\wfigure[#1,#2,#3]{
\begin{figure*}
\vspace*{0mm}
\begin{center}
 \includegraphics[width=\textwidth]{#1} 
 \vspace*{-3mm}\caption[]{#2
} \label{#3}
 
\end{center}
%\horizbar
\end{figure*}}

%%%%% 3 FIGURES IN A ROW
\def\threefigure[#1,#2,#3,#4,#5]{
\begin{figure*}
\vspace*{0mm}
\begin{center}

\begin{tabular}{ccc}
\includegraphics[width=2in]{#1} & \includegraphics[width=2in]{#2} &  \includegraphics[width=2in]{#3} \\
(a) & (b) & (c) \\
\end{tabular}

\vspace*{-3mm}\caption[]{#4
} \label{#5}

\vspace*{-5mm}
\end{center}
%\horizbar
\vspace*{-2mm}
\end{figure*}}

%%%%%% DOUBLE FIGURE
\def\dcfigure[#1,#2,#3,#4,#5,#6]{
{
\begin{figure*}
\begin{center}
\begin{minipage}[c]{\columnwidth}{
\includegraphics[width=\columnwidth]{#1} 
\vspace*{0mm}\caption[]{#2} \label{#3} \
}\end{minipage}\hspace*{\columnsep}\
\begin{minipage}[c]{\columnwidth}{
\includegraphics[width=\columnwidth]{#4} 
\vspace*{0mm}\caption[]{#5}\label{#6} \
}\end{minipage}
\end{center}
\end{figure*}
}
}


\def\tableByTable[#1,#2,#3,#4,#5,#6]{
{
\begin{table*}
\begin{center}
\begin{minipage}[c]{3in}{
\centering
{#1}
\vspace*{0mm}\tabcaption[]{#2}\label{#3} \
}\end{minipage}\hspace*{\columnsep}\
\begin{minipage}[c]{3in}{
\centering
{#4}
\vspace*{0mm}\tabcaption[]{#5}\label{#6} \
}\end{minipage}
\end{center}
\end{table*}
}
}


\def\figureByTable[#1,#2,#3,#4,#5,#6]{
{
\begin{figure*}
\begin{center}
\begin{minipage}[c]{3in}{
\centering
\includegraphics[width=\textwidth]{#1}
\vspace*{0mm}\figcaption[]{#2} \label{#3} \
}\end{minipage}\hspace*{\columnsep}\
\begin{minipage}[c]{3.3in}{
\centering
{#4}
\vspace*{0mm}\tabcaption[]{#5}\label{#6} \
}\end{minipage}
\end{center}
\end{figure*}
}
}

\def\tableByFigure[#1,#2,#3,#4,#5,#6]{
{
\begin{figure*}
\begin{center}
\begin{minipage}[c]{4.3in}{
\centering
{#1}
\vspace*{0mm}\tabcaption[]{#2} \label{#3} \
}\end{minipage}\hspace*{\columnsep}\
\begin{minipage}[c]{2.2in}{
\centering
\includegraphics[width=\textwidth]{#4}
\vspace*{-0.35in}\caption[]{#5}\label{#6} \
}\end{minipage}
\end{center}
\end{figure*}
}
}

% two figs pdfs in one column fig
\def\doublecfigure[#1,#2,#3,#4]{
{
\begin{figure}
\begin{center}
\begin{minipage}[c]{1.5in}{
\begin{center}
\includegraphics[width=1.5in]{#1}%\\(a)
\end{center}
}\end{minipage}\hspace*{1em}\
\begin{minipage}[c]{1.5in}{
\begin{center}
\includegraphics[width=1.5in]{#2}%\\(b)
\end{center}
}\end{minipage}
\vspace*{0mm}\caption[]{#3} \label{#4} \
\end{center}
\end{figure}
}
}

\def\qcfigure[#1,#2,#3,#4,#5,#6]{
{
\begin{figure*}
\vspace*{0.2in}\
\begin{center}
\begin{minipage}[c]{3in}{
\includegraphics[width=3in]{#1} 
\vspace*{-3mm}
}
\end{minipage}\hspace*{0.5in}\
\begin{minipage}[c]{3in}{
\includegraphics[width=3in]{#2} 
\vspace*{-3mm}
}\end{minipage}

\begin{minipage}[c]{3in}{
\includegraphics[width=3in]{#3} 
\vspace*{-3mm}
}
\end{minipage}\hspace*{0.5in}\
\begin{minipage}[c]{3in}{
\includegraphics[width=3in]{#4} 
\vspace*{-3mm}
}\end{minipage}
\end{center}
\caption[]{#5}\label{#6}
\end{figure*}
}
}

\def\twfigure[#1,#2,#3,#4,#5]{
{
\begin{figure*}
\vspace*{0.2in}\
\begin{center}
\begin{minipage}[c]{6.5in}{
\includegraphics[width=6.5in]{#1} 
\vspace*{-3mm}
}
\end{minipage}

\begin{minipage}[c]{6.5in}{
\includegraphics[width=6.5in]{#2} 
\vspace*{-3mm}
}\end{minipage}

\begin{minipage}[c]{6.5in}{
\includegraphics[width=6.5in]{#3} 
\vspace*{-3mm}
}
\end{minipage}
\end{center}
\caption[]{#4}\label{#5}
\end{figure*}
}
}

\def\dwfigure[#1,#2,#3,#4]{
{
\begin{figure*}
\vspace*{0.2in}\
\begin{center}
\begin{minipage}[c]{6.5in}{
\includegraphics[width=6.5in]{#1} 
\vspace*{-3mm}
}
\end{minipage}

\begin{minipage}[c]{6.5in}{
\includegraphics[width=6.5in]{#2} 
\vspace*{-3mm}
}\end{minipage}

\end{center}
\caption[]{#3}\label{#4}
\end{figure*}
}
}



\def\dssfigure[#1,#2,#3,#4,#5,#6]{
{
\begin{figure*}
\vspace*{0.2in}\
\begin{center}
\begin{minipage}[c]{4in}{
\includegraphics[width=4in]{#1}
\vspace*{-3mm}\caption[]{#2} \label{#3} \
}\end{minipage}\hspace*{0.5in}\
\begin{minipage}[c]{2in}{
\includegraphics[width=2in]{#4}
\vspace*{-3mm}\caption[]{#5}\label{#6} \
}\end{minipage}
\end{center}
\vspace*{-0.4in}\
\end{figure*}
}
}




\def\dsfigure[#1,#2,#3,#4,#5,#6]{
{
\begin{figure*}
\vspace*{0.2in}\
\begin{center}
\begin{minipage}[c]{3in}{
\includegraphics[width=3in]{#1}
\vspace*{-3mm}\caption[]{#2} \label{#3} \
}\end{minipage}\hspace*{0.5in}\
\begin{minipage}[c]{3in}{
\hspace*{0.5in}\
\includegraphics[height=3in]{#4}
\vspace*{-3mm}\caption[]{#5}\label{#6} \
}\end{minipage}
\end{center}
\vspace*{-0.4in}\
\end{figure*}
}
}


\def\dsyfigure[#1,#2,#3,#4,#5,#6]{
{
\begin{figure*}
\vspace*{0.2in}\
\begin{center}
\begin{minipage}[c]{2.5in}{
\includegraphics[height=2.5in]{#1}
\vspace*{-3mm}\caption[]{#2} \label{#3} \
}\end{minipage}\hspace*{0.5in}\
\begin{minipage}[c]{2.5in}{
\includegraphics[height=2.5in]{#4}
\vspace*{-3mm}\caption[]{#5}\label{#6} \
}\end{minipage}
\end{center}
\vspace*{-0.4in}\
\end{figure*}
}
}

\def\dyfigure[#1,#2,#3,#4,#5,#6]{
{
\begin{figure*}
\vspace*{0.2in}\
\begin{center}
\begin{minipage}[c]{3in}{
\includegraphics[height=3in]{#1} 
\vspace*{-3mm}\caption[]{#2} \label{#3} \
}\end{minipage}\hspace*{0.5in}\
\begin{minipage}[c]{3in}{
\includegraphics[height=3in]{#4} 
\vspace*{-3mm}\caption[]{#5}\label{#6} \
}\end{minipage}
\end{center}
\vspace*{-0.4in}\
\end{figure*}
}
}

%%%%%% DOUBLE FIGURE Y
\def\dyoldfigure[#1,#2,#3,#4,#5,#6]{
{
\begin{figure*}
\vspace*{0.2in}\
\begin{center}
\begin{minipage}[c]{3in}{
\epsfysize=2.0in\
\hspace{0.5in}\
\epsfbox{#1}
\vspace*{-3mm}\caption[]{#2} \label{#3} \
}\end{minipage}\hspace*{0.25in}\
\begin{minipage}[c]{3in}{
\epsfysize=2.0in\
\hspace{0.5in}\
\epsfbox{#4}
\vspace*{-3mm}\caption[]{#5}\label{#6} \
}\end{minipage}
\end{center}
\vspace*{-0.4in}\
\end{figure*}
}
}

%%%%%% DOUBLE FIGURE Y IN A COLUMN!!
\def\cfiguredouble[#1,#2,#3,#4]{
\begin{figure}
\vspace*{0.2in}\
\begin{center}
\begin{minipage}[c]{1.5in}{
\epsfxsize=1.5in\
\epsfbox{#1}
}\end{minipage}\hspace*{0.1in}\
\begin{minipage}[c]{1.5in}{
\epsfxsize=1.5in\
\vspace{0.1in}\epsfbox{#2}
}\end{minipage}\vspace*{-0.10in} \caption[]{#3}\label{#4}
\end{center}
\vspace*{-0.4in}\
\end{figure}
}


%%%%% Single programmable size figure
\def\wpfigure[#1,#2,#3,#4]{
\begin{figure*}
\vspace*{4mm}
\begin{center}

\includegraphics[width=#4]{#1} 

\vspace*{-3mm}\caption[]{#2
} \label{#3}

\vspace*{-5mm}
\end{center}
%\horizbar
\end{figure*}}

%%%%% Single programmable size figure, rotated
\def\wprfigure[#1,#2,#3,#4,#5]{
\begin{figure*}
\vspace*{4mm}
\begin{center}

\includegraphics[width=#4, angle=#5]{#1} 

\vspace*{-3mm}\caption[]{#2
} \label{#3}

\vspace*{-5mm}
\end{center}
%\horizbar
\end{figure*}}




%%%%% Adjacent, programmable-width figures, slid vertically by 9th
%%%%% parameter
\def\DoubleFigureWSlide[#1,#2,#3,#4,#5,#6,#7,#8,#9]{
\begin{figure*}
\vspace*{#9}
\begin{center}
\begin{minipage}{#4}
\includegraphics[width=#4]{#1}
\vspace*{-3mm}\caption{#2
}\label{#3}
\end{minipage}
\hspace{2em}
\begin{minipage}{#8}
\includegraphics[width=#8]{#5}
\vspace*{-3mm}\caption{#6
}\label{#7}
\end{minipage}
\vspace*{-5mm}
\end{center}
\end{figure*}
}


%%%%% Adjacent, programmable-width figures
\def\DoubleFigureW[#1,#2,#3,#4,#5,#6,#7,#8]{
\begin{figure*}
\vspace*{0in}
\begin{center}
\begin{minipage}{#4}
\includegraphics[width=#4]{#1}
\vspace*{-3mm}\caption{#2
}\label{#3}
\end{minipage}
\hspace{2em}
\begin{minipage}{#8}
\includegraphics[width=#8]{#5}
\vspace*{-3mm}\caption{#6
}\label{#7}
\end{minipage}
\vspace*{-5mm}
\end{center}
\end{figure*}
}



\def\DoubleFigureWHack[#1,#2,#3,#4,#5,#6,#7,#8]{
\begin{figure*}
\vspace*{0in}
\begin{center}
\begin{minipage}{3in}
\includegraphics[width=#4]{#1}
\vspace*{-3mm}\caption{#2
}\label{#3}
\end{minipage}
\hspace{2em}
\begin{minipage}{3in}
\includegraphics[width=#8]{#5}
\vspace*{-3mm}\caption{#6
}\label{#7}
\end{minipage}
\vspace*{-5mm}
\end{center}
\end{figure*}
}






%%%%%% DOUBLE FIGURE
\def\ddcfigure[#1,#2,#3,#4]{
\begin{figure*}
\vspace*{0.2in}\
\begin{center}
\begin{minipage}[c]{\columnwidth}{
\includegraphics[width=\columnwidth]{#1} 
}\end{minipage}\hspace{0.5in}\
\begin{minipage}[c]{\columnwidth}{
\includegraphics[width=\columnwidth]{#2} 
}\end{minipage} \caption[]{#3}\label{#4}
\end{center}
\end{figure*}
}

\def\ddcfigureSlide[#1,#2,#3,#4,#5]{
\begin{figure*}
\vspace*{#5}\
\begin{center}
\begin{minipage}[c]{3in}{
\includegraphics[height=3in]{#1} 
}\end{minipage}\hspace{0.5in}\
\begin{minipage}[c]{3in}{
\includegraphics[height=3in]{#2} 
}\end{minipage}\vspace*{-0.10in} \caption[]{#3}\label{#4}
\end{center}
\vspace*{-0.4in}\
\end{figure*}
}

\def\cxfigure[#1,#2,#3]{
\begin{figure}
\vspace*{4mm}
\begin{center}
 
\epsfxsize=2.5in\
\epsfbox{#1}\
 
\vspace*{-0.10in}\caption[]{#2
} \label{#3}
 
\vspace*{-5mm}
\end{center}
%\horizbar
\vspace*{-2mm}
\end{figure}}

\newenvironment{panefigure}{\begin{figure}\begin{center}}{\end{center}\end{figure}}

\newcommand{\pdfpane}[3]{
\begin{minipage}{#1}
\begin{center}
\includegraphics[width=#1]{#2}\\(#3)
\end{center}
\end{minipage}
}

\newcommand{\figWidth}{\columnwidth}
\newcommand{\figSep}{0.05in} 
%\newcommand{\figSep}{\columnsep} 
\newcommand{\figWidthOne}{3.05in} 
\newcommand{\figWidthHalf}{5.85in} 
\newcommand{\figWidthTwo}{3.7in} 
\newcommand{\figWidthThree}{2in} 
\newcommand{\figWidthFour}{1.3in} 
\newcommand{\figWidthFive}{2.3in} 
\newcommand{\figWidthSix}{2.3in} 
\newcommand{\figHeight}{2.0in}
\newcommand{\figHeightOne}{2.6in}
\newcommand{\captionText}[2]{\textbf{#1} \textit{\small{#2}}}

\newcommand{\beforecaption}{\vspace{-.15cm}\begin{spacing}{0.85}}
\newcommand{\aftercaption}{\vspace{-.45cm}\end{spacing}}
% \newcommand{\mycaption}[3]{{\beforecaption\caption{\label{#1}\footnotesize{\textbf{#2}} {\em #3}}\aftercaption}}
% haryadi, change mycaption three to mycaptionthree
%\newcommand{\mycaption}[3]{{\caption[#2]{{\bf #2.} {\em #3}}\label{#1}}}
%\newcommand{\mycaption}[3]{\beforecaption\caption{\label{#1}{\small \bf #2} \em\scriptsize #3}\aftercaption}
%\newcommand{\mycaption}[3]{\beforecaption\caption{\label{#1}{\bf #2} \em\footnotesize #3}\aftercaption}
\newcommand{\mycaption}[3]{\caption{\label{#1}{\bf #2} \em\small #3}}


%%%%% general

% only foreign words should be italicized... (example given should not)
\newcommand{\eg}{\textit{e.g.}}
\newcommand{\ie}{\textit{i.e.}}
\newcommand{\etal}{\textit{et al.}}
\newcommand{\etc}{\textit{etc.}}
\newcommand{\adhoc}{\textit{ad hoc}}

% units
\newcommand{\KB}{\,KB}
\newcommand{\MB}{\,MB}
\newcommand{\GB}{\,GB}
\newcommand{\TB}{\,TB}
\newcommand{\GBs}{\,GB/s}
\newcommand{\MBs}{\,MB/s}
\newcommand{\KBs}{\,KB/s}
\newcommand{\Kbs}{~Kbit/s}
\newcommand{\gbps}{\,Gbps}
\newcommand{\mbs}{~Mbit/s}
\newcommand{\mus}{\mbox{$\mu s$}}
\newcommand{\ms}{\mbox{$ms$}}

%\newcommand{\fsync}{\texttt{fsync}}

% axes
\newcommand{\xaxis}{x-axis}
\newcommand{\yaxis}{y-axis}


\newcommand{\unix}{{\sc Unix}}
\newcommand{\NULL}{{\sc NULL}}
\newcommand{\sysread}{\texttt{read}}
\newcommand{\syssync}{\texttt{sync}}
\newcommand{\fsync}{\texttt{fsync}}
\newcommand{\syswrite}{\texttt{write}}
\newcommand{\sysseek}{\texttt{lseek}}
\newcommand{\sysstat}{\texttt{stat}}
\newcommand{\make}{\texttt{make}}
\newcommand{\ioctl}{\texttt{ioctl}}
\newcommand{\panic}{\texttt{panic}}
\newcommand{\truncate}{\texttt{truncate}}
\newcommand{\rmdir}{\texttt{rmdir}}
\newcommand{\unlink}{\texttt{unlink}}
\newcommand{\open}{\textit{open}}
\newcommand{\close}{\textit{close}}
\newcommand{\linkscount}{\texttt{linkscount}}
\newcommand{\msync}{\textit{msync}}
\newcommand{\mmap}{\textit{mmap}}
\newcommand{\unmap}{\textit{munmap}}
\newcommand{\map}{\textit{map}}
\newcommand{\fetch}{\textit{gfetch}}
\newcommand{\acquire}{\textit{acquire}}
\newcommand{\commitxact}{\textit{commit}}
\newcommand{\commit}{\textit{commit}}
\newcommand{\barrier}{\textit{thread-barrier}}


% dsnvm
\newcommand{\dsnvm}{DSPM}
\newcommand{\dsm}{DSM}
\newcommand{\nvm}{PM}
\newcommand{\hotpot}{Hotpot}
\newcommand{\mrmw}{MRMW}
\newcommand{\mrsw}{MRSW}
\newcommand{\wfetch}{FETCH}
\newcommand{\cd}{CD}
\newcommand{\dr}{DR}
\newcommand{\on}{ON}
\newcommand{\dn}{DN}
\newcommand{\xn}{CN}
\newcommand{\master}{MN}
\newcommand{\xactid}{CID}
\newcommand{\dirty}{dirty}
\newcommand{\committed}{committed}
\newcommand{\redundant}{redundant}
\newcommand{\ib}{IB}
\newcommand{\sendreply}{\texttt{send-reply}}
\newcommand{\atomicsendreply}{\texttt{atomic-send-reply}}
\newcommand{\multisendreply}{\texttt{multicast-send-reply}}
\newcommand{\journaled}{JOURNALED}
\newcommand{\fsyncsafe}{FSYNC\_SAFE}
\newcommand{\X}{{$\times$}}
\newcommand{\pmfs}{PMFS}
\newcommand{\tmpfs}{tmpfs}
\newcommand{\Octopus}{Octopus}
\newcommand{\Mojim}{Mojim}
\newcommand{\dsmnoxact}{DSM-NoXact}
\newcommand{\dsmxact}{DSM-Xact}
\newcommand{\clflush}{\texttt{clflush}}
\newcommand{\pcommit}{\texttt{pcommit}}
\newcommand{\mfence}{\texttt{mfence}}
\newcommand{\sfence}{\texttt{sfence}}
\newcommand{\ra}{\textbf{R1.a}}
\newcommand{\rb}{\textbf{R1.b}}
\newcommand{\rcs}{\textbf{R2.a}}
\newcommand{\rcm}{\textbf{R2.b}}
\newcommand{\rdr}{\textbf{R3.r}}
\newcommand{\rdu}{\textbf{R3.u}}
\newcommand{\re}{\textbf{R3}}
\newcommand{\rf}{\textbf{R4}}

%\newcommand{\ignore}[1]{}
\input{remark}
%% Leave this on, so we can see them!!!
\remarktrue
\newcommand{\shortenum}{\vspace*{-0.1in}}
%\newcommand{\shortsec}{\vspace*{-0.2in}}
%\newcommand{\sparagraph}[1]{\vspace*{-0.2in}\paragraph{#1}}
%\newcommand{\sparagraph}[1]{\vspace*{-0.15in}\paragraph{#1}}
\newcommand{\sparagraph}[1]{\vspace*{0.0in}\paragraph{#1}}

% add below for confidential distribution
%\usepackage{draftwatermark}
%\SetWatermarkText{DO NOT DISTRIBUTE}
%\SetWatermarkScale{0.4}

\usepackage{url}
\def\UrlBreaks{\do\/\do-}

\begin{document}

\pagestyle{plain}


%\input{summary}
%\clearpage
%\pagestyle{myheadings}
%\pagenumbering{arabic}

%\newenvironment{smallitemize}{\begin{list}{$\bullet$}{\topsep0.0in\itemsep0.0in\parsep0.0in\partopsep0.0in\itemindent0.1in\leftmargin0.1in}}{\end{list}}
%\newenvironment{smallitemize}{\begin{list}{$\bullet$}{\topsep0.1in\itemsep0.1in\parsep0.1in\partopsep0.1in\itemindent0.1in\leftmargin0.1in}}{\end{list}}
%\newenvironment{smallitemize}{\begin{list}{$\bullet$}{\topsep0.05in\itemsep0.05in\parsep0.0in\partopsep0.05in\itemindent0.05in\leftmargin0.05in}}{\end{list}}
%{\topsep{0in}\itemsep{0.1in}\itemindent{0.1in}}
%%%%%%%%%%%%%%%%%%%%%%%%




\newcommand{\horizbar}{\rule{\linewidth}{.5mm}}
\newcommand{\app}[1]{{\sc #1}}
 
\renewcommand{\em}{\it}

  
\newcommand{\BigO}[1]{${\cal O}(#1)$}
\newcommand{\BigOmega}[1]{$\Omega(#1)$}
\newcommand{\BigTheta}[1]{$\Theta(#1)$}
 
\newcommand{\ceiling}[1]{\left\lceil #1 \right\rceil}
\newcommand{\faM}{\lfloor \alpha M \rfloor}
%\newcommand{\C}[2]{{#1 \choose #2}}

\newcommand{\x}{$\times$}
 
%\newcommand{\comment}[1]{}
\newcommand{\ignore}[1]{}


%\newcommand{\boldparagraph}[1]{\vspace*{-0ex}\paragraph{#1}}
\newcommand{\boldparagraph}[1]{\vspace*{1ex}\noindent\textit{#1}\hspace{1em}}

%%%%% SINGLE FIGURE
\def\cfigure[#1,#2,#3]{
\begin{figure}
\vspace*{0mm}
\begin{center}

\includegraphics[width=3in]{#1} 
 
\vspace*{-3mm}\caption[]{#2
} \label{#3}
 
\vspace*{-5mm}
\end{center}
%\horizbar
%\vspace*{-2mm}
\end{figure}}

%%%%% SINGLE FIGURE 4in wide
\def\cfigurefour[#1,#2,#3]{
\begin{figure}
\vspace*{0mm}
\begin{center}

\includegraphics[width=4in]{#1} 
 
\vspace*{-3mm}\caption[]{#2
} \label{#3}
 
\vspace*{-5mm}
\end{center}
%\horizbar
%\vspace*{-2mm}
\end{figure}}

%%%%% SINGLE FIGURE
\def\cfiguretemp[#1,#2,#3]{
\begin{figure}
\vspace*{0mm}
\begin{center}

\includegraphics[width=3.5in]{#1} 
 
\vspace*{-3mm}\caption[]{#2
} \label{#3}
 
\vspace*{-5mm}
\end{center}
%\horizbar
\vspace*{-2mm}
\end{figure}}

%%%%% SINGLE WIDE FIGURE
\def\wfigure[#1,#2,#3]{
\begin{figure*}
\vspace*{0mm}
\begin{center}
 \includegraphics[width=\textwidth]{#1} 
 \vspace*{-3mm}\caption[]{#2
} \label{#3}
 
\end{center}
%\horizbar
\end{figure*}}

%%%%% 3 FIGURES IN A ROW
\def\threefigure[#1,#2,#3,#4,#5]{
\begin{figure*}
\vspace*{0mm}
\begin{center}

\begin{tabular}{ccc}
\includegraphics[width=2in]{#1} & \includegraphics[width=2in]{#2} &  \includegraphics[width=2in]{#3} \\
(a) & (b) & (c) \\
\end{tabular}

\vspace*{-3mm}\caption[]{#4
} \label{#5}

\vspace*{-5mm}
\end{center}
%\horizbar
\vspace*{-2mm}
\end{figure*}}

%%%%%% DOUBLE FIGURE
\def\dcfigure[#1,#2,#3,#4,#5,#6]{
{
\begin{figure*}
\begin{center}
\begin{minipage}[c]{\columnwidth}{
\includegraphics[width=\columnwidth]{#1} 
\vspace*{0mm}\caption[]{#2} \label{#3} \
}\end{minipage}\hspace*{\columnsep}\
\begin{minipage}[c]{\columnwidth}{
\includegraphics[width=\columnwidth]{#4} 
\vspace*{0mm}\caption[]{#5}\label{#6} \
}\end{minipage}
\end{center}
\end{figure*}
}
}


\def\tableByTable[#1,#2,#3,#4,#5,#6]{
{
\begin{table*}
\begin{center}
\begin{minipage}[c]{3in}{
\centering
{#1}
\vspace*{0mm}\tabcaption[]{#2}\label{#3} \
}\end{minipage}\hspace*{\columnsep}\
\begin{minipage}[c]{3in}{
\centering
{#4}
\vspace*{0mm}\tabcaption[]{#5}\label{#6} \
}\end{minipage}
\end{center}
\end{table*}
}
}


\def\figureByTable[#1,#2,#3,#4,#5,#6]{
{
\begin{figure*}
\begin{center}
\begin{minipage}[c]{3in}{
\centering
\includegraphics[width=\textwidth]{#1}
\vspace*{0mm}\figcaption[]{#2} \label{#3} \
}\end{minipage}\hspace*{\columnsep}\
\begin{minipage}[c]{3.3in}{
\centering
{#4}
\vspace*{0mm}\tabcaption[]{#5}\label{#6} \
}\end{minipage}
\end{center}
\end{figure*}
}
}

\def\tableByFigure[#1,#2,#3,#4,#5,#6]{
{
\begin{figure*}
\begin{center}
\begin{minipage}[c]{4.3in}{
\centering
{#1}
\vspace*{0mm}\tabcaption[]{#2} \label{#3} \
}\end{minipage}\hspace*{\columnsep}\
\begin{minipage}[c]{2.2in}{
\centering
\includegraphics[width=\textwidth]{#4}
\vspace*{-0.35in}\caption[]{#5}\label{#6} \
}\end{minipage}
\end{center}
\end{figure*}
}
}

% two figs pdfs in one column fig
\def\doublecfigure[#1,#2,#3,#4]{
{
\begin{figure}
\begin{center}
\begin{minipage}[c]{1.5in}{
\begin{center}
\includegraphics[width=1.5in]{#1}%\\(a)
\end{center}
}\end{minipage}\hspace*{1em}\
\begin{minipage}[c]{1.5in}{
\begin{center}
\includegraphics[width=1.5in]{#2}%\\(b)
\end{center}
}\end{minipage}
\vspace*{0mm}\caption[]{#3} \label{#4} \
\end{center}
\end{figure}
}
}

\def\qcfigure[#1,#2,#3,#4,#5,#6]{
{
\begin{figure*}
\vspace*{0.2in}\
\begin{center}
\begin{minipage}[c]{3in}{
\includegraphics[width=3in]{#1} 
\vspace*{-3mm}
}
\end{minipage}\hspace*{0.5in}\
\begin{minipage}[c]{3in}{
\includegraphics[width=3in]{#2} 
\vspace*{-3mm}
}\end{minipage}

\begin{minipage}[c]{3in}{
\includegraphics[width=3in]{#3} 
\vspace*{-3mm}
}
\end{minipage}\hspace*{0.5in}\
\begin{minipage}[c]{3in}{
\includegraphics[width=3in]{#4} 
\vspace*{-3mm}
}\end{minipage}
\end{center}
\caption[]{#5}\label{#6}
\end{figure*}
}
}

\def\twfigure[#1,#2,#3,#4,#5]{
{
\begin{figure*}
\vspace*{0.2in}\
\begin{center}
\begin{minipage}[c]{6.5in}{
\includegraphics[width=6.5in]{#1} 
\vspace*{-3mm}
}
\end{minipage}

\begin{minipage}[c]{6.5in}{
\includegraphics[width=6.5in]{#2} 
\vspace*{-3mm}
}\end{minipage}

\begin{minipage}[c]{6.5in}{
\includegraphics[width=6.5in]{#3} 
\vspace*{-3mm}
}
\end{minipage}
\end{center}
\caption[]{#4}\label{#5}
\end{figure*}
}
}

\def\dwfigure[#1,#2,#3,#4]{
{
\begin{figure*}
\vspace*{0.2in}\
\begin{center}
\begin{minipage}[c]{6.5in}{
\includegraphics[width=6.5in]{#1} 
\vspace*{-3mm}
}
\end{minipage}

\begin{minipage}[c]{6.5in}{
\includegraphics[width=6.5in]{#2} 
\vspace*{-3mm}
}\end{minipage}

\end{center}
\caption[]{#3}\label{#4}
\end{figure*}
}
}



\def\dssfigure[#1,#2,#3,#4,#5,#6]{
{
\begin{figure*}
\vspace*{0.2in}\
\begin{center}
\begin{minipage}[c]{4in}{
\includegraphics[width=4in]{#1}
\vspace*{-3mm}\caption[]{#2} \label{#3} \
}\end{minipage}\hspace*{0.5in}\
\begin{minipage}[c]{2in}{
\includegraphics[width=2in]{#4}
\vspace*{-3mm}\caption[]{#5}\label{#6} \
}\end{minipage}
\end{center}
\vspace*{-0.4in}\
\end{figure*}
}
}




\def\dsfigure[#1,#2,#3,#4,#5,#6]{
{
\begin{figure*}
\vspace*{0.2in}\
\begin{center}
\begin{minipage}[c]{3in}{
\includegraphics[width=3in]{#1}
\vspace*{-3mm}\caption[]{#2} \label{#3} \
}\end{minipage}\hspace*{0.5in}\
\begin{minipage}[c]{3in}{
\hspace*{0.5in}\
\includegraphics[height=3in]{#4}
\vspace*{-3mm}\caption[]{#5}\label{#6} \
}\end{minipage}
\end{center}
\vspace*{-0.4in}\
\end{figure*}
}
}


\def\dsyfigure[#1,#2,#3,#4,#5,#6]{
{
\begin{figure*}
\vspace*{0.2in}\
\begin{center}
\begin{minipage}[c]{2.5in}{
\includegraphics[height=2.5in]{#1}
\vspace*{-3mm}\caption[]{#2} \label{#3} \
}\end{minipage}\hspace*{0.5in}\
\begin{minipage}[c]{2.5in}{
\includegraphics[height=2.5in]{#4}
\vspace*{-3mm}\caption[]{#5}\label{#6} \
}\end{minipage}
\end{center}
\vspace*{-0.4in}\
\end{figure*}
}
}

\def\dyfigure[#1,#2,#3,#4,#5,#6]{
{
\begin{figure*}
\vspace*{0.2in}\
\begin{center}
\begin{minipage}[c]{3in}{
\includegraphics[height=3in]{#1} 
\vspace*{-3mm}\caption[]{#2} \label{#3} \
}\end{minipage}\hspace*{0.5in}\
\begin{minipage}[c]{3in}{
\includegraphics[height=3in]{#4} 
\vspace*{-3mm}\caption[]{#5}\label{#6} \
}\end{minipage}
\end{center}
\vspace*{-0.4in}\
\end{figure*}
}
}

%%%%%% DOUBLE FIGURE Y
\def\dyoldfigure[#1,#2,#3,#4,#5,#6]{
{
\begin{figure*}
\vspace*{0.2in}\
\begin{center}
\begin{minipage}[c]{3in}{
\epsfysize=2.0in\
\hspace{0.5in}\
\epsfbox{#1}
\vspace*{-3mm}\caption[]{#2} \label{#3} \
}\end{minipage}\hspace*{0.25in}\
\begin{minipage}[c]{3in}{
\epsfysize=2.0in\
\hspace{0.5in}\
\epsfbox{#4}
\vspace*{-3mm}\caption[]{#5}\label{#6} \
}\end{minipage}
\end{center}
\vspace*{-0.4in}\
\end{figure*}
}
}

%%%%%% DOUBLE FIGURE Y IN A COLUMN!!
\def\cfiguredouble[#1,#2,#3,#4]{
\begin{figure}
\vspace*{0.2in}\
\begin{center}
\begin{minipage}[c]{1.5in}{
\epsfxsize=1.5in\
\epsfbox{#1}
}\end{minipage}\hspace*{0.1in}\
\begin{minipage}[c]{1.5in}{
\epsfxsize=1.5in\
\vspace{0.1in}\epsfbox{#2}
}\end{minipage}\vspace*{-0.10in} \caption[]{#3}\label{#4}
\end{center}
\vspace*{-0.4in}\
\end{figure}
}


%%%%% Single programmable size figure
\def\wpfigure[#1,#2,#3,#4]{
\begin{figure*}
\vspace*{4mm}
\begin{center}

\includegraphics[width=#4]{#1} 

\vspace*{-3mm}\caption[]{#2
} \label{#3}

\vspace*{-5mm}
\end{center}
%\horizbar
\end{figure*}}

%%%%% Single programmable size figure, rotated
\def\wprfigure[#1,#2,#3,#4,#5]{
\begin{figure*}
\vspace*{4mm}
\begin{center}

\includegraphics[width=#4, angle=#5]{#1} 

\vspace*{-3mm}\caption[]{#2
} \label{#3}

\vspace*{-5mm}
\end{center}
%\horizbar
\end{figure*}}




%%%%% Adjacent, programmable-width figures, slid vertically by 9th
%%%%% parameter
\def\DoubleFigureWSlide[#1,#2,#3,#4,#5,#6,#7,#8,#9]{
\begin{figure*}
\vspace*{#9}
\begin{center}
\begin{minipage}{#4}
\includegraphics[width=#4]{#1}
\vspace*{-3mm}\caption{#2
}\label{#3}
\end{minipage}
\hspace{2em}
\begin{minipage}{#8}
\includegraphics[width=#8]{#5}
\vspace*{-3mm}\caption{#6
}\label{#7}
\end{minipage}
\vspace*{-5mm}
\end{center}
\end{figure*}
}


%%%%% Adjacent, programmable-width figures
\def\DoubleFigureW[#1,#2,#3,#4,#5,#6,#7,#8]{
\begin{figure*}
\vspace*{0in}
\begin{center}
\begin{minipage}{#4}
\includegraphics[width=#4]{#1}
\vspace*{-3mm}\caption{#2
}\label{#3}
\end{minipage}
\hspace{2em}
\begin{minipage}{#8}
\includegraphics[width=#8]{#5}
\vspace*{-3mm}\caption{#6
}\label{#7}
\end{minipage}
\vspace*{-5mm}
\end{center}
\end{figure*}
}



\def\DoubleFigureWHack[#1,#2,#3,#4,#5,#6,#7,#8]{
\begin{figure*}
\vspace*{0in}
\begin{center}
\begin{minipage}{3in}
\includegraphics[width=#4]{#1}
\vspace*{-3mm}\caption{#2
}\label{#3}
\end{minipage}
\hspace{2em}
\begin{minipage}{3in}
\includegraphics[width=#8]{#5}
\vspace*{-3mm}\caption{#6
}\label{#7}
\end{minipage}
\vspace*{-5mm}
\end{center}
\end{figure*}
}






%%%%%% DOUBLE FIGURE
\def\ddcfigure[#1,#2,#3,#4]{
\begin{figure*}
\vspace*{0.2in}\
\begin{center}
\begin{minipage}[c]{\columnwidth}{
\includegraphics[width=\columnwidth]{#1} 
}\end{minipage}\hspace{0.5in}\
\begin{minipage}[c]{\columnwidth}{
\includegraphics[width=\columnwidth]{#2} 
}\end{minipage} \caption[]{#3}\label{#4}
\end{center}
\end{figure*}
}

\def\ddcfigureSlide[#1,#2,#3,#4,#5]{
\begin{figure*}
\vspace*{#5}\
\begin{center}
\begin{minipage}[c]{3in}{
\includegraphics[height=3in]{#1} 
}\end{minipage}\hspace{0.5in}\
\begin{minipage}[c]{3in}{
\includegraphics[height=3in]{#2} 
}\end{minipage}\vspace*{-0.10in} \caption[]{#3}\label{#4}
\end{center}
\vspace*{-0.4in}\
\end{figure*}
}

\def\cxfigure[#1,#2,#3]{
\begin{figure}
\vspace*{4mm}
\begin{center}
 
\epsfxsize=2.5in\
\epsfbox{#1}\
 
\vspace*{-0.10in}\caption[]{#2
} \label{#3}
 
\vspace*{-5mm}
\end{center}
%\horizbar
\vspace*{-2mm}
\end{figure}}

\newenvironment{panefigure}{\begin{figure}\begin{center}}{\end{center}\end{figure}}

\newcommand{\pdfpane}[3]{
\begin{minipage}{#1}
\begin{center}
\includegraphics[width=#1]{#2}\\(#3)
\end{center}
\end{minipage}
}

\newcommand{\figWidth}{\columnwidth}
\newcommand{\figSep}{0.05in} 
%\newcommand{\figSep}{\columnsep} 
\newcommand{\figWidthOne}{3.05in} 
\newcommand{\figWidthHalf}{5.85in} 
\newcommand{\figWidthTwo}{3.7in} 
\newcommand{\figWidthThree}{2in} 
\newcommand{\figWidthFour}{1.3in} 
\newcommand{\figWidthFive}{2.3in} 
\newcommand{\figWidthSix}{2.3in} 
\newcommand{\figHeight}{2.0in}
\newcommand{\figHeightOne}{2.6in}
\newcommand{\captionText}[2]{\textbf{#1} \textit{\small{#2}}}

\newcommand{\beforecaption}{\vspace{-.15cm}\begin{spacing}{0.85}}
\newcommand{\aftercaption}{\vspace{-.45cm}\end{spacing}}
% \newcommand{\mycaption}[3]{{\beforecaption\caption{\label{#1}\footnotesize{\textbf{#2}} {\em #3}}\aftercaption}}
% haryadi, change mycaption three to mycaptionthree
%\newcommand{\mycaption}[3]{{\caption[#2]{{\bf #2.} {\em #3}}\label{#1}}}
%\newcommand{\mycaption}[3]{\beforecaption\caption{\label{#1}{\small \bf #2} \em\scriptsize #3}\aftercaption}
%\newcommand{\mycaption}[3]{\beforecaption\caption{\label{#1}{\bf #2} \em\footnotesize #3}\aftercaption}
\newcommand{\mycaption}[3]{\caption{\label{#1}{\bf #2} \em\small #3}}


%%%%% general

% only foreign words should be italicized... (example given should not)
\newcommand{\eg}{\textit{e.g.}}
\newcommand{\ie}{\textit{i.e.}}
\newcommand{\etal}{\textit{et al.}}
\newcommand{\etc}{\textit{etc.}}
\newcommand{\adhoc}{\textit{ad hoc}}

% units
\newcommand{\KB}{\,KB}
\newcommand{\MB}{\,MB}
\newcommand{\GB}{\,GB}
\newcommand{\TB}{\,TB}
\newcommand{\GBs}{\,GB/s}
\newcommand{\MBs}{\,MB/s}
\newcommand{\KBs}{\,KB/s}
\newcommand{\Kbs}{~Kbit/s}
\newcommand{\gbps}{\,Gbps}
\newcommand{\mbs}{~Mbit/s}
\newcommand{\mus}{\mbox{$\mu s$}}
\newcommand{\ms}{\mbox{$ms$}}

%\newcommand{\fsync}{\texttt{fsync}}

% axes
\newcommand{\xaxis}{x-axis}
\newcommand{\yaxis}{y-axis}


\newcommand{\unix}{{\sc Unix}}
\newcommand{\NULL}{{\sc NULL}}
\newcommand{\sysread}{\texttt{read}}
\newcommand{\syssync}{\texttt{sync}}
\newcommand{\fsync}{\texttt{fsync}}
\newcommand{\syswrite}{\texttt{write}}
\newcommand{\sysseek}{\texttt{lseek}}
\newcommand{\sysstat}{\texttt{stat}}
\newcommand{\make}{\texttt{make}}
\newcommand{\ioctl}{\texttt{ioctl}}
\newcommand{\panic}{\texttt{panic}}
\newcommand{\truncate}{\texttt{truncate}}
\newcommand{\rmdir}{\texttt{rmdir}}
\newcommand{\unlink}{\texttt{unlink}}
\newcommand{\open}{\textit{open}}
\newcommand{\close}{\textit{close}}
\newcommand{\linkscount}{\texttt{linkscount}}
\newcommand{\msync}{\textit{msync}}
\newcommand{\mmap}{\textit{mmap}}
\newcommand{\unmap}{\textit{munmap}}
\newcommand{\map}{\textit{map}}
\newcommand{\fetch}{\textit{gfetch}}
\newcommand{\acquire}{\textit{acquire}}
\newcommand{\commitxact}{\textit{commit}}
\newcommand{\commit}{\textit{commit}}
\newcommand{\barrier}{\textit{thread-barrier}}


% dsnvm
\newcommand{\dsnvm}{DSPM}
\newcommand{\dsm}{DSM}
\newcommand{\nvm}{PM}
\newcommand{\hotpot}{Hotpot}
\newcommand{\mrmw}{MRMW}
\newcommand{\mrsw}{MRSW}
\newcommand{\wfetch}{FETCH}
\newcommand{\cd}{CD}
\newcommand{\dr}{DR}
\newcommand{\on}{ON}
\newcommand{\dn}{DN}
\newcommand{\xn}{CN}
\newcommand{\master}{MN}
\newcommand{\xactid}{CID}
\newcommand{\dirty}{dirty}
\newcommand{\committed}{committed}
\newcommand{\redundant}{redundant}
\newcommand{\ib}{IB}
\newcommand{\sendreply}{\texttt{send-reply}}
\newcommand{\atomicsendreply}{\texttt{atomic-send-reply}}
\newcommand{\multisendreply}{\texttt{multicast-send-reply}}
\newcommand{\journaled}{JOURNALED}
\newcommand{\fsyncsafe}{FSYNC\_SAFE}
\newcommand{\X}{{$\times$}}
\newcommand{\pmfs}{PMFS}
\newcommand{\tmpfs}{tmpfs}
\newcommand{\Octopus}{Octopus}
\newcommand{\Mojim}{Mojim}
\newcommand{\dsmnoxact}{DSM-NoXact}
\newcommand{\dsmxact}{DSM-Xact}
\newcommand{\clflush}{\texttt{clflush}}
\newcommand{\pcommit}{\texttt{pcommit}}
\newcommand{\mfence}{\texttt{mfence}}
\newcommand{\sfence}{\texttt{sfence}}
\newcommand{\ra}{\textbf{R1.a}}
\newcommand{\rb}{\textbf{R1.b}}
\newcommand{\rcs}{\textbf{R2.a}}
\newcommand{\rcm}{\textbf{R2.b}}
\newcommand{\rdr}{\textbf{R3.r}}
\newcommand{\rdu}{\textbf{R3.u}}
\newcommand{\re}{\textbf{R3}}
\newcommand{\rf}{\textbf{R4}}

\input{remark}
%
% Generic Defines
%

\newcommand{\mm}{mm$^2$}
\newcommand{\figtitle}[1]{\textbf{#1}}
\newcommand{\us}{$\mu$s}
\newcommand{\fixme}[1]{{\color{red}\textbf{\fbox{FIXME} #1}}}
\newcommand{\FIXME}[1]{{\color{red}\textbf{\fbox{FIXME} #1}}}
\newcommand{\TODO}[1]{{\color{red}\textbf{\fbox{TODO} #1}}}
\newcommand{\NOTE}[1]{{\color{blue}\textbf{\fbox{NOTE} #1}}}
\newcommand{\note}[1]{{\color{blue}\textbf{\fbox{NOTE} #1}}}

\newcommand{\yiying}[1]{{\color{cyan}\textbf{\fbox{Yiying} #1}}}
\newcommand{\arvind}[1]{{\color{orange}\textbf{\fbox{Arvind} #1}}}
\newcommand{\yizhou}[1]{{\color{red}\textbf{\fbox{Yizhou} #1}}}
\newcommand{\ryan}[1]{{\color{green}\textbf{\fbox{Ryan} #1}}}
\newcommand{\will}[1]{{\color{olive}\textbf{\fbox{Will} #1}}}
%\fbox{Zac} #1}}}

%\newcommand{\note}[2]{\fixme{$\ll$ #1 $\gg$ #2}}

\newcommand{\myitem}[1]{\item \textbf{#1}}
\newcommand{\myitemit}[1]{\item \textit{#1}}

\begin{document}

%\title{Disaggregating and Consolidating Network Functionalities with SuperNIC}
%\date{}
%\maketitle
%\thispagestyle{empty}


\twocolumn[
    \begin{@twocolumnfalse}
    \begin{center}
	{\Large\bf Disaggregating and Consolidating Network Functionalities with SuperNIC}
    \end{center}
    \centerline{
    Yizhou Shan\textsuperscript{\textdagger},
    Will Lin\textsuperscript{\textdagger},
    Ryan Kosta\textsuperscript{\textdagger},
    Arvind Krishnamurthy\textsuperscript{*},
    Yiying Zhang\textsuperscript{\textdagger}}
    \centerline{
    \it{
    \textsuperscript{\textdagger}University of California San Diego, \textsuperscript{*}University of Washington}
    }
    \vspace{0.2in}
    %\smallskip
    \end{@twocolumnfalse}
]
\thispagestyle{empty}

\setcounter{page}{1}

\begin{abstract}
This is thesis abstract. fill me in.
\end{abstract}

\section{Introduction}
\label{sec:snic:intro}

{\em Hardware resource disaggregation} is a solution that decomposes full-blown, general-purpose servers into segregated, network-attached hardware resource pools, each of which can be built, managed, and scaled independently. With disaggregation, different resources can be allocated from any device in their corresponding pools, exposing vast amounts of resources to applications and at the same time improving resource utilization. Disaggregation also allows data-center providers to independently deploy, manage, and scale different types of resources.
Because of these benefits, disaggregation has gained significant traction from both academia~\cite{LegoOS,FireBox-FASTKeynote,ATC20-pDPM,Nitu18-EUROSYS,DDC-hotcloud20,aifm-osdi20,Semeru,kona,InfiniSwap,FastSwap} and industry~\cite{HP-TheMachine,IntelRackScale,alibaba-polardb,facebook-disaggregation,SnowFlake-NSDI20}.

While increasing amounts of effort go into disaggregating compute~\cite{LegoOS,disagg-gpu}, memory (or persistent memory)~\cite{LegoOS,HP-TheMachine,Lim09-disaggregate,remote-region-atc18,ATC20-pDPM,Semeru,InfiniSwap,FastSwap,hotpot-socc17}, and storage~\cite{PolarFS-VLDB18,SnowFlake-NSDI20,hailstorm-asplos20,ana-eurosys16,gimbal}, the fourth major resource, \textit{network}, has been completely left out.
At first glance, ``network'' cannot be disaggregated from either a traditional monolithic server or a disaggregated device (in this paper collectively called {\em endpoints}), as they both need to be attached to the network.        
%To answer this question, we explore the minimal network functionalities an endpoint needs to have for its connectivity.
%\bolditpara{Proposal: what can be disaggregated?}
However, we observe that even though endpoints need basic connectivity, it is not necessary to run {\em network-related tasks} at the endpoints.
These network tasks, or {\em \nt}s, include the transport layer and all high-level layers such as network virtualization, packet filtering and encryption, and application-specific functions.
%everything including and above the transport layer can 
%each endpoint only needs to manage the connectivity and reliability of the {\em last hop} --- between the endpoint to its direct connection point, and thus only needs a link layer that can handle problems happening within the last hop.
%\noteys{the above reasoning does not make sense to me. we don't have enough context to setup "last hop".}
%Everything else can be disaggregated, including a transport layer for reliable end-to-end delivery, network functions like packet filtering and network virtualization, and application-specific functionalities such as data caching. We collectively call all these ``detachable'' functionalities {\em network tasks}, or {\em \nt}s.

This paper, for the first time, proposes the concept of {\em network disaggregation} and builds a real disaggregated network system to segregate \nt{}s from endpoints.
%systematically answers a set of key questions in network disaggregation.

%\bolditpara{Proposal: disaggregated network resource pool.}
At the core of our network-disaggregation proposal is the concept of a rack-scale disaggregated {\em network resource pool}, which consists of a set of hardware devices that can execute \nt{}s and collectively provide ``network'' as a service (Figure~\ref{fig-snic-topology}), similar to how today's disaggregated storage pool provides data storage service to compute nodes. 
Endpoints can offload (\ie, disaggregate) part or all of their \nt{}s to the network resource pool.
After \nt{}s are disaggregated, we further propose to {\em consolidate} them by aggregating a rack's endpoint \nt{}s onto a small set of network devices.
%\notearvind{might need to generalize to a network pool}
%, thereby reducing the total number of network .

We foresee two architectures of the network resource pool within a rack. The first architecture inserts a network pool between endpoints and the ToR switch by attaching a small set of endpoints to one network device, which is then connected to the ToR switch (Figure~\ref{fig-snic-topology} (a)). The second architecture attaches the pool of network devices to the ToR switch, which then connects to all the endpoints (Figure~\ref{fig-snic-topology} (b)). 

%\bolditpara{Motivating: what are the potential benefits of disaggregating and consolidating \nt{}s?}
%Same as disaggregating other resources like storage, 
Network disaggregation and consolidation have several key benefits.
(1) Disaggregating \nt{}s into a separate pool allows data center providers to build and manage network functionalities only at one place instead of at each endpoint. 
This is especially helpful for heterogeneous disaggregated clusters where a full network stack would otherwise need to be developed and customized for each type of endpoint.
(2) Disaggregating \nt{}s into a separate pool allows the {\em independent scaling} of hardware resources used for network functionalities without the need to change endpoints.
(3) Each endpoint can use more network resources than what can traditionally fit in a single NIC. 
(4) With \nt\ consolidation, the total number of network devices can be reduced, allowing a rack to host more endpoints.
%The final and important benefit comes from consolidation.
(5) The network pool only needs to provision hardware resources for the peak \textit{aggregated} bandwidth in a rack instead of each endpoint provisioning for its own peak, reducing the overall CapEx cost.

Before these benefits can be exploited in a real data center, network disaggregation needs to first meet several goals, which no existing solutions fully support (see \S\ref{sec:snic:related}).

{
\begin{figure}
\begin{center}
\centerline{\includegraphics[width=\textwidth]{snic/Figures/fig-topology.pdf}}
\mycaption{fig-snic-topology}{Overall Architectures of \sysname.}
{
Two ways of connecting \snic{}s to form a disaggregated network resource pool. In (a), dashed lines represent links that are optional.
}
\end{center}
\end{figure}
}

%\bolditpara{Building: what are the key requirements of network disaggregation and consolidation?}
First, each disaggregated network device should meet endpoints' original performance goals even when handling a much larger (aggregated) load than what each endpoint traditionally handles.
The aggregated load will also likely require many different \nt{}s, ranging from transports to application-specific functionalities.
Moreover, after aggregating traffic, there are likely more load spikes (each coming from a different endpoint) that the device needs to handle.

Second, using a disaggregated network pool should reduce the total cost of a cluster. This means that each disaggregated network device should provision the right amount of hardware resources (CapEx) and use as little of them as needed at run time (OpEx). At the same time, the remaining part of a rack (\eg, endpoints, ToR switch, cables) needs to be carefully designed to be low cost.

Third, as we are consolidating \nt{}s from multiple endpoints, in a multi-tenant environment, there would be more entities that need to be isolated. We should ensure that they fairly and safely share various hardware resources in a disaggregated network pool. 

Finally, network devices in a pool need to work together so that lightly loaded devices can handle traffic for other devices that are overloaded.
This load balancing would allow each device to provision less hardware resources as long as the entire pool can handle the peak aggregated load of the rack.

%key challenges consolidation
%sharing, autoscaling, dist
%control plane scalability

Meeting these requirements together is not easy as they imply that the disaggregated network devices need to use minimal and properly isolated hardware resources to handle large loads with high variation, while achieving application performance as if there is no disaggregation.

To tackle these challenges and to demonstrate the feasibility of network disaggregation, we built \textit{\textbf{SuperNIC}} (or \textit{\snic} for short), a new hardware-based programmable network device designed for network disaggregation.
%why new hardware-based sNIC. functions like transport need high speed parallel processing, and software is too slow for that. however, traditional NIC hardware or hardware-based SmartNIC does not offer the autoscaling or fair sharing feature we need for consolidation.
An \snic\ device consists of an ASIC for fixed systems logic, FPGA for running and reconfiguring \nt{}s, and software cores for executing the control plane.
We further built a distributed \snic\ platform that serves as a disaggregated network pool.
Users can deploy a single \nt\ written for FPGA or a directed acyclic graph (DAG) execution plan of \nt{}s to the pool.

To tightly \textbf{consolidate} \nt{}s within an \snic, we support three types of resource sharing: (1) splitting an \snic's hardware resources across different \nt{}s ({\em space sharing}), (2) allowing multiple applications to use the same \nt{} at different times ({\em time sharing}), and (3) configuring the same hardware resources to run different \nt{}s at different times ({\em time sharing with context switching}).
For space sharing, we partition the FPGA space into {\em region}s, with each hosting one or more \nt{}s.
Each region could be individually {\em reconfigured} (via FPGA partial reconfiguration, or {\em PR}) for starting new \nt{}s or to context switch \nt{}s.
Different from traditional software systems, hardware context switching with PR is orders of magnitude slower, which could potentially impact application performance significantly.
To solve this unique challenge, we propose a set of policies and mechanisms to reduce the need to perform PR or to move it off the performance-critical path, \eg, by keeping de-scheduled \nt{}s around like a traditional victim cache, by not over-reacting to load spikes, and by utilizing other \snic{}s when one \snic\ is overloaded.
%\notearvind{Might be worth saying that we also rely on other sNICs' resources if a local sNIC is overloaded.}

To achieve high \textbf{performance} under large, varying load with minimal cost, we automatically scale (auto-scale) an \nt{} by adding/removing instances of it and sending different flows in an application across these instances.
%\noteyiying{@Yizhou, do we send different flows to different instances or it's packet level? --- YS: We use flows. We cannot do individual packet LB, because there are states associated with each flow.}
We further launch different \nt{}s belonging to the same application in parallel and send forked packets to them in parallel for faster processing.
%We achieve high throughput using two levels of parallelism:
%{\em \nt{} parallelism} where a packet goes through multiple \nt{}s in parallel and {\em instance parallelism} where we launch multiple instances of the same \nt{} to handle different packets in an application.
%Apart from the above data-plane designs, we build a scalable control plane.
%To achieve low scheduling latency and scalability, we 
%propose a scheduler that centers around a new notion, {\em \nt\ chaining}.
%The idea is to 
To achieve low scheduling latency and improve scalability, we group \nt{}s that are likely to be executed in a sequence into a chain.
% and to have our central scheduler schedule packets only once for the entire chain. 
Our scheduler reserves credits for the entire chain as much as possible so that packets execute the chain as a whole without involving the scheduler in between.
%Doing so improves both packet-processing latency and scheduler scalability.

To provide \textbf{fairness}, we adopt a fine-grained approach that treats each internal hardware resource separately, \eg, ingress/egress bandwidth, internal bandwidth of each shared \nt, payload buffer space, and on-board memory, as doing so allows a higher degree of consolidation.
%Our context is unique in that the packet processing system itself requires multi-dimensional resource sharing. 
We adopt Dominant Resource Fairness (DRF)~\cite{DRF} for this multi-dimensional resource sharing.
%For the first time in networking systems, we consider multi-dimensional resource sharing and provide Dominant Resource Fairness (DRF)~\cite{DRF}. 
Instead of user-supplied, static per-resource demands as in traditional DRF systems, we monitor the actual load demands at run time and use them as the target in the DRF algorithm.
Furthermore, we propose to use ingress bandwidth throttling to control the allocation of other types of resources.
We also build a simple virtual memory system to \textbf{isolate and protect} accesses to on-board memory. %All \nt's memory accesses use 

Finally, for \textbf{distributed \snic{}s}, we automatically scale out \nt{}s beyond a single \snic\ when load increases and support different mechanisms for balancing loads across \snic{}s depending on the network pool architectures.
For example, with the switch-attached pool architecture, we use the ToR switch to balance all traffic across \snic{}s.
With the intermediate pool architecture, we further support a peer-to-peer, \snic-initiated load migration when one \snic\ is overloaded.

We prototype \snic\ with FPGA using two 100\Gbps, multi-port HiTech Global HTG-9200 boards~\cite{htg9200}.
%The data plane runs on FPGA directly, while the control plane runs in software cores deployed on FPGA.
We build three types of \nt{}s to run on \snic:
reliable transport, traditional network functions, and application-specific tasks, and port two end-to-end use cases to \snic.
The first use case is a key-value store we built on top of real disaggregated memory devices~\cite{Clio}.
We explore using \snic{}s for traditional \nt{}s like the transport layer and customized \nt{}s like key-value data replication and caching.
%For the latter, the client only needs to send one copy to the \snic, which will send copies of the data to multiple memory devices.
%customized network abstraction for disaggregated memory device: a key-value store interface (rather than the standard messaging interface).
%Furthermore, we 
The second use case is a Virtual Private Cloud application we built on top of regular servers by connecting \snic{}s at both the sender and the receiver side.
We disaggregate \nt{}s like encapsulation, firewall, and encryption to the \snic{}s.
%a go-back-N reliable transport; a set of network functions including firewall, AES encryption, and VPN Gateway; and a set of application-specific functions including key-value store replication and caching.
We evaluate \snic\ and the ported applications with micro- and macro-benchmarks and compare \snic\ with no network disaggregation and disaggregation using alternative solutions such as multi-host NICs and a recent multi-tenant SmartNIC~\cite{panic-osdi20}.
Overall, \snic\ achieves 52\% to 56\% CapEx and OpEx cost savings with only 4\% performance overhead compared to a traditional non-disaggregated per-endpoint SmartNIC scenario.
%Our results running a Facebook key-value trace~\cite{Atikoglu12-SIGMETRICS} show that \snic's consolidation of four endhosts and two \nt{}s saves 64\% costs compared to no consolidation, with only 1.3\% performance overhead.
Furthermore, the customized key-value store caching and replication functionalities on \snic\ improves throughput by 1.31\x\ to 3.88\x\ and latency by 1.21\x\ to 1.37\x\ when compared to today's remote memory systems with no \snic.
\section{Disaggregate Hardware Resource}
\label{sec:lego:motivation}

{    
\begin{figure}[h]
\begin{subfigure}{3in}
    \begin{center}
    \centerline{\includegraphics[width=3in]{lego/Figures/g_plot_google_util.pdf}}
    \caption[Google Cluster.]{Google Cluster.}
    \label{fig-googleutil}    
    \end{center}
\end{subfigure}
\begin{subfigure}{3in}
    \begin{center}    
    \centerline{\includegraphics[width=3in]{lego/Figures/g_plot_ali_util.pdf}}    
    \caption[Alibaba Cluster.]{Alibaba Cluster.}
    \label{fig-aliutil}
    \end{center}    
\end{subfigure}
\caption[Data center resource utilization.]{Data center resource utilization.}
\label{fig-resource-anal}
\end{figure}
}
{
\begin{figure*}[t]
\begin{subfigure}{1.7in}
\begin{center}
\centerline{\includegraphics[width=1.7in]{lego/Figures/monolithic-arch.pdf}}
\caption[Monolithic OS.]{OSes Designed for Monolithic Servers.}
\label{fig-monolithic}
\end{center}
\end{subfigure}
\begin{minipage}{0.05in}
\hspace{0.05in}
\end{minipage}
\begin{subfigure}{1.8in}
\begin{center}
\centerline{\includegraphics[width=1.8in]{lego/Figures/multikernel-arch.pdf}}
\caption[Multikernel Architecture.]{Multi-kernel Architecture. \small{P-NIC: programmable NIC.}}
\label{fig-multikernel}
\end{center}
\end{subfigure}
\begin{minipage}{0.05in}
\hspace{0.05in}
\end{minipage}
\begin{subfigure}{2.5in}
\begin{center}
\centerline{\includegraphics[width=2.6in]{lego/Figures/lego-arch.pdf}}
\caption[Splitkernel Architecture.]{Splitkernel Architecture.}
\label{fig-splitkernel}
\end{center}
\end{subfigure}
\caption[Operating System Architecture.]{Operating System Architecture.}
\end{figure*}
}

This section
motivates the hardware resource disaggregation architecture
and discusses the challenges in managing disaggregated hardware.

\subsection{Limitations of Monolithic Servers}
\label{sec:lego:monolimit}
A monolithic server has been the unit of deployment and operation in datacenters for decades.
This long-standing {\em server-centric} architecture has several key limitations.

\noindent{\textit{\uline{Inefficient resource utilization.}}}
With a server being the physical boundary of resource allocation, 
it is difficult to fully utilize all resources in a datacenter~\cite{Barroso-COMPUTER,Quasar-ASPLOS,PowerNap}.
We analyzed two production cluster traces: a 29-day Google one~\cite{GoogleTrace}
and a 12-hour Alibaba one~\cite{AliTrace}.
Figure~\ref{fig-resource-anal} plots the aggregated CPU and memory utilization in the two clusters.
For both clusters, only around half of the CPU and memory are utilized.
Interestingly,
a significant amount of jobs are being evicted at the same time in these traces
(\eg, evicting low-priority jobs to make room for high-priority ones~\cite{Borg}).
One of the main reasons for resource under-utilization in these production clusters is 
the constraint that CPU and memory for a job have to be allocated from 
the same physical machine.

\noindent{\textit{\uline{Poor hardware elasticity.}}}
It is difficult to add, move, remove, or reconfigure hardware components
after they have been installed in a monolithic server~\cite{FB-Wedge100}. %, and
Because of this rigidity, datacenter owners have to plan out server configurations in advance.
However, with today's speed of change in application requirements, such plans have to be adjusted frequently,
and when changes happen, it often comes with waste in existing server hardware.

\noindent{\textit{\uline{Coarse failure domain.}}}
The failure unit of monolithic servers is coarse.
When a hardware component within a server fails, %(\eg, processor, memory chip, RAID controller), 
the whole server is often unusable and applications running on it can all crash.
Previous analysis~\cite{Failure-Disk-FAST07} found that motherboard, memory, CPU, power supply failures account for 
50\% to 82\% of hardware failures in a server.
Unfortunately, monolithic servers cannot continue to operate when any of these devices fail.

\noindent{\textit{\uline{Bad support for heterogeneity.}}}
Driven by application needs, new hardware technologies are finding their ways into modern datacenters~\cite{sigarch-dc}.
Datacenters no longer host only commodity servers with CPU, DRAM, and hard disks. 
They include non-traditional and specialized hardware like GPGPU~\cite{GPU-google,GPU-aws}, 
TPU~\cite{TPU}, 
DPU~\cite{DPU},
FPGA~\cite{Putnam14-FPGA,Amazon-F1}, %,SmartNIC-nsdi18},
non-volatile memory~\cite{Intel3DXpoint}, %,facebook-eurosys18},
and NVMe-based SSDs~\cite{everspin}.
The monolithic server model tightly couples hardware devices with each other and with a motherboard.
As a result, making new hardware devices work with existing servers is a painful and lengthy process~\cite{Putnam14-FPGA}.
%The current practice of making new hardware work is not only slow but also expensive.
Mover, datacenters often need to purchase new servers to host certain hardware.
Other parts of the new servers can go underutilized 
and old servers need to retire to make room for new ones.

\subsection{Hardware Resource Disaggregation}
The server-centric architecture is a bad fit for the fast-changing datacenter hardware, software, and cost needs.
There is an emerging interest in utilizing resources beyond a local machine~\cite{Gao16-OSDI},
such as distributed memory~\cite{Dragojevic14-FaRM,Nelson15-ATC,Aguilera17-SOCC,Novakovic16-SOCC} and network swapping~\cite{GU17-NSDI}. 
These solutions improve resource utilization over traditional systems.
However, they cannot solve all the issues of monolithic servers (\eg, the last three issues in \S\ref{sec:lego:monolimit}), 
since their hardware model is still a monolithic one.
To fully support the growing heterogeneity in hardware and to provide elasticity and flexibility at the hardware level, 
we should {\em break the monolithic server model.}% into flexible resource components.

We envision a {\em hardware resource disaggregation} architecture 
where hardware resources in traditional servers are disseminated into network-attached {\em hardware components}.
Each component has a controller and a network interface,
can operate on its own,
and is an {\em independent, failure-isolated} entity.

The disaggregated approach largely increases the flexibility of a datacenter.
Applications can freely use resources from any hardware component,
which makes resource allocation easy and efficient.
Different types of hardware resources can {\em scale independently}.
It is easy to add, remove, or reconfigure components.
New types of hardware components can easily be deployed in a datacenter ---
by simply enabling the hardware to talk to the network and adding a new network link to connect it.
Finally, hardware resource disaggregation enables fine-grain failure isolation, % because of decomposed hardware resources.
since one component failure will not affect the rest of a cluster.

Three hardware trends are making resource disaggregation feasible in datacenters.
First, network speed has grown by more than an order of magnitude and has become more scalable in the past decade % faster both in bandwidth and latency
with new technologies like Remote Direct Memory Access ({\it RDMA})~\cite{ibverbs} 
and new topologies and switches~\cite{FireBox-FASTKeynote,costa15-r2c2,Costa-WRSC14},
enabling fast accesses of hardware components that are disaggregated across the network.
InfiniBand will soon reach 200Gbps and sub-600 nanosecond speed~\cite{Mellanox-ConnectX6-IB},
being only 2\x\ to 4\x\ slower than main memory bus in bandwidth.
With main memory bus facing a bandwidth wall~\cite{BW-Wall-ISCA09},
future network bandwidth (at line rate) is even projected to exceed local DRAM bandwidth~\cite{CacheCloud-hotcloud18}.

Second, network interfaces are moving closer to hardware components,
with technologies like Intel OmniPath~\cite{OmniPath},
RDMA~\cite{ibverbs},
and NVMe over Fabrics~\cite{NVMe-fabrics-Inteltalk,NVMe-fabrics}.
As a result, hardware devices will be able to access network directly 
without the need to attach any processors. 

Finally, hardware devices are incorporating more processing power~\cite{Ahn15-PIM,Bojnordi12,Mellanox-SmartNIC,Mellanox-SmartNIC2,Agilio-SmartNIC,Junwhan-ISCA17},
allowing application and OS logics to be offloaded to hardware~\cite{Willow,Kaufmann16-ASPLOS}.
On-device processing power will enable system software to manage disaggregated hardware components locally.

With these hardware trends and the limitations of monolithic servers,
we believe that future datacenters will be able to largely benefit from hardware resource disaggregation.
In fact, there have already been several initial hardware proposals in resource disaggregation~\cite{OCP},
including disaggregated memory~\cite{Lim09-disaggregate,Scaleout-numa,Nitu18-EUROSYS}, 
disaggregated flash~\cite{FlashDisaggregation,ReFlex},
%new power state for disaggregated memory~\cite{Nitu18-EUROSYS},
Intel Rack-Scale System~\cite{IntelRackScale}, 
HP ``The Machine''~\cite{HP-TheMachine,HP-MemoryOS}, 
IBM Composable System~\cite{IBM-Composable},
and Berkeley Firebox~\cite{FireBox-FASTKeynote}.

\subsection{OSes for Resource Disaggregation}
Despite various benefits hardware resource disaggregation promises, 
it is still unclear how to manage or utilize disaggregated hardware in a datacenter.
Unfortunately, existing OSes and distributed systems cannot work well with this new architecture.
Single-node OSes like Linux view a server as the unit of management and assume all hardware components are local (Figure~\ref{fig-monolithic}).
A potential approach is to run these OSes on processors
and access memory, storage, and other hardware resources remotely.
Recent disaggregated systems like soNUMA~\cite{Scaleout-numa} take this approach.
However, this approach incurs high network latency and bandwidth consumption with remote device management,
misses the opportunity of exploiting device-local computation power,
and makes processors the single point of failure.

Multi-kernel solutions~\cite{Baumann-SOSP09,Barrelfish-DC,Helios-SOSP,fos-SOCC,Hive-SOSP} (Figure~\ref{fig-multikernel}) 
view different cores, processors, or programmable devices within a server separately 
by running a kernel on each core/device and using message passing to communicate across kernels.
These kernels still run in a single server and all access some common hardware resources in the server like memory and the network interface.
Moreover, they do not manage distributed resources or handle failures in a disaggregated cluster. 

There have been various distributed OS proposals,
most of which date decades back~\cite{Amoeba-Experience,Sprite,MOSIX}. %,V-System,Accent-SOSP,DEMOS-SOSP,Charlotte}.
Most of these distributed OSes manage a set of monolithic servers
instead of hardware components.

Hardware resource disaggregation is fundamentally different from the traditional monolithic server model.
A complete disaggregation of processor, memory, and storage 
means that when managing one of them, there will be no local accesses to the other two.
For example, processors will have no local memory or storage to store user or kernel data.
%Memory and storage components will only have limited processing power. %not have no local memory to serve as cache.
An OS also needs to manage distributed hardware resource and handle hardware component failure.
We summarize the following key challenges in building an OS for resource disaggregation,
some of which have previously been identified~\cite{HP-MemoryOS}.

\begin{itemize}
\item How to deliver good performance when application execution involves the access of network-partitioned disaggregated hardware
and current network is still slower than local buses?

\item How to locally manage individual hardware components with limited hardware resources?

%\item How to communicate across components?

\item How to manage distributed hardware resources?

\item How to handle a component failure without affecting other components or running applications?

\item What abstraction should be exposed to users and how to support existing datacenter applications?

\end{itemize}

Instead of retrofitting existing OSes to confront these challenges,
we take the approach of designing a new OS architecture from the ground up for hardware resource disaggregation.

\section{SuperNIC Overview}
\label{sec:overview}

%As discussed in \S\ref{sec:related}, although different existing network solutions provide some features of network disaggregation and consolidation, none of them meet all our target goals, thus necessitating the design of a new network solution.
This section gives a high-level overview of the overall architecture of the \snic\ platform and how to use it. %We defer the detailed description of \snic\ design to \S\ref{sec:design} and \S\ref{sec:dist}.

\bolditpara{Overall Architectures.}~~
We support two ways of attaching an \snic\ pool in a rack (Figure~\ref{fig-topology}).
In the first architecture, the \snic\ pool is an intermediate layer between endpoints (servers or devices) and the ToR switch.
Each \snic\ uses one port to connect to the ToR switch.
Optionally, all the \snic{}s can be directly connected to each other, \eg, with a ring topology.
All remaining ports in the \snic\ connect endpoints.
We expect each of these endpoint-connecting links to have high bandwidth (\eg, 100\Gbps) and the uplink to the switch to have the same or slightly higher bandwidth (\eg, 100\Gbps\ or 200\Gbps). 
%Differently, today's multi-host NICs break one link into several sub-links each with a fixed portion of the original link's bandwidth.\notearvind{This is the thing that the reviewer complained about.  Apparently, BF3 has a pcie switch at the ingress that removes this problem. We could skip making this point - but I think that reviewer is not on the SIGCOMM PC!}
%The sum of the link bandwidth at each endpoint that connects to an \snic\ can and should exceed the link bandwidth between the \snic\ and the ToR switch. 
%This is because different endpoints' loads peak at different times (\S\ref{sec:motivation-server}), and after \snic's consolidation, the aggregated traffic would mostly fit the \snic's uplink, as shown in Figure~\ref{fig-fb-alibaba}.
%
The second architecture attaches \snic{}s to the ToR switch, and endpoints directly attach to the ToR switch.
In this architecture, the ToR switch re-directs incoming or outgoing traffic to one or more \snic{}s. 
%and balances load when doing the redirection.
Note that for both architectures, the actual implementation could either package the network pool with the ToR switch to form a new ``switch box'' or be separated out as an pluggable pool. 


\if 0
\snic\ is a data-center-scale solution. %Any endpoint in a data center can be the sender and/or the receiver, and any 
An endpoint could either connect to an \snic\ or directly to a ToR switch.
A given \snic\ can connect different types of endpoints.
However, there is a potential benefit in connecting similar endpoints to an \snic.
Doing so offers more opportunity for resource consolidation, as similar endpoints (\eg, memory devices) are likely to use the same set of \nt{}s (\eg, encryption).
%On the other hand, the same type of endpoints are more likely to receive similar workloads (\eg, a replicated write sent to two memory devices) that could result in synchronized traffic peak and burden the \snic.

When an \snic\ fails, or its link to the ToR switch fails, if other links and the basic switching functionalities are still alive, the \snic\ would turn into a passthrough device, forwarding \nt{}s to other \snic{}s for processing.
When the entire \snic\ fails, the endpoints connected to it will be disconnected to the rest of the data center.
This failure could be viewed as equivalent to traditional ToR switch failure but with a smaller failure domain (only the endpoints under the failed \snic\ instead of the whole rack).
Data centers that desire stronger reliability~\cite{pangu-nsdi21} could use a multi-homed solution by connecting each endpoint to two \snic{}s.
\fi

\bolditpara{Requirements for endpoints and the last hop.}~~
For basic connectivity, an endpoint needs to have the equivalence of physical and link layers.
For reliable transmission, the link layer needs to provide basic reliability functionality if the reliable transport is offloaded to \snic.
This is because packets could still be corrupted or dropped during the point-to-point transmission between an endpoint and its connected \snic/switch (the last hop).
Thus, the endpoint's link layer should be able to detect corrupted or dropped packets. It will either correct the corruption or treat it as a lost packet and retransmit it.
%Since the connection is point-to-point, the reliable link layer only needs one logical flow and requires a small retransmission buffer.
%Our implementation uses only \fixme{XXX} more resource than an unreliable link layer.
%Since the connection is point-to-point, t
The link layer also requires a simple flow control to slow down packet sending when the \snic\ pool is overloaded or the application's fair share is exceeded.
%In addition, an endpoint should perform simple flow control of the last hop (\eg, by slowing down the transmission when receiving back pressure or using PFC).
%This addition is the only change to today's endpoints that use an unreliable link layer, and it is only needed when the reliable transport is offloaded to \snic.
%We choose 64\KB\ buffer size, which is more than sufficient in the worst case.
%Overall, our reliable link layer only uses 37\% more resources than an unreliable link layer.

%The above requirements are all that is needed for disaggregating network functionalities, and any endpoints that meets these requirements can work with \snic.
Any interconnect fabric that meets the above requirements can be used as the last-hop link.
PCIe is one such example, as it supports reliable data transfer and flow control.
%Our \snic\ prototype uses Ethernet and extends standard non-reliable link layer to handle the reliability and rely on Priority Flow Control (PFC) for flow control.
Our \snic\ prototype uses Ethernet as it is more flexible.
We use Priority Flow Control (PFC) for the one-hop flow control and add simple retransmission support.
%extend the unreliable Ethernet link layer with a small one-hop reliable retransmission.
%By design, \snic\ can work with different types of physical links between endpoints and the \snic. Our prototype uses regular Ethernet. Future extensions could use faster/tighter links like PCIe to further reduce latency overhead. 
Unlike a traditional reliable link layer, our {\em point-to-point} reliable link layer is lightweight, %(only 37\% more resources than an unreliable link layer with our implementation), 
as it only needs to maintain one logical flow and a small retransmission buffer for the small Bandwidth-Delay Product (BDP) of the last hop (64\KB\ in our prototype).

\if 0
%\fixme{TODO: Need to revisit the following three paragraphs depending on how much is implemented and evaluated. Also this is a place to shorten if we need more space.}
We envision three types of endpoints and different network features for them.
The first type is regular servers.
Servers could choose to offload a transport protocol, network functions, and/or application-specific tasks to \snic\ to save CPU cycles and/or to accelerate performance.
Since servers have plenty of memory (larger than or similar to what an \snic\ has), they are more fit to store data than \snic{}s.
One interesting architecture we explore is to offload a reliable transport to \snic\ but to have the server still buffer un-acknowledged packets until receiving an ACK from the receiver (which we refer to as an {\em end-to-end buffer}).
%In this case, \snic\ will discard packets after they leave the \snic, saving its memory for other tasks.

The second type is disaggregated devices that only serve as a request handler (\eg, a memory device that accepts memory alloc/read/write operations~\cite{clio-arxiv,ATC20-pDPM}). 
Such a device often has limited processing power and would offload most tasks such as a transport protocol, network functions like encryption, and device-specific functionalities like replication to \snic. 
Since it never serves as a request originator, there is no need to maintain any packet store, and a failure could be handled by having the client retry the entire request~\cite{clio-arxiv,homa-sigcomm18,1RMA-sigcomm20}.

The final type is disaggregated devices that could serve as request initiators (\eg, a disaggregated CPU or GPU device) but have little memory (because memory is disaggregated to memory devices~\cite{LegoOS}).
When offloading \nt{}s to \snic{}s, they do not have significant memory to maintain end-to-end buffers like regular servers.
On the other hand, if they do not buffer packets at all and rely on \snic\ to buffer un-acknowledged packets, errors can still happen when packets are not successfully delivered to the \snic. 
We propose a {\em one-hop buffer}---buffering a packet at the device only until the next hop (\ie, the \snic) acknowledges.
Doing so reduces the amount of time each packet is maintained and the overall memory consumption.
\fi

\bolditpara{Using SuperNIC.}~~
To use the \snic\ platform, users first write and deploy \nt{}s.
%The user can use any endpoints in the data center as the sender and the receiver (even if the endpoint is not connected to an \snic\ and connects directly to a ToR switch).
They specify which \snic\ (sender side or receiver side) to deploy an \nt.
Users also specify whether an \nt\ needs to access the packet payload and whether it needs to use on-board memory.
For the latter, we provide a virtual memory interface that gives each \nt\ its own virtual address space.
Optionally, users can specify which applications share the same \nt{}(s).
Currently, our FPGA prototype only supports \nt{}s written on FPGA (deployed as netlists).
Future implementation could extend \snic{}s to support p4 programs running on RMT pipelines~\cite{p4fpga-sosr17} and generic software programs running on a processor.

After all the \nt{}s that a user desires have been deployed, the user specifies one or multiple user-written or compiler-generated~\cite{clicknp-sigcomm16,NFP-sigcomm17} DAGs of the execution order of deployed \nt{}s. Users could also add more DAGs at run time. Compared to existing works which let users specify an NF DAG when deploying NFs~\cite{e2-sosp15,flowtags-nsdi14,clicknp-sigcomm16}, we allow more flexible usages and sharing of deployed \nt{}s. %Different from traditional NF execution flows that execute NFs in sequence, we allow multiple \nt{}s to execute in parallel. 
The \snic\ stores user-specified DAGs in its memory and assigns a unique identifier (UID) to each DAG.
At run time, each packet carries a UID, which \snic\ uses to fetch the DAG.


\section{\lego\ Design}
\label{sec:lego:design}

Based on the \splitkernel\ architecture,
we built {\em \lego}, the first OS designed for hardware resource disaggregation.
\lego\ is a research prototype that demonstrates the feasibility of the \splitkernel\ design,
but it is not the only way to build a \splitkernel.
\lego' design targets three types of hardware components:
processor, memory, and storage,
and we call them {\em \pcomponent, \mcomponent}, and {\em \scomponent}.

This section first introduces the abstraction \lego\ exposes to users
and then describes the hardware architecture of components \lego\ runs on.
Next, we explain the design of \lego' process, memory, and storage \microos{}s.
Finally, we discuss \lego' global resource management and failure handling mechanisms.

Overall, \lego\ achieves the following design goals:

\begin{itemize}

\item Clean separation of process, memory, and storage functionalities.

\item Monitors run at hardware components and fit device constraints.

\item Comparable performance to monolithic Linux servers.

\item Efficient resource management and memory failure handling, both in space and in performance. % and performance-efficient memory replication scheme.

\item Easy-to-use, backward compatible user interface.

\item Supports common Linux system call interfaces.

\end{itemize}

\subsection{Abstraction and Usage Model}
\lego\ exposes a distributed set of {\em virtual nodes}, or {\em \vnode}, to users.
From users' point of view, a \vnode\ is like a virtual machine. 
Multiple users can run in a \vnode\ and each user can run multiple processes.
Each \vnode\ has a unique ID, a unique virtual IP address, %({\em \vip}),
and its own storage mount point. % ({\em \vmount}).
\lego\ protects and isolates the resources given to each \vnode\ from others.
Internally, one \vnode\ can run on multiple \pcomponent{}s, multiple \mcomponent{}s,
and multiple \scomponent{}s.
At the same time, each hardware component can host resources for more than one \vnode.
The internal execution status is transparent to \lego\ users;
they do not know which physical components their applications run on.

With \splitkernel's design principle of components not being coherent,
\lego\ does not support writable shared memory across processors. %execute application threads that need to have shared write access to common memory.
\lego\ assumes that threads within the same process access shared memory
and threads belonging to different processes do not share writable memory,
and \lego\ makes scheduling decision based on this assumption (\S\ref{sec:lego:proc-scheduling}).
Applications that use shared writable memory across processes (\eg, with MAP\_SHARED)
will need to be adapted to use message passing across processes.
We made this decision because writable shared memory across processes is rare 
(we have not seen a single instance in the datacenter applications we studied),
and supporting it makes both hardware and software more complex 
(in fact, we have implemented this support but later decided not to include it because of its complexity).

One of the initial decisions we made when building \lego\ is to support the Linux system call interface 
and unmodified Linux ABI,
because doing so can greatly ease the adoption of \lego.
Distributed applications that run on Linux can seamlessly run on a \lego\ cluster
by running on a set of \vnode{}s. % and using their virtual IP addresses to communicate.

\subsection{Hardware Architecture}
\label{sec:lego:hardware}
{
\begin{figure}[th]
\begin{center}
\centerline{\includegraphics[width=0.8\textwidth]{lego/Figures/hwarch.pdf}}
\caption[\lego\ \pcomponent\ and \mcomponent\ Architecture.]{\lego\ \pcomponent\ and \mcomponent\ Architecture.}
\label{fig-lego-hw-arch}
\end{center}
\end{figure}
}

\lego\ \pcomponent, \mcomponent, and \scomponent\ are independent devices,
each having their own hardware controller and network interface (for \pcomponent, the hardware controller is the processor itself).
Our current hardware model uses CPU in \pcomponent, 
DRAM in \mcomponent, and SSD or HDD in \scomponent.
We leave exploring other hardware devices for future work.

To demonstrate the feasibility of hardware resource disaggregation,
we propose a \pcomponent{} and an \mcomponent\ architecture designed 
within today's network, processor, and memory performance and hardware constraints
(Figure~\ref{fig-lego-hw-arch}).

\noindent{\textit{\uline{Separating process and memory functionalities.}}}
\lego\ moves all hardware memory functionalities to \mcomponent{}s 
(e.g., page tables, TLBs) and leaves {\em only} caches at the \pcomponent{} side. 
With a clean separation of process and memory hardware units, 
the allocation and management of memory can be completely transparent to \pcomponent{}s.
Each \mcomponent{} can choose its own memory allocation technique
and virtual to physical memory address mappings (\eg, segmentation). 

\noindent{\textit{\uline{Processor virtual caches.}}}
After moving all memory functionalities to \mcomponent{}s,  
\pcomponent{}s will only see virtual addresses and have to use virtual memory addresses to access its caches. 
Because of this, \lego\ organizes all levels of \pcomponent{} caches as {\em virtual caches}~\cite{Goodman-ASPLOS87,Wang-ISCA89},
\ie, virtually-indexed and virtually-tagged caches.

A virtual cache has two potential problems, commonly known as synonyms and homonyms~\cite{CacheMemory82}.
Synonyms happens when a physical address maps to multiple virtual addresses (and thus multiple virtual cache lines) 
as a result of memory sharing across processes,
and the update of one virtual cache line will not reflect to other lines that share the data.
Since \lego\ does not allow writable inter-process memory sharing,
it will not have the synonym problem.
The homonym problem happens when two address spaces use the same virtual address for their own different data.
Similar to previous solutions~\cite{OVC}, we solve homonyms by storing an address space ID (ASID) with each cache line,
and differentiate a virtual address in different address spaces using ASIDs.

\noindent{\textit{\uline{Separating memory for performance and for capacity.}}}
Previous studies~\cite{Gao16-OSDI,GU17-NSDI} and our own show that today's network speed 
cannot meet application performance requirements if all memory accesses are across the network. 
Fortunately, many modern datacenter applications exhibit strong memory access temporal locality.
For example, we found 90\% of memory accesses in PowerGraph~\cite{Gonzalez12-OSDI} go to just 0.06\% of total memory
and 95\% go to 3.1\% of memory
(22\% and 36\% for TensorFlow~\cite{TensorFlow} respectively,
5.1\% and 6.6\% for Phoenix~\cite{Ranger07-HPCA}).
%PG 90% 0.0063G 95% 0.301G 100% 9.68G
%TF 90% 0.608G 95% 0.968G 100% 2.7G

With good memory-access locality, we propose to %separate hardware memory into two categories and organize them differently:
leave a small amount of memory (\eg, 4\GB) at each \pcomponent{}
and move most memory across the network (\eg, few TBs per \mcomponent{}).
\pcomponent{}s' local memory can be regular DRAM 
or the on-die HBM~\cite{HBM-JEDEC,Knights-Landing},
and \mcomponent{}s use DRAM or NVM.

Different from previous proposals~\cite{Lim09-disaggregate}, 
we propose to organize \pcomponent{}s' DRAM/HBM as cache rather than main memory
for a clean separation of process and memory functionalities.
We place this cache under the current processor Last-Level Cache (LLC)
and call it an extended cache, or {\em \excache}.
\excache\ serves as another layer in the memory hierarchy between LLC and memory across the network.
With this design, \excache\ can serve hot memory accesses fast, while \mcomponent{}s can provide the capacity applications desire. 

\excache\ is a virtual, inclusive cache,
and we use a combination of hardware and software to manage \excache.
Each \excache\ line has a (virtual-address) tag and two access permission bits (one for read/write and one for valid).
These bits are set by software when a line is inserted to \excache\ and checked by hardware at access time.
For best hit performance, the hit path of \excache\ is handled purely by hardware
--- the hardware cache controller maps a virtual address to an \excache\ set, 
fetches and compares tags in the set, and on a hit, fetches the hit \excache\ line.
Handling misses of \excache\ is more complex than with traditional CPU caches, 
and thus we use \lego\ to handle the miss path of \excache\ (see \S\ref{sec:lego:excachemgmt}).

Finally, we use a small amount of DRAM/HBM at \pcomponent{} for \lego' own kernel data usages,
accessed directly with physical memory addresses and managed by \lego. 
\lego\ ensures that all its own data fits in this space to avoid going to \mcomponent{}s.

With our design, \pcomponent{}s do not need any address mappings:
\lego\ accesses all \pcomponent{}-side DRAM/HBM using physical memory addresses
and does simple calculations to locate the \excache\ set for a memory access.
Another benefit of not handling address mapping at \pcomponent{}s and moving TLBs to \mcomponent{}s 
is that \pcomponent{}s do not need to access TLB or suffer from TLB misses,
potentially making \pcomponent{} cache accesses faster~\cite{Kaxiras-ISCA13}.
%We use software~\cite{softvm-HPCA97,Tsai-ISCA17} (\lego) to manage \excache\ and the kernel physical memory,
%although they can all be implemented in hardware too.

\subsection{Process Management}
The \lego\ {\em process \microos{}} runs in the kernel space of a \pcomponent\
and manages the \pcomponent's CPU cores and \excache. 
\pcomponent{}s run user programs in the user space.

\subsubsection{Process Management and Scheduling}
\label{sec:lego:proc-scheduling}
At every \pcomponent, \lego\ uses a simple local thread scheduling model 
that targets datacenter applications 
(we will discuss global scheduling in \S~\ref{sec:lego:grm}).
\lego\ dedicates a small amount of cores for kernel background threads 
(currently two to four)
and uses the rest of the cores for application threads.
When a new process starts, \lego\ uses a global policy to choose a \pcomponent{} for it (\S~\ref{sec:lego:grm}).
Afterwards, \lego\ schedules new threads the process spawns on the same \pcomponent{} 
by choosing the cores that host fewest threads.
After assigning a thread to a core, 
we let it run to the end with no scheduling or kernel preemption under common scenarios.
For example, we do not use any network interrupts 
and let threads busy wait on the completion of outstanding network requests, 
since a network request in \lego\ is fast 
(\eg, fetching an \excache\ line from an \mcomponent\ takes around 6.5\mus).
\lego\ improves the overall processor utilization in a disaggregated cluster,
since it can freely schedule processes on any \pcomponent{}s without considering memory allocation.
Thus, we do not push for perfect core utilization when scheduling individual threads
and instead aim to minimize scheduling and context switch performance overheads.
Only when a \pcomponent{} has to schedule 
more threads than its cores will
\lego\ start preempting threads on a core.

\subsubsection{\excache\ Management}
\label{sec:lego:excachemgmt}
\lego\ process \microos\ configures and manages \excache.
During the \pcomponent{}'s boot time, \lego\ configures the set associativity of \excache\
and its cache replacement policy.
While \excache\ hit is handled completely in hardware, 
\lego\ handles misses in software.
When an \excache\ miss happens, 
the process \microos\ fetches the corresponding line from an \mcomponent\ and inserts it to \excache.
If the \excache\ set is full, the process \microos\ first evicts a line in the set.
It throws away the evicted line if it is clean
and writes it back to an \mcomponent{} if it is dirty.
\lego\ currently supports two eviction policies: FIFO and LRU.
For each \excache\ set, \lego\ maintains a FIFO queue (or an approximate LRU list)
and chooses \excache\ lines to evict based on the corresponding policy (see \S\ref{sec:lego:procimpl} for details).

\subsubsection{Supporting Linux Syscall Interface}
One of our early decisions is to support Linux ABIs for backward compatibility
and easy adoption of \lego.
A challenge in supporting the Linux system call interface is that 
many Linux syscalls are associated with {\em states},
information about different Linux subsystems that is stored with each process 
and can be accessed by user programs across syscalls.
For example, Linux records the states of a running process' open files, socket connections, and several other entities,
and it associates these states with file descriptors ({\em fd}s) that are exposed to users.
In contrast, \lego\ aims at the clean separation of OS functionalities.
With \lego' stateless design principle, each component only stores information about its own resource
and each request across components contains all the information that the destination component needs to handle the request.
To solve this discrepancy between the Linux syscall interface and \lego' design, 
we add a layer on top of \lego' core process \microos\ at each \pcomponent\ to store Linux states
and translate these states and the Linux syscall interface to \lego' internal interface.

\subsection{Memory Management}

We use \mcomponent{}s for three types of data:
anonymous memory (\ie, heaps, stacks), 
memory-mapped files, and storage buffer caches.
The \lego\ {\em memory \microos{}}
manages both the virtual and physical memory address spaces,
their allocation, deallocation, and memory address mappings.
It also performs the actual memory read and write.
No user processes run on \mcomponent{}s 
and they run completely in the kernel mode
(same is true for \scomponent{}s). 

\lego\ lets a process address space span multiple \mcomponent{}s
to achieve efficient memory space utilization and high parallelism.
Each application process uses one or more \mcomponent{}s to host its data
and a {\em home \mcomponent},
an \mcomponent\ that initially loads the process, 
accepts and oversees all system calls related to virtual memory space management
(\eg, \brk, \mmap, \munmap, and \mremap).
\lego\ uses a global memory resource manager ({\em \gmm}) to assign a home \mcomponent{} to each new process at its creation time.
A home \mcomponent\ can also host process data.

\subsubsection{Memory Space Management}
\noindent{\textit{\uline{Virtual memory space management.}}}
We propose a two-level approach to manage distributed virtual memory spaces,
where the home \mcomponent\ of a process makes coarse-grained, high-level virtual memory allocation decisions
and other \mcomponent{}s perform fine-grained virtual memory allocation.
This approach minimizes network communication during both normal memory accesses and virtual memory operations,
while ensuring good load balancing and memory utilization.
Figure~\ref{fig-dist-vma} demonstrates the data structures used. % in virtual memory space management.

At the higher level, we split each virtual memory address space into coarse-grained, fix-sized {\em virtual regions},
or {\em \vregion{}s} (\eg, of 1\GB).
Each \vregion\ that contains allocated virtual memory addresses (an active \vregion) is {\em owned} by an \mcomponent{}.
The owner of a \vregion\ handles all memory accesses and virtual memory requests within the \vregion.

{
\begin{figure}[th]
\begin{minipage}{\figWidth}
\begin{center}
\centerline{\includegraphics[width=2.8in]{Figures/dist-vma.pdf}}
%\vspace{-0.1in}
\mycaption{fig-dist-vma}{Distributed Memory Management.}
{
}
\end{center}
\end{minipage}
\vspace{-0.15in}
\end{figure}
}

The lower level stores user process virtual memory area ({\em vma}) information,
such as virtual address ranges and permissions, in {\em vma trees}.
The owner of an active \vregion\ stores a vma tree for the \vregion,
with each node in the tree being one vma.
A user-perceived virtual memory range can split across multiple \mcomponent{}s,
but only one \mcomponent{} owns a \vregion.

\vregion\ owners perform the actual virtual memory allocation and vma tree set up.
A home \mcomponent{} can also be the owner of \vregion{}s,
but the home \mcomponent{} does not maintain any information about memory that belongs to \vregion{}s owned by other \mcomponent{}s.
It only keeps the information of which \mcomponent{} owns a \vregion\ (in a {\em \vregion\ array})
and how much free virtual memory space is left in each \vregion.
These metadata can be easily reconstructed if a home \mcomponent{} fails.

When an application process wants to allocate a virtual memory space,
the \pcomponent{} forwards the allocation request 
to its home \mcomponent{} (\circled{1} in Figure~\ref{fig-dist-vma}).
The home \mcomponent{} uses its stored information of available virtual memory space in \vregion{}s
to find one or more \vregion{}s that best fit the requested amount of virtual memory space.
If no active \vregion\ can fit the allocation request, the home \mcomponent{} makes a new \vregion\ active and 
contacts the \gmm\ (\circled{2} and \circled{3}) to find a candidate \mcomponent{} to own the new \vregion.
\gmm\ makes this decision based on available physical memory space and access load on different \mcomponent{}s (\S~\ref{sec:lego:grm}).
If the candidate \mcomponent\ is not the home \mcomponent{}, the home \mcomponent{} next forwards the request to that \mcomponent\ (\circled{4}),
which then performs local virtual memory area allocation and sets up the proper vma tree. 
Afterwards, the \pcomponent{} directly sends memory access requests to the owner of the \vregion\ where the memory access falls into
(\eg, \circled{a} and \circled{c} in Figure~\ref{fig-dist-vma}).


\lego' mechanism of distributed virtual memory management is efficient and it cleanly separates memory operations from \pcomponent{}s.
\pcomponent{}s hand over all memory-related system call requests to \mcomponent{}s
and only cache a copy of the \vregion\ array for fast memory accesses.
To fill a cache miss or to flush a dirty cache line, 
a \pcomponent{} looks up the cached \vregion\ array to find its owner \mcomponent{} and sends the request to it.

\noindent{\textit{\uline{Physical memory space management.}}}
Each \mcomponent\ manages the physical memory allocation for data that falls into the
\vregion\ that it owns.
Each \mcomponent{} can choose their own way of physical memory allocation
and own mechanism of virtual-to-physical memory address mapping.


\subsubsection{Optimization on Memory Accesses}
\label{sec:lego:zerofill}
With our strawman memory management design, 
all \excache\ misses will go to \mcomponent{}s.
We soon found that a large performance overhead in running real applications 
is caused by filling empty \excache, \ie, {\em cold misses}.
To reduce the performance overhead of cold misses, we propose a technique 
to avoid accessing \mcomponent\ on first memory accesses.

The basic idea is simple: since the initial content of anonymous memory 
(non-file-backed memory) is zero, %undefined and can be any data, 
\lego\ can directly allocate a cache line with empty content
in \excache\ for the first access to 
anonymous memory instead of going to \mcomponent\
(we call such cache lines {\em p-local lines}).
When an application creates a new anonymous memory region, the process \microos\ records its address range and permission.
The application's first access to this region will be an \excache\ miss and it will trap to \lego.
\lego\ process \microos\ then allocates an \excache\ line, fills it with zeros, 
and sets its R/W bit according to the recorded memory region's permission.
Before this p-local line is evicted, it only lives in the \excache.
No \mcomponent{}s are aware of it or will allocate physical memory or a virtual-to-physical memory mapping for it.
When a p-local cache line becomes dirty and needs to be flushed, 
the process \microos\ sends it to its owner \mcomponent, which then
allocates physical memory space and establishes a virtual-to-physical memory mapping.
Essentially, \lego\ {\em delays physical memory allocation until write time}.
Notice that it is safe to only maintain p-local lines at a \pcomponent{} \excache\ 
without any other \pcomponent{}s knowing them, 
since \pcomponent{}s in \lego\ do not share any memory
and other \pcomponent{}s will not access this data.

\subsection{Storage Management}
\lego\ supports a hierarchical file interface that is backward compatible with POSIX 
through its \vnode\ abstraction. 
Users can store their directories and files under their \vnode{}s' mount points
and perform normal read, write, and other accesses to them.

\lego\ implements core storage functionalities at \scomponent{}s.
To cleanly separate storage functionalities, \lego\ uses a stateless storage server design, 
where each I/O request to the storage server contains all the information needed to 
fulfill this request, \eg, full path name, absolute file offset,
similar to the server design in NFS v2~\cite{Sandberg-NFS-85}.

While \lego\ supports a hierarchical file use interface,
internally, \lego\ storage \microos\ treats (full) directory and file paths just as unique names of a file
and place all files of a \vnode\ under one internal directory at the \scomponent{}.
To locate a file, \lego\ storage \microos\ maintains a simple hash table with the full paths of files (and directories) as keys.
From our observation, most datacenter applications only have a few hundred files or less.
Thus, a simple hash table for a whole \vnode\ is sufficient to achieve good lookup performance.
Using a non-hierarchical file system implementation largely reduces the complexity of \lego' file system,
making it possible for a storage \microos\ to fit in storage devices controllers that have limited processing power~\cite{Willow}.

\lego\ places the storage buffer cache at \mcomponent{}s
rather than at \scomponent{}s, because \scomponent{}s can only host a limited amount of internal memory.
\lego\ memory \microos\ manages the storage buffer cache by simply performing insertion, lookup, and deletion of buffer cache entries.
For simplicity and to avoid coherence traffic, we currently place the buffer cache of one file
under one \mcomponent{}.
When receiving a file read system call, the \lego\ process \microos\ first uses its extended Linux state layer to 
look up the full path name, then passes it with the requested offset and size to the \mcomponent\ that holds the file's buffer cache.
This \mcomponent\ will look up the buffer cache and returns the data to \pcomponent\ on a hit.
On a miss, \mcomponent\ will forward the request to the \scomponent\ that stores the file, 
which will fetch the data from storage device and return it to the \mcomponent.
The \mcomponent\ will then insert it into the buffer cache and returns it to the \pcomponent.
Write and fsync requests work in a similar fashion.

\subsection{Global Resource Management}
\label{sec:lego:grm}
\lego\ uses a two-level resource management mechanism.
At the higher level, \lego\ uses three global resource managers for process, memory, and storage resources, 
{\em \gpm, \gmm}, and {\em \gsm}.
These global managers perform coarse-grained global resource allocation and load balancing,
and they can run on one normal Linux machine.
Global managers only maintain approximate resource usage and load information.
They update their information either when they make allocation decisions 
or by periodically asking \microos{}s in the cluster.
At the lower level, each \microos\ can employ its own policies and mechanisms to manage its local resources.

For example, process \microos{}s allocate new threads locally 
and only ask \gpm\ when they need to create a new process.
\gpm\ chooses the \pcomponent{} that has the least amount of threads based on its maintained approximate information.
Memory \microos{}s allocate virtual and physical memory space on their own.
Only home \mcomponent{} asks \gmm\ when it needs to allocate a new \vregion.
\gmm\ maintains approximate physical memory space usages and memory access load by periodically asking \mcomponent{}s
and chooses the memory with least load among all the ones that have at least \vregion\ size of free physical memory.

\lego\ decouples the allocation of different resources and 
can freely allocate each type of resource from a pool of components.
Doing so largely improves resource packing compared to a monolithic server cluster
that packs all type of resources a job requires within one physical machine.
Also note that \lego\ allocates hardware resources only {\em on demand}, 
\ie, when applications actually create threads or access physical memory.
This on-demand allocation strategy further improves \lego' resource packing efficiency
and allows more aggressive over-subscription in a cluster.

\subsection{Reliability and Failure Handling}
\label{sec:lego:failure}
After disaggregation, \pcomponent{}s, \mcomponent{}s, and \scomponent{}s can all fail independently.
Our goal is to build a reliable disaggregated cluster that has the same or lower application failure rate
than a monolithic cluster.
As a first (and important) step towards achieving this goal, %building a reliable disaggregated cluster,
we focus on providing memory reliability by handling \mcomponent\ failure in the current version of \lego\ because of three observations.
First, when distributing an application's memory to multiple \mcomponent{}s, 
the probability of memory failure increases and not handling \mcomponent\ failure will cause applications to fail more often 
on a disaggregated cluster than on monolithic servers.
Second, since most modern datacenter applications
already provide reliability to their distributed storage data %(usually through some form of redundancy)
and the current version of \lego\ does not split a file across \scomponent,
we leave providing storage reliability to applications.
Finally, since \lego\ does not split a process across \pcomponent{}s,
the chance of a running application process being affected by the failure of a \pcomponent\ is similar to 
one affected by the failure of a processor in a monolithic server.
Thus, we currently do not deal with \pcomponent\ failure and leave it for future work.

A naive approach to handle memory failure is to perform a full replication of memory content over two or more \mcomponent{}s.
This method would require at least 2\x\ memory space,
making the monetary and energy cost of providing reliability prohibitively high (the same reason why RAMCloud~\cite{Ongaro11-RamCloud} does not replicate in memory).
Instead, we propose a space- and performance-efficient approach to provide in-memory data reliability in a best-effort way.
Further, since losing in-memory data will not affect user persistent data,
we propose to provide memory reliability in a best-effort manner.

We use one primary \mcomponent, one secondary \mcomponent, and a backup file in \scomponent\ for each vma.
A \mcomponent{} can serve as the primary for some vma and the secondary for others.
The primary stores all memory data and metadata.
\lego\ maintains a small append-only log at the secondary \mcomponent{}
and also replicates the vma tree there.
When \pcomponent{} flushes a dirty \excache\ line, 
\lego\ sends the data to both primary and secondary in parallel (step \circled{a} and \circled{b} in Figure~\ref{fig-dist-vma})
and waits for both to reply (\circled{c} and \circled{d}).
In the background, the secondary \mcomponent\ flushes the backup log to a \scomponent{},
which writes it to an append-only file.

If the flushing of a backup log to \scomponent\ is slow and the log is full, 
we will skip replicating application memory.
If the primary fails during this time, \lego\ simply reports an error to application.
Otherwise when a primary \mcomponent\ fails, we can recover memory content 
by replaying the backup logs on \scomponent\ and in the secondary \mcomponent.
When a secondary \mcomponent\ fails, we do not reconstruct anything 
and start replicating to a new backup log on another \mcomponent{}.


\section{Distributed SuperNIC}
\label{sec:dist}

The design discussion so far focused on a single \snic. To enable better consolidation and network as a service, we 
%When we use a single \snic\ to consolidate \nt{}s of its connected endpoints, the \snic\ needs to be provisioned with the aggregated peak load of these endpoints.
%To further reduce cost, we 
build a rack-scale distributed \snic\ platform that enables one \snic\ to use other \snic{}s' resources.
With this platform, a rack's \snic{}s can collectively provision for the maximum aggregated load of all the endpoints in the rack.

As discussed in \S\ref{sec:overview}, we support two types of \snic\ pool topology.
For the switch-attached topology, the ToR switch serves as the load balancer across different \snic{}s.
It also decides which \snic\ to launch a new instance of an \nt\ with the goal of balancing traffic and efficiently utilizing \snic\ hardware resources.
%Specifically, it chooses the \snic\ that has enough free regions and is lightly loaded to launch new instances of \nt\ chains.
%Afterwards, the switch simply directs incoming flows to the \snic{}s that contain their target \nt{}s.
Supporting the intermediate-pool topology where the ToR switch cannot perform the above tasks is more complex. Below we discuss our design for it.

%\boldpara{Distributed Control Plane.}~~
SoftCores on the \snic{}s in the intermediate pool form a distributed control plane. 
They communicate with each other to exchange metadata and cooperate in performing distributed tasks. % like \nt\ migration and memory swapping.
We choose this peer-to-peer design instead of a centralized one, because the latter requires another global manager and adds complexity and cost to the rack architecture. %\zac{I am not sure about this argument -- I have heard that other places (e.g. google) have used the centralized manager architecture because it's easier to build and deploy than P2P ones. I personally don't buy this argument either. Do you have stronger support?}
Every \snic\ collects its FPGA space, on-board memory, and port bandwidth consumption, and it periodically sends this information to all the other \snic{}s in the rack.
Each \snic\ thus has a global view of the rack and can redirect traffic to other \snic{}s if it is overloaded.
%make decisions like \nt\ migration independently.
To redirect traffic, the \snic's SoftCore sets a rule in the parser MAT to forward certain packets (\eg, ones to be processed by an \nt\ chain on another \snic) to the remote \snic.

%\notearvind{We could also talk about something more basic - if the same NT is loaded on many snics, we can balance the load across all instances and achieve good consolidation. Maybe that would be easier for reviewers to accept before we talk about NT migration.}

%\boldpara{\nt\ Migration.}~~
If an \snic\ is overloaded and no other \snic{}s currently have the \nt\ chain that needs to be launched, the \snic\ tries to launch the chain at another \snic.
Specifically, the \snic's SoftCore first identifies the set of \snic{}s in the same rack that have available resources to host the \nt\ chain.
Among them, it picks one that is closest in distance to it (\ie, fewest hops).
The \snic's SoftCore then sends the bitstreams of the \nt{} chain to this picked remote \snic, which launches the chain in one of its own free regions.
When the original \snic\ has a free region, it moves back the migrated \nt\ chain. 
%It does so by first launching the \nt\ chain locally, then removing the MAT tunneling rule, and finally instructing the remote \snic\ to remove its \nt\ chain.
If the \nt\ chain is stateful, then the SoftCore manages a state migration process after launching the \nt\ chain locally, by first pausing new traffic, then migrating the \nt's states (if any) from the remote \snic\ to the local \snic. %and finally removing the MAT rule.


\section{Case Studies}
\label{sec:snic:application}

We now present two use cases of \snic\ that we implemented, one for disaggregated memory and one for regular servers.

\subsection{Disaggregated Key-Value Store}
\label{sec:snic:kvstore}
We first demonstrate the usage of \snic\ in a disaggregated environment by adapting a
recent open-source FPGA-based disaggregated memory device called {\em Clio}~\cite{Clio}.
The original Clio device hosts standard physical and link layers, a Go-Back-N reliable transport, and a system that maps keys to physical addresses of the corresponding values.
Clients running at regular servers send key-value load/store/delete requests to Clio devices over the network.
When porting to \snic, we do not change the client-side or Clio's core key-value mapping functionality.

\bolditpara{Disaggregating transport.}~~
The Go-Back-N transport consumes a fair amount of on-chip resources %ß(5.8\% LUTs and 2.6\% BRAM of the Clio device and 
(roughly the same amount as Clio's core key-value functionality~\cite{clio-arxiv}).
%(mainly on-chip memory used to store states for retransmission). 
We move the Go-Back-N stack from multiple Clio devices to an \snic\ and consolidate them by handling the aggregated load.
After moving the Go-Back-N stack, we extend each Clio device's link layer to a reliable one (\S\ref{sec:snic:overview}).

\bolditpara{Disaggregating KV-store-specific functionalities.}~~
A unique opportunity that \snic\ offers is its centralized position when connecting a set of endpoints, which users could potentially use to more efficiently coordinate the endpoints.
We explore this opportunity by building a replication service and a caching service as two \nt{}s in the \snic.
%that connects the Clio devices.

For \textbf{replication}, the client sends a replicated write request with a replication degree $K$, which the \snic\ handles by replicating the data and sending them to $K$ Clio devices. 
In comparison, the original Clio client needs to send $K$ copies of data to $K$ Clio devices or send one copy to a primary device, which then sends copies to the secondary device(s).
The former increases the bandwidth consumption at the client side, and the latter increases end-to-end latency.

For \textbf{caching}, the \snic\ maintains recently written/read key-value pairs in a small buffer. It checks this cache on every read request. If there is a cache hit, the \snic\ directly returns the value to the client, avoiding the cost of accessing Clio devices. Our current implementation that uses simple FIFO replacement already yields good results. Future improvements like LRU could perform even better.

\subsection{Virtual Private Cloud}
\label{sec:snic:vpc}

Cloud vendors offer Virtual Private Cloud (VPC) for customers to have an isolated network environment where their traffic is not affected by others and where they can deploy their own network functions such as firewall, network address translation (NAT), and encryption.
Today's VPC functionalities are implemented either in software~\cite{andromeda-google-nsdi18,ovs-nsdi15,ovs-sigcomm21} or offloaded to specialized hardware at the server~\cite{vfp-nsdi17,SmartNIC-nsdi18,aws-nitro}.
As cloud workloads experience dynamic loads and do not always use all the network functions (\S\ref{sec:snic:motivation}), VPC functionalities are a good fit for offloading to \snic.
Our baseline here is regular servers running Open vSwitch (OVS) with three NFs, firewall, NAT, and AES encryption/decryption. %Both senders and receivers servers employ the same setting and are connected to a physical switch directly.
We connect \snic{}s to both sender and receiver servers and then offload these three NFs to each \snic\ as one \nt\ chain. 

{
\begin{figure*}[th]
\begin{center}
\centerline{\includegraphics[width=0.5\textwidth]{clio/Figures/g_plot_scalability_conn.pdf}}
\mycaption{fig-conn}{Process (Connection) Scalability.}
{
}
\end{center}
\end{figure*}
}
{
\begin{figure*}[h]
\begin{center}
\centerline{\includegraphics[width=0.5\textwidth]{clio/Figures/g_plot_scalability_pte.pdf}}
\mycaption{fig-pte-mr}{PTE and MR Scalability.}
{
RDMA fails beyond $2^{18}$ MRs. 
}
\end{center}
\end{figure*}
}
{
\begin{figure*}[h]
\begin{center}
\centerline{\includegraphics[width=0.5\textwidth]{clio/Figures/g_plot_latency_comparison.pdf}}
\mycaption{fig-miss-hit}{Comparison of TLB Miss and page fault.}
{
\sys-ASIC are projected values of TLB hit.
}
\end{center}
\end{figure*}
}
{
\begin{figure*}[h]
\begin{center}
\centerline{\includegraphics[width=0.5\textwidth]{clio/Figures/clio_rdma_lat_cdf.pdf}}
\mycaption{fig-tail-latency}{Latency CDF.}
{
}
\end{center}
\end{figure*}
}
{
\begin{figure*}[th]
\begin{center}
\centerline{\includegraphics[width=0.5\textwidth]{clio/Figures/g_plot_throughput.pdf}}
\mycaption{fig-read-write-throughput}{End-to-End Goodput.}
{
1\KB\ requests. % between 1 \CN\ and 1 \MN.
}
\end{center}
\end{figure*}
}
{
\begin{figure*}[h]
\begin{center}
\centerline{\includegraphics[width=0.5\textwidth]{clio/Figures/g_plot_onboard_throughput.pdf}}
\mycaption{fig-onboard-throughput}{On-board Goodput.}
{
FPGA test module generates requests at maximum speed.
}
\end{center}
\end{figure*}
}
{
\begin{figure*}[h]
\begin{center}
\centerline{\includegraphics[width=0.5\textwidth]{clio/Figures/g_plot_read_latency.pdf}}
\mycaption{fig-read-lat}{Read Latency.}
{
HERD-BF: HERD running on BlueField. %SmartNIC.
}
\end{center}
\end{figure*}
}
{
\begin{figure*}[h]
\begin{center}
\centerline{\includegraphics[width=0.5\textwidth]{clio/Figures/g_plot_write_latency.pdf}}
\mycaption{fig-write-lat}{Write Latency.}
{
Clover requires $\ge$ 2 RTTs for write.
}
\end{center}
\end{figure*}
}

\section{Evaluation}
\label{sec:clio:results}


Our evaluation reveals the scalability, throughput, median and tail latency, energy and resource consumption of \sys.
%, and how it compares with state-of-the-art systems. 
We compare \sys's end-to-end performance with industry-grade NICs (ASIC) and well-tuned RDMA-based software systems.
All \sys's results are FPGA-based, which would be improved with ASIC implementation.
%Nonetheless, \sys\ significantly outperforms RDMA on scalability and tail latency, while being similar on other measurements.

\ulinebfpara{Environment.}
We evaluated \sys\ in our local cluster of four \CN{}s and four \MN{}s (Xilinx ZCU106 boards),
%\footnote{Unfortunately, our process of purchasing and setting up a bigger cluster was significantly delayed because of COVID-19},
all connected to an Nvidia 40\Gbps\ VPI switch.
Each \CN\ is a Dell PowerEdge R740 server equipped with a Xeon Gold 5128 CPU and a 40\Gbps\ Nvidia ConnectX-3 NIC,
with two of them also having an Nvidia BlueField SmartNIC~\cite{BlueField}.
We also include results from CloudLab~\cite{CloudLab} with the Nvidia ConnectX-5 NIC.


\subsection{Basic Microbenchmark Performance}

{
\begin{figure*}[th]
\begin{minipage}{\figWidthSix}
\begin{center}
\centerline{\includegraphics[width=\columnwidth]{Figures/g_plot_alloc_free.pdf}}
\vspace{-0.1in}
\captionsetup{width=.9\columnwidth}
\mycaption{fig-alloc-free}{Alloc/Free Latency.}
{
ODP means On-Demand-Paging mode
}
\end{center}
\end{minipage}
\begin{minipage}{\figWidthSix}
\begin{center}
\centerline{\includegraphics[width=\columnwidth]{Figures/g_plot_alloc_conflict.pdf}}
\vspace{-0.1in}
\captionsetup{width=.9\columnwidth}
\mycaption{fig-alloc-conflict}{Alloc Retry Rate.}
{
%Alloc's number of retries when vary physical memory utilization.
}
\end{center}
\end{minipage}
\begin{minipage}{\figWidthSix}
\begin{center}
\centerline{\includegraphics[width=\columnwidth]{Figures/g_plot_latency_breakdown.pdf}}
\vspace{-0.1in}
\captionsetup{width=.9\columnwidth}
\mycaption{fig-lat-break}{Latency Breakdown.}
{
Breakdown of time spent at \sysboard.
}
\end{center}
\end{minipage}
\begin{minipage}{\figWidthSix}
\begin{center}
\centerline{\includegraphics[width=\columnwidth]{Figures/g_plot_ycsb_mn.pdf}}
\vspace{-0.1in}
\captionsetup{width=.9\columnwidth}
\mycaption{fig-ycsb-mn}{\syskv\ Scalability against \MN{}s.}
{
}
\end{center}
\end{minipage}
\vspace{-0.15in}
\end{figure*}
}

{
\begin{figure*}[th]
\begin{minipage}{\figWidthSix}
\begin{center}
\centerline{\includegraphics[width=\columnwidth]{Figures/g_plot_image_compression.pdf}}
\vspace{-0.1in}
\captionsetup{width=.9\columnwidth}
\mycaption{fig-photo}{Image Compression.}
{
}
\end{center}
\end{minipage}
\begin{minipage}{\figWidthSix}
\begin{center}
\centerline{\includegraphics[width=\columnwidth]{Figures/g_plot_radix_tree.pdf}}
\vspace{-0.1in}
\captionsetup{width=.9\columnwidth}
\mycaption{fig-radix}{Radix Tree Search Latency.}
{
}
\end{center}
\end{minipage}
\begin{minipage}{\figWidthSix}
\begin{center}
\centerline{\includegraphics[width=\columnwidth]{Figures/g_plot_ycsb_cn.pdf}}
\vspace{-0.1in}
\captionsetup{width=.9\columnwidth}
\mycaption{fig-kvstore}{Key-Value Store YCSB Latency.}
{
}
\end{center}
\end{minipage}
%\if 0 g_plot_ycsb_mn
%\fi
\begin{minipage}{\figWidthSix}
\begin{center}
\centerline{\includegraphics[width=\columnwidth]{Figures/g_plot_mvstore.pdf}}
\vspace{-0.1in}
\captionsetup{width=.9\columnwidth}
\mycaption{fig-mvstore}{\sysmv\ Object Read/Write Latency.}
{
}
\end{center}
\end{minipage}
%\vspace{-0.15in}
\end{figure*}
}


\ulinebfpara{Scalability.}
We first compare the scalability of \sys\ and RDMA.
Figure~\ref{fig-conn} measures the latency of \sys\ and RDMA as the number of client processes increases.
For RDMA, each process uses its own QP.
Since \sys\ is connectionless, it scales perfectly with the number of processes.
RDMA scales poorly with its QP, and the problem persists with newer generations of RNIC,
which is also confirmed by our previous works~\cite{Pythia,Storm}.

Figure~\ref{fig-pte-mr} evaluates the scalability with respect to PTEs and memory regions.
For the memory region test, we register multiple MRs using the same physical memory for RDMA.
For \sys, we map a large range of VAs (up to 4\TB) to a small physical memory space, as our testbed only has 2\GB\ physical memory.
However, the number of PTEs and the amount of processing needed are the same for \sysboard\ as if it had a real 4\TB\ physical memory.
Thus, this workload stress tests \sysboard's scalability.
%For \sys\ (which gets rid of the MR concept), we use multiple processes to share the same memory,
%resulting in one PTE per process.
RDMA's performance starts to degrade when there are more than $2^8$ (local cluster) or $2^{12}$ (CloudLab),
and the scalability wrt MR is worse than wrt PTE.
In fact, RDMA fails to run beyond $2^{18}$ MRs.
In contrast, \sys\ scales well and never fails (at least up to 4\TB\ memory).
It has two levels of latency that are both stable: a lower latency below $2^4$ for TLB hit and a higher latency above $2^4$ for TLB miss (which always involves one DRAM access).
A \sysboard\ could use a larger TLB if optimal performance is desired.

These experiments confirm that \textbf{\sys\ can handle thousands of concurrent clients and TBs of memory}.



\ulinebfpara{Latency variation.}
Figure~\ref{fig-miss-hit} plots the latency of reading/writing 16\,B data 
when the operation results in a TLB hit, a TLB miss, a first-access page fault, and MR miss (for RDMA only, when the MR metadata is not in RNIC).
RDMA's performance degrades significantly with misses.
Its page fault handling is extremely slow (16.8\ms).
We confirm the same effect on CloudLab with the newer ConnectX-5 NICs.
\sys\ only incurs a small TLB miss cost and \textbf{no additional cost of page fault handling}.

We also include a projection of \sys's latency if it was to be implemented using a real ASIC-based \sysboard.
Specifically, we collect the latency breakdown of time spent on the network wire and at \CN, time spent on third-party FPGA IPs,
number of cycles on FPGA, and time on accessing on-board DRAM.
We maintain the first two parts, scale the FPGA part to ASIC's frequency (2\,GHz), use DDR access time collected on our server to replace the access time to on-board DRAM (which 
goes through a slow board memory controller).
This estimation is conservative, as a real ASIC implementation of the third-party IPs would make the total latency lower.
Our estimated read latency is better than RDMA, while write latency is worse.
We suspect the reason being Nvidia RNIC's optimization of replying a write before it is fully written to DRAM, which \sys\ could also potentially adopt.

Figure~\ref{fig-tail-latency} plots the request latency CDF of continuously running read/write 16\,B data while not triggering page faults.
Even without page faults, \sys\ has much less latency variation and a much shorter tail than RDMA.
%Thanks to our bounded address translation and deterministic hardware design, \sys\ has much less latency variation and a much shorter tail than RDMA.

{
\begin{figure*}[th]
\begin{center}
\centerline{\includegraphics[width=0.5\textwidth]{clio/Figures/g_plot_dp.pdf}}
\mycaption{fig-dataframe}{Select-Aggregate-Shuffle.}
{
Y axis starts at 4 sec. 
CN represents computation done at \CN.
%, as histogram  time is the same.
}
\end{center}
\end{figure*}
}
{
\begin{figure*}[h]
\begin{center}
\centerline{\includegraphics[width=0.5\textwidth]{clio/Figures/g_plot_ycsb_energy.pdf}}
\mycaption{fig-energy}{Energy Comparison.}
{
Darker/lighter shades represent energy spent at \MN{}s and \CN{}s.
}
\end{center}
\end{figure*}
}
{
\begin{table}\small
\begin{center}
\begin{center}
\begin{tabular}{ p{1.2in} | p{0.5in} |p{0.6in} }
 & \textbf{Logic} & \textbf{Memory} \\
\textbf{System/Module} & \textbf{(LUT)} & \textbf{(BRAM)} \\
\hline
\hline
StRoM-RoCEv2 & 39\% & 76\% \\
Tonic-SACK & 48\% & 40\% \\
\hline
\sys\ (Total) & 31\% & 31\% \\
VirtMem & 5.5\% & 3\% \\
NetStack & 2.3\% & 1.7\% \\
\hline
Go-Back-N & 5.8\% & 2.6\% \\
\end{tabular}
\end{center}
\mycaption{fig-fpga-resource}{Clio FPGA Utilization.}
{
}
\end{center}
\end{table}
}

\ulinebfpara{Read/write throughput.}
We measure \sys's throughput by varying the number of concurrent client threads (Figure~\ref{fig-read-write-throughput}).
\sys's default asynchronous APIs quickly reach the line rate of our testbed (9.4\Gbps\ maximum throughput).
Its synchronous APIs could also reach line rate fairly quickly.

Figure~\ref{fig-onboard-throughput} measures the maximum throughput of \sys's FPGA implementation without the bottleneck of the board's 10\Gbps\ port, by generating traffic on board.
Both read and write can reach more than 110\Gbps\ when request size is large.
Read throughput is lower than write when request size is smaller.
We found the throughput bottleneck to be the third-party non-pipelined DMA IP
(which could potentially be improved).

\ulinebfpara{Comparison with other systems.}
We compare \sys\ with native one-sided RDMA, Clover~\cite{Tsai20-ATC}, HERD~\cite{Kalia14-RDMAKV}, and LegoOS~\cite{Shan18-OSDI}.
We ran HERD on both CPU and BlueField (HERD-BF).
%Native RDMA can be considered as a baseline (optimal performance but low-level, restrictive interface).
Clover is a passive disaggregated persistent memory system which we adapted as a passive disaggregated memory (PDM) system.
HERD is an RDMA-based system that supports a key-value interface with an RPC-like architecture.
LegoOS builds its virtual memory system in software at \MN.
%It uses one RDMA read for its read and one RDMA write plus one 
%it can be considered as a software-based active disaggregated memory system. 

\sys's performance is similar to HERD and close to native RDMA.
%\sys's write performance is better than Clover and similar to HERD. %but has a constant overhead over native RDMA.
Clover's write is the worst because it uses at least 2 RTTs for writes to deliver its consistency guarantees without any processing power at \MN{}s.
HERD-BF's latency is much higher than when HERD runs on CPU
due to the slow communication between BlueField's ConnectX-5 chip and ARM processor chip.
LegoOS's latency is almost two times higher than \sys's when request size is small.
In addition, from our experiment, LegoOS can only reach a peak throughput of 77\Gbps, while \sys\ can reach 110\Gbps.
LegoOS' performance overhead comes from its software approach, demonstrating the necessity of a hardware-based solution like \sys.
%due to the slow communication between BlueField's Connect-X5 chip and ARM processor %chip..
%\sys's write overhead can be attributed to \fixme{XXX}.

\ulinebfpara{Allocation performance.}
Figure~\ref{fig-alloc-free} shows \sys's VA and PA allocation and RDMA's MR registration performance.
%Physical memory allocation includes the time to perform an allocation with the buddy algorithm and to insert the allocated address into the free page list.
%It is very fast, indicating that our asynchronous free physical page generation could keep up with most workloads' page fault speed.
%Virtual memory allocation and free (measured from client on \CN) are slower,
\sys's PA allocation takes less than 20\mus, and the VA allocation is much faster than RDMA MR registration,
although both get slower with larger allocation/registration size.
%since these operations involve the costly crossing between FPGA and ARM.
%They are also slower with larger sizes, as searching the VMA tree for a big free region takes more time.
Figure~\ref{fig-alloc-conflict} shows the number of retries at allocation time with three allocation sizes as the physical memory fills up.
%running at on-board ARM processor.
%The hash-based page table is proportional to the physical memory size. 
%Hence higher its utilization, higher the overflow probability therefore higher number of retries.
%The page table has 2\x\ extra slots by default.
There is no retry when memory is below half utilized. Even when memory is close to full, there are at most 60 retries per allocation request, with roughly 0.5\ms\ per retry. This confirms that our design of avoiding hash overflows at allocation time is practical.
%co-design of overflow-free hash-based page table and allocation retry scheme is practical.


\ulinebfpara{Close look at \sysboard{} components.}
To further understand \sys's performance, % and to determine the reason for worse large-read performance,
we profile different parts of \sys's processing for read and write of 4\,B to 1\KB.
\syslib\ adds a very small overhead (250\ns\ in total), 
thanks to our efficient threading model and network stack implementation.
Figure~\ref{fig-lat-break} shows the latency breakdown at \sysboard.
Time to fetch data from DRAM (DDRAccess) and to transfer it over the wire (WireDelay) are the main 
contributor to read latency, especially with large read size.
Both could be largely improved in a real \sysboard\ with better memory controller and higher frequency.
TLB miss (which takes one DRAM read) is the other main part of the latencies.


\subsection{Application Performance}

\ulinebfpara{Image Compression.}
We run a workload where each client 
compresses and decompresses 1000 256*256-pixel images with increasing number of concurrently running clients.
Figure~\ref{fig-photo} shows the total runtime per client.
We compare \sys\ with RDMA, with both performing computation at the \CN\ side and the RDMA using one-sided operations instead of \sys\ APIs to read/write images in remote memory.
\sys's performance stays the same as the number of clients increase.
RDMA's performance does not scale because it requires each client to register a different MR to have protected memory accesses.
With more MRs, RDMA runs into the case where the RNIC cannot hold all the MR metadata and many accesses would involve a slow read to host main memory.

\ulinebfpara{Radix Tree.}
Figure~\ref{fig-radix} shows the latency of searching a key in pre-populated radix trees when varying the tree size. 
We again compare with RDMA which uses one-sided read operations to perform the tree traversal task.
RDMA's performance is worse than \sys,
because it requires multiple RTTs to traverse the tree,
while \sys\ only needs one RTT for each pointer chasing (each tree level).
In addition, RDMA also scales worse than \sys.

\ulinebfpara{Key-value store.}
Figure~\ref{fig-kvstore} evaluates \syskv\ using the YCSB benchmark~\cite{YCSB} and compares it to Clover, HERD, and HERD-BF.
We run two \CN{}s and 8 threads per \CN.
We use 100K key-value entries and run 100K operations per test,
with YCSB's default key-value size of 1\KB. %where the key size is 8 bytes and the value size is 1\KB.
The accesses to keys follow the Zipf distribution ($\theta=0.99$).
We use three YCSB workloads with different {\em get-set} ratios: 
100\% {\em get} (workload C), 5\% {\em set} (B), and 50\% {\em set} (A).
\syskv\ performs the best.
HERD running on BlueField performs the worst, mainly because BlueField's slower crossing between its NIC chip and ARM chip.




Figures~\ref{fig-ycsb-mn} shows the throughput of \syskv\ when varying the number of MNs. Similar to our
\sys\ scalability results, \syskv\ can reach a CN’s maximum
throughput and can handle concurrent get/set requests even
under contention. These results are similar to or better than
previous FPGA-based and RDMA-based key-value stores that
are fine-tuned for just key-value workloads (Table 3 in \cite{KVDIRECT}),
while we got our results without any performance tuning.

\ulinebfpara{Multi-version data store.}
%\subsubsection{Multi-Version Data Store}
We evaluate \sysmv\ by varying the number of \CN{}s that concurrently access data objects (of 16\,B) on an \MN\ using workloads of 50\% read (of different versions) and 50\% write under uniform and Zipf distribution of objects (Figure~\ref{fig-mvstore}). 
\sysmv's read and write have the same performance, and reading any version has the 
same performance, since we use an array-based version design. 
%Running multiple \MN{}s have similar performance and we omit for space.




\ulinebfpara{Data analytics.}
We run a simple workload which first \texttt{select} rows in a table whose field-A matches a value (\eg, gender is female)
and calculate \texttt{avg} of field-B (\eg, final score) of all the rows.
Finally, it calculates the histogram of the selected rows (\eg, score distribution), which can be presented to the user together with the avg value. %(\eg, how female students' scores compare to the whole class).
\sys\ executes the first two steps at \MN\ offloads and the final step at \CN,
while RDMA always reads rows to \CN\ and then does each operation.
Figure~\ref{fig-dataframe} plots the total run time as the select ratio decreases (fewer rows selected).
% When the select ratio is high, \sys\ and RDMA send a similar amount of data across the network,
% and as the CPU computation is faster than our FPGA implementation for these operations, \sys's overall performance is worse than RDMA.
When the select ratio is low, \sys\ transfers much less data than RDMA, resulting in its better performance.




%To put \sys\ in respective with other existing RDMA-based and FPGA-based key-value stores that we couldn't directly compare with (\eg, close-sourced), we compare 
%\syskv's latency results with reported latencies in ~\cite{KVDIRECT}. 
%\syskv\ has {\bf lower end-to-end latency than all these existing systems}.

%then sends the data to \CN, which shuffles the data and sends the shuffled 
%data back to \MN\ for aggregation.

\subsection{CapEx, Energy, and FPGA Utilization}
\label{sec:clio:results-cost}


We estimate the cost of server and \sysboard\ using market prices of different hardware units. When using 1\TB\ DRAM, a server-based \MN\ costs 1.1-1.5\x\ and consumes 1.9-2.7\x\ power compared to \sysboard. These numbers become 1.4-2.5\x\ and 5.1-8.6\x\ with OptaneDimm~\cite{optane-dcpm}, which we expect to be the more likely remote memory media in future systems.


We measure the total energy used for running YCSB workloads
by collecting the total CPU (or FPGA) cycles and the Watt of a CPU core~\cite{gold5128}, ARM processor~\cite{armpower}, and FPGA (measured).
We omit the energy used by DRAM and NICs in all the calculations. 
Clover, a system that centers its design around low cost, has slightly higher energy than \sys.
Even though there is no processing at \MN{}s for Clover, its \CN{}s use more cycles to process and manage memory.
HERD consumes 1.6\x\ to 3\x\ more energy than \sys, mainly because of its CPU overhead at \MN{}s.
Surprisingly, HERD-BF consumes the most energy, even though it is a low-power ARM-based SmartNIC.
This is because of its worse performance and longer total runtime.

Figure~\ref{fig-fpga-resource} compares the FPGA utilization among Clio, StRoM's RoCEv2~\cite{StRoM}, and Tonic's selective ack stack~\cite{TONIC}.
%With our design that is tailored to save resources, 
%\sys\ consumes roughly one third of the total resources.
Both StRoM and Tonic include only a network stack but they consume more resources than \sys.
Within \sys, the virtual memory (VirtMem) and
the network stack (NetStack) consume a small fraction of the total resources,
with the rest being vendor IPs (PHY, MAC, DDR4, and interconnect).
%To put things in perspective, we implement a Go-back-N network stack which supports 1K connections. It uses 2.5\x\ more logic than what our current network stack consumes. 
Overall, our efficient hardware implementation leaves most FPGA resources available for application offloads.

%% https://homes.cs.washington.edu/~arvind/papers/pcp.pdf

{
\begin{table*}[th]\footnotesize
\begin{center}
\begin{tabular}{ p{1in} | p{1.5in} |p{3.3in} | p{0.5in}}

\textbf{Goal} &
\textbf{Mechanism} &
\textbf{Description} &
\textbf{Section} \\
\hline
\hline

Connectivity            & Reliable link layer & enable connection between supernic and endpoints & XX \\
\hline
Endpoint Integration    & Endpoint shim layer & APIs and lighteweight states & XX \\
\hline
\hline

Perf & Partial NF Chaining & XXXX  & XXX \\
\hline

Perf & Service-level Parallelism & XXXX  & XXX \\
\hline

Perf & Instance-level Parallelism & XXXX  & XXX \\
\hline

Correctness & Reorder Buffer & parallel services and instances & XXX \\
\hline

Perf \& Security & Resource Partition & Partition resource to improve aggregated on-board throughput & XXX \\
\hline
\hline

Management & Softcore & Runs management plane (all policies) & XX \\
\hline

Consolidation & Packet Store & Shared central packet buffer & XX \\
\hline

Fairness & Credit Allocation & Eager and \& on-demand policies & XXX \\
\hline

Fairness & Header/PIFO Queues & XX & XXX \\
\hline

XX & NF Wrapper & XXXX & XXX \\
\hline

Security & Segmented VM & XXXX & XXX \\
\hline


PR Facility & PR Region Decoupler & Logically isolates static and dynamic PR regions, with quiescence API & XXX \\
\hline

Computation Scaling & PR-based Auto Scaling & PR can replace, add, and remove NFs during runtime & XXX \\
\hline

Computation Scaling & Bitstream Store & A table of bitstreams saved in on-board DRAM & XXX \\
\hline

Better Deployment & Bitstream AutoGen & A framework to generate wrapped bitstreams using user netlists & XXX \\
\hline


\end{tabular}
\end{center}
\mycaption{table-techniques}{Technique Summary.}
{
PR: Partial Reconfiguration.
NF: Network Function.
}
\end{table*}
}


\if 0
\section{\sysname\ Overview}
\label{sec:overview}

We walk through the key challenges and present out solutions.

\boldunderpara{Key Challenges:}
\begin{itemize}[leftmargin=0cm,itemindent=.35cm]

\myitem{C1:}
After decoupling the network, what's left at the endpoints?
And how can endpoints communicate with the NetPool?
%
\textbf{Answer:} \TODO{}
We observe that a reliable link layer is sufficient
to connect endpoints and NetPool.
We design a shim library at the endpoints.

\myitem{C2:}
How to efficiently and safely consolidate applications onto NetPool
with limited amount of resource?
%
\textbf{Answer:} \TODO{}
Scheduling: space and time sharing limited resource.
Auto Scaling: xxx.
Isolation: segmented virtual memory, XXX.
Fairness: PIFO, XXX.



\myitem{C3:}
How to design a flexible management interface so that
we can deploy, upgrade, scale applications easily?
\textbf{Answer:}
Although FPGA enables in field reconfiguration,
it is far from a complete solution.
In response, \sysname\ makes a clean separation between the management and data plane.
The management plane makes decisions based on runtime policies
and react to real time traffic load. The data plane simply follows
the management plane's instructions and adapts its computation layout.
We build the management plane using C running at softcores.
As a result, \sysname\ has software-alike configurability
while able to run at raw hardware speed.

\myitem{C4:} The NetPool adds another hop in the path and is
shared by many applications, how to ensure low latency and high throughput?
%
\textbf{Answer:}
Latency: we make two technical contributions.
a) to optimize the common case, we propose \textif{partial chaining}
to shorten the connection latency between chained functions.
b) we use service-level parallelism to shorten critical path.
Throughput:
we use \textit{resource partitioning} and ensure SuperNIC's system
component (e.g., packet scheduler) is able to sustain peak throughput.
\yizhou{partitioning maybe beneficial for side channels}.
When consolidated computation exceeds SuperNIC's capacity,
we use either time- or space-sharing to multiplex resource.
As a result, the execution latency build up.
We trade some extra latency for the consolidation benefits.
Luckily, we obverse that datacenter traffic is bursty and mostly underutilized,
thus peak load scenario is rare.

\myitem{C5:} Failure.
The failure domain is enlarged.
Shall we handle it? If so, how?
Make endpoints multi-homed. Add another link
to switch or to another sNIC.

\yizhou{other challenges: tail latency. security, side channels, programming framework.}

\yizhou{for latency part: it would be great to say:
we obverse the average chain length is XXX,
and we find XX\% traffic will traverse N NFs.
Also, in our eval, during low load, we found offloading transport and NF
to SuperNIC actually improves latency compared to baselin (running transport
on CPU or smartnic like bluefield.)
}

\end{itemize}
%
\\


\textbf{Features}

1. Distributed setup.
A resource pool has set of connected SuperNIC, in mesh or others.
3 planes.
Traffic can bounce among SuperNICs.
Load balancing. (\TODO{depends.})

2. Resource consolidation and abstraction groups.
Describe how end points can choose to offload whatever to supernic,
they could enjoy flexible combo. (this is essentially our Abstraction discussion earlier.)

3. New smartnic architecture.
Highlight prior archs, esp PANIC, show the difference.
Our arch has the benefits of both worlds (we have a diagram for this).

4. Packet Processing Parallelism.
We introduce service-level and instance-level parallelism.
And to ensure ordering, we introduce Reorder Buffer, a mechanism that XXX.

5. Runtime NF Scheduling and scaling, time and space sharing.
We closely monitor NF status and does automatic NF scheduling and scaling

6. Performance Isolation and Protection.
We have virtual memory interface for on-board memory.
We use PIFO and other things to control bandwidth etc.
\fi

\section{\sysname\ Design Yizhou}
\label{sec:design}

\subsection{Overview}

SuperNIC's innovation comes from two parts, the board-level design and the distributed design.

\subsubsection{At Board-Level}

We adopt PANIC’s architecture as our baseline.
Its design enables resource consolidation and flexible NF chaining.
Despite its advantages over traditional RMT, Pipeline, and SoC-based SmartNICs, PANIC suffers from long chaining latency and cannot scale resources.

We propose three techniques to improve this state-of-the-art SmartNIC architecture, namely, 1) traffic-aware NF merging, 2) eager credit allocation scheduling, 3) on-demand scheduling and auto-scaling. These techniques are crucial for reducing latency and enable flexible consolidation. We will cover the details later. (NF merging has two benefits: reduce latency, and increase slot resource utilization while reducing the pressure on the xbar. We need experiments to show.)

In addition, we introduce a softcore (could be a real SoC in the future)
to run the management and control plane tasks.
Our microservice-style software is partitioned into \texttt{agents}.
Each agent is responsible for one particular task, e.g., we currently have monitoring agent, auto-scaling agent, Inter-SuperNIC agent, PR agent, and so on.
It presents a shell-alike interface hence administrates can control SuperNIC via simple commands. 

In all, a single SuperNIC supports multi-tenancy chaining,
traffic-aware NF merging, on-demand scheduling, and auto scaling.
It presents flexible policies with a clean separation of data and management/control planes.

\subsubsection{At Distributed-Level}

A single SuperNIC has limited resources and may become the computation bottleneck during peak load.
To tackle this issue, we leverage the fact that SuperNICs under one ToR are
connected in a ring or a mesh.
We \textit{logically} group such a pool of physically distributed SuperNICs as one giant computation entity.
Within the pool, SuperNICs are able to auto-scale, migrate, and balance computation among each other.

We offer two-level auto-scaling, in which a certain computation offload
can not only scale-up within one SuperNIC
but also scale-out across SuperNICs.
On top of it, SuperNICs can migrate computation and balance traffic on demand.

These designs are possible because of the following components.
First, each SuperNIC board has extensive monitoring facilities collecting runtime statistics for software agents to consume and make decisions using flexible policies.
Second, our fast PR mechanism serves as the basis.
Third, we deploys a hardware routing agent to help re-route traffic.

\subsection{Physical Connections}

\textit{Endpoints and Link Layer}.
We need to describe what's the minimum setting required at the endpoint side. The minimum setting is used when the whole transport and NF are moved to SuperNIC.
The minimum setting should include: a) a reliable link layer, with one-hop flow control, error detection and correction. b) a shim layer exposed to applications. this layer needs to save at least the connection IDs.
I'm a bit worried about the error correction part - what if we cannot guarantee
100\% error-free, then what's the guarantee we offer to endpoints.
We had discussion on this part before, e.g., expose bit error as network failure.

\subsection{On-chip Architecture}

\yizhou{Overview. Add a figure.}

\subsubsection{Parser}
Each RX port has a Parser attached. All parsers share a service chain store.
The parser uses Match-Action-Table style to generate a runtime
descriptor for each packet.
This descriptor is a metadata placeholder, includes XXX.

\subsubsection{Central Scheduler}

\yizhou{Deserves a zoom-in figure.}

\textbf{On-demand and eager credit management}:
The classical pull and push model proposed by
Click (used by PANIC) is designed for single element/NF.
With a NF chain, especially consider the extra cost
of going back to Scheduler or going across xbar,
it is beneficial to favor push over pull.
Mapping to our model, we propose on-demand
and eager credit management.

On-demand credit allocation:
If a packet is going to use a NF, the packet must allocate credit right before traversing the NF.
Hence, if a packet needs to traverse a NF chain, the packet must go back to the central scheduler to
allocate credit for the next NF in the chain. 

Eager credit allocation:
The packet can pre-allocate credits for services it going to use.
Hence, if a packet wants to traverse a NF chain, the packet would 
allocate credits for all NFs in the chain.
Implication: a packet will wait until all NFs in a chain have available credits.

Eager credit pre-allocate resources, lower latency, but may have poor resource utilization.
We think the eager should be the default policy during low load.
We should switch to on-demand when load spikes.

\subsubsection{Parallelism}
Reorder buffer.

Three-level parallelism.
1) pipeline level.
2) fine-grained, needs help from a PL framework.
3) service-level, coarse-grained. Either manual linking, or use a high-level scripting language like Click.
4) instance-level, among instances of the same service.

For now, 1) come from user, 2) is our future work,
We can do 3), manually. We have to do 4).
Both 3 and 4 need the help of ROB.

\yizhou{we could briefly mention that this can benefit from a programming framework. The PL part is deferred to the Discussion section.}

\subsubsection{Service Scheduling}
Mechanism: we use counters. rely on fpga PR, a softcore.
Policy: based on counters? once it exceeds/lowers than XXX, we do YYY.

\subsubsection{Abstraction Groups}

0. Wrapper, and other facilities: slot, PR wrapper, etc.

1. Transports (what about congestion control?)
2. Network Function

\if 0
\subsubsection{NF Shell}
We expose a standard shell abstraction to each running NF.
The shell is analogous to a system call interface,
through which NF will communicate with the rest of SuperNIC.
The shell has several standard signals such as fixed clock (250 MHz),
a pair of network interface, memory access interface, and control signals. Amazon F1 has a similar shell.
\fi

\subsubsection{Security Measures}
Consolidating resource means we need security measures.
1) NFs are only allowed to access shared resource via a fixed set of interfaces provided by shell. This reduced the exposed attack surface.
2) On-board memory employs a simple segment-based permission checking.
3) From FPGA's perspective, the generation framework (we should have a section describing what's this) would do security checks during compilation time, to filter FPGA side-channels designs (bunch papers on this).

\subsection{Distributed SuperNICs}

\subsubsection{Scale-out}
\subsubsection{Migration}
\subsubsection{Hardware Routing Agent}


%\section{Discussion}

\subsection{Programming Model}
Couple points.
1) why we need such a PL model?
2) what's in this PL model?
3) the challenges and benefits of designing/using this PL model?

We could benefit from having a programming framework, like ClickNP, the ones
that provide APIs to do packet processing. Through these APIs,
SuperNIC can do better scheduling etc.
We do not have such model yet.
Ideally, such model can be:
1) a way to describe services, their linking etc.
And SuperNIC is able to infer better NF merging and runtime scheduling policies.
2) a set of shell interfaces through which NFs interact with SuperNIC.

Also its worth mentioning: it is user/system's responsibility to partition
the function among endpoints and SuperNIC. 

\subsection{Failure}

We restrict out discussion to transient failures.
We assume permanent failures are rare.
SuperNIC adds back fate-sharing failure domain (all the connected devices and the SuperNIC). It is making failure handling a bit worse than a full disaggregation model.

Ideally, we want SuperNIC be able to physically isolate its on-board
failure domains, possibly by using different power supplies and chips.
For example, the board has two domains, one pass-through and one computation domain.
The latter hosts all logic and softcore.
If the latter fails, SuperNIC is still able to route traffic via the pass-through domain hence the connected
devices are still reachable from others.

\subsection{Security and FPGA Side-Channels}

We ensure security by properly isolate resources.
FPGA side-channels are possible, but they can be prevented
during compile time by checking user logic.
Most malicious FPGA programs have a certain signature logic.

%% https://docs.google.com/spreadsheets/d/1JODWoEtDBxeOTr-ZqEqm3D-9wfADyNqgMoIOSpukBL0/edit#gid=0

\section{Related Work}
\label{sec:related}


Datacenter network topology.
Disaggregation.
Devices: pswitch, circuit switch, multi-host nic.

Intel IPU










\if 0
In this section, we review emerging network devices
and investigate whether they can be used to implement the
network resource pool, which, in turn provides network-as-a-service
for both disaggregated devices and regular servers.
Our focus is the support for disaggregated devices.
As it requires more functionalities and hence a super set of 
the ones required for regular server case.
Solutions work for the disaggregation setting would naturally
work for the regular-server cluster setting.

To this end, we propose a set of goals that a particular
solution must meet:
\boldpara{1) Port Count}.
The solution must be able to
support the exploded number of endpoints with
a cost-effective network topology.
\boldpara{2) Heterogeneous Endpoints}.
The solution should support various known computation mediums
such as FPGA, ASIC, and CPU, as well as any new ones in the future.
\boldpara{3) Transports and Network Functions}.
As we discussed earlier, the network resource pool is 
consolidating three types of resources: packet processing
in NIC, software network stack and advanced application-specific
network functions. The first type naturally comes with hardware.
The solution must have a mean to support the latter two.
\boldpara{4) Programmability}.
One of the key requirements for any current or future datacenters
is the ability to upgrade or re-program after deployment.
\boldpara{5) Consolidation, Manageability, and Multi-Tenancy}.
The core of pool is resource consolidation,
which relies on good management and multi-tenancy support.

Table~\ref{tabel-related-work} presents all the goals
and whether each reviewed network device or system can meet them.
Next, we will take a deep dive into each one.


\subsection{Programmable Switch}

\subsubsection{Primer}
Unlike traditional switches, the programmable switches
allow users to install specific actions on the switch data path.
thereby enable line-rate packet processing.
The core of programmable switch is Reconfigurable Match Table (RMT),
pioneered by a seminal SIGCOMM'13 paper~\cite{RMT-SIGCOMM13}.
RMT was first proposed to enhance OpenSDN deployment.
Since then, the programmable switches have seen great success in both industry and academic. 
P4~\cite{p4-paper}, a young domain-specific language specifically designed for packet processing, is the de-facto programming language for programmable switches. P4 greatly simplifies
packet manipulation and has helped the wide adoption of programmable switches.

The arise of programmable switch shifted the network computation paradigm:
it breaks the common belief held by distributed system designers
and opens doors to improve, redesign, or create distributed systems in unimaginable ways.
In essence, programmable switch is a centralized computation point
that can mitigate synchronization and consistency issues,
acting as cache front end, or simply be a network function offloading unit. 
Recent systems demonstrate performance improvements in domains like
caching for KV~\cite{netcache-sosp17, incbricks-asplos17},
caching for load-balancing~\cite{distcache-fast19},
in-network coherence directories~\cite{pegasus-osdi20},
congestion control~\cite{hpcc-sigcomm19},
distributed lock management~\cite{netlock-sigcomm20},
databases~\cite{cheetah-sigmod20},
scheduling~\cite{racksched-osdi20},
and network function processing~\cite{tea-sigcomm20}.
%consensus,
%machine learning,

Most recently, researchers started using programmable switches
to consolidate computation resource.
Wang et al.~\cite{wang-hotcloud20} observes programmable switches are
heavily under-utilized, hence use a set of
compile/run-time techniques to deploy multiple p4 programs
onto one programmable switch, thereby enabling multi-tenancy and consolidation.
TEA~\cite{tea-sigcomm20} consolidates NFs at rack-scale,
providing NF-as-a-service to the servers under the ToR switch.
Das et al.~\cite{active-hotnets20} takes a fresh look at
active networking and uses p4 to turn a programmable switch
into a physical computing device akin to a virtual machine.

\subsubsection{Feasibility}

Programmable switch has limited number of ports,
e.g., 64 ports for Intel Tofino2, hence not able to
accommodate the exploded number of disaggregated devices.
Further, with the diminishing of Dennard Scaling and Moore's Law,
the merchant chip is not likely to see dramatic computation power increase,
which in turn limits the number of ports a certain chip can support.

Most of commodity programmable switches are Ethernet-based.
To communicate with such switches,
an endpoint requires an Ethernet gear
(could be any physical form: a chip, a device, or integrated IPs)
with at least L1 and L2 functionalities (i.e., a PHY and a MAC). 
Since all disaggregated devices are directly attached to the network,
it is reasonable to assume they have such Ethernet gear equipped.
As for regular servers, they already have NIC installed.
In all, programmable switches are able to support heterogeneous devices.
Note that both the switch and the endpoints are free to
use other physical and link layer protocols (e.g., Infiniband),
there is nothing fundamental about using Ethernet except it is
already widely used so its beneficial continue using it.

As we mentioned earlier, several work~\cite{tea-sigcomm20,active-hotnets20,wang-hotcloud20}
have tried to consolidate NFs and applications onto programmable switches.
No work has tried to build transport on it, though we believe it is doable.
However, the multi-tenancy support is still at its infancy.
Most of the existing work leverage compile-time tricks to
overcome hardware limitations, which result in some inevitable cost.
With enough momentum, the vendors might wight in and develop certain
virtualization features on programmable switches.
It would be interesting to explore what those features might be.

\subsubsection{Summary}

For both regular server's network disaggregation and consolidation,
programmable switches can partially meet their goals.
Servers can offload their transport processing,
network functions, and advanced application-specific functions to
the programmable switches -  this is already possible now.
However, current commodity programmable switches
are not able to meet the goals for disaggregated datacenter.
Specifically, it is not able to solve the exploded port counts
without a significant increase in cost.

%What about line cards.

\subsection{Circuit Switch}

\subsubsection{Primer}

Circuit switch establishes a dedicated channel between
endpoints connected to it. It guarantees the full bandwidth
of the channel and remains connected for the duration of a
certain session. It creates an illusion as if endpoints are
directly and physically connected.

Circuit switch operates at the physical layer with no buffers,
no arbitration, and no packet inspection mechanisms.
Thus, they are cheaper and more power-efficient than
traditional electrical packet switch. As a result,
circuit switches could support hundreds or even thousands of
ports with lower CAPEX and OPEX than equivalent packet switches,
making it a good candidate to interconnect disaggregated devices in a rack.

Circuit switch has seen great improvement over the previous decade.
Around 2010, Helios~\cite{helios-sigcomm10} first proposes to integrate circuit switches into
the datacenter network and uses a hybrid packet and circuit switching mode.
Mordia~\cite{mordia-sigcomm13} improves the switching time from
tens of milliseconds to microsecond level. In response, Mordia proposes
a proactive scheduling mechanism instead of a reactive one.
REACToR~\cite{reactor-nsdi14} leverages Mordia's fast switching and builds a hybrid
ToR using both packet and circuit switches, enjoy the benefits of both.
But REACToR is sensitive to the traffic pattern.
Until then, circuit switch solutions were tightly coupling
their data plane with the control plane. The control plane
reconfigures the switches in response to traffic demands.
Such solutions cannot scale well.
Hence, RotorNet~\cite{rotornet-sigcomm17} proposes a fully decentralized
control plane solution using specialized hardware and a round-robin policy.
In addition, RotorNet can scale to 1000s of ports with 10 us switching delay.
The latest work, Sirius~\cite{sirius-sigcomm20} achieves nanosecond-level
switching time (3.84 ns for end-to-end reconfiguration).
Overall, the state-of-the-art circuit switch is able to
achieve fast nanosecond-level switching, works with decentralized control plane,
while still able to provide high port counts and consumes very little energy.
Given the foreseeable limitations of electrical packet switch,
circuit switch is gradually making its way into datacenters (e.g., Shoal~\cite{shoal-nsdi19} and dRedBox~\cite{dRedBox-DATE}).

\subsubsection{Feasibility}

Clearly, circuit switch is a good candidate to deploy networks in disaggregated datacenters.
As it supports high port counts hence able to accommodate the exploded number of devices.
Further, circuit switch is able to overcome the over-subscription problem while operating
with very low energy consumption compared to traditional packet switch.

To use circuit switch,
endpoints need to use specialized physical and link layer protocols,
with companion upper layer software~\cite{alex-thesis2020}.
This is relatively easier to achieve in servers with regular NICs
than the heterogeneous devices.
Past work has built an FPGA-based NIC~\cite{alex-thesis2020} for this purpose.
Similar to the programmable switch case,
we anticipate devices using circuit switch would incorporate customized network gear (e.g., a device, a chip, or IPs).

However, circuit switch is not able to
run any computation other than the scheduling algorithm (if any).
Unlike electrical packet switch which encapsulates the control complexity
within the switch, circuit switch \textit{exposes} the control complexity
to the rest of the network~\cite{rotornet-sigcomm17}.
Hence, circuit switch is not able to meet any other goals requiring computation.

\subsubsection{Summary}

Circuit switch has several appealing traits
such as high port counts, power efficient, and excellent scalability.
These make it a good candidate to \textit{build} disaggregated datacenters,
but not necessary for network consolidation.
Its lack of computation power is the key limitation.
If future circuit switch technologies are able to incorporate
any form of computation, then it would become one of the best choices
to build network resource pool.

\subsection{Coherent Fabrics}
\subsubsection{Primer}

In recent years, there are several industry proposals to build new interconnect
fabrics across endpoints in a server or in a rack.
They include Gen-Z~\cite{GenZ}, CXL~\cite{CXL}, OpenCAPI~\cite{OpenCAPI}, and CCIX~\cite{CCIX}.
These fabrics usually provide a universal memory interface and hardware-level memory/cache coherence across different endpoints.

Gen-Z~\cite{GenZ} is a new datacenter-scale fabric providing low latency
and high bandwidth accesses to remote resources.
It supports byte-addressable memory access, block memory access, and
accelerator-specific messaging interface.
Gen-Z is a full-stack solution, it specifies the physical layer,
link layer, network layer, transport layer, and above virtual memory interfaces.
It also defines its own routers and switches.
Each Gen-Z compliant device has a Gen-Z controller attached.
This controller translates user requests into Gen-Z requests
and sends to remote.

OpenCAPI~\cite{OpenCAPI}, CCIX~\cite{CCIX}, and CXL~\cite{CXL}
are all intra-server memory coherent interconnects.
OpenCAPI attaches CPUs to accelerators and I/O devices with minimal overhead.
It provides coherent memory interface across CPU and various devices.
CCIX provides a similar set of features.
On top of those capabilities, CXL further exposes a window directly
into the processor caching hierarchy.
All of them use the widely available PCIe physical and link layer to transmit data within the chassis.

Recently, CXL and Gen-Z consortium announced that
they will bridge their protocols and improve compatibility.
It is likely that in the near future,
CXL could be extended out beyond server boundary
and have coherent access to remote memory (or accelerators) via Gen-Z.
%Likewise, IBM has OpenCAPI has been extended to access disaggregated memory.
Those emerging coherent fabric protocols are gradually
making their ways into datacenters and being used for disaggregation purpose.

\subsubsection{Feasibility}

We will focus on Gen-Z as it is the only datacenter-scale fabric for now.
Unfortunately, there are no commercial Gen-Z products available, so we will
draw our discussion purely based on its latest specification~\cite{GenZ}.

In theory, Gen-Z's topology is able to support high port count.
It uses a combination of routers and switches, and they can be
customized for high port count.
However, they run at link layer or network layer,
thereby lacking any other computation power.
Hence, they are not able to support packet processing offload, nor resource consolidation.

As Gen-Z attaches its own controller to each device,
it is able to support any types of heterogeneous devices.
Also, it supports per-device customization.

\subsubsection{Summary}

Most of the emerging coherent fabrics are still under heavy development
(except OpenCAPI, which is already used in IBM Power series), no commercial
products are available. All of them have great potentials but with high uncertainties.

The main obstacle in adopting these fabrics is the requirement to
replace existing network infrastructure with new switches and new hardware network controllers (one at each endpoint).
These controllers cannot be easily managed or reconfigured, and they are not programmable.

Although these emerging coherent fabrics
are beneficial for traditional disaggregation on compute or memory,
they cannot satisfy the requirements for network disaggregation.

\subsection{Middleboxes and NFV}
\subsubsection{Primer}

Middleboxes, also known as hardware-based network appliances, originally
resided in specialized hardware boxes from various vendors.
They provide network functionalities such as firewall, packet filtering,
NAT, load balancing, and so on.
They are mostly black boxes for network operators.

In the early 2000s, people were still championing middleboxes~\cite{walfish-osdi04}.
But starting form early 2010s, as the workloads were rapidly changing,
people started questioning middleboxes' black box nature
and proposed software-centric middlebox deployment,
which resulted in consolidated middlebox architecture~\cite{comb-nsdi12}.
Around the same time, APLOMB~\cite{aplomb-sigcomm20} took a step further
by outsouring enterprise middleboxes processing all together to the cloud.
Despite middleboxes' usefulness and ubiquitousness, they come with
a set of problems, many of which arise from the fact that they are
hardware-based: they are costly, difficult to manage, and their
functionality is hard or impossible to change.

In response, also in the early 2010s,
the Network Function Virtualization (NFV) concept was proposed.
NFV advocates moving traditional network functions out of
proprietary middleboxes into virtualized software applications
that can be run on commodity, general purpose processors.

The past decade was a golden age for NFV.
Along the timeline, it is very clear what researchers were
focusing on at their time.
In the beginning, single-machine solutions such as ClickOS~\cite{clickos-nsdi14} arise just to enable virtualized
NF development. Not soon after, E2~\cite{e2} was proposed
to help distributed NF deployment. E2 deals with a set of typical
distributed system issues such as failure handling and scaling.
However, during 2016, despite all the promised benefits of NFV,
there has been little progress towards large-scale deployment.
One of the reasons is performance degradation due to virtualization.
Hence, NetBricks~\cite{netbricks} uses safe language Rust to avoid that.
In the same vein, Metron~\cite{metron-nsdi18} and ResQ~\cite{resq-nsdi18}
also propose low-level processor hacks to improve single machine efficiency.

It was clear that CPU will not be able to keep up with the
fast growing network speed.
As a result, there was a renewed interest in moving NFV
back to specialized hardware.
Notably, early work ClickNP~\cite{clickos-nsdi14} deploys NF to
FPGA-based programmable NICs.
FPGA provides massive cheap parallelism and is an ideal medium to run NFs.
Many work followed~\cite{flowblaze-nsdi19,panic-osdi20} and Microsoft
has deployed FPGA-based NF platform in their Azure cloud~\cite{azure-nsdi18}.

Over the years, the whole space moved from specialized middleboxes
to consolidated software-based NFs, and finally find their
way back to using specialized hardware.
But unlike original closed middleboxes, these new hardware (e.g., programmable switch or programmable NIC) are open, programmable,
and supported by the community at large.

\subsubsection{Feasibility}

Both middleboxes and NFV are not able to provide
high number of ports. As the former being a specialized box
and the latter mostly runs on commodity hardware.
Likewise, both of them cannot support heterogeneous endpoints
, nor can they support offloaded transports.
By design, both of them are able to run offloaded NFs. 

Similarly, both of them could support resource consolidation.
Prior work has tried to consolidate middleboxes~\cite{comb-nsdi12}.
As for NFV consolidation, TEA~\cite{tea-sigcomm20} accomplish that using programmable switches,
PANIC~\cite{panic-osdi20} uses programmable NIC,
and SNF~\cite{snf-socc20} uses a serverless framework.
For manageability, middleboxes' closed system nature makes it hard to manage and scale.
On the contrary, NFV is relatively easier to manage and has much more mature systems.

\subsubsection{Summary}

Overall, middlebox is no longer considered a good
solution for future datacenter development,
as its black box nature cannot fit in.
NFV systems are more diverse and open,
in which both hardware and software have open standard and
backed by the community.
For regular server datacenters,
NFV has already been consolidated and provided as a service~\cite{tea-sigcomm20,snf-socc20,panic-osdi20}.
However, none of these systems is able
to do so for disaggregated datacenters.


\subsection{Multi-host NIC}
\subsubsection{Primer}
Multi-host NIC~\cite{Intel-RedRockCanyon,Mellanox-Multihost}, as its name suggests, is a physical NIC shared by multiple hosts.
It connects to hosts via extended PCIe cables.
It appears as independent NIC to each host.
Internally, it can partition its uplink bandwidth among connected hosts,
follow a certain policy (e.g., fair partition).
To host, it is no different than using a normal exclusive NIC,
hence each host runs its own network stack.
Multi-host NIC is proposed to consolidate network resources
in virtualized environment, but it has never been widely deployed.

\subsubsection{Feasibility}

Multi-host NICs reduces the number of ports ToR switch needs,
and it consolidates traditional NIC functionalities.
However, it still requires each end host to run its transport and network
functions in software.
Also, multi-host NICs do not support programmability or rapid on-field upgradability.

\subsubsection{Summary}

Multi-host NIC can meet the port count goal.
Since it is using a more general PCIe interface, it could
potentially support heterogeneous devices.
Although it is designed to consolidate resources,
it only does so for physical and link layer resources.
Higher level protocols such as transports and network functions
cannot be offloaded to the multi-host NIC.

\subsection{Summary of Network Device Review}

We have now reviewed programmable switch,
circuit switch, coherent fabrics, middleboxes, NFV, and multi-host NIC.
Table~\ref{tabel-related-work} summarizes whether each system can meet
the goals for network disaggregation and consolidation.
Unfortunately, none of them can make the cut.
Programmable switch and NFV are the closest solution for regular-server
datacenter, but they are not able to solve exploded port count and
to accommodate heterogeneous devices. Coherent fabrics and multi-host NIC
are able to meet specific goals for disaggregated datacenter, but
lack the support for computation offload and consolidation.

We find that programmable switch, circuit switch, and NFV
all made their first appearance in the early 2010s.
The past decade has witnessed their rapid growth:
programmable switch and NFV are widely deployed in
datacenters~\cite{hpcc-sigcomm19,azure-nsdi18};
though circuit switch has received less adoption,
it is gaining its momentum~\cite{dRedBox-DATE,sirius-sigcomm20}.

\textit{But are they the right devices to build next decade's datacenter network?}
Our answer is no.
We think the next-generation datacenter,
including regular-server and disaggregated,
should use a disaggregated and consolidated network,
for the reasons in Section~\ref{sec:motivation}.
But none of the above device is able to meet our goals.
As a result, we propose a new device called \sysname,
which meets all the goals in Table~\ref{tabel-related-work}
\fi
\section{Conclusion}
\label{sec:snic:conclude}

We propose network disaggregation and consolidation by building SuperNIC, a new networking device specifically for a disaggregated datacenter.
Our FPGA prototype demonstrates the performance and cost benefits of \snic.
Our experience also reveals many new challenges in a new networking design space that could guide future researchers.

\section{Acknowledgments}
Chapter 5, in part, has been submitted for publication of the material as it may appear in Yizhou Shan, Will Lin, Ryan Kosta, Arvind Krishnamurthy, Yiying Zhang, ``Disaggregating and Consolidating Network Functionalities with SuperNIC'', \textit{arXiv, 2022}. The dissertation author was the primary investigator and author of this paper.

%%\documentclass[pageno]{jpaper}
%\newcommand{\asplossubmissionnumber}{234}
%\usepackage[normalem]{ulem}
%\begin{document}

\title{Disaggregating and Consolidating Network Functionalities with SuperNIC}
\date{}
\maketitle
\thispagestyle{empty}

\if 0
\twocolumn[
    \begin{@twocolumnfalse}
    \begin{center}
	{\Large\bf Disaggregating and Consolidating Network Functionalities with SuperNIC}
    \end{center}
    \smallskip
%    \centerline{Yizhou Shan}
%    \centerline{\em UCSD}
    \end{@twocolumnfalse}
]
\fi

{\em Resource disaggregation}, a concept that organizes hardware resources as separate types of network-attached pools instead of as monolithic servers, has quickly gained popularity in both academia and industry in recent years~\cite{Shan18-OSDI,HP-TheMachine,IntelRackScale,FireBox-FASTKeynote,Tsai20-ATC,Nitu18-EUROSYS,DDC-hotcloud20,Ali-SinglesDay}.
Existing production and research disaggregated systems have focused on separating three types of resources: compute (processing)~\cite{Shan18-OSDI,any-gpu-disagg-work}, memory (or persistent memory)~\cite{Shan18-OSDI,HP-TheMachine,Lim09-disaggregate,Aguilera-FarMemory,Tsai20-ATC}, and storage~\cite{cao2018polarfs,SnowFlake-NSDI20}.
%These efforts have seen real success, and more data centers have started to adopt disaggregation at the production-scale~\cite{Ali-SinglesDay}.
%For example, Alibaba listed their disaggregated storage solution as a key enabling factor of serving 544,000 orders per second during their shopping festival~\cite{Ali-SinglesDay}.

The fourth major resource in computing, \textit{network}, has been completely left out in resource disaggregation research.
No work has attempted to disaggregate the network.
At first glance, ``network'' cannot be disaggregated from either a traditional monolithic server or a disaggregated device, as they both need to be attached to the network.

\section{Key Insights and Proposal}
Our insight is that even though each endpoint still needs its own network interface, its \emph{network-related tasks} such as a layer in the network stack or a customized network function do not have to run at the endpoint, \ie, they could potentially be disaggregated and run as a separate resource pool.  

With this insight, we propose to \textit{segregate} network tasks (or {\em \nt{}s}) from individual endpoints and \textit{consolidate} them in a network resource pool.
This pool consists of a distributed set of \textit{SuperNIC}s (or \textit{\snic} for short), a new programmable device that executes \nt{}s for the endpoints.
Each \snic\ connects to a small set of endpoints (\eg, 4 to 8) and a ToR switch,
and all \snic{}s are connected, \eg, as a ring.
%Together, the disaggregated \snic\ pool offers \textit{Network Functionality as a Service} ({\em NFaaS}).
In building \snic, we answer six key research questions.
%it is an essential part after disaggregation,

\vspace{0.05in}
\noindent \textbf{\textit{What is the benefit of disaggregating and consolidating network functionalities?}} 
To answer this question, we study network communication patterns and needs in the traditional server-based data-center environment and in the disaggregated data-center environment by understanding the unique architecture of these environments and by analyzing real-world network traces.
We discover three potential benefits of network disaggregation and consolidation for the disaggregated environment: it enables an existing rack and ToR switch to host a large number of disaggregated devices; it avoids the need to implement hardware and/or software units for \nt{}s at each type of device; and it uses less hardware resources to run \nt{}s.
Server-based data centers share the last and to some extent the second benefits.
To elaborate on the last benefit, the cost saving comes from the fact that both server-based and disaggregated data centers exhibits a highly bursty pattern.
Thus, after consolidation, an \snic{} only needs to provision the \textit{aggregated} peak bandwidth and hardware utilization,
while traditional endpoints need to each provision for its own peak.


\vspace{0.05in}
\noindent \textbf{\textit{What network functionalities should be disaggregated?}}
To answer this question, we explore two dimensions.
In the first dimension, we consider what type of \nt{}s that would benefit from disaggregation and consolidation for different types of endpoints. 
For example, regular servers could offload their TCP transport to \snic\ and potentially share one transport at \snic\ for consolidation.
Servers and disaggregated devices could also offload various network functions like encryption and filtering or application-specific functions like caching and replication.

In the second dimension, we explore where different parts of an \nt{} should be located.
We separate each network functionality into three parts: computation logic (\eg, transmission protocol), states (\eg, packet sequence number, connection information), and data (\eg, un-acknowledged send buffer, or un-ack buffer).
While traditional network solutions bundle all three at one location (host CPU or host NIC), with \sysname's flexible offloading choice, each part of each network functionality could sit at different locations.
For example, the link-layer logic and states and transport-layer states could be at an endpoint, while the link-layer un-ack buffer, transport logic and un-ack buffer for this end host could be maintained at its connected \snic.
Such flexibility could potentially unlock new ways to build network stacks and functions and new mechanisms to provide end-to-end reliability, and it will be especially useful when exploring customized communication abstractions for disaggregated devices.

\vspace{0.05in}
\noindent \textbf{\textit{How to consolidate network functionalities in a way that maximizes the utilization of \snic{} hardware resources while ensuring fairness and minimizing impact on application execution?}}
Instead of provisioning for the peak load at each individual end point, we leverage statistical multiplexing to provision for the aggregated peak load of all connected end points at an \snic.
We achieve this goal with three techniques: 1) sharing an \snic's hardware resources across different \nt{}s ({\em space sharing}), 2) sharing the same \nt{} across different applications ({\em time sharing}), and 3) configuring the same hardware resources for different \nt{}s at different time ({\em time sharing with context switching}).
%Our goal of network disaggregation is to reduce the total amount of hardware resources (CapEx and OpEx) compared to no disaggregation (\ie, each endpoint hosts its own network device for all its network functionalities) while achieving application performance on par with no disaggregation. 
%For 1), we partition the reconfigurable hardware resource of an \snic\ (\eg, FPGA logic cells, on-chip memory, etc.) into fix-sized {\em regions}, each of which can be reconfigured independently. 
%A region can be configured to run different \nt{}s over time.
Space sharing is relatively straightforward and has no side effect; we partition the FPGA space to chunks of different sizes, with each hosting one or more \nt{}s.
On the other hand, time sharing is more challenging, as it could potentially impact application performance and fairness.
Moreover, context switching would hurt application performance a lot more as reconfiguring an area on reconfigurable hardware is orders of magnitude slower than processing a packet.

We solve these challenges by designing policies and mechanisms that leverage the flexiblity network disaggregation and our \snic\ brings.
Our policies include heuristics that favor space sharing over time sharing and fallback to context switching as the last resort. Before starting context switching, we first try migrating the \nt\ to another \snic\ and give applications a chance to mitigate the problem (\eg, by slowing down the sending speed or by switching to software implementation of the \nt). We also frame \nt\ allocation as a multi-dimensional resource allocation problem and considers Dominant Resource Fairness~\cite{DRF} when making the allocation decision.
Our mechanisms focus on reducing the need to reconfigure spaces and on {\em hiding} the reconfiguration performance overhead. We use a technique similar to victim cache for the former. For the latter, we always leave one or more empty slots to quickly start new \nt{}s, while in the background we evict least used \nt{}s to make more empty slots.

\vspace{0.05in}
\noindent \textbf{\textit{How to achieve high-throughput, low-latency, and scalable performance?}}
When we consolidate the \nt{}s of multiple endpoints to one \snic, the \snic\ should provide high throughput for all the endpoints, minimize latency overhead spent at \snic, and be able to handle many concurrent \nt{}s.
We provide high throughput by two levels of parallelism:
{\em \nt{} parallelism} where a packet goes through multiple \nt{}s in parallel and {\em instance parallelism} where we launch multiple instances of the same \nt{} to handle different packets in an application.
\snic\ automatically scales an \nt{} out/down as its load increases/decreases. 
To achieve low latency and scalability, we propose a scheduling system that centers around a new notion, {\em \nt\ chaining}.
The idea is to group \nt{}s that are likely to be executed in a sequence into a chain and to have our central scheduler schedule packets only once for the entire chain. 
Our scheduler reserves credits for the entire chain and only sends packets over when there are credits.
Doing so avoids going through scheduler after each \nt\ and thus improves both packet-processing latency and scheduler scalability.
Finally, to accomodate more types of \nt\ combinations, we support skipping arbitrary \nt{}s in a chain (\eg, a chain falls back to a single \nt\ when all other \nt{}s are skipped).


\vspace{0.05in}
\noindent \textbf{\textit{How to build and utilize a distributed \snic{} framework?}}
When one \snic\ is overloaded and need to accommodate too much network bandwidth, on-chip hardware resources, or off-chip memory space, traditional solutions would drop packets or an entire \nt{} to make room for others or to provision a larger \snic. The former impacts application performance and the latter results in resource wastes when loads are not at their peaks. 
Our idea is to utilize other \snic{}s to handle load spikes, based on the observation that not all \snic{}s under a rack would run at their peak load at the same time. Essentially, with distributed \snic{}s, we provision for the maximum aggregated bandwidth in a rack instead of the sum of peak loads under each \snic.
Specifically, to support overloaded network bandwidth or on-chip hardware resources at an \snic, we create an \nt\ at another \snic\ (one with light load). 
The current \snic\ then only serves as a simple pass-through device to redirect packets to the new \snic, with the latter sending packets to the next hop after processing them.
To support overloaded memory space at an \snic, we deploy a {\em transparent} swapping mechanism that swaps colder data to another \snic's memory.

\vspace{0.05in}
\noindent \textbf{\textit{How to best exploit \snic{}s to build disaggregated applications?}}
While network disaggregation has its benefits in resource and port consolidation, independent network resource scaling and management, it offers additional unique benefits to disaggregated applications. 
We made an initial exploration in this direction with two examples.
First, we explore a customized network abstraction for disaggregated memory device: a key-value store interface (rather than the standard messaging interface).
Second, we explore using \snic{}s as centralized, in-network computation for tasks like data replication, caching, and \fixme{XXX}.


In building \snic\ and its distributed framework, we separate data and control plane, with the former performed in ASIC and FPGA and the latter done in software.
We prototyped \snic\ with FPGA (Xilinx \fixme{XXX}) and connect \snic{}s with both regular server and disaggregated devices.
We demonstrate \snic's benefits and tradeoffs by disaggregating and consolidating three types of network functionalities:
a go-back-N reliable transport, a set of network functions including firewall, AES, and \fixme{XXX}, and a set application-specific functions including \fixme{XXX}.
We evaluated these scenarios with micro- and macro-benchmarks and compared \snic\ with no disaggregation and disaggregation using a device similar to a recent programmable NIC~\cite{PANIC}.
Our results show that \fixme{XXX}.
We will open source \snic\ upon the publication of this paper.



%\pagebreak
%\bibliographystyle{plain}
%\bibliography{references}
%\end{document}



  %%%%%%%%%%%%%%%%%%%%%%%%5
%\clearpage
%\pagestyle{empty}
%\input{department}
%\clearpage

%\setstretch{0.8}
%\titlespacing*{\section}{0em}{1ex}{1ex}
%\begin{small}

%\begin{spacing}{0.3}
%\bibliographystyle{abbrv}
\bibliographystyle{plain}
\bibliography{all-defs,all,personal,all-confs,local,paper}

%\end{spacing}
%\end{small}

%\clearpage
%\input{bio}
%\clearpage
%\input{budget}
%\clearpage
%\input{facilities}

\end{document}



\clearpage

\appendix

\section{Appendix}

\subsection{FPGA Resource Utilization}

The following table shows the FPGA resources used by \snic{} shell.
Most of the resources are left for running \nt{}s.

\begin{center}
\scriptsize
\begin{tabular}{ p{0.6in} | p{0.2in} |p{0.27in} }
 & \textbf{Logic} & \textbf{Memory} \\
\textbf{Module} & \textbf{(LUT)} & \textbf{(BRAM)} \\
\hline
\hline
%Firewall     & 2.8\% & 0.5\% \\
%AES-256       & 0.4\% & 0 \\
%Transport    & 1.3\% & 0.42\% \\
%\hline
%\hline
\snic{} Core & 4.36\%   & 4.74\% \\
Packet Store & 0.91\%   & 9.17\% \\
PHY+MAC      & 0.72\%   & 0.35\% \\
DDR4Controller         & 1.57\%   & 0.29\% \\
MicroBlaze   & 0.25\%   & 1.81\% \\
Misc         & 1.52\%   & 0.75\% \\
%\textbf{Total (w/o \nt{})}        & \textbf{9.33\%}   & \textbf{17.11\%} \\
\hline
\textbf{Total}        & \textbf{9.33\%}   & \textbf{17.11\%} \\
\end{tabular}
\end{center}



\subsection{Cost Calculation}
We explain the different deployment models and the cost calculation formulas behind our CapEx comparisons.
We limit our scope to rack-scale as the higher-level network hierarchies
are orthogonal to the resource pool deployment models.
We calculate that, to deploy a certain number of endpoints, what's the
network cost (i.e., the network interface card, cable, and switch port costs).

We compare the following models:
1) Non-disaggregation model, or the traditional model, termed \texttt{traditional}.
2) Disaggregation model, in which we insert the network pool between endpoints and the ToR switch (Figure~\ref{fig-topology} (a)), termed \texttt{ring}.
3) Disaggregattion model, in which we connect the pool of network devices directly to the ToR switch (Figure~\ref{fig-topology} (b)), termed \texttt{direct}.
For both disaggregation models, we further compare two type of devices: sNIC which has auto-scaling capability and multi-host NIC which can only provision for max resource usage. With runtime dynamic scaling and load balancing features, sNICs can provision for less than the max required resource , the specific ratio is calculated by comparing a particular workload's the sum-of-peak versus the peak-of-sum.

In all, we have the following models under comparison:
\texttt{traditionl, sNIC-direct, sNIC-ring, mhnic-direct, mhnic-ring}.

We now detail the cost calculations.
In the traditional non-disaggregation model,
each endpoint has a full-fledged NIC and a normal high-speed cable for connection to the ToR switch.
In both disaggregation models, since most network tasks are offloaded to the network resource pool, each endpoint can uses a down-scaled NIC.
Furthermore, the last hop link layer between endpoints and the network resource pool is reliable, we can leverage down-scaled, cheaper and less reliable physical cable~\cite{RAIL-NSDI}.

We use the following parameters in our calculation:
\begin{itemize}
\item Deploy \texttt{N} devices.
\item Each switch port has a cost of \texttt{costSwitchPort}
\item A full-fledged NIC's cost is \texttt{costNIC}. A down-scaled NIC cost is \texttt{costDSNIC}.
\item A normal high-speed cable cost is \texttt{costCable}.
A down-scaled less reliable physical cable cost is \texttt{costDSCable}.
\item A consolidation ratio \texttt{consolidRatio} determines how many endpoints are sharing one network resource pool device. We can calculate the number of network pool devices by \texttt{M = N / consolidRatio}.
\item For a network device, only a certain portion is dedicated to running network task, other parts are used as shell. We define the cost ratio used by network task to be \texttt{NTCostRatio}.
\item The peak-of-sum versus the sum-of-peak yields the auto-scaling potentials. A multi-host NIC (mhnic) provisions for the sum-of-peak while an sNIC provisions for the peak-of-sum. We call this ratio \texttt{capExConsolidRatio}.
\item The multi-host NIC's cost can be calculated as \texttt{costMHNIC = costNIC * N}.
\item The sNIC's cost can be calculated as \texttt{costsNIC = costMHNIC * capExRatio}, in which \texttt{capExRatio = (1 - NTCostRatio) + NTCostRatio * capExConsolidRatio}.
\end{itemize}

We now define each model's cost.

The traditional deployment model's cost is straightforward, it includes NIC, cable and switch ports:
\begin{gather}
N * (costNIC + costCable + costSwitchPort)
\end{gather}

The disaggregation models' cost has more moving parts than the traditional. It includes the down-scaled NICs and cables, network pool devices, the cables to the ToR switch, and switch ports.

The first disaggregation model (Figure~\ref{fig-topology} (a)) can be calculated as follows (for both \texttt{sNIC-ring, mhnic-ring}). 
\begin{align}
N * (costDSNIC + costDSCable) + \\
M * (costsNIC + costCable + costSwitchPort)
\end{align}

The second disaggregation model (Figure~\ref{fig-topology} (b)) can be calculated as follows (for both \texttt{sNIC-direct, mhnic-direct}).
\begin{align}
N * (costDSNIC + costCable + costSwitchPort) + \\
M * (costsNIC + costCable + costSwitchPort)
\end{align}

This tables shows the real-world numbers we use.

\begin{center}
\scriptsize
\begin{tabular}{|l|l|l|} 
 \hline
 Parameters & Value & Note \\
 \hline\hline
 costSwitchPort & \$250 & FS 100Gbps switch~\cite{fs-64port-switch} \\
 costNIC & \$500 & Mellanox Connect-X5 \\
 costCable & \$100 & FS DAC 100Gbps cable \\
 costDSNIC & costNIC * 0.2 & Numbers from our prototpe \\
 costDSCable & costCable * 0.6 & ~\cite{RAIL-NSDI} \\
 consolidRatio & 4 & Current model\\
 NTCostRatio & 0.9 & Numbers from our prototype \\
 capExConslidRatio & 0.23 & Facebook Hadoop trace~\cite{facebook-sigcomm15} \\
 \hline
\end{tabular}
\end{center}

%\subsection{Extended Evaluation Results}

{
\begin{figure*}[th]
\begin{minipage}{\figWidthSix}
\begin{center}
\centerline{\includegraphics[width=\columnwidth]{Figures/g_plot_conslid_perf.pdf}}
\vspace{-0.1in}
\mycaption{fig-kv-consolid}{Consolidation Performance w/ FB Key-Value.}
{
}
\end{center}
\end{minipage}
\begin{minipage}{\figWidthSix}
\begin{center}
\centerline{\includegraphics[width=\columnwidth]{Figures/g_plot_conslid_cost.pdf}}
\vspace{-0.1in}
\mycaption{fig-kv-cost}{Consolidation Resource Usage w/ FB KV.}
{
}
\end{center}
\end{minipage}
\vspace{-0.1in}
\end{figure*}
}
\subsection{End-to-End Application Performance and Cost with Consolidation}

To evaluate the benefit and tradeoff of consolidation, we deploy a testbed with four sender and four receiving servers with four setups:
each endhost connects to a ToR switch with 100\Gbps\ or 40\Gbps\ link (baseline, no consolidation), and four endhosts connect to an \snic, each with 100\Gbps\ or 40\Gbps\ link, and the \snic\ connects to the ToR switch with a 100\Gbps\ or 40\Gbps\ link (\snic\ consolidation).
%, and 3) four endhosts connect to an emulated multi-host NIC, each with a 25\Gbps\ link (\S\ref{sec:related}), and the multi-host NIC connects to the ToR switch with a 100\Gbps\ link. %(multi-host NIC, statically partitioned link bandwidth).
For both settings, we execute two \nt{}s, firewall and NAT, in FPGA. 
For the baseline, each endhost has its own set of \nt{}s, while %the multi-host NIC uses one set of \nt{}s in total and 
\snic\ autoscales \nt{}s as described in \S\ref{sec:policy}.
On each server, we generate traffic to follow inter-arrival and size distribution reported in the Facebook 2012 key-value store trace~\cite{Atikoglu12-SIGMETRICS}.
%the Hadoop load distribution reported in the 2015 Facebook workloads~\cite{facebook-sigcomm15}.
%Since there is no reported inter-arrival time for these workloads, we use the inter-arrival time reported by the 2012 Facebook workloads~\cite{Atikoglu12-SIGMETRICS}.
%We measure the application throughput (IOPS) every 10\ms\ time unit to evaluate the throughput changes over time.

Figure~\ref{fig-kv-consolid} reports the throughput comparison of \snic\ and the baseline.
%average IOPS and 95-percentile IOPS across all time units for the three settings. 
\snic\ only adds 1.3\% performance overhead to the baseline under 100\Gbps\ network and 18\% overhead under 40\Gbps\ network. 
We further analyze the workload and found its median and 95-percentile loads to be 24\Gbps\ and 32\Gbps.
With four senders/receivers, the aggregated load is mostly under 100\Gbps\ but often exceeds 40\Gbps.
Note that a multi-host NIC would not be able to achieve \snic's performance, as it subdivides the 100\Gbps\ or 40\Gbps\ into four 25\Gbps\ or 10\Gbps\ sub-links, which would result in each endhost exceeding its sub-link capacity.


We then calculate the amount of FPGA used for running the \nt{}s multiplied by the duration they are used for, to capture the run-time resource consumption with \snic's autoscaling mechanism. The baseline has one set of \nt{}s per endhost for the whole duration.
Figure~\ref{fig-kv-cost} shows this comparison when consolidating two and four endhosts to an \snic\ and using \nt{}s of different performance metrics.
For a slower \nt{} (\eg, one that can only sustain 20\Gbps\ max load), the \snic\ auto-scales more instances of it, resulting in less cost saving.
Our implementation of firewall \nt{} reaches 100\Gbps, while the AES \nt\ is 30\Gbps, resulting in a 64\% cost saving when deploying both of them.



%On the other hand, multi-host NIC incurs higher performance overhead, especially for the tail.
%Whenever any endhost exceeds 25\Gbps\ load, multi-host NIC will have a bottleneck link.
%On the other hand, \snic\ can sustain the peak of aggregated traffic, which is mostly under 100\Gbps, demonstrating the benefit of run-time, dynamic consolidation.

\if 0
\subsubsection{Distributed \snic{}}
To run an \nt\ at a remote \snic,
an \snic{}'s SoftCore first sends a control message to the remote \snic{} to launch the \nt{} and then installs forwarding rules to its parser. This process takes 2.3\mus\ in our testbed.
Afterwards, packets are forwarded to the remote \snic. We observe an addition of 1.3\mus\ latency when packets go through the remote \snic.
\fi

\end{document}