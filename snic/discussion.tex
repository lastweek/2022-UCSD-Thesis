\section{Discussion}

\subsection{Programming Model}
Couple points.
1) why we need such a PL model?
2) what's in this PL model?
3) the challenges and benefits of designing/using this PL model?

We could benefit from having a programming framework, like ClickNP, the ones
that provide APIs to do packet processing. Through these APIs,
SuperNIC can do better scheduling etc.
We do not have such model yet.
Ideally, such model can be:
1) a way to describe services, their linking etc.
And SuperNIC is able to infer better NF merging and runtime scheduling policies.
2) a set of shell interfaces through which NFs interact with SuperNIC.

Also its worth mentioning: it is user/system's responsibility to partition
the function among endpoints and SuperNIC. 

\subsection{Failure}

We restrict out discussion to transient failures.
We assume permanent failures are rare.
SuperNIC adds back fate-sharing failure domain (all the connected devices and the SuperNIC). It is making failure handling a bit worse than a full disaggregation model.

Ideally, we want SuperNIC be able to physically isolate its on-board
failure domains, possibly by using different power supplies and chips.
For example, the board has two domains, one pass-through and one computation domain.
The latter hosts all logic and softcore.
If the latter fails, SuperNIC is still able to route traffic via the pass-through domain hence the connected
devices are still reachable from others.

\subsection{Security and FPGA Side-Channels}

We ensure security by properly isolate resources.
FPGA side-channels are possible, but they can be prevented
during compile time by checking user logic.
Most malicious FPGA programs have a certain signature logic.
