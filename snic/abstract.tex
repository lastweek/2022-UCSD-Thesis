%\NOTE{Check defs.tex for comment marcos.}

\begin{abstract}

Resource disaggregation has gained huge popularity in recent years.  %in both academia and industry. 
Existing works demonstrate how to disaggregate compute, memory, and storage resources. We, for the first time, demonstrate how to disaggregate {\em network resources} by proposing a network resource pool that consists of a new hardware-based network device called {\em SuperNIC}. 
Each SuperNIC consolidates network functionalities from multiple endpoints by fairly sharing limited hardware resources, and it achieves its performance goals by an auto-scaled, highly parallel data plane and a scalable control plane.
We prototyped SuperNIC with FPGA and demonstrate its performance and cost benefits with real network functions and customized disaggregated applications.

\if 0
For decades, the unit of deployment, operation, and failure in datacenters has been a monolithic server,
one that contains all the hardware resources needed to run a user program.
This decade-old server architecture is seeing its limits in the face of today’s datacenter needs.
Hardware resource disaggregation is a solution that breaks full-blown,
general-purpose servers into segregated, network-attached hardware resource pools,
each of which can be built, managed, and scaled independently.

While increasing amounts of effort go into disaggregating compute, memory, and storage,
the fourth major resource in computing, \textit{network},
has been completely left out.
No work has attempted to disaggregate the network.
At first glance, network cannot be disaggregated from either
a traditional monolithic server or a disaggregated device,
as they both need to be attached to the network and
each endpoint is provisioned with its own network interface and the associated network stack.

In this research exam,
I will first present our preliminary study
to motivate network disaggregation and consolidation.
Then I will present a detailed survey on all possible solutions and show why they fall short.
Those solutions include programmable switch, circuit switch, coherent fabrics, middlebox, NFV, and multi-host NIC.
Finally, I will present our initial work, an FPGA-based new network device that meets
all our goals for network disaggregation and consolidation.
\fi

\end{abstract}
