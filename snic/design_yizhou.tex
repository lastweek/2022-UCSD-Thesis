% https://homes.cs.washington.edu/~arvind/papers/pcp.pdf

{
\begin{table*}[th]\footnotesize
\begin{center}
\begin{tabular}{ p{1in} | p{1.5in} |p{3.3in} | p{0.5in}}

\textbf{Goal} &
\textbf{Mechanism} &
\textbf{Description} &
\textbf{Section} \\
\hline
\hline

Connectivity            & Reliable link layer & enable connection between supernic and endpoints & XX \\
\hline
Endpoint Integration    & Endpoint shim layer & APIs and lighteweight states & XX \\
\hline
\hline

Perf & Partial NF Chaining & XXXX  & XXX \\
\hline

Perf & Service-level Parallelism & XXXX  & XXX \\
\hline

Perf & Instance-level Parallelism & XXXX  & XXX \\
\hline

Correctness & Reorder Buffer & parallel services and instances & XXX \\
\hline

Perf \& Security & Resource Partition & Partition resource to improve aggregated on-board throughput & XXX \\
\hline
\hline

Management & Softcore & Runs management plane (all policies) & XX \\
\hline

Consolidation & Packet Store & Shared central packet buffer & XX \\
\hline

Fairness & Credit Allocation & Eager and \& on-demand policies & XXX \\
\hline

Fairness & Header/PIFO Queues & XX & XXX \\
\hline

XX & NF Wrapper & XXXX & XXX \\
\hline

Security & Segmented VM & XXXX & XXX \\
\hline


PR Facility & PR Region Decoupler & Logically isolates static and dynamic PR regions, with quiescence API & XXX \\
\hline

Computation Scaling & PR-based Auto Scaling & PR can replace, add, and remove NFs during runtime & XXX \\
\hline

Computation Scaling & Bitstream Store & A table of bitstreams saved in on-board DRAM & XXX \\
\hline

Better Deployment & Bitstream AutoGen & A framework to generate wrapped bitstreams using user netlists & XXX \\
\hline


\end{tabular}
\end{center}
\mycaption{table-techniques}{Technique Summary.}
{
PR: Partial Reconfiguration.
NF: Network Function.
}
\end{table*}
}


\if 0
\section{\sysname\ Overview}
\label{sec:overview}

We walk through the key challenges and present out solutions.

\boldunderpara{Key Challenges:}
\begin{itemize}[leftmargin=0cm,itemindent=.35cm]

\myitem{C1:}
After decoupling the network, what's left at the endpoints?
And how can endpoints communicate with the NetPool?
%
\textbf{Answer:} \TODO{}
We observe that a reliable link layer is sufficient
to connect endpoints and NetPool.
We design a shim library at the endpoints.

\myitem{C2:}
How to efficiently and safely consolidate applications onto NetPool
with limited amount of resource?
%
\textbf{Answer:} \TODO{}
Scheduling: space and time sharing limited resource.
Auto Scaling: xxx.
Isolation: segmented virtual memory, XXX.
Fairness: PIFO, XXX.



\myitem{C3:}
How to design a flexible management interface so that
we can deploy, upgrade, scale applications easily?
\textbf{Answer:}
Although FPGA enables in field reconfiguration,
it is far from a complete solution.
In response, \sysname\ makes a clean separation between the management and data plane.
The management plane makes decisions based on runtime policies
and react to real time traffic load. The data plane simply follows
the management plane's instructions and adapts its computation layout.
We build the management plane using C running at softcores.
As a result, \sysname\ has software-alike configurability
while able to run at raw hardware speed.

\myitem{C4:} The NetPool adds another hop in the path and is
shared by many applications, how to ensure low latency and high throughput?
%
\textbf{Answer:}
Latency: we make two technical contributions.
a) to optimize the common case, we propose \textif{partial chaining}
to shorten the connection latency between chained functions.
b) we use service-level parallelism to shorten critical path.
Throughput:
we use \textit{resource partitioning} and ensure SuperNIC's system
component (e.g., packet scheduler) is able to sustain peak throughput.
\yizhou{partitioning maybe beneficial for side channels}.
When consolidated computation exceeds SuperNIC's capacity,
we use either time- or space-sharing to multiplex resource.
As a result, the execution latency build up.
We trade some extra latency for the consolidation benefits.
Luckily, we obverse that datacenter traffic is bursty and mostly underutilized,
thus peak load scenario is rare.

\myitem{C5:} Failure.
The failure domain is enlarged.
Shall we handle it? If so, how?
Make endpoints multi-homed. Add another link
to switch or to another sNIC.

\yizhou{other challenges: tail latency. security, side channels, programming framework.}

\yizhou{for latency part: it would be great to say:
we obverse the average chain length is XXX,
and we find XX\% traffic will traverse N NFs.
Also, in our eval, during low load, we found offloading transport and NF
to SuperNIC actually improves latency compared to baselin (running transport
on CPU or smartnic like bluefield.)
}

\end{itemize}
%
\\


\textbf{Features}

1. Distributed setup.
A resource pool has set of connected SuperNIC, in mesh or others.
3 planes.
Traffic can bounce among SuperNICs.
Load balancing. (\TODO{depends.})

2. Resource consolidation and abstraction groups.
Describe how end points can choose to offload whatever to supernic,
they could enjoy flexible combo. (this is essentially our Abstraction discussion earlier.)

3. New smartnic architecture.
Highlight prior archs, esp PANIC, show the difference.
Our arch has the benefits of both worlds (we have a diagram for this).

4. Packet Processing Parallelism.
We introduce service-level and instance-level parallelism.
And to ensure ordering, we introduce Reorder Buffer, a mechanism that XXX.

5. Runtime NF Scheduling and scaling, time and space sharing.
We closely monitor NF status and does automatic NF scheduling and scaling

6. Performance Isolation and Protection.
We have virtual memory interface for on-board memory.
We use PIFO and other things to control bandwidth etc.
\fi

\section{\sysname\ Design Yizhou}
\label{sec:design}

\subsection{Overview}

SuperNIC's innovation comes from two parts, the board-level design and the distributed design.

\subsubsection{At Board-Level}

We adopt PANIC’s architecture as our baseline.
Its design enables resource consolidation and flexible NF chaining.
Despite its advantages over traditional RMT, Pipeline, and SoC-based SmartNICs, PANIC suffers from long chaining latency and cannot scale resources.

We propose three techniques to improve this state-of-the-art SmartNIC architecture, namely, 1) traffic-aware NF merging, 2) eager credit allocation scheduling, 3) on-demand scheduling and auto-scaling. These techniques are crucial for reducing latency and enable flexible consolidation. We will cover the details later. (NF merging has two benefits: reduce latency, and increase slot resource utilization while reducing the pressure on the xbar. We need experiments to show.)

In addition, we introduce a softcore (could be a real SoC in the future)
to run the management and control plane tasks.
Our microservice-style software is partitioned into \texttt{agents}.
Each agent is responsible for one particular task, e.g., we currently have monitoring agent, auto-scaling agent, Inter-SuperNIC agent, PR agent, and so on.
It presents a shell-alike interface hence administrates can control SuperNIC via simple commands. 

In all, a single SuperNIC supports multi-tenancy chaining,
traffic-aware NF merging, on-demand scheduling, and auto scaling.
It presents flexible policies with a clean separation of data and management/control planes.

\subsubsection{At Distributed-Level}

A single SuperNIC has limited resources and may become the computation bottleneck during peak load.
To tackle this issue, we leverage the fact that SuperNICs under one ToR are
connected in a ring or a mesh.
We \textit{logically} group such a pool of physically distributed SuperNICs as one giant computation entity.
Within the pool, SuperNICs are able to auto-scale, migrate, and balance computation among each other.

We offer two-level auto-scaling, in which a certain computation offload
can not only scale-up within one SuperNIC
but also scale-out across SuperNICs.
On top of it, SuperNICs can migrate computation and balance traffic on demand.

These designs are possible because of the following components.
First, each SuperNIC board has extensive monitoring facilities collecting runtime statistics for software agents to consume and make decisions using flexible policies.
Second, our fast PR mechanism serves as the basis.
Third, we deploys a hardware routing agent to help re-route traffic.

\subsection{Physical Connections}

\textit{Endpoints and Link Layer}.
We need to describe what's the minimum setting required at the endpoint side. The minimum setting is used when the whole transport and NF are moved to SuperNIC.
The minimum setting should include: a) a reliable link layer, with one-hop flow control, error detection and correction. b) a shim layer exposed to applications. this layer needs to save at least the connection IDs.
I'm a bit worried about the error correction part - what if we cannot guarantee
100\% error-free, then what's the guarantee we offer to endpoints.
We had discussion on this part before, e.g., expose bit error as network failure.

\subsection{On-chip Architecture}

\yizhou{Overview. Add a figure.}

\subsubsection{Parser}
Each RX port has a Parser attached. All parsers share a service chain store.
The parser uses Match-Action-Table style to generate a runtime
descriptor for each packet.
This descriptor is a metadata placeholder, includes XXX.

\subsubsection{Central Scheduler}

\yizhou{Deserves a zoom-in figure.}

\textbf{On-demand and eager credit management}:
The classical pull and push model proposed by
Click (used by PANIC) is designed for single element/NF.
With a NF chain, especially consider the extra cost
of going back to Scheduler or going across xbar,
it is beneficial to favor push over pull.
Mapping to our model, we propose on-demand
and eager credit management.

On-demand credit allocation:
If a packet is going to use a NF, the packet must allocate credit right before traversing the NF.
Hence, if a packet needs to traverse a NF chain, the packet must go back to the central scheduler to
allocate credit for the next NF in the chain. 

Eager credit allocation:
The packet can pre-allocate credits for services it going to use.
Hence, if a packet wants to traverse a NF chain, the packet would 
allocate credits for all NFs in the chain.
Implication: a packet will wait until all NFs in a chain have available credits.

Eager credit pre-allocate resources, lower latency, but may have poor resource utilization.
We think the eager should be the default policy during low load.
We should switch to on-demand when load spikes.

\subsubsection{Parallelism}
Reorder buffer.

Three-level parallelism.
1) pipeline level.
2) fine-grained, needs help from a PL framework.
3) service-level, coarse-grained. Either manual linking, or use a high-level scripting language like Click.
4) instance-level, among instances of the same service.

For now, 1) come from user, 2) is our future work,
We can do 3), manually. We have to do 4).
Both 3 and 4 need the help of ROB.

\yizhou{we could briefly mention that this can benefit from a programming framework. The PL part is deferred to the Discussion section.}

\subsubsection{Service Scheduling}
Mechanism: we use counters. rely on fpga PR, a softcore.
Policy: based on counters? once it exceeds/lowers than XXX, we do YYY.

\subsubsection{Abstraction Groups}

0. Wrapper, and other facilities: slot, PR wrapper, etc.

1. Transports (what about congestion control?)
2. Network Function

\if 0
\subsubsection{NF Shell}
We expose a standard shell abstraction to each running NF.
The shell is analogous to a system call interface,
through which NF will communicate with the rest of SuperNIC.
The shell has several standard signals such as fixed clock (250 MHz),
a pair of network interface, memory access interface, and control signals. Amazon F1 has a similar shell.
\fi

\subsubsection{Security Measures}
Consolidating resource means we need security measures.
1) NFs are only allowed to access shared resource via a fixed set of interfaces provided by shell. This reduced the exposed attack surface.
2) On-board memory employs a simple segment-based permission checking.
3) From FPGA's perspective, the generation framework (we should have a section describing what's this) would do security checks during compilation time, to filter FPGA side-channels designs (bunch papers on this).

\subsection{Distributed SuperNICs}

\subsubsection{Scale-out}
\subsubsection{Migration}
\subsubsection{Hardware Routing Agent}

