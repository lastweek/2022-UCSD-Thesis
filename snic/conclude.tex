\section{Conclusion}
\label{sec:conclude}

We propose network disaggregation and consolidation by building SuperNIC, a new networking device specifically for a disaggregated datacenter.
Our FPGA prototype demonstrates the performance and cost benefits of \snic.
Our experience also reveals many new challenges in a new networking design space that could guide future researchers.
%Our experience also reveals new challenges in network hardware design, packet and network-task scheduling, multi-tenant sharing, and distributed network systems.

\if 0
Resource disaggregation breaks the monolithic
servers into segregated hardware resource pools,
each of which can be built, managed, and scaled independently.
Based on our preliminary study,
we find that network would be the major obstacle 
to deploy either disaggregated datacenters,
or high-speed regular server datacenters.
Network, the fourth major computing resource in datacenter,
has been completely left out in the disaggregation process.
In response, we propose to disaggregate network from
endpoints, then further consolidate them into a datacenter-scale
network resource pool. This pool provides \textit{network-as-a-service}
to the rest of the datacenter, consisting of three types of network computation.

In this paper, we have reviewed all emerging network devices
that are possible solutions for network consolidation.
Unfortunately, none of them satisfies our goals.
As a result, we take a step further and propose
\sysname, a new programmable network device.
\sysname\ connects to endpoints via the down link
and connects to the ToR switch via the up link.
A set of \sysname{}s form the network resource pool
and provide dynamic services to the endpoints connected to it.

We believe that datacenters will undergo major renovations
in the next decade to accommodate the diverse hardware
and high-demand applications.
Network disaggregation and consolidation is one of the keys to achieve that.
Programmable switch, circuit switch, and NFV all have enjoyed
their rapid growth in the past decade, but as we our review shows,
they are not the right tools for the next decade.
\fi

\textbf{\textit{This work does not raise any ethical issues.}}