%
%
% UCSD Doctoral Dissertation Template
% -----------------------------------
% https://github.com/ucsd-thesis/ucsd-thesis
%

% Setup the documentclass 
% default options: 12pt, oneside, final
%
% fonts: 10pt, 11pt, 12pt -- are valid for UCSD dissertations.
% sides: oneside, twoside -- note that two-sided theses are not accepted 
%                            by OGS.
% mode: draft, final      -- draft mode switches to single spacing, 
%                            removes hyperlinks, and places a black box
%                            at every overfull hbox (check these before
%                            submission).
% chapterheads            -- Include this if you want your chapters to read:
%                              Chapter 1
%                              Title of Chapter
%
%                            instead of
%                              1 Title of Chapter
\documentclass[12pt,chapterheads]{ucsd}



% Include all packages you need here.  
% Some standard options are suggested below.
%
% See the project wiki for information on how to use 
% these packages. Other useful packages are also listed there.
%
%   http://code.google.com/p/ucsd-thesis/wiki/GettingStarted



%% AMS PACKAGES - Chances are you will want some or all 
%    of these if writing a dissertation that includes equations.
%  \usepackage{amsmath, amscd, amssymb, amsthm}

%% BONUS MATH
%  \usepackage{mathtools} 

%% MARGIN REQUIREMENTS IN TITLES - Hyphenation in a Section Title does not always respect margin settings in Latex.  To force no hyphentation, uncomment the package below.
%  \usepackage[raggedright]{titlesec} 

%% GRAPHICX - This is the standard package for 
%    including graphics for latex/pdflatex.
\usepackage{scrextend}
\usepackage{pslatex}
\usepackage{graphicx}

%% CAPTION
% This overrides some of the ugliness in ucsd.cls and
% allows the text to be double-spaced while letting figures,
% tables, and footnotes to be single-spaced--all OGS requirements.
% NOTE: Must appear after graphics and ams math
\makeatletter
\gdef\@ptsize{2}% 12pt documents
\let\@currsize\normalsize
\makeatother
\usepackage{setspace}
\doublespace
\usepackage[font=small, width=0.9\textwidth]{caption}
\usepackage{subcaption}
\usepackage[capposition=bottom]{floatrow} %force captions below figure per OGS requirement

\usepackage{soul}
\usepackage{ulem}

%% SUBFIG - Use this to place multiple images in a
%    single figure.  Subfig will handle placement and
%    proper captioning (e.g. Figure 1.2(a))
% \usepackage{subfig}

%% TIMES FONT - replacements for Computer Modern
%%   This package will replace the default font with a
%%   Times-Roman font with math support.
% \usepackage[T1]{fontenc}
% \usepackage{mathptmx}

%% INDEX
%   Uncomment the following two lines to create an index: 
% \usepackage{makeidx}
% \makeindex
%   You will need to uncomment the \printindex line near the
%   bibliography to display the index.  Use the command
% \index{keyword} 
%   within the text to create an entry in the index for keyword.
%   To compile a LaTeX document with an index the 'makeindex'
%   command will need to be run.  See the wiki for more details.

%% HYPERLINKS
%   To create a PDF with hyperlinks, you need to include the hyperref package.
%   THIS HAS TO BE THE LAST PACKAGE INCLUDED!
%   Note that the options plainpages=false and pdfpagelabels exist
%   to fix indexing associated with having both (ii) and (2) as pages.
%   Also, all links must be black according to OGS.
%   See: http://www.tex.ac.uk/cgi-bin/texfaq2html?label=hyperdupdest
%   Note: This may not work correctly with all DVI viewers (i.e. Yap breaks).
%   NOTE: hyperref will NOT work in draft mode, as noted above.
\usepackage[colorlinks=true, pdfstartview=FitV,
            linkcolor=black, citecolor=black,
            urlcolor=black, plainpages=false,
            pdfpagelabels]{hyperref}
% \hypersetup{ pdfauthor = {Your Name Here}, 
%              pdftitle = {The Title of The Dissertation}, 
%              pdfkeywords = {Keywords for Searching}, 
%              pdfcreator = {pdfLaTeX with hyperref package}, 
%              pdfproducer = {pdfLaTeX} }
% \urlstyle{same}
% \usepackage{bookmark}


%% CITATIONS
% Sets citation format
% and fixes up citations madness
\usepackage{microtype}  % avoids citations that hang into the margin


%% FOOTNOTE-MAGIC
% Enables footnotes in tables, re-referencing the same footnote multiple times.
\usepackage{footnote}
\makesavenoteenv{tabular}
\makesavenoteenv{table}


%% TABLE FORMATTING MADNESS
% Enable all sorts of fun table tricks
\usepackage{rotating}  % Enables the sideways environment (NCPW)
\usepackage{array}  % Enables "m" tabular environment http://ctan.org/pkg/array
\usepackage{booktabs}  % Enables \toprule  http://ctan.org/pkg/array

\newcommand{\horizbar}{\rule{\linewidth}{.5mm}}
\newcommand{\app}[1]{{\sc #1}}
 
\renewcommand{\em}{\it}

  
\newcommand{\BigO}[1]{${\cal O}(#1)$}
\newcommand{\BigOmega}[1]{$\Omega(#1)$}
\newcommand{\BigTheta}[1]{$\Theta(#1)$}
 
\newcommand{\ceiling}[1]{\left\lceil #1 \right\rceil}
\newcommand{\faM}{\lfloor \alpha M \rfloor}
%\newcommand{\C}[2]{{#1 \choose #2}}

\newcommand{\x}{$\times$}
 
%\newcommand{\comment}[1]{}
\newcommand{\ignore}[1]{}


%\newcommand{\boldparagraph}[1]{\vspace*{-0ex}\paragraph{#1}}
\newcommand{\boldparagraph}[1]{\vspace*{1ex}\noindent\textit{#1}\hspace{1em}}

%%%%% SINGLE FIGURE
\def\cfigure[#1,#2,#3]{
\begin{figure}
\vspace*{0mm}
\begin{center}

\includegraphics[width=3in]{#1} 
 
\vspace*{-3mm}\caption[]{#2
} \label{#3}
 
\vspace*{-5mm}
\end{center}
%\horizbar
%\vspace*{-2mm}
\end{figure}}

%%%%% SINGLE FIGURE 4in wide
\def\cfigurefour[#1,#2,#3]{
\begin{figure}
\vspace*{0mm}
\begin{center}

\includegraphics[width=4in]{#1} 
 
\vspace*{-3mm}\caption[]{#2
} \label{#3}
 
\vspace*{-5mm}
\end{center}
%\horizbar
%\vspace*{-2mm}
\end{figure}}

%%%%% SINGLE FIGURE
\def\cfiguretemp[#1,#2,#3]{
\begin{figure}
\vspace*{0mm}
\begin{center}

\includegraphics[width=3.5in]{#1} 
 
\vspace*{-3mm}\caption[]{#2
} \label{#3}
 
\vspace*{-5mm}
\end{center}
%\horizbar
\vspace*{-2mm}
\end{figure}}

%%%%% SINGLE WIDE FIGURE
\def\wfigure[#1,#2,#3]{
\begin{figure*}
\vspace*{0mm}
\begin{center}
 \includegraphics[width=\textwidth]{#1} 
 \vspace*{-3mm}\caption[]{#2
} \label{#3}
 
\end{center}
%\horizbar
\end{figure*}}

%%%%% 3 FIGURES IN A ROW
\def\threefigure[#1,#2,#3,#4,#5]{
\begin{figure*}
\vspace*{0mm}
\begin{center}

\begin{tabular}{ccc}
\includegraphics[width=2in]{#1} & \includegraphics[width=2in]{#2} &  \includegraphics[width=2in]{#3} \\
(a) & (b) & (c) \\
\end{tabular}

\vspace*{-3mm}\caption[]{#4
} \label{#5}

\vspace*{-5mm}
\end{center}
%\horizbar
\vspace*{-2mm}
\end{figure*}}

%%%%%% DOUBLE FIGURE
\def\dcfigure[#1,#2,#3,#4,#5,#6]{
{
\begin{figure*}
\begin{center}
\begin{minipage}[c]{\columnwidth}{
\includegraphics[width=\columnwidth]{#1} 
\vspace*{0mm}\caption[]{#2} \label{#3} \
}\end{minipage}\hspace*{\columnsep}\
\begin{minipage}[c]{\columnwidth}{
\includegraphics[width=\columnwidth]{#4} 
\vspace*{0mm}\caption[]{#5}\label{#6} \
}\end{minipage}
\end{center}
\end{figure*}
}
}


\def\tableByTable[#1,#2,#3,#4,#5,#6]{
{
\begin{table*}
\begin{center}
\begin{minipage}[c]{3in}{
\centering
{#1}
\vspace*{0mm}\tabcaption[]{#2}\label{#3} \
}\end{minipage}\hspace*{\columnsep}\
\begin{minipage}[c]{3in}{
\centering
{#4}
\vspace*{0mm}\tabcaption[]{#5}\label{#6} \
}\end{minipage}
\end{center}
\end{table*}
}
}


\def\figureByTable[#1,#2,#3,#4,#5,#6]{
{
\begin{figure*}
\begin{center}
\begin{minipage}[c]{3in}{
\centering
\includegraphics[width=\textwidth]{#1}
\vspace*{0mm}\figcaption[]{#2} \label{#3} \
}\end{minipage}\hspace*{\columnsep}\
\begin{minipage}[c]{3.3in}{
\centering
{#4}
\vspace*{0mm}\tabcaption[]{#5}\label{#6} \
}\end{minipage}
\end{center}
\end{figure*}
}
}

\def\tableByFigure[#1,#2,#3,#4,#5,#6]{
{
\begin{figure*}
\begin{center}
\begin{minipage}[c]{4.3in}{
\centering
{#1}
\vspace*{0mm}\tabcaption[]{#2} \label{#3} \
}\end{minipage}\hspace*{\columnsep}\
\begin{minipage}[c]{2.2in}{
\centering
\includegraphics[width=\textwidth]{#4}
\vspace*{-0.35in}\caption[]{#5}\label{#6} \
}\end{minipage}
\end{center}
\end{figure*}
}
}

% two figs pdfs in one column fig
\def\doublecfigure[#1,#2,#3,#4]{
{
\begin{figure}
\begin{center}
\begin{minipage}[c]{1.5in}{
\begin{center}
\includegraphics[width=1.5in]{#1}%\\(a)
\end{center}
}\end{minipage}\hspace*{1em}\
\begin{minipage}[c]{1.5in}{
\begin{center}
\includegraphics[width=1.5in]{#2}%\\(b)
\end{center}
}\end{minipage}
\vspace*{0mm}\caption[]{#3} \label{#4} \
\end{center}
\end{figure}
}
}

\def\qcfigure[#1,#2,#3,#4,#5,#6]{
{
\begin{figure*}
\vspace*{0.2in}\
\begin{center}
\begin{minipage}[c]{3in}{
\includegraphics[width=3in]{#1} 
\vspace*{-3mm}
}
\end{minipage}\hspace*{0.5in}\
\begin{minipage}[c]{3in}{
\includegraphics[width=3in]{#2} 
\vspace*{-3mm}
}\end{minipage}

\begin{minipage}[c]{3in}{
\includegraphics[width=3in]{#3} 
\vspace*{-3mm}
}
\end{minipage}\hspace*{0.5in}\
\begin{minipage}[c]{3in}{
\includegraphics[width=3in]{#4} 
\vspace*{-3mm}
}\end{minipage}
\end{center}
\caption[]{#5}\label{#6}
\end{figure*}
}
}

\def\twfigure[#1,#2,#3,#4,#5]{
{
\begin{figure*}
\vspace*{0.2in}\
\begin{center}
\begin{minipage}[c]{6.5in}{
\includegraphics[width=6.5in]{#1} 
\vspace*{-3mm}
}
\end{minipage}

\begin{minipage}[c]{6.5in}{
\includegraphics[width=6.5in]{#2} 
\vspace*{-3mm}
}\end{minipage}

\begin{minipage}[c]{6.5in}{
\includegraphics[width=6.5in]{#3} 
\vspace*{-3mm}
}
\end{minipage}
\end{center}
\caption[]{#4}\label{#5}
\end{figure*}
}
}

\def\dwfigure[#1,#2,#3,#4]{
{
\begin{figure*}
\vspace*{0.2in}\
\begin{center}
\begin{minipage}[c]{6.5in}{
\includegraphics[width=6.5in]{#1} 
\vspace*{-3mm}
}
\end{minipage}

\begin{minipage}[c]{6.5in}{
\includegraphics[width=6.5in]{#2} 
\vspace*{-3mm}
}\end{minipage}

\end{center}
\caption[]{#3}\label{#4}
\end{figure*}
}
}



\def\dssfigure[#1,#2,#3,#4,#5,#6]{
{
\begin{figure*}
\vspace*{0.2in}\
\begin{center}
\begin{minipage}[c]{4in}{
\includegraphics[width=4in]{#1}
\vspace*{-3mm}\caption[]{#2} \label{#3} \
}\end{minipage}\hspace*{0.5in}\
\begin{minipage}[c]{2in}{
\includegraphics[width=2in]{#4}
\vspace*{-3mm}\caption[]{#5}\label{#6} \
}\end{minipage}
\end{center}
\vspace*{-0.4in}\
\end{figure*}
}
}




\def\dsfigure[#1,#2,#3,#4,#5,#6]{
{
\begin{figure*}
\vspace*{0.2in}\
\begin{center}
\begin{minipage}[c]{3in}{
\includegraphics[width=3in]{#1}
\vspace*{-3mm}\caption[]{#2} \label{#3} \
}\end{minipage}\hspace*{0.5in}\
\begin{minipage}[c]{3in}{
\hspace*{0.5in}\
\includegraphics[height=3in]{#4}
\vspace*{-3mm}\caption[]{#5}\label{#6} \
}\end{minipage}
\end{center}
\vspace*{-0.4in}\
\end{figure*}
}
}


\def\dsyfigure[#1,#2,#3,#4,#5,#6]{
{
\begin{figure*}
\vspace*{0.2in}\
\begin{center}
\begin{minipage}[c]{2.5in}{
\includegraphics[height=2.5in]{#1}
\vspace*{-3mm}\caption[]{#2} \label{#3} \
}\end{minipage}\hspace*{0.5in}\
\begin{minipage}[c]{2.5in}{
\includegraphics[height=2.5in]{#4}
\vspace*{-3mm}\caption[]{#5}\label{#6} \
}\end{minipage}
\end{center}
\vspace*{-0.4in}\
\end{figure*}
}
}

\def\dyfigure[#1,#2,#3,#4,#5,#6]{
{
\begin{figure*}
\vspace*{0.2in}\
\begin{center}
\begin{minipage}[c]{3in}{
\includegraphics[height=3in]{#1} 
\vspace*{-3mm}\caption[]{#2} \label{#3} \
}\end{minipage}\hspace*{0.5in}\
\begin{minipage}[c]{3in}{
\includegraphics[height=3in]{#4} 
\vspace*{-3mm}\caption[]{#5}\label{#6} \
}\end{minipage}
\end{center}
\vspace*{-0.4in}\
\end{figure*}
}
}

%%%%%% DOUBLE FIGURE Y
\def\dyoldfigure[#1,#2,#3,#4,#5,#6]{
{
\begin{figure*}
\vspace*{0.2in}\
\begin{center}
\begin{minipage}[c]{3in}{
\epsfysize=2.0in\
\hspace{0.5in}\
\epsfbox{#1}
\vspace*{-3mm}\caption[]{#2} \label{#3} \
}\end{minipage}\hspace*{0.25in}\
\begin{minipage}[c]{3in}{
\epsfysize=2.0in\
\hspace{0.5in}\
\epsfbox{#4}
\vspace*{-3mm}\caption[]{#5}\label{#6} \
}\end{minipage}
\end{center}
\vspace*{-0.4in}\
\end{figure*}
}
}

%%%%%% DOUBLE FIGURE Y IN A COLUMN!!
\def\cfiguredouble[#1,#2,#3,#4]{
\begin{figure}
\vspace*{0.2in}\
\begin{center}
\begin{minipage}[c]{1.5in}{
\epsfxsize=1.5in\
\epsfbox{#1}
}\end{minipage}\hspace*{0.1in}\
\begin{minipage}[c]{1.5in}{
\epsfxsize=1.5in\
\vspace{0.1in}\epsfbox{#2}
}\end{minipage}\vspace*{-0.10in} \caption[]{#3}\label{#4}
\end{center}
\vspace*{-0.4in}\
\end{figure}
}


%%%%% Single programmable size figure
\def\wpfigure[#1,#2,#3,#4]{
\begin{figure*}
\vspace*{4mm}
\begin{center}

\includegraphics[width=#4]{#1} 

\vspace*{-3mm}\caption[]{#2
} \label{#3}

\vspace*{-5mm}
\end{center}
%\horizbar
\end{figure*}}

%%%%% Single programmable size figure, rotated
\def\wprfigure[#1,#2,#3,#4,#5]{
\begin{figure*}
\vspace*{4mm}
\begin{center}

\includegraphics[width=#4, angle=#5]{#1} 

\vspace*{-3mm}\caption[]{#2
} \label{#3}

\vspace*{-5mm}
\end{center}
%\horizbar
\end{figure*}}




%%%%% Adjacent, programmable-width figures, slid vertically by 9th
%%%%% parameter
\def\DoubleFigureWSlide[#1,#2,#3,#4,#5,#6,#7,#8,#9]{
\begin{figure*}
\vspace*{#9}
\begin{center}
\begin{minipage}{#4}
\includegraphics[width=#4]{#1}
\vspace*{-3mm}\caption{#2
}\label{#3}
\end{minipage}
\hspace{2em}
\begin{minipage}{#8}
\includegraphics[width=#8]{#5}
\vspace*{-3mm}\caption{#6
}\label{#7}
\end{minipage}
\vspace*{-5mm}
\end{center}
\end{figure*}
}


%%%%% Adjacent, programmable-width figures
\def\DoubleFigureW[#1,#2,#3,#4,#5,#6,#7,#8]{
\begin{figure*}
\vspace*{0in}
\begin{center}
\begin{minipage}{#4}
\includegraphics[width=#4]{#1}
\vspace*{-3mm}\caption{#2
}\label{#3}
\end{minipage}
\hspace{2em}
\begin{minipage}{#8}
\includegraphics[width=#8]{#5}
\vspace*{-3mm}\caption{#6
}\label{#7}
\end{minipage}
\vspace*{-5mm}
\end{center}
\end{figure*}
}



\def\DoubleFigureWHack[#1,#2,#3,#4,#5,#6,#7,#8]{
\begin{figure*}
\vspace*{0in}
\begin{center}
\begin{minipage}{3in}
\includegraphics[width=#4]{#1}
\vspace*{-3mm}\caption{#2
}\label{#3}
\end{minipage}
\hspace{2em}
\begin{minipage}{3in}
\includegraphics[width=#8]{#5}
\vspace*{-3mm}\caption{#6
}\label{#7}
\end{minipage}
\vspace*{-5mm}
\end{center}
\end{figure*}
}






%%%%%% DOUBLE FIGURE
\def\ddcfigure[#1,#2,#3,#4]{
\begin{figure*}
\vspace*{0.2in}\
\begin{center}
\begin{minipage}[c]{\columnwidth}{
\includegraphics[width=\columnwidth]{#1} 
}\end{minipage}\hspace{0.5in}\
\begin{minipage}[c]{\columnwidth}{
\includegraphics[width=\columnwidth]{#2} 
}\end{minipage} \caption[]{#3}\label{#4}
\end{center}
\end{figure*}
}

\def\ddcfigureSlide[#1,#2,#3,#4,#5]{
\begin{figure*}
\vspace*{#5}\
\begin{center}
\begin{minipage}[c]{3in}{
\includegraphics[height=3in]{#1} 
}\end{minipage}\hspace{0.5in}\
\begin{minipage}[c]{3in}{
\includegraphics[height=3in]{#2} 
}\end{minipage}\vspace*{-0.10in} \caption[]{#3}\label{#4}
\end{center}
\vspace*{-0.4in}\
\end{figure*}
}

\def\cxfigure[#1,#2,#3]{
\begin{figure}
\vspace*{4mm}
\begin{center}
 
\epsfxsize=2.5in\
\epsfbox{#1}\
 
\vspace*{-0.10in}\caption[]{#2
} \label{#3}
 
\vspace*{-5mm}
\end{center}
%\horizbar
\vspace*{-2mm}
\end{figure}}

\newenvironment{panefigure}{\begin{figure}\begin{center}}{\end{center}\end{figure}}

\newcommand{\pdfpane}[3]{
\begin{minipage}{#1}
\begin{center}
\includegraphics[width=#1]{#2}\\(#3)
\end{center}
\end{minipage}
}

\newcommand{\figWidth}{\columnwidth}
\newcommand{\figSep}{0.05in} 
%\newcommand{\figSep}{\columnsep} 
\newcommand{\figWidthOne}{3.05in} 
\newcommand{\figWidthHalf}{5.85in} 
\newcommand{\figWidthTwo}{3.7in} 
\newcommand{\figWidthThree}{2in} 
\newcommand{\figWidthFour}{1.3in} 
\newcommand{\figWidthFive}{2.3in} 
\newcommand{\figWidthSix}{2.3in} 
\newcommand{\figHeight}{2.0in}
\newcommand{\figHeightOne}{2.6in}
\newcommand{\captionText}[2]{\textbf{#1} \textit{\small{#2}}}

\newcommand{\beforecaption}{\vspace{-.15cm}\begin{spacing}{0.85}}
\newcommand{\aftercaption}{\vspace{-.45cm}\end{spacing}}
% \newcommand{\mycaption}[3]{{\beforecaption\caption{\label{#1}\footnotesize{\textbf{#2}} {\em #3}}\aftercaption}}
% haryadi, change mycaption three to mycaptionthree
%\newcommand{\mycaption}[3]{{\caption[#2]{{\bf #2.} {\em #3}}\label{#1}}}
%\newcommand{\mycaption}[3]{\beforecaption\caption{\label{#1}{\small \bf #2} \em\scriptsize #3}\aftercaption}
%\newcommand{\mycaption}[3]{\beforecaption\caption{\label{#1}{\bf #2} \em\footnotesize #3}\aftercaption}
\newcommand{\mycaption}[3]{\caption{\label{#1}{\bf #2} \em\small #3}}


%%%%% general

% only foreign words should be italicized... (example given should not)
\newcommand{\eg}{\textit{e.g.}}
\newcommand{\ie}{\textit{i.e.}}
\newcommand{\etal}{\textit{et al.}}
\newcommand{\etc}{\textit{etc.}}
\newcommand{\adhoc}{\textit{ad hoc}}

% units
\newcommand{\KB}{\,KB}
\newcommand{\MB}{\,MB}
\newcommand{\GB}{\,GB}
\newcommand{\TB}{\,TB}
\newcommand{\GBs}{\,GB/s}
\newcommand{\MBs}{\,MB/s}
\newcommand{\KBs}{\,KB/s}
\newcommand{\Kbs}{~Kbit/s}
\newcommand{\gbps}{\,Gbps}
\newcommand{\mbs}{~Mbit/s}
\newcommand{\mus}{\mbox{$\mu s$}}
\newcommand{\ms}{\mbox{$ms$}}

%\newcommand{\fsync}{\texttt{fsync}}

% axes
\newcommand{\xaxis}{x-axis}
\newcommand{\yaxis}{y-axis}


\newcommand{\unix}{{\sc Unix}}
\newcommand{\NULL}{{\sc NULL}}
\newcommand{\sysread}{\texttt{read}}
\newcommand{\syssync}{\texttt{sync}}
\newcommand{\fsync}{\texttt{fsync}}
\newcommand{\syswrite}{\texttt{write}}
\newcommand{\sysseek}{\texttt{lseek}}
\newcommand{\sysstat}{\texttt{stat}}
\newcommand{\make}{\texttt{make}}
\newcommand{\ioctl}{\texttt{ioctl}}
\newcommand{\panic}{\texttt{panic}}
\newcommand{\truncate}{\texttt{truncate}}
\newcommand{\rmdir}{\texttt{rmdir}}
\newcommand{\unlink}{\texttt{unlink}}
\newcommand{\open}{\textit{open}}
\newcommand{\close}{\textit{close}}
\newcommand{\linkscount}{\texttt{linkscount}}
\newcommand{\msync}{\textit{msync}}
\newcommand{\mmap}{\textit{mmap}}
\newcommand{\unmap}{\textit{munmap}}
\newcommand{\map}{\textit{map}}
\newcommand{\fetch}{\textit{gfetch}}
\newcommand{\acquire}{\textit{acquire}}
\newcommand{\commitxact}{\textit{commit}}
\newcommand{\commit}{\textit{commit}}
\newcommand{\barrier}{\textit{thread-barrier}}


% dsnvm
\newcommand{\dsnvm}{DSPM}
\newcommand{\dsm}{DSM}
\newcommand{\nvm}{PM}
\newcommand{\hotpot}{Hotpot}
\newcommand{\mrmw}{MRMW}
\newcommand{\mrsw}{MRSW}
\newcommand{\wfetch}{FETCH}
\newcommand{\cd}{CD}
\newcommand{\dr}{DR}
\newcommand{\on}{ON}
\newcommand{\dn}{DN}
\newcommand{\xn}{CN}
\newcommand{\master}{MN}
\newcommand{\xactid}{CID}
\newcommand{\dirty}{dirty}
\newcommand{\committed}{committed}
\newcommand{\redundant}{redundant}
\newcommand{\ib}{IB}
\newcommand{\sendreply}{\texttt{send-reply}}
\newcommand{\atomicsendreply}{\texttt{atomic-send-reply}}
\newcommand{\multisendreply}{\texttt{multicast-send-reply}}
\newcommand{\journaled}{JOURNALED}
\newcommand{\fsyncsafe}{FSYNC\_SAFE}
\newcommand{\X}{{$\times$}}
\newcommand{\pmfs}{PMFS}
\newcommand{\tmpfs}{tmpfs}
\newcommand{\Octopus}{Octopus}
\newcommand{\Mojim}{Mojim}
\newcommand{\dsmnoxact}{DSM-NoXact}
\newcommand{\dsmxact}{DSM-Xact}
\newcommand{\clflush}{\texttt{clflush}}
\newcommand{\pcommit}{\texttt{pcommit}}
\newcommand{\mfence}{\texttt{mfence}}
\newcommand{\sfence}{\texttt{sfence}}
\newcommand{\ra}{\textbf{R1.a}}
\newcommand{\rb}{\textbf{R1.b}}
\newcommand{\rcs}{\textbf{R2.a}}
\newcommand{\rcm}{\textbf{R2.b}}
\newcommand{\rdr}{\textbf{R3.r}}
\newcommand{\rdu}{\textbf{R3.u}}
\newcommand{\re}{\textbf{R3}}
\newcommand{\rf}{\textbf{R4}}


\begin{document}

%% FRONT MATTER
%
%  All of the front matter.
%  This includes the title, degree, dedication, vita, abstract, etc..
%  Modify the file template_frontmatter.tex to change these pages.
%
%
% UCSD Doctoral Dissertation Template
% -----------------------------------
% http://ucsd-thesis.googlecode.com
%
%


%% REQUIRED FIELDS -- Replace with the values appropriate to you

% No symbols, formulas, superscripts, or Greek letters are allowed
% in your title.
\title{Distributing and Disaggregating Hardware Resources in Data Centers}

\author{Yizhou Shan}
\degreeyear{\the\year}

% Master's Degree theses will NOT be formatted properly with this file.
\degreetitle{Doctor of Philosophy}

\field{Computer Science}

\chair{Professor Yiying Zhang}
% Uncomment the next line iff you have a Co-Chair
% \cochair{Professor Cochair Semimaster}
%
% Or, uncomment the next line iff you have two equal Co-Chairs.
%\cochairs{Professor Chair Masterish}{Professor Chair Masterish}

%  The rest of the committee members  must be alphabetized by last name.
\othermembers{
Professor George C. Papen\\
Professor Alex C. Snoeren\\
Professor Stefan Savage\\
Professor Geoffrey M. Voelker\\
}
\numberofmembers{5} % |chair| + |cochair| + |othermembers|



%% START THE FRONTMATTER
%
\begin{frontmatter}

%% TITLE PAGES
%
%  This command generates the title, copyright, and signature pages.
%
\makefrontmatter

%% DEDICATION
%
%  You have three choices here:
%    1. Use the ``dedication'' environment.
%       Put in the text you want, and everything will be formated for
%       you. You'll get a perfectly respectable dedication page.
%
%
%    2. Use the ``mydedication'' environment.  If you don't like the
%       formatting of option 1, use this environment and format things
%       however you wish.
%
%    3. If you don't want a dedication, it's not required.
%
%
\begin{dedication}
Dedicated to my family for their love and support.
\end{dedication}


% \begin{mydedication} % You are responsible for formatting here.
%   \vspace{1in}
%   \begin{flushleft}
% 	To me.
%   \end{flushleft}
%
%   \vspace{2in}
%   \begin{center}
% 	And you.
%   \end{center}
%
%   \vspace{2in}
%   \begin{flushright}
% 	Which equals us.
%   \end{flushright}
% \end{mydedication}



%% EPIGRAPH
%
%  The same choices that applied to the dedication apply here.
%
\begin{epigraph} % The style file will position the text for you.
\begin{center}
  Don’t adventures ever have an end? I suppose not.\\
  Someone else always has to carry on the story.\\
  -- The Fellowship of the Ring
\end{center}
\end{epigraph}

% \begin{myepigraph} % You position the text yourself.
%   \vfil
%   \begin{center}
%     {\bf Think! It ain't illegal yet.}
%
% 	\emph{---George Clinton}
%   \end{center}
% \end{myepigraph}


%% SETUP THE TABLE OF CONTENTS
%
\tableofcontents
\listoffigures  % Comment if you don't have any figures
\listoftables   % Comment if you don't have any tables



%% ACKNOWLEDGEMENTS
%
%  While technically optional, you probably have someone to thank.
%  Also, a paragraph acknowledging all coauthors and publishers (if
%  you have any) is required in the acknowledgements page and as the
%  last paragraph of text at the end of each respective chapter. See
%  the OGS Formatting Manual for more information.
%
\section*{Acknowledgments}

We would like to thank the anonymous reviewers and our shepherd Michael Swift
for their tremendous feedback and comments, which have
substantially improved the content and presentation of this paper.
We are also thankful to Sumukh H. Ravindra for his contribution in the early
stage of this work.

This material is based upon work supported by the National
Science Foundation under the following grant: NSF 1719215.
Any opinions, findings, and conclusions or recommendations
expressed in this material are those of the authors and do not 
necessarily reflect the views of NSF or other institutions.


%% VITA
%
%  A brief vita is required in a doctoral thesis. See the OGS
%  Formatting Manual for more information.
%
\begin{vitapage}
\begin{vita}
  \item[2014] B.~S. in Computer Engineering, Beihang University, Beijing 
  \item[2014-2016] Research Assistant, University of Chinese Academy of Sciences, Institution of Computing Technology, Beijing
  \item[2022] Ph.~D. in Computer Science, University of California San Diego
\end{vita}
\begin{publications}
  \item \underline{Yizhou Shan}, Will Lin, Ryan Kosta, Arvind Krishnamurthy, Yiying Zhang, ``Disaggregating and Consolidating Network Functionalities with SuperNIC'', \textit{arXiv, 2022}.
  \item \underline{Yizhou Shan}, Zhiyuan Guo (co-first authors), Xuhao Luo, Yutong Huang, Yiying Zhang, ``Clio: A Hardware-Software Co-Designed Disaggregated Memory System'', \textit{ASPLOS, 2022}
  \item Shin-Yeh Tsai, \underline{Yizhou Shan}, Yiying Zhang, ``Disaggregating Persistent Memory and Controlling Them Remotely: An Exploration of Passive Disaggregated Key-Value Stores'', \textit{ATC, 2020}
  \item Stanko Novakovic, \underline{Yizhou Shan}, Aasheesh Kolli, Michael Cui, Yiying Zhang, Haggai Eran, Liran Liss, Michael Wei, Dan Tsafrir, Marcos Aguilera, ``Storm: a fast transactional dataplane for remote data structures'', \textit{SYSTOR, 2019}, \textbf{Best Paper Award}
  \item \underline{Yizhou Shan}, Yutong Huang, Yilun Chen, Yiying Zhang, ``LegoOS: A Disseminated, Distributed OS for Hardware Resource Disaggregation'', \textit{OSDI, 2018}, \textbf{Best Paper Award}
  \item \underline{Yizhou Shan}, Shin-Yeh Tsai, Yiying Zhang, ``Distributed Shared Persistent Memory'', \textit{SoCC, 2017}
\end{publications}
\end{vitapage}


%% ABSTRACT
%
%  Doctoral dissertation abstracts should not exceed 350 words.
%   The abstract may continue to a second page if necessary.
%
\begin{abstract}
This is thesis abstract. fill me in.
\end{abstract}


\end{frontmatter}






%% DISSERTATION

% A common strategy here is to include files for each of the chapters. I.e.,
% Place the chapters is separate files: 
%   chapter1.tex, chapter2.tex
% Then use the commands:
%   \include{chapter1}
%   \include{chapter2}
%
% Of course, if you prefer, you can just start with
%   \chapter{My First Chapter Name}
% and start typing away.  
\chapter{Introduction}
This is only a test.
\section{A section}
Test.

\subsection{A Figure Example}
\label{ssec:figure_example}

This subsection shows a sample figure.

\begin{figure}[h] 
  \centering
  \includegraphics[width=0.5\textwidth]{sandiego}
  \caption[A picture of San Diego. Short figure caption must be \protect{$< 4$} lines in the list of figures]
{A picture of San Diego.  Short figure caption must be \protect{$< 4$} lines in the list of figures and match the start of the main figure caption verbatim. Note that figures must be on their own line (no neighboring text) and captions must be single-spaced and appear \protect\textit{below} the figure.  Captions can be as long as you want, but if they are longer than 4 lines in the list of figures, you must provide a short figure caption.\index{SanDiego}}
  \label{fig:sandiego}
\end{figure}

\subsection{A Table Example}

While in Section \ref{ssec:figure_example} Figure \ref{fig:sandiego} we had a majestic figure, here we provide a crazy table example.


%%%% TABLE 1 %%%%
\vspace{0.25in}
\begin{table}[!ht]
\caption[A table of when I get hungry.  Short table caption must be \protect{$< 4$} lines in the list of tables]{A table of when I get hungry. Short table caption must be \protect{$< 4$} lines in the list of tables and match the start of the main table caption verbatim.  Note that tables must be on their own line (no neighboring text) and captions must be single-spaced and appear \protect\textit{above} the table.  Captions can be as long as you want, but if they are longer than 4 lines in the list of figures, you must provide a short figure caption.}

\vspace{-0.25in}
\begin{center}
\begin{tabular}{|p{1in}|p{2in}|p{3in}|}

\hline
Time of day & Hunger Level & Preferred Food \\

\hline
8am & high & IHOP (French Toast) \\

\hline
noon & medium & Croutons (Tomato Basil Soup \& Granny Smith Chicken Salad) \\

\hline
5pm & high & Bombay Coast (Saag Paneer) or Hi Thai (Pad See Ew) \\

\hline
8pm & medium & Yogurt World (froyo!) \\

\hline
\end{tabular}
\end{center}
\label{tab:analysis3}
\end{table}

\chapter{Distributed Shared Persistent Memory}

\section{Introduction}
\label{sec:introduction}

Next-generation non-volatile memories ({\em NVMs}), 
such as 3DXpoint~\cite{Intel3DXpoint}, 
phase change memory ({\em PCM}),
spin-transfer torque magnetic memories ({\em STTMs}),  and the memristor
will provide byte addressability, persistence, high density, and DRAM-like performance~\cite{NVMDB}.
These developments are poised to radically alter the landscape of memory and storage technologies
and have already inspired a host of research 
projects~\cite{Bailey10-OSImpl,Coburn11-ASPLOS, sosp09:bpfs, Dulloor14-EuroSys, hotos09:mogul, Volos11-ASPLOS, Xiaojian11-SC,Zhang15-Mojim,Octopus}.
However, most previous research on NVMs has focused on using them in a single machine environment.
Even though NVMs have the potential to greatly improve the performance and reliability of large-scale applications,
it is still unclear how to best utilize them in distributed, datacenter environments. 

This paper takes a significant step towards the goal of using NVMs in distributed datacenter environments.
We propose {\em Distributed Shared Persistent Memory (\dsnvm)},
a framework that provides a global, shared, and persistent memory space 
using a pool of machines with NVMs attached at the main memory bus.
Applications can perform native memory load and store instructions to access both local and remote data in this global memory space 
and can at the same time make their data persistent and reliable.
\dsnvm\ can benefit both single-node persistent-data applications that want to scale out efficiently
and shared-memory applications that want to add durability to their data.

Unlike traditional systems with separate memory and storage layers~\cite{Larchant,Perdis00,Larchant94,PerDis},
we propose to use just one layer that incorporates both distributed memory and 
distributed storage in \dsnvm.
\dsnvm's one-layer approach eliminates the performance overhead of data marshaling and unmarshaling,
and the space overhead of storing data twice. 
With this one-layer approach, \dsnvm\ can potentially provide the low-latency performance, 
vast persistent memory space, data reliability, and high availability
that many modern datacenter applications demand. 


Building a \dsnvm\ system presents its unique challenges.
Adding ``Persistence'' to Distributed Shared Memory (DSM)
is not as simple as just making in-memory data durable.
Apart from data durability, \dsnvm\ needs to provide two key features that DSM does not have:
persistent naming and data reliability.
In addition to accessing data in \nvm\ via native memory loads and stores,
applications should be able to easily
name, close, and re-open their in-memory data structures.
User data should also be reliably stored in NVM and sustain various types of failures; %(\eg, to have $N$ copies of persistent data).
they need to be consistent both within a node and across distributed nodes after crashes.
To make it more challenging, 
\dsnvm\ has to deliver these guarantees without sacrificing application performance
in order to preserve the low-latency performance of NVMs.

We built {\em \hotpot}, a \dsnvm\ system in the Linux kernel.
\hotpot\ offers low-latency, direct memory access, data persistence, reliability, and
high availability to datacenter applications.
It exposes a global virtual memory address space to each user application
and provides a new persistent naming mechanism that is both easy-to-use and efficient.
Internally, \hotpot\ organizes and manages data in a flexible way
and uses a set of adaptive resource management techniques to improve performance and scalability.

\hotpot\ builds on two main ideas to efficiently provide data reliability with distributed shared memory access.
Our first idea is to integrate distributed memory caching and data replication 
by imposing {\em morphable} states on persistent memory ({\em \nvm}) pages.

In DSM systems, when an application on a node accesses shared data in remote memory {\em on demand},
DSM caches these data copies in its local memory for fast accesses
and later evicts them when reclaiming local memory space.
Like DSM, \hotpot\ caches application-accessed data in local \nvm\
and ensures the coherence of multiple cached copies on different nodes.
But \hotpot\ also uses these cached data as {\em persistent data replicas}
and ensures their reliability and crash consistency.

On the other hand, unlike distributed storage systems, which {\em creates} extra data replicas 
to meet user-specified reliability requirements, 
\hotpot\ makes use of data copies that {\em already exist} in the system when
they were fetched to a local node due to application memory accesses.
 
In essence, every local copy of data serves two simultaneous purposes.
First, applications can access it locally without any network delay.
Second, by placing the fetched copy in \nvm, it can be treated as a persistent replica 
for data reliability.

This seemingly-straightforward integration is not simple. 
Maintaining wrong or outdated versions of data can result in inconsistent data.
To make it worse, these inconsistent data will be persistent in \nvm.
We carefully designed a set of protocols to deliver data reliability and crash consistency guarantees 
while integrating memory caching and data replication.

Our second idea is to exploit application behaviors and intentions in the \dsnvm\ setting. 
Unlike traditional memory-based applications, persistent-data-based applications,
\dsnvm's targeted type of application, have well-defined data {\em commit points}
where they specify what data they want to make persistent.
When a process in such an application makes data persistent,
it usually implies that the data can be {\em visible} outside the process (\eg, to other processes or other nodes). 
\hotpot\ utilizes these data commit points to also push updates to cached copies on distributed nodes
to avoid maintaining coherence on every \nvm\ write. %~\cite{XXX},
Doing so greatly improves the performance of \hotpot, 
while still ensuring correct memory sharing and data reliability.

To demonstrate the benefits of \hotpot, we ported the MongoDB~\cite{MongoDB} NoSQL database to \hotpot\
and built a distributed graph engine based on PowerGraph~\cite{Gonzalez12-OSDI} on \hotpot. 
Our MongoDB evaluation results show that \hotpot\ outperforms a \nvm-based replication system~\cite{Zhang15-Mojim} by up to 3.1\x{}, 
a recent \nvm-based distributed file systems~\cite{Octopus} by up to 3.0\x{}, and a DRAM-based file system by up to 53\x{}. 
\hotpot\ outperforms PowerGraph by 2.3\x{} to 5\x{}, a recent DSM system~\cite{Nelson15-ATC} by 1.3\x{} to 3.2\x{},
and two DSM systems that we built by TODO.
Moreover, \hotpot\ delivers stronger data reliability and availability guarantees than these alternative systems.

Overall, this paper makes the following key contributions:

\begin{itemize}
\item We are the first to introduce the Distributed Shared Persistent Memory (DSPM) model
and among the first to build distributed \nvm-based systems.
The DSPM model provides direct and shared memory accesses to a distributed set of \nvm{}s 
and is an easy and efficient way for datacenter applications to use \nvm.

\item We propose a one-layer approach to build \dsnvm\ by 
integrating memory coherence and data replication.
The one-layer approach avoids the performance cost of two or more indirection layers.

\item We designed two distributed data commit protocols with different consistency levels 
and corresponding recovery protocols to 
ensure data durability, reliability, and availability.

\item We built the first \dsnvm\ system, \hotpot, in the Linux kernel, 
and two traditional kernel-level DSM systems (as comparison to \hotpot). 
\hotpot\ and the two DSM systems are both open sourced.

\item We demonstrated \hotpot's performance benefits and ease of use with two real datacenter applications
and extensive microbenchmark evaluation. 
We compared \hotpot\ with five existing file systems and distributed memory systems, 
and two in-house DSM systems.

\end{itemize}

The rest of the paper is organized as follows.
Section 2 presents and analyzes several recent datacenter trends that motivated our design of DSPM.
We discuss the benefits and challenges of DSPM in Section 3.
Section 4 presents the architecture and abstraction of Hotpot.
We then discuss Hotpot's data management in Section 5.
We present our protocols and mechanisms to ensure data durability, consistency, reliability, and availability in Section 6.
Section 7 briefly discusses the network layer we built underlying \hotpot,
and Section 8 presents detailed evaluation of Hotpot.
We cover related work in Section 9 and conclude in Section 10.

\section{Motivation}
\label{sec:motivation}

\dsnvm\ is motivated by three datacenter trends: 
emerging hardware \nvm\ technologies, 
modern data-intensive applications' data sharing, persistence, and reliability needs, 
and the availability of fast datacenter network.

\subsection{Persistent Memory and PM Apps}
Next-generation non-volatile memories ({\em NVMs}), 
such as 3DXpoint~\cite{Intel3DXpoint}, phase change memory ({\em PCM}),
spin-transfer torque magnetic memories ({\em STTMs}), and the memristor
will provide byte addressability, persistence, and latency that is within 
an order of magnitude of 
DRAM~\cite{hosomi2005novel,Lee10-pcmquest,lee2010phase,lee2011fast,qureshi2010morphable,NVMDB,yang2013memristive,Octopus}.
These developments are poised to radically alter the landscape of memory and storage technologies.

NVMs can attach directly to the main memory bus to form Persistent Memory, or \nvm. 
If applications want to exploit all the low latency and byte-addressability benefits of \nvm,
they should directly access it via memory load and store instructions without any software 
overheads~\cite{Coburn11-ASPLOS,Volos11-ASPLOS,Zhang15-Mojim,Memory-Persistency,Kamino-EuroSys17,pmxact-asplos16} 
(we call this model {\em durable in-memory computation}),
rather than accessing it via a file system~\cite{sosp09:bpfs,Dragojevic14-NSDI,Dulloor14-EuroSys,Xiaojian11-SC,HiNFS-Eurosys16,Octopus}.

{
\begin{figure}[th]
\begin{center}
\centerline{\includegraphics[width=0.8\textwidth]{hotpot/Figures/g_plot_pagerank_average.pdf}}
\caption[PowerGraph Sharing Analysis.]{PowerGraph Sharing Analysis.
Results of running PageRank~\cite{PageRank} on a Twitter graph~\cite{Kwak10-WWW}.
Black lines represent total amount of sharing.
Green lines represent sharing within five seconds.
}
\label{fig-pagerank}
\end{center}
\end{figure}
}
Unfortunately, most previous durable in-memory systems were designed for the single-node environment.
With modern datacenter applications' computation scale, 
we have to be able to scale out these single-node \nvm\ systems.

\subsection{Shared Memory Applications}
Modern data-intensive applications increasingly need
to access and share vast amounts of data fast. 
We use PIN~\cite{Luk05-PLDI} to collect memory access traces of two popular data-intensive applications, 
TensorFlow~\cite{TensorFlow} and PowerGraph~\cite{Gonzalez12-OSDI}.
Figures~\ref{fig-pagerank} and \ref{fig-tensorflow} show the total number of reads and writes performed to the same memory location 
by $N$ threads and the amount of these shared locations.
There are a significant amount of shared read accesses in these applications,
especially across a small set of threads.
We further divided the memory traces into smaller time windows 
and found that there is still a significant amount of sharing, 
indicating that many shared accesses occur at similar times. 

Distributed Shared Memory ({\em \dsm}) takes the shared memory concept a step further 
by organizing a pool of machines into a globally shared memory space.
Researchers and system builders have developed a host of software and hardware \dsm\ systems in the past few 
decades~\cite{Bennett90-PPOPP,Bisiani90-ISCA,Black89-COMPCON,Delp:1988:AIM:59505,Fleisch89-SOSP,Gibbons91-SPAA,Kontothanassis97-ISCA,Lo94-AC,Kessler89-ACM,Keleher92-ISCA,Ramachandran91-Wiley,Zhou92-IEEE,Stumm90-IEEE,Stumm90-IPDPS,HLRC,Shasta}.
Recently, there is a new interest in \dsm~\cite{Nelson15-ATC} to support modern data-intensive applications.

However, although DSM scales out shared-memory applications, 
there has been no persistent-memory support for DSM.
DSM systems all had to checkpoint to disks~\cite{Stumm90,Richard93,Neves94}.
Memory persistence
can allow these applications to checkpoint fast and recover fast~\cite{Narayanan12-ASPLOS}.

\subsection{Fast Network and RDMA}
Datacenter network performance has improved significantly over the past decades.
InfiniBand ({\em \ib}) NICs and switches support high bandwidth ranging from 40 to 100\gbps.
Remote Direct Memory Access ({\em RDMA}) technologies that provide low-latency remote memory accesses
have become more mature for datacenter uses in recent years~\cite{FaSST,Dragojevic14-NSDI,Kalia14-SIGCOMM,Guo16-SIGCOMM}.
These network technology advances
make remote-memory-based systems~\cite{Nelson15-ATC,GU17-NSDI,OSDI-Disaggregate,Chen16-EUROSYS,Binnig16-VLDB,Zamanian17-VLDB} more attractive than decades ago.

\subsection{Lack of Distributed PM Support}

{
\begin{figure}[th]
\begin{center}
\centerline{\includegraphics[width=0.8\textwidth]{hotpot/Figures/g_plot_tensorflow_average.pdf}}
\caption[TensorFlow Sharing Analysis.]
{
TensorFlow Sharing Analysis.
Results of running a hand-writing recognition workloads provided by TensorFlow.
Black lines represent total amount of sharing.
Green lines represent sharing within five seconds.
}
\label{fig-tensorflow}
\end{center}
\end{figure}
}

Many large-scale datacenter applications require fast access to vast amounts of persistent data
and could benefit from \nvm's performance, durability, and capacity benefits.
For \nvm{}s to be successful in datacenter environments, they have to support these applications.
However, neither traditional distributed storage systems or DSM systems are designed for \nvm.
Traditional distributed storage systems~\cite{AdyaEtAl-Farsite,calder11-azure,DeCandia+07-Dynamo,Ghemawat03-GoogleFS,KubiEtAl00-Ocean,Petersen97-Bayou}
target slower, block-based storage devices.
Using them on \nvm{}s will result in excessive software and network overheads that outstrip \nvm's low latency performance~\cite{Zhang15-Mojim}.
DSM systems were designed for fast, byte-addressable memory, but lack the support for data durability and reliability.

Octopus~\cite{Octopus} is a recent RDMA-enabled distributed file system built for \nvm.
Octopus and our previous work Mojim~\cite{Zhang15-Mojim} are the only distributed \nvm-based systems that we are aware of.
Octopus was developed in parallel with \hotpot\ and has a similar goal as \hotpot: 
to manage and expose distributed PM to datacenter applications. 
However, Octopus uses a file system abstraction and is built in the user level.
These designs add significant performance overhead to native PM accesses (Section~\ref{sec:mongodb}).
Moreover, Octopus does not provide any data reliability or high availability, 
both of which are key requirements in datacenter environments.

\if 0
\section{Distributed Shared Persistent Memory}
\label{sec:dspm}

The datacenter application and hardware trends described in Section~\ref{sec:motivation} 
clearly point to one promising direction of using \nvm\ in datacenter environments --- 
as distributed, shared, persistent memory (\dsnvm).
A \dsnvm\ system manages a distributed set of \nvm{}-equipped machines  
and provides the abstraction of a global virtual address space and a data persistence interface to applications.
This section gives a brief discussion on the \dsnvm\ model.

\subsection{\dsnvm\ Benefits and Usage Scenarios}
\dsnvm\ offers low-latency, shared access to vast amount of durable data in distributed \nvm,
and the reliability and high availability of these data.
Application developers can build in-memory data structures with the global virtual address space 
and decide how to name their data and when to make data persistent.

Applications that fit \dsnvm\ well have two properties:
accessing data with memory instructions and making data durable explicitly.
We call the time when an application makes its data persistent a {\em commit point}.
There are several types of datacenter applications that meet the above two descriptions and can benefit from running on \dsnvm.

First, applications that are built for single-node \nvm\
can be easily ported to \dsnvm\ and scale out to distributed environments.
These applications store persistent data as in-memory data structures 
and already express their commit points explicitly.
Similarly, storage applications that use memory-mapped files also fit \dsnvm\ well,
since they operate on in-memory data and explicitly make them persistent at well-defined commit points (\ie, \msync).
Finally, \dsnvm\ fits shared-memory or DSM-based applications that desire to incorporate durability.
These applications do not yet have durable data commit points,
but we expect that when developers want to make their applications durable, 
they should have the knowledge of when and what data they want make durable.

\subsection{\dsnvm\ Challenges}
\label{sec:challenges}
Building a \dsnvm\ system presents several new challenges.

First, {\em what type of abstraction should \dsnvm\ offer to support both direct memory accesses and data persistence (Section~\ref{sec:abstraction})}?
To perform native memory accesses, application processes should use virtual memory addresses. 
But virtual memory addresses are not a good way to {\em name} persistent data.
\dsnvm\ needs a naming mechanism that applications can easily use to retrieve their in-memory data after reboot or crashes (Section~\ref{sec:naming}).
Allowing direct memory accesses to \dsnvm\ also brings another new problem:
pointers need to be both persistent in \nvm\ and consistent across machines (Section~\ref{sec:addressing}).

Second, {\em how to efficiently organize data in \dsnvm\ to deliver good application performance (Section~\ref{sec:data})?}
To make \dsnvm's interface easy to use and transparent, 
\dsnvm\ should manage the physical \nvm\ space for applications and handle \nvm\ allocation.
\dsnvm\ needs a flexible and efficient data management mechanism to deliver good performance to different types of applications.

Finally, {\em \dsnvm\ needs to ensure both distributed cache coherence and data reliability at the same time} (Section~\ref{sec:xact}).
The former requirement ensures the coherence of multiple cached copies at different machines under concurrent accesses and is usually enforced in a distributed memory layer.
The latter provides data reliability and availability when crashes happen and is implemented in distributed storage systems or distributed databases.
\dsnvm\ needs to incorporate both these two different requirements in one layer in a correct and efficient way.
%Note that PM is attached to main memory bus directly, hence we assume PM share the same CPU cache coherence mechanism with DRAM.
%Hotpot focus on cache coherence among different cached copies across nodes.

\section{\lego\ Design}
\label{sec:lego:design}

Based on the \splitkernel\ architecture,
we built {\em \lego}, the first OS designed for hardware resource disaggregation.
\lego\ is a research prototype that demonstrates the feasibility of the \splitkernel\ design,
but it is not the only way to build a \splitkernel.
\lego' design targets three types of hardware components:
processor, memory, and storage,
and we call them {\em \pcomponent, \mcomponent}, and {\em \scomponent}.

This section first introduces the abstraction \lego\ exposes to users
and then describes the hardware architecture of components \lego\ runs on.
Next, we explain the design of \lego' process, memory, and storage \microos{}s.
Finally, we discuss \lego' global resource management and failure handling mechanisms.

Overall, \lego\ achieves the following design goals:

\begin{itemize}

\item Clean separation of process, memory, and storage functionalities.

\item Monitors run at hardware components and fit device constraints.

\item Comparable performance to monolithic Linux servers.

\item Efficient resource management and memory failure handling, both in space and in performance. % and performance-efficient memory replication scheme.

\item Easy-to-use, backward compatible user interface.

\item Supports common Linux system call interfaces.

\end{itemize}

\subsection{Abstraction and Usage Model}
\lego\ exposes a distributed set of {\em virtual nodes}, or {\em \vnode}, to users.
From users' point of view, a \vnode\ is like a virtual machine. 
Multiple users can run in a \vnode\ and each user can run multiple processes.
Each \vnode\ has a unique ID, a unique virtual IP address, %({\em \vip}),
and its own storage mount point. % ({\em \vmount}).
\lego\ protects and isolates the resources given to each \vnode\ from others.
Internally, one \vnode\ can run on multiple \pcomponent{}s, multiple \mcomponent{}s,
and multiple \scomponent{}s.
At the same time, each hardware component can host resources for more than one \vnode.
The internal execution status is transparent to \lego\ users;
they do not know which physical components their applications run on.

With \splitkernel's design principle of components not being coherent,
\lego\ does not support writable shared memory across processors. %execute application threads that need to have shared write access to common memory.
\lego\ assumes that threads within the same process access shared memory
and threads belonging to different processes do not share writable memory,
and \lego\ makes scheduling decision based on this assumption (\S\ref{sec:lego:proc-scheduling}).
Applications that use shared writable memory across processes (\eg, with MAP\_SHARED)
will need to be adapted to use message passing across processes.
We made this decision because writable shared memory across processes is rare 
(we have not seen a single instance in the datacenter applications we studied),
and supporting it makes both hardware and software more complex 
(in fact, we have implemented this support but later decided not to include it because of its complexity).

One of the initial decisions we made when building \lego\ is to support the Linux system call interface 
and unmodified Linux ABI,
because doing so can greatly ease the adoption of \lego.
Distributed applications that run on Linux can seamlessly run on a \lego\ cluster
by running on a set of \vnode{}s. % and using their virtual IP addresses to communicate.

\subsection{Hardware Architecture}
\label{sec:lego:hardware}
{
\begin{figure}[th]
\begin{center}
\centerline{\includegraphics[width=0.8\textwidth]{lego/Figures/hwarch.pdf}}
\caption[\lego\ \pcomponent\ and \mcomponent\ Architecture.]{\lego\ \pcomponent\ and \mcomponent\ Architecture.}
\label{fig-lego-hw-arch}
\end{center}
\end{figure}
}

\lego\ \pcomponent, \mcomponent, and \scomponent\ are independent devices,
each having their own hardware controller and network interface (for \pcomponent, the hardware controller is the processor itself).
Our current hardware model uses CPU in \pcomponent, 
DRAM in \mcomponent, and SSD or HDD in \scomponent.
We leave exploring other hardware devices for future work.

To demonstrate the feasibility of hardware resource disaggregation,
we propose a \pcomponent{} and an \mcomponent\ architecture designed 
within today's network, processor, and memory performance and hardware constraints
(Figure~\ref{fig-lego-hw-arch}).

\noindent{\textit{\uline{Separating process and memory functionalities.}}}
\lego\ moves all hardware memory functionalities to \mcomponent{}s 
(e.g., page tables, TLBs) and leaves {\em only} caches at the \pcomponent{} side. 
With a clean separation of process and memory hardware units, 
the allocation and management of memory can be completely transparent to \pcomponent{}s.
Each \mcomponent{} can choose its own memory allocation technique
and virtual to physical memory address mappings (\eg, segmentation). 

\noindent{\textit{\uline{Processor virtual caches.}}}
After moving all memory functionalities to \mcomponent{}s,  
\pcomponent{}s will only see virtual addresses and have to use virtual memory addresses to access its caches. 
Because of this, \lego\ organizes all levels of \pcomponent{} caches as {\em virtual caches}~\cite{Goodman-ASPLOS87,Wang-ISCA89},
\ie, virtually-indexed and virtually-tagged caches.

A virtual cache has two potential problems, commonly known as synonyms and homonyms~\cite{CacheMemory82}.
Synonyms happens when a physical address maps to multiple virtual addresses (and thus multiple virtual cache lines) 
as a result of memory sharing across processes,
and the update of one virtual cache line will not reflect to other lines that share the data.
Since \lego\ does not allow writable inter-process memory sharing,
it will not have the synonym problem.
The homonym problem happens when two address spaces use the same virtual address for their own different data.
Similar to previous solutions~\cite{OVC}, we solve homonyms by storing an address space ID (ASID) with each cache line,
and differentiate a virtual address in different address spaces using ASIDs.

\noindent{\textit{\uline{Separating memory for performance and for capacity.}}}
Previous studies~\cite{Gao16-OSDI,GU17-NSDI} and our own show that today's network speed 
cannot meet application performance requirements if all memory accesses are across the network. 
Fortunately, many modern datacenter applications exhibit strong memory access temporal locality.
For example, we found 90\% of memory accesses in PowerGraph~\cite{Gonzalez12-OSDI} go to just 0.06\% of total memory
and 95\% go to 3.1\% of memory
(22\% and 36\% for TensorFlow~\cite{TensorFlow} respectively,
5.1\% and 6.6\% for Phoenix~\cite{Ranger07-HPCA}).
%PG 90% 0.0063G 95% 0.301G 100% 9.68G
%TF 90% 0.608G 95% 0.968G 100% 2.7G

With good memory-access locality, we propose to %separate hardware memory into two categories and organize them differently:
leave a small amount of memory (\eg, 4\GB) at each \pcomponent{}
and move most memory across the network (\eg, few TBs per \mcomponent{}).
\pcomponent{}s' local memory can be regular DRAM 
or the on-die HBM~\cite{HBM-JEDEC,Knights-Landing},
and \mcomponent{}s use DRAM or NVM.

Different from previous proposals~\cite{Lim09-disaggregate}, 
we propose to organize \pcomponent{}s' DRAM/HBM as cache rather than main memory
for a clean separation of process and memory functionalities.
We place this cache under the current processor Last-Level Cache (LLC)
and call it an extended cache, or {\em \excache}.
\excache\ serves as another layer in the memory hierarchy between LLC and memory across the network.
With this design, \excache\ can serve hot memory accesses fast, while \mcomponent{}s can provide the capacity applications desire. 

\excache\ is a virtual, inclusive cache,
and we use a combination of hardware and software to manage \excache.
Each \excache\ line has a (virtual-address) tag and two access permission bits (one for read/write and one for valid).
These bits are set by software when a line is inserted to \excache\ and checked by hardware at access time.
For best hit performance, the hit path of \excache\ is handled purely by hardware
--- the hardware cache controller maps a virtual address to an \excache\ set, 
fetches and compares tags in the set, and on a hit, fetches the hit \excache\ line.
Handling misses of \excache\ is more complex than with traditional CPU caches, 
and thus we use \lego\ to handle the miss path of \excache\ (see \S\ref{sec:lego:excachemgmt}).

Finally, we use a small amount of DRAM/HBM at \pcomponent{} for \lego' own kernel data usages,
accessed directly with physical memory addresses and managed by \lego. 
\lego\ ensures that all its own data fits in this space to avoid going to \mcomponent{}s.

With our design, \pcomponent{}s do not need any address mappings:
\lego\ accesses all \pcomponent{}-side DRAM/HBM using physical memory addresses
and does simple calculations to locate the \excache\ set for a memory access.
Another benefit of not handling address mapping at \pcomponent{}s and moving TLBs to \mcomponent{}s 
is that \pcomponent{}s do not need to access TLB or suffer from TLB misses,
potentially making \pcomponent{} cache accesses faster~\cite{Kaxiras-ISCA13}.
%We use software~\cite{softvm-HPCA97,Tsai-ISCA17} (\lego) to manage \excache\ and the kernel physical memory,
%although they can all be implemented in hardware too.

\subsection{Process Management}
The \lego\ {\em process \microos{}} runs in the kernel space of a \pcomponent\
and manages the \pcomponent's CPU cores and \excache. 
\pcomponent{}s run user programs in the user space.

\subsubsection{Process Management and Scheduling}
\label{sec:lego:proc-scheduling}
At every \pcomponent, \lego\ uses a simple local thread scheduling model 
that targets datacenter applications 
(we will discuss global scheduling in \S~\ref{sec:lego:grm}).
\lego\ dedicates a small amount of cores for kernel background threads 
(currently two to four)
and uses the rest of the cores for application threads.
When a new process starts, \lego\ uses a global policy to choose a \pcomponent{} for it (\S~\ref{sec:lego:grm}).
Afterwards, \lego\ schedules new threads the process spawns on the same \pcomponent{} 
by choosing the cores that host fewest threads.
After assigning a thread to a core, 
we let it run to the end with no scheduling or kernel preemption under common scenarios.
For example, we do not use any network interrupts 
and let threads busy wait on the completion of outstanding network requests, 
since a network request in \lego\ is fast 
(\eg, fetching an \excache\ line from an \mcomponent\ takes around 6.5\mus).
\lego\ improves the overall processor utilization in a disaggregated cluster,
since it can freely schedule processes on any \pcomponent{}s without considering memory allocation.
Thus, we do not push for perfect core utilization when scheduling individual threads
and instead aim to minimize scheduling and context switch performance overheads.
Only when a \pcomponent{} has to schedule 
more threads than its cores will
\lego\ start preempting threads on a core.

\subsubsection{\excache\ Management}
\label{sec:lego:excachemgmt}
\lego\ process \microos\ configures and manages \excache.
During the \pcomponent{}'s boot time, \lego\ configures the set associativity of \excache\
and its cache replacement policy.
While \excache\ hit is handled completely in hardware, 
\lego\ handles misses in software.
When an \excache\ miss happens, 
the process \microos\ fetches the corresponding line from an \mcomponent\ and inserts it to \excache.
If the \excache\ set is full, the process \microos\ first evicts a line in the set.
It throws away the evicted line if it is clean
and writes it back to an \mcomponent{} if it is dirty.
\lego\ currently supports two eviction policies: FIFO and LRU.
For each \excache\ set, \lego\ maintains a FIFO queue (or an approximate LRU list)
and chooses \excache\ lines to evict based on the corresponding policy (see \S\ref{sec:lego:procimpl} for details).

\subsubsection{Supporting Linux Syscall Interface}
One of our early decisions is to support Linux ABIs for backward compatibility
and easy adoption of \lego.
A challenge in supporting the Linux system call interface is that 
many Linux syscalls are associated with {\em states},
information about different Linux subsystems that is stored with each process 
and can be accessed by user programs across syscalls.
For example, Linux records the states of a running process' open files, socket connections, and several other entities,
and it associates these states with file descriptors ({\em fd}s) that are exposed to users.
In contrast, \lego\ aims at the clean separation of OS functionalities.
With \lego' stateless design principle, each component only stores information about its own resource
and each request across components contains all the information that the destination component needs to handle the request.
To solve this discrepancy between the Linux syscall interface and \lego' design, 
we add a layer on top of \lego' core process \microos\ at each \pcomponent\ to store Linux states
and translate these states and the Linux syscall interface to \lego' internal interface.

\subsection{Memory Management}

We use \mcomponent{}s for three types of data:
anonymous memory (\ie, heaps, stacks), 
memory-mapped files, and storage buffer caches.
The \lego\ {\em memory \microos{}}
manages both the virtual and physical memory address spaces,
their allocation, deallocation, and memory address mappings.
It also performs the actual memory read and write.
No user processes run on \mcomponent{}s 
and they run completely in the kernel mode
(same is true for \scomponent{}s). 

\lego\ lets a process address space span multiple \mcomponent{}s
to achieve efficient memory space utilization and high parallelism.
Each application process uses one or more \mcomponent{}s to host its data
and a {\em home \mcomponent},
an \mcomponent\ that initially loads the process, 
accepts and oversees all system calls related to virtual memory space management
(\eg, \brk, \mmap, \munmap, and \mremap).
\lego\ uses a global memory resource manager ({\em \gmm}) to assign a home \mcomponent{} to each new process at its creation time.
A home \mcomponent\ can also host process data.

\subsubsection{Memory Space Management}
\noindent{\textit{\uline{Virtual memory space management.}}}
We propose a two-level approach to manage distributed virtual memory spaces,
where the home \mcomponent\ of a process makes coarse-grained, high-level virtual memory allocation decisions
and other \mcomponent{}s perform fine-grained virtual memory allocation.
This approach minimizes network communication during both normal memory accesses and virtual memory operations,
while ensuring good load balancing and memory utilization.
Figure~\ref{fig-dist-vma} demonstrates the data structures used. % in virtual memory space management.

At the higher level, we split each virtual memory address space into coarse-grained, fix-sized {\em virtual regions},
or {\em \vregion{}s} (\eg, of 1\GB).
Each \vregion\ that contains allocated virtual memory addresses (an active \vregion) is {\em owned} by an \mcomponent{}.
The owner of a \vregion\ handles all memory accesses and virtual memory requests within the \vregion.

{
\begin{figure}[th]
\begin{minipage}{\figWidth}
\begin{center}
\centerline{\includegraphics[width=2.8in]{Figures/dist-vma.pdf}}
%\vspace{-0.1in}
\mycaption{fig-dist-vma}{Distributed Memory Management.}
{
}
\end{center}
\end{minipage}
\vspace{-0.15in}
\end{figure}
}

The lower level stores user process virtual memory area ({\em vma}) information,
such as virtual address ranges and permissions, in {\em vma trees}.
The owner of an active \vregion\ stores a vma tree for the \vregion,
with each node in the tree being one vma.
A user-perceived virtual memory range can split across multiple \mcomponent{}s,
but only one \mcomponent{} owns a \vregion.

\vregion\ owners perform the actual virtual memory allocation and vma tree set up.
A home \mcomponent{} can also be the owner of \vregion{}s,
but the home \mcomponent{} does not maintain any information about memory that belongs to \vregion{}s owned by other \mcomponent{}s.
It only keeps the information of which \mcomponent{} owns a \vregion\ (in a {\em \vregion\ array})
and how much free virtual memory space is left in each \vregion.
These metadata can be easily reconstructed if a home \mcomponent{} fails.

When an application process wants to allocate a virtual memory space,
the \pcomponent{} forwards the allocation request 
to its home \mcomponent{} (\circled{1} in Figure~\ref{fig-dist-vma}).
The home \mcomponent{} uses its stored information of available virtual memory space in \vregion{}s
to find one or more \vregion{}s that best fit the requested amount of virtual memory space.
If no active \vregion\ can fit the allocation request, the home \mcomponent{} makes a new \vregion\ active and 
contacts the \gmm\ (\circled{2} and \circled{3}) to find a candidate \mcomponent{} to own the new \vregion.
\gmm\ makes this decision based on available physical memory space and access load on different \mcomponent{}s (\S~\ref{sec:lego:grm}).
If the candidate \mcomponent\ is not the home \mcomponent{}, the home \mcomponent{} next forwards the request to that \mcomponent\ (\circled{4}),
which then performs local virtual memory area allocation and sets up the proper vma tree. 
Afterwards, the \pcomponent{} directly sends memory access requests to the owner of the \vregion\ where the memory access falls into
(\eg, \circled{a} and \circled{c} in Figure~\ref{fig-dist-vma}).


\lego' mechanism of distributed virtual memory management is efficient and it cleanly separates memory operations from \pcomponent{}s.
\pcomponent{}s hand over all memory-related system call requests to \mcomponent{}s
and only cache a copy of the \vregion\ array for fast memory accesses.
To fill a cache miss or to flush a dirty cache line, 
a \pcomponent{} looks up the cached \vregion\ array to find its owner \mcomponent{} and sends the request to it.

\noindent{\textit{\uline{Physical memory space management.}}}
Each \mcomponent\ manages the physical memory allocation for data that falls into the
\vregion\ that it owns.
Each \mcomponent{} can choose their own way of physical memory allocation
and own mechanism of virtual-to-physical memory address mapping.


\subsubsection{Optimization on Memory Accesses}
\label{sec:lego:zerofill}
With our strawman memory management design, 
all \excache\ misses will go to \mcomponent{}s.
We soon found that a large performance overhead in running real applications 
is caused by filling empty \excache, \ie, {\em cold misses}.
To reduce the performance overhead of cold misses, we propose a technique 
to avoid accessing \mcomponent\ on first memory accesses.

The basic idea is simple: since the initial content of anonymous memory 
(non-file-backed memory) is zero, %undefined and can be any data, 
\lego\ can directly allocate a cache line with empty content
in \excache\ for the first access to 
anonymous memory instead of going to \mcomponent\
(we call such cache lines {\em p-local lines}).
When an application creates a new anonymous memory region, the process \microos\ records its address range and permission.
The application's first access to this region will be an \excache\ miss and it will trap to \lego.
\lego\ process \microos\ then allocates an \excache\ line, fills it with zeros, 
and sets its R/W bit according to the recorded memory region's permission.
Before this p-local line is evicted, it only lives in the \excache.
No \mcomponent{}s are aware of it or will allocate physical memory or a virtual-to-physical memory mapping for it.
When a p-local cache line becomes dirty and needs to be flushed, 
the process \microos\ sends it to its owner \mcomponent, which then
allocates physical memory space and establishes a virtual-to-physical memory mapping.
Essentially, \lego\ {\em delays physical memory allocation until write time}.
Notice that it is safe to only maintain p-local lines at a \pcomponent{} \excache\ 
without any other \pcomponent{}s knowing them, 
since \pcomponent{}s in \lego\ do not share any memory
and other \pcomponent{}s will not access this data.

\subsection{Storage Management}
\lego\ supports a hierarchical file interface that is backward compatible with POSIX 
through its \vnode\ abstraction. 
Users can store their directories and files under their \vnode{}s' mount points
and perform normal read, write, and other accesses to them.

\lego\ implements core storage functionalities at \scomponent{}s.
To cleanly separate storage functionalities, \lego\ uses a stateless storage server design, 
where each I/O request to the storage server contains all the information needed to 
fulfill this request, \eg, full path name, absolute file offset,
similar to the server design in NFS v2~\cite{Sandberg-NFS-85}.

While \lego\ supports a hierarchical file use interface,
internally, \lego\ storage \microos\ treats (full) directory and file paths just as unique names of a file
and place all files of a \vnode\ under one internal directory at the \scomponent{}.
To locate a file, \lego\ storage \microos\ maintains a simple hash table with the full paths of files (and directories) as keys.
From our observation, most datacenter applications only have a few hundred files or less.
Thus, a simple hash table for a whole \vnode\ is sufficient to achieve good lookup performance.
Using a non-hierarchical file system implementation largely reduces the complexity of \lego' file system,
making it possible for a storage \microos\ to fit in storage devices controllers that have limited processing power~\cite{Willow}.

\lego\ places the storage buffer cache at \mcomponent{}s
rather than at \scomponent{}s, because \scomponent{}s can only host a limited amount of internal memory.
\lego\ memory \microos\ manages the storage buffer cache by simply performing insertion, lookup, and deletion of buffer cache entries.
For simplicity and to avoid coherence traffic, we currently place the buffer cache of one file
under one \mcomponent{}.
When receiving a file read system call, the \lego\ process \microos\ first uses its extended Linux state layer to 
look up the full path name, then passes it with the requested offset and size to the \mcomponent\ that holds the file's buffer cache.
This \mcomponent\ will look up the buffer cache and returns the data to \pcomponent\ on a hit.
On a miss, \mcomponent\ will forward the request to the \scomponent\ that stores the file, 
which will fetch the data from storage device and return it to the \mcomponent.
The \mcomponent\ will then insert it into the buffer cache and returns it to the \pcomponent.
Write and fsync requests work in a similar fashion.

\subsection{Global Resource Management}
\label{sec:lego:grm}
\lego\ uses a two-level resource management mechanism.
At the higher level, \lego\ uses three global resource managers for process, memory, and storage resources, 
{\em \gpm, \gmm}, and {\em \gsm}.
These global managers perform coarse-grained global resource allocation and load balancing,
and they can run on one normal Linux machine.
Global managers only maintain approximate resource usage and load information.
They update their information either when they make allocation decisions 
or by periodically asking \microos{}s in the cluster.
At the lower level, each \microos\ can employ its own policies and mechanisms to manage its local resources.

For example, process \microos{}s allocate new threads locally 
and only ask \gpm\ when they need to create a new process.
\gpm\ chooses the \pcomponent{} that has the least amount of threads based on its maintained approximate information.
Memory \microos{}s allocate virtual and physical memory space on their own.
Only home \mcomponent{} asks \gmm\ when it needs to allocate a new \vregion.
\gmm\ maintains approximate physical memory space usages and memory access load by periodically asking \mcomponent{}s
and chooses the memory with least load among all the ones that have at least \vregion\ size of free physical memory.

\lego\ decouples the allocation of different resources and 
can freely allocate each type of resource from a pool of components.
Doing so largely improves resource packing compared to a monolithic server cluster
that packs all type of resources a job requires within one physical machine.
Also note that \lego\ allocates hardware resources only {\em on demand}, 
\ie, when applications actually create threads or access physical memory.
This on-demand allocation strategy further improves \lego' resource packing efficiency
and allows more aggressive over-subscription in a cluster.

\subsection{Reliability and Failure Handling}
\label{sec:lego:failure}
After disaggregation, \pcomponent{}s, \mcomponent{}s, and \scomponent{}s can all fail independently.
Our goal is to build a reliable disaggregated cluster that has the same or lower application failure rate
than a monolithic cluster.
As a first (and important) step towards achieving this goal, %building a reliable disaggregated cluster,
we focus on providing memory reliability by handling \mcomponent\ failure in the current version of \lego\ because of three observations.
First, when distributing an application's memory to multiple \mcomponent{}s, 
the probability of memory failure increases and not handling \mcomponent\ failure will cause applications to fail more often 
on a disaggregated cluster than on monolithic servers.
Second, since most modern datacenter applications
already provide reliability to their distributed storage data %(usually through some form of redundancy)
and the current version of \lego\ does not split a file across \scomponent,
we leave providing storage reliability to applications.
Finally, since \lego\ does not split a process across \pcomponent{}s,
the chance of a running application process being affected by the failure of a \pcomponent\ is similar to 
one affected by the failure of a processor in a monolithic server.
Thus, we currently do not deal with \pcomponent\ failure and leave it for future work.

A naive approach to handle memory failure is to perform a full replication of memory content over two or more \mcomponent{}s.
This method would require at least 2\x\ memory space,
making the monetary and energy cost of providing reliability prohibitively high (the same reason why RAMCloud~\cite{Ongaro11-RamCloud} does not replicate in memory).
Instead, we propose a space- and performance-efficient approach to provide in-memory data reliability in a best-effort way.
Further, since losing in-memory data will not affect user persistent data,
we propose to provide memory reliability in a best-effort manner.

We use one primary \mcomponent, one secondary \mcomponent, and a backup file in \scomponent\ for each vma.
A \mcomponent{} can serve as the primary for some vma and the secondary for others.
The primary stores all memory data and metadata.
\lego\ maintains a small append-only log at the secondary \mcomponent{}
and also replicates the vma tree there.
When \pcomponent{} flushes a dirty \excache\ line, 
\lego\ sends the data to both primary and secondary in parallel (step \circled{a} and \circled{b} in Figure~\ref{fig-dist-vma})
and waits for both to reply (\circled{c} and \circled{d}).
In the background, the secondary \mcomponent\ flushes the backup log to a \scomponent{},
which writes it to an append-only file.

If the flushing of a backup log to \scomponent\ is slow and the log is full, 
we will skip replicating application memory.
If the primary fails during this time, \lego\ simply reports an error to application.
Otherwise when a primary \mcomponent\ fails, we can recover memory content 
by replaying the backup logs on \scomponent\ and in the secondary \mcomponent.
When a secondary \mcomponent\ fails, we do not reconstruct anything 
and start replicating to a new backup log on another \mcomponent{}.


\section{Data Management and Access}
\label{sec:data}

This section presents how \hotpot\ manages user data in \dsnvm. 
We postpone the discussion of data durability and reliability to Section~\ref{sec:xact}.

\subsection{\nvm\ Page Morphable States}
One of \hotpot's design philosophies is to use one layer for both memory and storage 
and to integrate distributed memory caching and data replication.
To achieve this goal, we propose to impose {\em morphable} states on \nvm\ pages,
where the same \nvm\ page in \hotpot\ can be used both as a local memory cached copy to improve performance
and as a redundant data page to improve data reliability and availability.

We differentiate three states of a \nvm\ page:
active and dirty, active and clean, and inactive and clean,
and we call these three states {\em \dirty}, {\em \committed}, and {\em \redundant} respectively.
A page being clean means that it has not been updated since the last commit point;
committing a dirty page moves it to the clean state.
A page being active means that it is currently being accessed by an application,
while an \redundant\ page is a page which the application process has not mapped or accessed.
Several \hotpot\ tasks can change page states,
including page read, page write, data commit, data replication, page migration, and page eviction.
We will discuss how page states change throughout the rest of this section.
Figure~\ref{fig-data-eg} illustrates two operations that cause \hotpot\ data state changes.

\subsection{Data Organization}
\hotpot\ aims to support large-scale, data-intensive applications
on a fairly large number of nodes. %(\eg, at least a few racks)
Thus, it is important to minimize \hotpot's performance and scalability bottlenecks.
In order to enable flexible load balancing and resource management,
\hotpot\ splits the virtual address range of each dataset 
into {\em chunks} of a configurable size (\eg, 4\MB).
\nvm\ pages in a chunk do not need to be physically consecutive
and not all pages in a chunk need to exist on a node.

Each chunk in \hotpot\ is owned by an {\em owner node (\on)},
similar to the ``home'' node in home-based DSM systems~\cite{HLRC}.
%and to the primary node of distributed storage systems,
An \on\ maintains all the data and metadata of the chunk it owns.
%and serves requests from other nodes.
Other nodes, called {\em data node} or {\em \dn}, always fetch data from the \on\
when they initially access the data.
A single \hotpot\ node can simultaneously be the \on\ for some data chunks and the \dn\ for other chunks.
When the application creates a dataset, 
\hotpot\ \cd\ performs an initial assignment of \on{}s to chunks of the dataset.

Two properties separate \hotpot\ \on{}s from traditional home nodes.
First, %besides serving read data,
\hotpot\ \on\ is responsible for the reliability and crash consistency of the pages it owns,
besides serving read data and ensure the coherence of cached copies.
Second, \hotpot\ does not fix which node owns a chunk
and the location of \on\ adapts to application workload behavior dynamically.
Such flexibility is important for load balancing and application performance (see Section~\ref{sec:migration}).

\subsection{Data Reads and Writes}
\label{sec:readwrite}

\hotpot\ minimizes software overhead to improve application performance.
It is invoked only when a page fault occurs or when 
applications execute data persistence operations (see Section~\ref{sec:xact} for details of data persistence operations).

{
\begin{figure}[th]
\centering
\begin{center}
\centerline{\includegraphics[width=\columnwidth]{Figures/data-eg.pdf}}
\end{center}
\vspace{-0.2in}
\mycaption{fig-data-eg}{Data State Change Example.}
{
White, black, and striped blocks represent \committed, \redundant, and \dirty\ states.
Before commit, Node 2 and Node 3 both have cached copies of 
data page $B$. Node 2 has written to $B$ and created a \dirty\ page, $B1$.
During commit, Node 2 pushes the content $B1$ to its \on, Node 1.
Node 1 updates its \committed\ copy to $B1$ and also sends this update to Node 3.
Figure (c) shows the state after migrating the \on\ of chunk 1 from Node 1 to Node 3. 
%page $A$ on Node 3 is \redundant.
After migration, Node 3 has all the pages of the chunk and all of them are in \committed\ states.
}
\vspace{-0.1in}
\end{figure}
}

When a page fault happens because of read, 
it means that there is no valid local page.
\hotpot\ first checks if there is any local \redundant\ page.
If so, it will move this page to the \committed\ state and establish a page table entry (PTE) for it.
Otherwise, there is no available local data and 
\hotpot\ will fetch it from the remote \on.
\hotpot\ writes the received data to a newly-allocated local physical \nvm\ page.
Afterwards, applications will use memory instructions to access this local page directly.


Writing to a \committed\ page also causes a page fault in \hotpot. 
This is because a \committed\ page can contribute towards user-specified degree of replication as one data replica,
and \hotpot\ needs to protect this committed version from being modified.
Thus, \hotpot\ write protects all \committed\ pages.
When these pages are written to (and generating a write page fault), 
\hotpot\ creates a local Copy-On-Write (COW) page
and marks the new page as dirty while leaving
the original page in \committed\ state.
\hotpot\ does not write protect this COW page, since it is already in the dirty state.

Following \hotpot's design philosophy to exploit hints from our targeted data-intensive applications,
we avoid propagating updates to cached copies at other nodes on each write and only do so at each application commit point.
Thus, all writes in \hotpot\ is local and only writing to a \committed\ page will generate a page fault.

Not updating remote cached copies on each write also has the benefit of reducing write amplification in \nvm.
In general, other software mechanisms and policies such as wear-aware \nvm\ allocation and reclamation 
and hardware techniques like Start-Gap~\cite{start-gap-micro09}
can further reduce \nvm\ wear.
We do not focus on \nvm\ wear in this paper and leave such optimizations for future work.

\subsection{\nvm\ Page Allocation and Eviction}
\label{sec:eviction}
Each \hotpot\ node manages its own physical \nvm\ space and performs \nvm\ page allocation and eviction.
Since physical pages do not need to be consecutive,
we use a simple and efficient allocation mechanism by maintaining a free page list
and allocating one page at a time.

\hotpot\ uses an approximate-LRU replacement algorithm that is similar to Linux's page replacement mechanism.
Different from Linux,
\hotpot\ distinguishes pages of different states.
\hotpot\ never evicts a \dirty\ page
and always tries to evict \redundant\ pages before evicting \committed\ pages.
We choose to first evict \redundant\ pages, 
because these are the pages that have not been accessed by applications
and less likely to be accessed in the future than \committed\ pages. %if a node accesses an \redundant\ page, it will become an \committed\ page.

Since both \redundant\ and \committed\ pages can serve as a redundant copy
for data reliability, \hotpot\ cannot simply throw them away during eviction.
The evicting node of a page will contact its \on{}, 
which will check the current degree of replication of the candidate pages 
and prioritize the eviction of pages that already have enough replicas. 
For pages that will drop below the user-defined replication degree after the eviction, 
the \on\ will make a new \redundant\ page at another node. %, if the eviction violates user-specified reliability requirements.

\subsection{Chunk \on\ Migration}
\label{sec:migration}
An \on{} serves both page read and data commit requests that belong to the chunks it owns.
Thus, the location of \on\ is important to \hotpot's performance.
Ideally, the node that performs the most reads and commits of data in a chunk 
should be its \on\ to avoid network communication.

By default, \hotpot\ initially spreads out a dataset's chunks to all \hotpot\ nodes in a round robin fashion
(other static placement policies can easily replace round robin).
Static placement alone cannot achieve optimal run-time performance.
\hotpot\ remedies this limitation by performing online chunk migration,
where one \on\ and one \dn\ of a chunk can switch their identities
and become the new \dn\ and new \on\ of the chunk.

\hotpot\ utilizes application behavior in recent history 
to decide how to migrate \on{}s.
Each \on\ records the number of page read requests
and the amount of committing data it receives in the most recent time window.

\on{}s make their migration decisions with a simple greedy algorithm based on the combination of two criteria:
maximizing the {\em benefit} while not exceeding a configurable {\em cost} of migration.
The benefit is the potential reduction in network traffic during remote data reads and commits.
The node that performs most data communication to the \on\ in recent history
is likely to benefit the most from being the new \on,
since after migration these operations will become local.
We model the cost of migration by the amount of data needed to copy to a node so that it has all the chunk data to become \on.

Once \hotpot\ has made a decision, it performs the actual chunk migration using 
a similar method as process and VM migration~\cite{OsmanEtAl02-Zap,Douglis87-Migration,Clark05-XenMigrate}
by temporary stopping commits to the chunk under migration
and resume them at the new \on\ after migration.

\if 0
    Once decided, a new ON node will be chosen. And the old ON will start to migrate pages to the
    the ON. During migration, the old ON is still the only valid ON in system. And, the old ON
    will keep serving page-fetch, but xact will be rejected.

    Once all pages are migrated from old ON to new ON, the old ON will 1) Tell CD that this region
    is migrated, hence CD can change its metadata. 2) Broadcast to nodes that currently have this
    region, that this region is migrated to new ON.
\fi

\if 0
\subsubsection{Replica Selection}
Apart from \on\ locations, the locations of data replicas can also affect application performance.
When a node has an \redundant\ page it can directly access it and avoid a remote page read. 
Thus, placing a data replica at a node that is likely to access the data in the future
can potentially improve performance.

\hotpot\ \on\ decides where to place an \redundant\ page during transaction commit with two criteria. 
%Currently, we use two criteria in selecting the location of a replica.
%replication gives us a chance to re-balance workloads
%happened in two occasions:
%when committing a transaction 
%and when evicting a \redundant\ page.
First, we use spatial locality to estimate the likelihood a node is going to access an \redundant\ page 
by the number of pages this node has read in the chunk that contains the \redundant\ page.
The second consideration is to prevent thrashing.
When a node runs out of space, it will first evict \redundant\ pages (Section~\ref{sec:eviction})
and assigning \redundant\ pages to such nodes will cause thrashing.
Thus, \hotpot\ compares the total \nvm\ free space of a node and avoids assigning 
\redundant\ pages to nodes with space pressure.
\fi

\section{Data Durability, Consistency, and Reliability}
\label{sec:xact}

Being distributed shared memory and distributed storage at the same time,
\dsnvm\ should ensure both correct shared memory accesses to \nvm\
and the persistence and reliability of in-\nvm\ data. 
\hotpot\ provides three guarantees: coherence among cached copies of in-\nvm\ data,
recovery from various types of failures into a consistent state,
and user data reliability and availability under concurrent failures.
Although each of these three properties have been explored before,
as far as we know, \hotpot\ is the first system that integrates all of them in one layer.
\hotpot\ also has the unique requirement of low software overhead to retain the performance benefit of \nvm.

\begin{itemize}[leftmargin=*]
\item{\em Cache coherence.} 
In \hotpot, application processes on different nodes cache remote data in their local \nvm\ for fast accesses.
\hotpot\ provides two consistency levels across cached copies: 
{\em \ra}, multiple readers and single writer ({\em MRSW}) 
and {\em \rb}, multiple readers and multiple writers ({\em MRMW}).
MRMW allows multiple nodes to concurrently write and commit their local cached copies.
With \mrmw, there can be multiple versions of dirty data in the system (but still one committed version),
while \mrsw\ guarantees only one dirty version at any time.
An application can use different modes for different datasets,
but only one mode with the same dataset.
This design allows flexibility at the dataset granularity while guaranteeing correctness.
 
\item{\em Crash consistency.} 
Data storage applications usually have well-defined {\em consistent} states and need to move from 
one consistent state to another atomically.
When a crash happens, 
user data should be recovered to a consistent state ({\ie, \em crash consistency}). 
\hotpot\ guarantees crash consistency both within a single node ({\em \rcs}) and across distributed nodes ({\em \rcm}).
Note that crash consistency is different and orthogonal to cache
coherence in \ra\ and \rb. 

\item{\em Reliability and availability.} 
To ensure that user persistent data can sustain $N-1$ concurrent node failures, 
where $N$ is a user defined value, \hotpot\ guarantees that {\em \re}, once data has 
been committed, there are always $N$ copies of clean, committed data.

\end{itemize}

This section first discusses how \hotpot\ ensures crash consistency within a single node,
then presents the \mrmw\ and \mrsw\ modes and their atomic commit protocols, %and the optional group fetch protocol.
and ends with the discussion of \hotpot's recovery mechanisms under different crash scenarios.

\subsection{Single-Node Persistence and Consistency}
\label{sec:singleconsistency}

Before ensuring user data's global reliability and consistency in \dsnvm,
\hotpot\ first needs to make sure that data on a single node can properly sustain power crashes (\rcs)~\cite{Memory-Persistency}.
\hotpot\ makes data persistent with the standard Intel persistent memory instructions~\cite{Delegated-persist},
\ie, \clflush, \mfence\ (note that we do not include the deprecated \pcommit\ instruction~\cite{Deprecating-PCOMMIT}).

After a node crashes, if its \nvm\ is still accessible, \hotpot\ will use the \nvm\ content to recover;
otherwise, \hotpot\ will use other nodes to reconstruct data on a new node (Section~\ref{sec:recovery}).
For the former case, \hotpot\ needs to guarantee that user data in \dsnvm\ is in a consistent state after crash.
\hotpot\ also needs to ensure that its own metadata is persistent and is consistent with user data.

\hotpot\ maintains metadata on a local node to find user data and record their morphable states (\ie, \committed, \dirty, or \redundant).
Since these metadata are only used within a single node, \hotpot\ does not need to replicate them on other nodes.
\hotpot\ makes these metadata persistent at known locations in \nvm\ ---
a pre-allocated beginning area of \nvm.
\hotpot\ also uses metadata to record online state of the system (\eg, \on\ maintains a list of active \dn{}s that have a cached copy of data).
These metadata can be reconstructed by re-examining system states after recovery.
Thus, \hotpot\ does not make these metadata persistent.

{
\begin{figure}[t]
\begin{center}
\centerline{\includegraphics[width=\textwidth]{hotpot/Figures/commit.pdf}}
\caption[\mrmw\ Commit Example.]
{
\mrmw\ Commit Example.
Solid arrows represent data communication.
Dashed arrows represent metadata communication.
Node 1 (\xn) commit data to \on{}s at Node 2 and 3 with replication degree four.
Black shapes represent old committed states before the update
and white shapes represent new states.
}
\label{fig-mrmw}
\end{center}
\end{figure}
}

Similar to traditional file systems and databases, 
it is important to enforce {\em ordering} of metadata and data persistence
in order to recover to a consistent state.
For single-node non-commit operations (we defer the discussion of commit operations to Sections \ref{sec:mrmw} and \ref{sec:mrsw}), 
\hotpot\ uses a simple shadow-paging mechanism to ensure that the consistency of metadata and data.
Specifically, we associate each physical memory page with a metadata slot
and use a single 8-byte index value to locate both the physical page and its metadata.
When an application performs a memory store to a \committed\ page,
\hotpot\ allocates a new physical page, writes the new data to it, and writes the new metadata 
(\eg, the state of the new page) to the metadata slot associated with this physical page.
After making all the above data and metadata persistent, \hotpot\ changes the index
from pointing to the old \committed\ page to pointing to the new \dirty\ page.
Since most architectures support atomic 8-byte writes, this operation atomically moves the system to a new consistent state with both the new data and the new metadata.
%and \hotpot\ can always recover local data to a consistent state after crashes.

\subsection{\mrmw\ Mode}
\label{sec:mrmw}
\hotpot\ supports two levels of concurrent shared-\nvm\ accesses and uses different protocols to commit data.
The \mrmw\ mode allows multiple concurrent versions of dirty, uncommitted data 
to support great parallelism.
\mrmw\ meets \rb, \rcm, and \re.

\mrmw\ uses a distributed atomic commit protocol at each commit point %(a user \commit\ call)
to make local updates globally visible, persistent, and replicated.
Since \mrmw\ supports concurrent commit operations 
and each commit operation can involve multiple remote \on{}s,
\hotpot\ needs to ensure that all the \on{}s reach consensus on the commit operation they serve. 
We designed a three-phase commit protocol for the \mrmw\ mode
based on traditional two-phase commit protocols~\cite{Samaras93,Gray78,Lampson81} but differs in that
\hotpot\ needs to ensure cache coherence, crash consistency, and data replication all in one protocol.
Figure~\ref{fig-mrmw} illustrates an example of \mrmw. 

{
\begin{figure}[th]
\begin{center}
\centerline{\includegraphics[width=0.42\textwidth]{Figures/mrsw.pdf}}
%\vspace{-0.05in}
\mycaption{fig-mrsw}{\mrsw\ Example.}
{
Node 1 (\xn) first acquires write permission from Node 2 (\master)
before writing data.
It then commits the new data to \on{}s at Node 2 and 3 with replication degree four
and finally releases the write permission to \master.
}
\end{center}
\end{figure}
%\vspace{-0.1in}
}


\noindent{\bf Commit phase 1.} 
When a node receives a \commitxact\ call (we call this node {\em \xn}), it checks if data specified in the \commitxact\ call is dirty
and commits only the dirty pages.
\xn\ persistently records the addresses of these dirty pages for recovery reasons (Section~\ref{sec:recovery}).
\xn\ also assigns a unique ID ({\em \xactid}) for this \commitxact\ request and persistently records the \xactid\ and its state of starting phase 1. 
 
Afterwards, \xn\ sends the committing data to its \on{}s 
to prepare these \on{}s for the commit.
Each \on{} accepts the commit request if it has not accepted other commit request to the same pages,
and it stores the committing data in a {\em persistent redo log} in \nvm.
The \on\ also persistently records the \xactid\ and its state (\ie, completed phase 1) persistently.
The \on{} will block future commit requests to these data until the whole commit process finishes.
The \xn\ can proceed to phase 2 only when all \on{}s return successfully.

\noindent {\bf Commit phase 2.}
In commit phase 2, \hotpot\ makes the committing data persistent, 
coherent, and replicated.
This is the phase that \hotpot\ differs most from traditional distributed commit protocols.

\xn\ sends a command to all the involving \on{}s to indicate the beginning of phase 2.  
%This command indicates that the \on{}s can safely begin commit phase 2 and specifies the application's desired replication degree.  
Each \on{} then performs two tasks in one multicast operation (Section~\ref{sec:network}): 
updating \dn{}s' cached copies of the committing data and making extra replicas.
\on\ looks up its metadata to find what \dn{}s have a cached copy.
If these \dn{}s alone cannot meet the replication degree, \on{} will choose new \dn{}s that do not have
a copy of the data and send the data to them.
%These \dn{}s mat have a \dirty, \committed, \redundant\ copy of the data, or they have no copy at all. 
%The \on{} does not differentiate these states and sends the updated data to all these \dn{}s. 

When a \dn{} receives the committing data from an \on,
it checks the state of its local data pages.
If a local page is in the \committed\ state or the \redundant\ state, 
the \dn\ will directly overwrite the local page with the received data.
In doing so, the \dn's cached \nvm\ data is updated.
If the local page is \dirty\  or if there is no corresponding local page,
the \dn\ allocates a new physical page and writes the new data to this page.
The new physical page will be in the \redundant\ state and will not affect the \dn's dirty data.
In this way, all \dn{}s that receive updated data from the \on\ will 
have a clean, committed copy, either in the \committed\ or the \redundant\ state.

After all \dn{}s have replied to the \on{} indicating that there are now $N$ copies of the committing data,
the \on\ commits data locally
by checkpointing (copying) data from the redo log to their home locations.
Unlike traditional databases and file systems that lazily checkpoint logged data, 
\hotpot\ checkpoints all committing data in this phase 
so that it can make the updated version of the data
visible to applications immediately, 
a requirement of shared-memory cache coherence.
During checkpointing, the \on{} will block both local and remote reads to the committing data
to prevent applications from reading intermediate, inconsistent data.

After the \xn\ receives successful replies from all the \on{}s, 
it deletes its old local data and moves to the new, committed version. 
At this point, the whole system has a coherent view of the new data
and has at least $N$ copies of it.
%The committing node can only proceed to phase 3 when all \on{}s returns successfully from phase 2.

\noindent {\bf Commit phase 3.}
In the last phase, the \xn\ informs all \on{}s that the \commitxact\ operation has succeeded.
The \on{}s then delete their redo logs.
%and make the new \committed\ data visible to applications.

\noindent {\bf Committing to a single \on\ and to local \on.}
When only one remote \on\ is involved in a \commitxact\ operation,
there is no need to coordinate multiple \on{}s
and \hotpot\ performs the above commit protocol in a single phase.

The \xn\ can also be the \on\ of committing data.
In this case, the \xn\ performs the \commitxact\ operation locally.
Since all local dirty pages are the COW of old \committed\ pages,
\xn\ already has an undo and a redo copy of the committing data
and does not need to create any other redo log as in remote \on's phase 1.

\subsection{MRSW Mode}
\label{sec:mrsw}

The \mrsw\ mode allows only one writer to a \nvm\ page at a time
to trade parallelism for stronger consistency. 
%compared to the \mrmw\ mode.
\mrsw\ meets \ra, \rcm, and \re.

Traditional \mrsw\ protocols in DSM systems are usually invoked at every memory store
(\eg, to update cached read copies, to revoke current writer's write permission).
Unlike DSM systems, \dsnvm\ applications store and manage persistent data;
they do not need to ensure coherence on every memory store,
since they have well-defined points of when they want to start updating data and when they want to commit.
To avoid the cost of invoking coherence events on each memory store
while ensuring only one writer at a time, 
\hotpot\ uses an {\em \acquire} API for applications to indicate the data areas they want to update.
Afterwards, applications can update any date that they have acquired and use the \commitxact\ call to both 
commit updates and release corresponding data areas.
Figure~\ref{fig-mrsw} shows an example of \mrsw. % acquire, commit, and release process.

\noindent{\bf Acquire write permission.}
\hotpot\ uses a master node ({\em \master}) to maintain the active writer of each page. 
An \master\ can be one of the \hotpot\ node, the \cd, or a dedicated node.
%The \master\ maintains a simple hash table of the virtual page numbers of the data that is currently being written to.
When a node receives the \acquire\ call, it sends the virtual addresses of the data specified in the call to the \master.
If the \master\ finds that at least one of these addresses are currently being written to, 
it will reject the \acquire\ request and let the requesting node retry later.

\noindent{\bf Commit and release data.}
\mrsw's commit protocol is simpler and more efficient than \mrmw's,
since there is no concurrent commit operations to the same data in \mrsw\ (concurrent commit to different data pages is still allowed).
\mrsw\ combines phase 1 and phase 2 of the \mrmw\ commit protocol into a single phase 
where the \xn\ sends committing data to all \on{}s and all \on{}s commit data on their own.
%without the need to coordinate with other \on{}s.
Each \on\ individually handles commit in the same way as in the \mrmw\ mode, 
except that it does not need to coordinate with any other \on{}s or the \xn. 
\on\ directly proceeds to propagating data to \dn{}s after it has written its own redo log.

At the end of the commit process, the \xn\ informs the \on{}s to delete their redo logs (same as \mrmw\ commit phase 3)
and the \master\ to release the data pages.



\subsection{Crash Recovery}
\label{sec:hotpot:recovery}

\hotpot\ can safely recover from different crash scenarios
without losing applications' data.
% as long as the number of concurrent
%node failures is less than the application-specified replication degree.
\hotpot\ detects node failures
%by detecting an unresponsive node during a network operation and 
by request timeout 
and by periodically sending heartbeat messages from 
the \cd\ to all \hotpot\ nodes.
We now explain \hotpot's crash recovery mechanism
in the following four crash scenarios.
Table~\ref{tbl-crash} summarizes various crash scenarios and \hotpot's recovery mechanisms.

%\subsubsubsection{Crash During Transaction Commit}

{
\begin{table}[t]\small
\begin{center}
\begin{center}
\begin{tabular}{ p{0.18in} | p{0.5in} | p{0.25in} | p{0.5in} | p{4in} }
 & \small Node & \small \nvm & \small Time & \small Action \\
\hline
\hline
& any & Y & any & resume normal operation after reboot \\
\hline
& \cd\ & N & any & reconstruct using mirrored copy \\
\hline
& \on\ & N & NC & promote an existing \dn\ to \on\ \\
& \dn\ & N & NC & reconstruct data to meet replication degree \\
\hline
\multirow{7}{*}{\rotatebox{90}{\mrmw\ Commit}} & \xn/\on\ & N & p1 & undo commit, \on{}s delete redo logs \\
%& any & Y & any & continue commit \\
%& \on\ & N & p1 & undo commit, \on{}s delete redo logs \\
& \xn\ & N & p2 & redo commit, \on{}s send new data to \dn{}s \\
& \on\ & N & p2 & redo commit, \xn\ sends new data to new \on\ \\
& \dn\ & N & p2 & continue commit, \on\ sends data to new \dn\ \\
& \xn\ & N & p3 & complete commit, \on{}s delete redo logs \\
& \on/\dn\ & N & p3 & complete commit, new chunk reconstructed using \committed\ data \\
\hline
\multirow{5}{*}{\rotatebox{90}{MRSW}} & \xn\ & N & commit & undo commit, \on{}s send old data to \dn{}s \\
& \on\ & N & commit & \xn\ redo commit from scratch\\
& \xn\ & N & release & complete commit, release data \\
& \on/\dn\ & N & release & complete commit, new chunk reconstructed using \committed\ data \\
%& \xn\ & N &  & \\
%& \xn\ & N &  & \\
%\hline
\end{tabular}
\end{center}
\caption[Crash and Recovery Scenarios.]
{Crash and Recovery Scenarios.
Columns represent crashing node, if \nvm\ is accessible after crash, time of crash, and actions taken at recovery.
NC represents non-commit time.
}
\label{tbl-crash}
\end{center}
\end{table}
}


\noindent{\bf Recovering \cd\ and \master.}
\cd\ maintains node membership and dataset name mappings.
\hotpot\ currently uses one \cd\ but can be easily extended to 
include a hot stand-by \cd\ (\eg, using Mojim~\cite{Zhang15-Mojim}).
%When the primary \cd\ fails, the stand-by \cd\ can resume immediately.

\master\ tracks which node has acquired write access to a page under the \mrsw\ mode.
\hotpot\ does not make this information persistent
and simply reconstructs it by contacting all other nodes during recovery.
%If the \cd\ fails, a new node will be selected as \cd.
%Since all the metadata that the \cd\ maintains can be reconstructed, 
%\hotpot\ simply rebuild them by contacting all \hotpot\ nodes.

\noindent{\bf Non-commit time crashes.}
Recovering from node crashes during non-commit time is fairly straightforward.
If the \nvm\ in the crashed node is accessible after the crash (we call it {\em with-\nvm\ failure}),
\hotpot\ directly restarts the node and
lets applications access data in \nvm.
% provides the data in \nvm\ to applications,
%which can choose to resume from where they left over~\cite{Narayanan12-ASPLOS}.
As described in Section~\ref{sec:hotpot:singleconsistency}, \hotpot\ ensures crash consistency of a single node.
Thus, \hotpot\ can always recover to a consistent state 
when \nvm\ survives a crash.
\hotpot\ can sustain arbitrary number of with-\nvm\ failures concurrently. 

When a crash results in corrupted or inaccessible \nvm\ (we call it {\em no-\nvm\ failure}),
\hotpot\ will reconstruct the lost data using redundant copies.
\hotpot\ can sustain $N-1$ concurrent no-\nvm\ failures, where $N$ is the user-defined degree of replication.

If a \dn\ chunk is lost, 
the \on\ of this chunk will check what data pages in the chunk
have dropped below user-defined replication degree
and replicating them on the new node that replaces the failed node. 
%If a node fails and loses all its \dn{} data, and \hotpot\ 
%no longer meets the degree of replication requested by an application,
%the \on\ will send a committed copy of the failed node's data to the new node.
There is no need to reconstruct the rest of the \dn\ data;
\hotpot\ simply lets the new node access them on demand.

%\noindent{\bf Reconstructing \on{} chunks.}
When an \on\ chunk is lost, it is critical to reconstruct it quickly,
since an \on\ serves both remote data read and 
commit operations.
%If a node fails and loses its \on\ chunks,
Instead of reconstructing a failed \on\ chunk from scratch,
\hotpot\ promotes an existing \dn\ chunk to an \on\ chunk
and creates a new \dn\ chunk.
%To promote a \dn\ chunk to an \on\ chunk,
%the newly promoted \dn\ will communicate with all the other \dn{}s of this chunk.
The new \on\ will fetch locally-missing committed data from other nodes
and reconstruct \on\ metadata for the chunk.
Our evaluation results show that it takes at most 2.3 seconds to promote
a 1GB \dn\ chunks to \on.

%\subsubsection{}
\noindent{\bf Crash during commit.}
If a with-\nvm\ failure happens during a \commitxact\ call,
\hotpot\ will just continue its commit process after restart.
When a no-\nvm\ failure happens during commit,
\hotpot\ takes different actions to recover depending on when the failure happens.

For \mrmw\ commit, if no-\nvm\ failure happens before all the \on{}s have created the persistent redo logs (\ie, before starting phase 2),
\hotpot\ will undo the commit and revert to the old committed state
by deleting the redo logs at \on{}s.
If a no-\nvm\ failure happens after all \on{}s have written the committing data to their persistent redo logs (\ie, after commit phase 1),
\hotpot\ will redo the commit by replaying redo logs.

For \mrsw, since we combine \mrmw's phase 1 and phase 2 into one commit phase,
%do not allow each \on\ to proceed with pushing updates to \dn{}s right after it has written its own persistent redo log,
we will not be able to tell whether or not an \on\ has pushed the committing data to \dn{}s 
when this \on\ experience a no-\nvm\ failure.
In this case, \hotpot\ will let \xn\ redo the commit from scratch. 
Even if the crashed \on\ has pushed updates to some \dn{}s,
the system is still correct after \xn\ redo the commit;
it will just have more \redundant\ copies.
When the \xn\ fails during \mrsw\ commit, \hotpot\ will undo the commit
by letting all \on{}s delete their redo logs and send old data to \dn{}s to overwrite \dn{}s' updated data.
%During this recovery process, \hotpot\ needs to know what are the \on{}s that are involved in the commit.
%Instead of contacting all nodes to discover which ones are the \on{}s,
%\hotpot\ uses the \master\ %to reduce the overhead of contacting all nodes
%to only contact the nodes that owns pages that have been acquired by the failed \xn.

During commit, \hotpot\ only supports either \xn\ no-\nvm\ failure or \on\ no-\nvm\ failure.
We choose not to support concurrent \xn\ and \on\ no-\nvm\ failures during commit,
because doing so largely simplifies \hotpot's commit protocol and improves its performance.
\hotpot's commit process is fast (under 250\mus\ with up to 16 nodes, see Section~\ref{sec:hotpot:results}).
Thus, the chance of \xn\ and \on\ both fail and lose their \nvm\ during commit is very small.
\hotpot\ always supports \dn\ no-\nvm\ failures during commit regardless of whether there are concurrent \xn\ or \on\ failure.


\section{Network Layer}
\label{sec:hotpot:network}

The networking delay in \dsnvm\ systems is crucial to their overall performance.
We implement \hotpot's network communication using RDMA. %(Remote Direct Memory Access), 
RDMA provides low-latency, high-bandwidth direct remote memory accesses with low CPU utilization.
\hotpot's network layer is based on LITE~\cite{Tsai17-SOSP}, an efficient RDMA software stack we
built in the Linux kernel on top of the RDMA native APIs, {\em Verbs}~\cite{ibverbs}.

Most of \hotpot's network communication is in the form of RPC. % uses this RPC interface to implement most of its functionality. 
We implemented a customized RPC-like interface in our RDMA layer based on the two-sided RDMA send and receive semantics.
We further built a multicast RPC interface where one node can send a request to multiple nodes in parallel and let them each 
perform their processing functions and reply with the return values to the sending node.
Similar to the findings from recent works~\cite{FaSST}, two-sided RDMA works better and is more flexible 
for these RPC-like interfaces than one-sided RDMA. 

To increase network bandwidth, our RDMA layer enables multiple connections between each pair of nodes. 
It uses only one busy polling thread per node to poll a shared ring buffer for all connections, 
which delivers low-latency performance while keeping CPU utilization low.
Our customized RDMA layer achieves an average latency of 7.9\us\ to perform a \hotpot\ remote page read. %an RPC of 8B outgoing and 4KB incoming message size (equivalent to a remote page read).
In comparison, IPoIB, a standard IP layer on top of Verbs, requires 77\us\ for a round trip with the same size.




\if 0 
%Our \ib\ layer provides a richer and more efficient abstraction of \ib\ operations than 
%existing kernel-level \ib\ layers such as IPoIB and RDS.
To provide low-latency network performance for \hotpot, 
we made several unique design and implementation decisions that are different from previous 
\ib- and RDMA-based network implementations~\cite{Dragojevic14-NSDI,Nelson15-ATC,Kalia14-SIGCOMM}.
%IB-Verbs requires the application to post send (receive) requests to send (receive) queues.
%It uses completion messages in the completion queue to indicate the completion of requests 
%and supports both polling and interrupts to detect completions.
%IB-Verbs offers native IB performance and outperforms alternative IB protocols such as IPoIB and RDS.

%uses a thin protocol based on the reliable transportation mode of IB-Verbs. 

The basic primitive of our \ib\ layer is a pair of send and reply messages
between two nodes. 
With this primitive, one node sends a message to a remote node,
usually a request for an operation, and then waits for a reply.
After the remote node handles this request, it sends a reply back to the calling node,
finishing a send-reply round. 

Interestingly, although \ib's one-sided RDMA operations allow direct read and write to remote memory 
without involving remote side's CPU~\cite{Dragojevic14-NSDI}, 
using them to implement \hotpot\ incurs higher latency than RDMA send and receive~\cite{Kalia14-SIGCOMM}.
This is because one-sided RDMA is mainly useful when accessing remote memory without any states managed by remote side.
\hotpot\ maintains various states and metadata at each node. 
%These metadata need to be consistently updated with their data.
%For example, when a \dn\ accesses a remote page, 
%\hotpot\ uses the send-reply primitive to send a request to the \on. 
%The \on\ updates its list of \dn{}s with committed data copies
%and then sends the page back to the \dn.
Implementing complex \hotpot\ operations using one-sided RDMA requires the combination of several RDMA commands.
Even though the latency of one send or reply is slightly higher than a single one-sided RDMA command,
using send-reply to implement \hotpot\ operations achieves better performance than using one-sided RDMA.

On top of the send-reply primitive, our \ib\ layer further provides an efficient implementation of 
two new interfaces:
atomically sending a group of messages and waiting for reply, 
and multicasting send messages to a set of nodes and waiting for all replies from them.
Both these operations are useful in implementing \hotpot's transaction system.

Our \ib\ layer implements a persistent, append-only log in \nvm\ for receiving messages.
with the help of the Linux kernel slab allocator.
The slab allocator performs object-based allocation efficiently 
by maintaining lists of free objects 
where it allocates new objects from and frees objects into.
%Thus, the slab allocator has v
%The slab allocator is more efficient than 
%Instead of preallocating and maintaining a circular log,
The \ib\ layer allocates new receiving buffers in \nvm\ using the slab allocator
and provides an interface to free these buffers back to the slab lists.
%The \ib\ layer supports atomic operations 
%in \nvm\ and provides a free let .
\hotpot\ uses this mechanism to maintain a persistent redo log for transactions.
\hotpot\ only frees buffers in this log after a transaction has been committed.
\fi

\section{Applications and Evaluation}
\label{sec:hotpot:app}

This section presents the performance evaluation of two applications and a set of microbenchmarks.
We ran all experiments on a cluster of 17 machines, each with two Intel Xeon CPU E5-2620 2.40GHz
processors, 128 GB DRAM, and one 40 Gbps Mellanox ConnectX-3 InfiniBand network adapter;
a Mellanox 40 Gbps InfiniBand switch connects all of the machines. 
All machines run the CentOS 7.1 distribution and the 3.11.1 Linux kernel.

The focus of our evaluation is to understand the performance of \dsnvm's distributed memory model,
its commit protocols, and its data persistence cost. As there is no real \nvm\ in production yet,
we use DRAM as stand-in for \nvm. A previous study~\cite{Zhang15-NVMMStudy} shows that even though
\nvm\ and DRAM can have some performance difference, the difference is small and has much lower impact
on application performance than the cost of flushing data from CPU caches to \nvm, which we have
included in \hotpot\ and can measure accurately.

\subsection{Systems in Comparison}
\label{sec:hotpot:comparesys}
We compare \hotpot\ with one in-memory file system, two \nvm-based file systems, 
one replicated \nvm-based system, and three distributed shared memory systems.
Below we briefly describe these systems in comparison.

\noindent{\textbf{Single-Node File Systems.}} 
Tmpfs is a Linux file system that stores all data in main memory and does not perform any I/Os to storage devices.
\pmfs~\cite{Dulloor14-EuroSys} is a file system designed for \nvm. 
The key difference between \pmfs\ and a conventional file system is that its implementation of
\mmap\ maps the physical \nvm\ pages directly into the applications' address spaces rather than moving them back and
forth between the file store and the buffer cache.
\pmfs\ ensures data persistence using \sfence\ and \clflush\ instructions.

\noindent{\textbf{Distributed \nvm-Based Systems}}
Octopus~\cite{Octopus} is a user-level RDMA-based distributed file system designed for \nvm.
\Octopus\ provides a set of customized file APIs including read and write,
but does not support memory-mapped I/Os or provide data reliability and availability.
%Octopus data servers access local PM without stacking a local file system layer. In Octopus,
%files are distributed to data servers in a hash-based way.

Mojim~\cite{Zhang15-Mojim} is our previous work that uses a primary-backup model to replicate \nvm\ data
over a customized IB layer.
Similar to \hotpot, \pmfs, and Octopus, Mojim maps \nvm\ pages directly into application virtual memory address spaces.
Mojim supports application reads and writes on the primary node but only reads on backup nodes. 

\noindent\textbf{Distributed Shared Memory Systems.} 
We implemented two kernel-level DSM systems, {\em \dsmxact} and {\em \dsmnoxact}, on top of the same network stack as \hotpot's.
Both of them support multiple readers and single writer (MRSW)
and use a home node for each memory page to serve remote read and to store which nodes are the current readers and writer of the page, 
similar to HLRC~\cite{Li89-ACM,HLRC}.
We open source both these DSM systems together with \hotpot.

\dsmxact\ guarantees release consistency using a transaction interface that is similar to \hotpot's \mrsw\ mode. 
Applications first call a transaction begin API to specify the data that they want to write.
Transaction begin only succeeds if no other writer is writing to any of the transaction data.
After beginning a transaction, applications can read and write to any transaction data
and use a transaction commit call to end a transaction. 
When committing a transaction, \dsmxact\ writes all updated transaction data to the home node, 
invalidates the read caches on all other nodes, 
and releases the write permission.

\dsmnoxact\ supports write (memory stores) without transactions and 
does not require applications to declare which data they want to write in advance.
On each write (memory store), \dsmnoxact\ revokes the write permission from the current writer, 
writes the current dirty data to the home node, and grants the write permission to the new writer.
Compared to \dsmxact, \dsmnoxact\ supports stronger consistency, requires less programmer efforts, 
but incurs higher performance overhead because of its more frequent writer invalidation.

Apart from the two DSM systems that we built, 
we also compare \hotpot\ with Grappa~\cite{Nelson15-ATC}, 
a recent DSM system that supports modern data-parallel applications. 
Different from traditional DSM systems and our DSM systems, Grappa moves computation to data instead of fetching data to where computation is.


{
\begin{table}[t]
\begin{center}
\begin{center}
\begin{tabular}{ c | c | c | c | c | c }\normalsize
\normalsize Workload & \normalsize Read & \normalsize Update & \normalsize Scan & \normalsize Insert & \normalsize R\&U \\
\hline
A & 50\% & 50\% & - & - & - \\
B & 95\% & 5\% & - & - & - \\
C & 100\% & - & - & - & - \\
D & 95\% & - & - & 5\% & - \\
E & - & - & 95\% & 5\% & - \\
F & 50\% & - & - & - & 50\% \\
\end{tabular}
\end{center}
\caption[YCSB Workload Properties.]
{
YCSB Workload Properties.
The percentage of operations in each YCSB workload. 
R\&U stands for Read and Update.
}
\label{tbl-ycsb}
\end{center}
\end{table}
}
{
\begin{figure*}[t]
\begin{center}
\centerline{\includegraphics[width=\textwidth]{hotpot/Figures/g_plot_YCSB_run_throughput.pdf}}
\caption[YCSB Workloads Throughput.]{YCSB Workloads Throughput.}
\label{fig-ycsbrun}
\end{center}
\end{figure*}
}


\subsection{In-Memory NoSQL Database}
\label{sec:hotpot:mongodb}
MongoDB~\cite{MongoDB} is a popular distributed NoSQL database that supports several different storage engines
including its own storage engine that is based on memory-mapped files (called MMAPv1).
Applications like MongoDB can largely benefit from having a fast means to store and access persistent data. 
We ported MongoDB v2.7.0 to \hotpot\ by modifying its storage engine to keep track of all writes to the memory-mapped data file.
We then group the written memory regions belonging to the same client request into a \hotpot\ \commit\ call.
In total, porting MongoDB to \hotpot\ requires modifying 120 lines of code. 

To use the ported MongoDB, administrators can simply configure several machines to share 
a \dsnvm\ space under \hotpot\ and run ported MongoDB on each machine.
Applications on top of the ported MongoDB can issue requests to any machine, 
since all machines access the same \dsnvm\ space.
In our experiments, we ran the ported MongoDB on three \hotpot\ nodes
and set data replication degree to three.

We compare this ported MongoDB with the default MongoDB running on \tmpfs, \pmfs, and \Octopus, 
and a ported MongoDB to \Mojim\ on three nodes connected with IB.
Because \Octopus\ does not memory-mapped operations and MongoDB's storage engine is based on memory-mapped files,
MongoDB cannot directly run on \Octopus.
We run MongoDB on top of FUSE~\cite{fuse-fs}, a full-fledged user-level file system, 
which in turn runs on \Octopus.

%All these systems have three replicas of all data.
%\tmpfs, \pmfs, and \Octopus\ use MongoDB's default replication mechanism, 
For \tmpfs\ and \pmfs, we use two consistency models (called MongoDB write concerns):
the \journaled\ write concern and the \fsyncsafe\ write concern. With the \journaled\ write concern, MongoDB
logs data in a journal file and checkpoints the data in a lazy fashion. MongoDB blocks a client call until the
updated data is written to the journal file. With \fsyncsafe, MongoDB does not perform journaling. Instead, it flushes
all the dirty pages to the data file after each write operation and blocks the client call until this operation completes.
We run \Octopus\ and \Mojim\ with the \fsyncsafe\ write concern.
\Octopus, \tmpfs, and \pmfs\ provide no replication,
while \Mojim\ and \hotpot\ use their own replication mechanisms to make three replicas of all data 
(\Mojim\ uses one node as the primary node and the other two nodes as backup nodes).

YCSB~\cite{Cooper10-CloudCom} is a key-value store benchmark 
that imitates web applications' data access models. 
Figure~\ref{tbl-ycsb} summarizes the number of different operations in the YCSB workloads.
Each workload performs 10,000 operations on a database with 100,000 1\KB\ records.
Figure~\ref{fig-ycsbrun} presents the throughput of MongoDB on \tmpfs, \pmfs, Octopus, Mojim, and \hotpot\ using YCSB workloads. 

For all workloads, \hotpot\ outperforms \tmpfs, \pmfs, \Octopus, and \Mojim\ for both the \journaled\ and the \fsyncsafe\ write concerns. 
The performance improvement is especially high for write-heavy workloads.
\pmfs\ performs worst mainly because of its inefficient process of making data persistent with default MongoDB.
The default MongoDB \fsync{}s the whole data file after each write under \fsyncsafe,
and \pmfs\ flushes all cache lines of the file to \nvm\ by performing one \clflush\ at a time.
\hotpot\ and Mojim only commit dirty data, largely improving MongoDB performance over \pmfs.
Compared to \tmpfs\ and \pmfs\ under \journaled, \hotpot\ and Mojim use their own mechanisms to 
ensure data reliability and avoid the performance cost of journaling.
Moreover, \hotpot\ and Mojim make three persistent replica for all data, while \pmfs\ makes only one.
Tmpfs is slower than \hotpot\ even though \tmpfs\ does not make any data persistent, 
because MongoDB's slower replication mechanism on IPoIB.
\hotpot's network layer is significantly better than IPoIB~\cite{lite-sosp17}.

\Octopus\ performs worse than \hotpot\ and \Mojim\ because it incurs significant overhead of additional {\em indirection layers}:
each memory operation within the memory-mapped file goes through the FUSE file system and then through \Octopus.
\hotpot\ and \Mojim\ both support native memory instructions and incurs no indirection overhead.
Finally, even though Mojim's replication protocol is simpler and faster than \hotpot's,
\hotpot\ outperforms Mojim because Mojim only supports write on one node while \hotpot\ supports write on all nodes.

{
\begin{figure*}[th]
\begin{minipage}{1.9in}
\begin{center}
\centerline{\includegraphics[width=1.9in]{Figures/g_plot_graph_ATC_runtime.pdf}}
\vspace{-0.05in}
\mycaption{fig-graph-runtime}{Pagerank Total Run Time.}
{
N stands for total number of nodes, T stands for number of threads running on a node.
}
\end{center}
\end{minipage}
\begin{minipage}{0.04in}
\hspace{0.04in}
\end{minipage}
\begin{minipage}{1.8in}
\begin{center}
\centerline{\includegraphics[width=1.8in]{Figures/g_plot_graph_ATC_network.pdf}}
\vspace{-0.05in}
\mycaption{fig-graph-traffic}{Pagerank Total Network Traffic.}
{
}
\end{center}
\end{minipage}
\begin{minipage}{0.04in}
\hspace{0.04in}
\end{minipage}
\begin{minipage}{3.1in}
\begin{center}
\centerline{\includegraphics[width=3.1in]{Figures/g_plot_combined_trace_timewindow.pdf}}
\vspace{-0.05in}
\mycaption{fig-graph-timeline}{Pagerank Network Traffic Over Time.}
{
}
\end{center}
\end{minipage}
\vspace{-0.3in}
\end{figure*}
}


\subsection{Distributed (Persistent) Graph}
Graph processing is an increasingly important type of applications in modern 
datacenters~\cite{Gonzalez12-OSDI,Gonzalez14-OSDI,Kyrola12-OSDI,Low10-UAI,Low12-VLDB,Malewicz10-SIGMOD}.
Most graph systems require large memory to run big graphs.
Running graph algorithms on \nvm\ not only enables them to exploit the big memory space the high-density \nvm\ provides,
but can also enable graph algorithms to stop and resume in the middle of a long run.

We implemented a distributed graph processing engine on top of \hotpot\ based on the PowerGraph design~\cite{Gonzalez12-OSDI}.
It stores graphs with vertex-centric representation in \dsnvm\ with random order of vertices
and distributes graph processing load to multiple threads across all \hotpot\ nodes.
Each thread performs graph algorithms on a set of vertices in three steps: gather, apply, and scatter, 
with the optimization of delta caching~\cite{Gonzalez12-OSDI}.
After each step, we perform a global synchronization with \barrier\ and only start the next step when all threads have finished the last step.
At the scatter step, the graph engine uses \hotpot's \mrsw\ \commitxact\ to make local changes of the scatter values 
visible to all nodes in the system. We implemented the \hotpot\ graph engine with only around 700 lines of code.
Similarly, we implemented two distributed graph engines on top of \dsmxact\ and \dsmnoxact;
these engines differ from \hotpot's graph engine only in the way they perform data write and commit.

We compare \hotpot's graph engine with \dsmxact, \dsmnoxact, PowerGraph, and Grappa~\cite{Nelson15-ATC} with two real datasets,
Twitter (41\,M vertices, 1\,B directed edges)~\cite{Kwak10-WWW} and LiveJournal (4\,M vertices, 34.7\,M undirected edges)~\cite{snapnets}.
For space reason, we only present the results of the Twitter graph, but the results of LiveJournal are similar.
Figure~\ref{fig-graph-runtime} shows the total run time of the PageRank~\cite{PageRank} algorithm with
\hotpot, \dsmxact, \dsmnoxact, PowerGraph, and Grappa under three system settings:
four nodes each running four graph threads, seven nodes each running four threads, and seven nodes each running eight threads.

\hotpot\ outperforms PowerGraph by 2.3\x\ to 5\x\ and Grappa by 1.3\x\ to 3.2\x.
In addition, \hotpot\ makes all intermediate results of graph persistent for fast restart. 
A major reason why \hotpot\ outperforms PowerGraph and Grappa even when \hotpot\
requires data persistence and replication is \hotpot's network stack.
Compare to the IPoIB used in PowerGraph and Grappa's own network stack,
\hotpot's RDMA stack is more efficient.

Our implementation of \dsmxact\ and \dsmnoxact\ use the same network stack as \hotpot,
but \hotpot\ still outperforms \dsmnoxact\.
\dsmnoxact\ ensures cache coherence on every write and thus incurs much higher performance overhead than \hotpot\ and \dsmxact.

To further understand the performance differences, we traced the network traffic of these three systems.
Figure~\ref{fig-graph-traffic} plots the total amount of traffic %in the PageRank run for PowerGraph, Grappa, and \hotpot. 
and Figure~\ref{fig-graph-timeline} plots a detailed trace of network activity of the 7Nx4T setting.
\hotpot\ sends less total traffic and achieves higher bandwidth than PowerGraph and Grappa.


{
\begin{figure*}[th]
\begin{center}
\centerline{\includegraphics[width=0.5\textwidth]{clio/Figures/g_plot_scalability_conn.pdf}}
\mycaption{fig-conn}{Process (Connection) Scalability.}
{
}
\end{center}
\end{figure*}
}
{
\begin{figure*}[h]
\begin{center}
\centerline{\includegraphics[width=0.5\textwidth]{clio/Figures/g_plot_scalability_pte.pdf}}
\mycaption{fig-pte-mr}{PTE and MR Scalability.}
{
RDMA fails beyond $2^{18}$ MRs. 
}
\end{center}
\end{figure*}
}
{
\begin{figure*}[h]
\begin{center}
\centerline{\includegraphics[width=0.5\textwidth]{clio/Figures/g_plot_latency_comparison.pdf}}
\mycaption{fig-miss-hit}{Comparison of TLB Miss and page fault.}
{
\sys-ASIC are projected values of TLB hit.
}
\end{center}
\end{figure*}
}
{
\begin{figure*}[h]
\begin{center}
\centerline{\includegraphics[width=0.5\textwidth]{clio/Figures/clio_rdma_lat_cdf.pdf}}
\mycaption{fig-tail-latency}{Latency CDF.}
{
}
\end{center}
\end{figure*}
}
{
\begin{figure*}[th]
\begin{center}
\centerline{\includegraphics[width=0.5\textwidth]{clio/Figures/g_plot_throughput.pdf}}
\mycaption{fig-read-write-throughput}{End-to-End Goodput.}
{
1\KB\ requests. % between 1 \CN\ and 1 \MN.
}
\end{center}
\end{figure*}
}
{
\begin{figure*}[h]
\begin{center}
\centerline{\includegraphics[width=0.5\textwidth]{clio/Figures/g_plot_onboard_throughput.pdf}}
\mycaption{fig-onboard-throughput}{On-board Goodput.}
{
FPGA test module generates requests at maximum speed.
}
\end{center}
\end{figure*}
}
{
\begin{figure*}[h]
\begin{center}
\centerline{\includegraphics[width=0.5\textwidth]{clio/Figures/g_plot_read_latency.pdf}}
\mycaption{fig-read-lat}{Read Latency.}
{
HERD-BF: HERD running on BlueField. %SmartNIC.
}
\end{center}
\end{figure*}
}
{
\begin{figure*}[h]
\begin{center}
\centerline{\includegraphics[width=0.5\textwidth]{clio/Figures/g_plot_write_latency.pdf}}
\mycaption{fig-write-lat}{Write Latency.}
{
Clover requires $\ge$ 2 RTTs for write.
}
\end{center}
\end{figure*}
}

\section{Evaluation}
\label{sec:clio:results}


Our evaluation reveals the scalability, throughput, median and tail latency, energy and resource consumption of \sys.
%, and how it compares with state-of-the-art systems. 
We compare \sys's end-to-end performance with industry-grade NICs (ASIC) and well-tuned RDMA-based software systems.
All \sys's results are FPGA-based, which would be improved with ASIC implementation.
%Nonetheless, \sys\ significantly outperforms RDMA on scalability and tail latency, while being similar on other measurements.

\ulinebfpara{Environment.}
We evaluated \sys\ in our local cluster of four \CN{}s and four \MN{}s (Xilinx ZCU106 boards),
%\footnote{Unfortunately, our process of purchasing and setting up a bigger cluster was significantly delayed because of COVID-19},
all connected to an Nvidia 40\Gbps\ VPI switch.
Each \CN\ is a Dell PowerEdge R740 server equipped with a Xeon Gold 5128 CPU and a 40\Gbps\ Nvidia ConnectX-3 NIC,
with two of them also having an Nvidia BlueField SmartNIC~\cite{BlueField}.
We also include results from CloudLab~\cite{CloudLab} with the Nvidia ConnectX-5 NIC.


\subsection{Basic Microbenchmark Performance}

{
\begin{figure*}[th]
\begin{minipage}{\figWidthSix}
\begin{center}
\centerline{\includegraphics[width=\columnwidth]{Figures/g_plot_alloc_free.pdf}}
\vspace{-0.1in}
\captionsetup{width=.9\columnwidth}
\mycaption{fig-alloc-free}{Alloc/Free Latency.}
{
ODP means On-Demand-Paging mode
}
\end{center}
\end{minipage}
\begin{minipage}{\figWidthSix}
\begin{center}
\centerline{\includegraphics[width=\columnwidth]{Figures/g_plot_alloc_conflict.pdf}}
\vspace{-0.1in}
\captionsetup{width=.9\columnwidth}
\mycaption{fig-alloc-conflict}{Alloc Retry Rate.}
{
%Alloc's number of retries when vary physical memory utilization.
}
\end{center}
\end{minipage}
\begin{minipage}{\figWidthSix}
\begin{center}
\centerline{\includegraphics[width=\columnwidth]{Figures/g_plot_latency_breakdown.pdf}}
\vspace{-0.1in}
\captionsetup{width=.9\columnwidth}
\mycaption{fig-lat-break}{Latency Breakdown.}
{
Breakdown of time spent at \sysboard.
}
\end{center}
\end{minipage}
\begin{minipage}{\figWidthSix}
\begin{center}
\centerline{\includegraphics[width=\columnwidth]{Figures/g_plot_ycsb_mn.pdf}}
\vspace{-0.1in}
\captionsetup{width=.9\columnwidth}
\mycaption{fig-ycsb-mn}{\syskv\ Scalability against \MN{}s.}
{
}
\end{center}
\end{minipage}
\vspace{-0.15in}
\end{figure*}
}

{
\begin{figure*}[th]
\begin{minipage}{\figWidthSix}
\begin{center}
\centerline{\includegraphics[width=\columnwidth]{Figures/g_plot_image_compression.pdf}}
\vspace{-0.1in}
\captionsetup{width=.9\columnwidth}
\mycaption{fig-photo}{Image Compression.}
{
}
\end{center}
\end{minipage}
\begin{minipage}{\figWidthSix}
\begin{center}
\centerline{\includegraphics[width=\columnwidth]{Figures/g_plot_radix_tree.pdf}}
\vspace{-0.1in}
\captionsetup{width=.9\columnwidth}
\mycaption{fig-radix}{Radix Tree Search Latency.}
{
}
\end{center}
\end{minipage}
\begin{minipage}{\figWidthSix}
\begin{center}
\centerline{\includegraphics[width=\columnwidth]{Figures/g_plot_ycsb_cn.pdf}}
\vspace{-0.1in}
\captionsetup{width=.9\columnwidth}
\mycaption{fig-kvstore}{Key-Value Store YCSB Latency.}
{
}
\end{center}
\end{minipage}
%\if 0 g_plot_ycsb_mn
%\fi
\begin{minipage}{\figWidthSix}
\begin{center}
\centerline{\includegraphics[width=\columnwidth]{Figures/g_plot_mvstore.pdf}}
\vspace{-0.1in}
\captionsetup{width=.9\columnwidth}
\mycaption{fig-mvstore}{\sysmv\ Object Read/Write Latency.}
{
}
\end{center}
\end{minipage}
%\vspace{-0.15in}
\end{figure*}
}


\ulinebfpara{Scalability.}
We first compare the scalability of \sys\ and RDMA.
Figure~\ref{fig-conn} measures the latency of \sys\ and RDMA as the number of client processes increases.
For RDMA, each process uses its own QP.
Since \sys\ is connectionless, it scales perfectly with the number of processes.
RDMA scales poorly with its QP, and the problem persists with newer generations of RNIC,
which is also confirmed by our previous works~\cite{Pythia,Storm}.

Figure~\ref{fig-pte-mr} evaluates the scalability with respect to PTEs and memory regions.
For the memory region test, we register multiple MRs using the same physical memory for RDMA.
For \sys, we map a large range of VAs (up to 4\TB) to a small physical memory space, as our testbed only has 2\GB\ physical memory.
However, the number of PTEs and the amount of processing needed are the same for \sysboard\ as if it had a real 4\TB\ physical memory.
Thus, this workload stress tests \sysboard's scalability.
%For \sys\ (which gets rid of the MR concept), we use multiple processes to share the same memory,
%resulting in one PTE per process.
RDMA's performance starts to degrade when there are more than $2^8$ (local cluster) or $2^{12}$ (CloudLab),
and the scalability wrt MR is worse than wrt PTE.
In fact, RDMA fails to run beyond $2^{18}$ MRs.
In contrast, \sys\ scales well and never fails (at least up to 4\TB\ memory).
It has two levels of latency that are both stable: a lower latency below $2^4$ for TLB hit and a higher latency above $2^4$ for TLB miss (which always involves one DRAM access).
A \sysboard\ could use a larger TLB if optimal performance is desired.

These experiments confirm that \textbf{\sys\ can handle thousands of concurrent clients and TBs of memory}.



\ulinebfpara{Latency variation.}
Figure~\ref{fig-miss-hit} plots the latency of reading/writing 16\,B data 
when the operation results in a TLB hit, a TLB miss, a first-access page fault, and MR miss (for RDMA only, when the MR metadata is not in RNIC).
RDMA's performance degrades significantly with misses.
Its page fault handling is extremely slow (16.8\ms).
We confirm the same effect on CloudLab with the newer ConnectX-5 NICs.
\sys\ only incurs a small TLB miss cost and \textbf{no additional cost of page fault handling}.

We also include a projection of \sys's latency if it was to be implemented using a real ASIC-based \sysboard.
Specifically, we collect the latency breakdown of time spent on the network wire and at \CN, time spent on third-party FPGA IPs,
number of cycles on FPGA, and time on accessing on-board DRAM.
We maintain the first two parts, scale the FPGA part to ASIC's frequency (2\,GHz), use DDR access time collected on our server to replace the access time to on-board DRAM (which 
goes through a slow board memory controller).
This estimation is conservative, as a real ASIC implementation of the third-party IPs would make the total latency lower.
Our estimated read latency is better than RDMA, while write latency is worse.
We suspect the reason being Nvidia RNIC's optimization of replying a write before it is fully written to DRAM, which \sys\ could also potentially adopt.

Figure~\ref{fig-tail-latency} plots the request latency CDF of continuously running read/write 16\,B data while not triggering page faults.
Even without page faults, \sys\ has much less latency variation and a much shorter tail than RDMA.
%Thanks to our bounded address translation and deterministic hardware design, \sys\ has much less latency variation and a much shorter tail than RDMA.

{
\begin{figure*}[th]
\begin{center}
\centerline{\includegraphics[width=0.5\textwidth]{clio/Figures/g_plot_dp.pdf}}
\mycaption{fig-dataframe}{Select-Aggregate-Shuffle.}
{
Y axis starts at 4 sec. 
CN represents computation done at \CN.
%, as histogram  time is the same.
}
\end{center}
\end{figure*}
}
{
\begin{figure*}[h]
\begin{center}
\centerline{\includegraphics[width=0.5\textwidth]{clio/Figures/g_plot_ycsb_energy.pdf}}
\mycaption{fig-energy}{Energy Comparison.}
{
Darker/lighter shades represent energy spent at \MN{}s and \CN{}s.
}
\end{center}
\end{figure*}
}
{
\begin{table}\small
\begin{center}
\begin{center}
\begin{tabular}{ p{1.2in} | p{0.5in} |p{0.6in} }
 & \textbf{Logic} & \textbf{Memory} \\
\textbf{System/Module} & \textbf{(LUT)} & \textbf{(BRAM)} \\
\hline
\hline
StRoM-RoCEv2 & 39\% & 76\% \\
Tonic-SACK & 48\% & 40\% \\
\hline
\sys\ (Total) & 31\% & 31\% \\
VirtMem & 5.5\% & 3\% \\
NetStack & 2.3\% & 1.7\% \\
\hline
Go-Back-N & 5.8\% & 2.6\% \\
\end{tabular}
\end{center}
\mycaption{fig-fpga-resource}{Clio FPGA Utilization.}
{
}
\end{center}
\end{table}
}

\ulinebfpara{Read/write throughput.}
We measure \sys's throughput by varying the number of concurrent client threads (Figure~\ref{fig-read-write-throughput}).
\sys's default asynchronous APIs quickly reach the line rate of our testbed (9.4\Gbps\ maximum throughput).
Its synchronous APIs could also reach line rate fairly quickly.

Figure~\ref{fig-onboard-throughput} measures the maximum throughput of \sys's FPGA implementation without the bottleneck of the board's 10\Gbps\ port, by generating traffic on board.
Both read and write can reach more than 110\Gbps\ when request size is large.
Read throughput is lower than write when request size is smaller.
We found the throughput bottleneck to be the third-party non-pipelined DMA IP
(which could potentially be improved).

\ulinebfpara{Comparison with other systems.}
We compare \sys\ with native one-sided RDMA, Clover~\cite{Tsai20-ATC}, HERD~\cite{Kalia14-RDMAKV}, and LegoOS~\cite{Shan18-OSDI}.
We ran HERD on both CPU and BlueField (HERD-BF).
%Native RDMA can be considered as a baseline (optimal performance but low-level, restrictive interface).
Clover is a passive disaggregated persistent memory system which we adapted as a passive disaggregated memory (PDM) system.
HERD is an RDMA-based system that supports a key-value interface with an RPC-like architecture.
LegoOS builds its virtual memory system in software at \MN.
%It uses one RDMA read for its read and one RDMA write plus one 
%it can be considered as a software-based active disaggregated memory system. 

\sys's performance is similar to HERD and close to native RDMA.
%\sys's write performance is better than Clover and similar to HERD. %but has a constant overhead over native RDMA.
Clover's write is the worst because it uses at least 2 RTTs for writes to deliver its consistency guarantees without any processing power at \MN{}s.
HERD-BF's latency is much higher than when HERD runs on CPU
due to the slow communication between BlueField's ConnectX-5 chip and ARM processor chip.
LegoOS's latency is almost two times higher than \sys's when request size is small.
In addition, from our experiment, LegoOS can only reach a peak throughput of 77\Gbps, while \sys\ can reach 110\Gbps.
LegoOS' performance overhead comes from its software approach, demonstrating the necessity of a hardware-based solution like \sys.
%due to the slow communication between BlueField's Connect-X5 chip and ARM processor %chip..
%\sys's write overhead can be attributed to \fixme{XXX}.

\ulinebfpara{Allocation performance.}
Figure~\ref{fig-alloc-free} shows \sys's VA and PA allocation and RDMA's MR registration performance.
%Physical memory allocation includes the time to perform an allocation with the buddy algorithm and to insert the allocated address into the free page list.
%It is very fast, indicating that our asynchronous free physical page generation could keep up with most workloads' page fault speed.
%Virtual memory allocation and free (measured from client on \CN) are slower,
\sys's PA allocation takes less than 20\mus, and the VA allocation is much faster than RDMA MR registration,
although both get slower with larger allocation/registration size.
%since these operations involve the costly crossing between FPGA and ARM.
%They are also slower with larger sizes, as searching the VMA tree for a big free region takes more time.
Figure~\ref{fig-alloc-conflict} shows the number of retries at allocation time with three allocation sizes as the physical memory fills up.
%running at on-board ARM processor.
%The hash-based page table is proportional to the physical memory size. 
%Hence higher its utilization, higher the overflow probability therefore higher number of retries.
%The page table has 2\x\ extra slots by default.
There is no retry when memory is below half utilized. Even when memory is close to full, there are at most 60 retries per allocation request, with roughly 0.5\ms\ per retry. This confirms that our design of avoiding hash overflows at allocation time is practical.
%co-design of overflow-free hash-based page table and allocation retry scheme is practical.


\ulinebfpara{Close look at \sysboard{} components.}
To further understand \sys's performance, % and to determine the reason for worse large-read performance,
we profile different parts of \sys's processing for read and write of 4\,B to 1\KB.
\syslib\ adds a very small overhead (250\ns\ in total), 
thanks to our efficient threading model and network stack implementation.
Figure~\ref{fig-lat-break} shows the latency breakdown at \sysboard.
Time to fetch data from DRAM (DDRAccess) and to transfer it over the wire (WireDelay) are the main 
contributor to read latency, especially with large read size.
Both could be largely improved in a real \sysboard\ with better memory controller and higher frequency.
TLB miss (which takes one DRAM read) is the other main part of the latencies.


\subsection{Application Performance}

\ulinebfpara{Image Compression.}
We run a workload where each client 
compresses and decompresses 1000 256*256-pixel images with increasing number of concurrently running clients.
Figure~\ref{fig-photo} shows the total runtime per client.
We compare \sys\ with RDMA, with both performing computation at the \CN\ side and the RDMA using one-sided operations instead of \sys\ APIs to read/write images in remote memory.
\sys's performance stays the same as the number of clients increase.
RDMA's performance does not scale because it requires each client to register a different MR to have protected memory accesses.
With more MRs, RDMA runs into the case where the RNIC cannot hold all the MR metadata and many accesses would involve a slow read to host main memory.

\ulinebfpara{Radix Tree.}
Figure~\ref{fig-radix} shows the latency of searching a key in pre-populated radix trees when varying the tree size. 
We again compare with RDMA which uses one-sided read operations to perform the tree traversal task.
RDMA's performance is worse than \sys,
because it requires multiple RTTs to traverse the tree,
while \sys\ only needs one RTT for each pointer chasing (each tree level).
In addition, RDMA also scales worse than \sys.

\ulinebfpara{Key-value store.}
Figure~\ref{fig-kvstore} evaluates \syskv\ using the YCSB benchmark~\cite{YCSB} and compares it to Clover, HERD, and HERD-BF.
We run two \CN{}s and 8 threads per \CN.
We use 100K key-value entries and run 100K operations per test,
with YCSB's default key-value size of 1\KB. %where the key size is 8 bytes and the value size is 1\KB.
The accesses to keys follow the Zipf distribution ($\theta=0.99$).
We use three YCSB workloads with different {\em get-set} ratios: 
100\% {\em get} (workload C), 5\% {\em set} (B), and 50\% {\em set} (A).
\syskv\ performs the best.
HERD running on BlueField performs the worst, mainly because BlueField's slower crossing between its NIC chip and ARM chip.




Figures~\ref{fig-ycsb-mn} shows the throughput of \syskv\ when varying the number of MNs. Similar to our
\sys\ scalability results, \syskv\ can reach a CN’s maximum
throughput and can handle concurrent get/set requests even
under contention. These results are similar to or better than
previous FPGA-based and RDMA-based key-value stores that
are fine-tuned for just key-value workloads (Table 3 in \cite{KVDIRECT}),
while we got our results without any performance tuning.

\ulinebfpara{Multi-version data store.}
%\subsubsection{Multi-Version Data Store}
We evaluate \sysmv\ by varying the number of \CN{}s that concurrently access data objects (of 16\,B) on an \MN\ using workloads of 50\% read (of different versions) and 50\% write under uniform and Zipf distribution of objects (Figure~\ref{fig-mvstore}). 
\sysmv's read and write have the same performance, and reading any version has the 
same performance, since we use an array-based version design. 
%Running multiple \MN{}s have similar performance and we omit for space.




\ulinebfpara{Data analytics.}
We run a simple workload which first \texttt{select} rows in a table whose field-A matches a value (\eg, gender is female)
and calculate \texttt{avg} of field-B (\eg, final score) of all the rows.
Finally, it calculates the histogram of the selected rows (\eg, score distribution), which can be presented to the user together with the avg value. %(\eg, how female students' scores compare to the whole class).
\sys\ executes the first two steps at \MN\ offloads and the final step at \CN,
while RDMA always reads rows to \CN\ and then does each operation.
Figure~\ref{fig-dataframe} plots the total run time as the select ratio decreases (fewer rows selected).
% When the select ratio is high, \sys\ and RDMA send a similar amount of data across the network,
% and as the CPU computation is faster than our FPGA implementation for these operations, \sys's overall performance is worse than RDMA.
When the select ratio is low, \sys\ transfers much less data than RDMA, resulting in its better performance.




%To put \sys\ in respective with other existing RDMA-based and FPGA-based key-value stores that we couldn't directly compare with (\eg, close-sourced), we compare 
%\syskv's latency results with reported latencies in ~\cite{KVDIRECT}. 
%\syskv\ has {\bf lower end-to-end latency than all these existing systems}.

%then sends the data to \CN, which shuffles the data and sends the shuffled 
%data back to \MN\ for aggregation.

\subsection{CapEx, Energy, and FPGA Utilization}
\label{sec:clio:results-cost}


We estimate the cost of server and \sysboard\ using market prices of different hardware units. When using 1\TB\ DRAM, a server-based \MN\ costs 1.1-1.5\x\ and consumes 1.9-2.7\x\ power compared to \sysboard. These numbers become 1.4-2.5\x\ and 5.1-8.6\x\ with OptaneDimm~\cite{optane-dcpm}, which we expect to be the more likely remote memory media in future systems.


We measure the total energy used for running YCSB workloads
by collecting the total CPU (or FPGA) cycles and the Watt of a CPU core~\cite{gold5128}, ARM processor~\cite{armpower}, and FPGA (measured).
We omit the energy used by DRAM and NICs in all the calculations. 
Clover, a system that centers its design around low cost, has slightly higher energy than \sys.
Even though there is no processing at \MN{}s for Clover, its \CN{}s use more cycles to process and manage memory.
HERD consumes 1.6\x\ to 3\x\ more energy than \sys, mainly because of its CPU overhead at \MN{}s.
Surprisingly, HERD-BF consumes the most energy, even though it is a low-power ARM-based SmartNIC.
This is because of its worse performance and longer total runtime.

Figure~\ref{fig-fpga-resource} compares the FPGA utilization among Clio, StRoM's RoCEv2~\cite{StRoM}, and Tonic's selective ack stack~\cite{TONIC}.
%With our design that is tailored to save resources, 
%\sys\ consumes roughly one third of the total resources.
Both StRoM and Tonic include only a network stack but they consume more resources than \sys.
Within \sys, the virtual memory (VirtMem) and
the network stack (NetStack) consume a small fraction of the total resources,
with the rest being vendor IPs (PHY, MAC, DDR4, and interconnect).
%To put things in perspective, we implement a Go-back-N network stack which supports 1K connections. It uses 2.5\x\ more logic than what our current network stack consumes. 
Overall, our efficient hardware implementation leaves most FPGA resources available for application offloads.

% https://docs.google.com/spreadsheets/d/1JODWoEtDBxeOTr-ZqEqm3D-9wfADyNqgMoIOSpukBL0/edit#gid=0

\section{Related Work}
\label{sec:related}


Datacenter network topology.
Disaggregation.
Devices: pswitch, circuit switch, multi-host nic.

Intel IPU










\if 0
In this section, we review emerging network devices
and investigate whether they can be used to implement the
network resource pool, which, in turn provides network-as-a-service
for both disaggregated devices and regular servers.
Our focus is the support for disaggregated devices.
As it requires more functionalities and hence a super set of 
the ones required for regular server case.
Solutions work for the disaggregation setting would naturally
work for the regular-server cluster setting.

To this end, we propose a set of goals that a particular
solution must meet:
\boldpara{1) Port Count}.
The solution must be able to
support the exploded number of endpoints with
a cost-effective network topology.
\boldpara{2) Heterogeneous Endpoints}.
The solution should support various known computation mediums
such as FPGA, ASIC, and CPU, as well as any new ones in the future.
\boldpara{3) Transports and Network Functions}.
As we discussed earlier, the network resource pool is 
consolidating three types of resources: packet processing
in NIC, software network stack and advanced application-specific
network functions. The first type naturally comes with hardware.
The solution must have a mean to support the latter two.
\boldpara{4) Programmability}.
One of the key requirements for any current or future datacenters
is the ability to upgrade or re-program after deployment.
\boldpara{5) Consolidation, Manageability, and Multi-Tenancy}.
The core of pool is resource consolidation,
which relies on good management and multi-tenancy support.

Table~\ref{tabel-related-work} presents all the goals
and whether each reviewed network device or system can meet them.
Next, we will take a deep dive into each one.


\subsection{Programmable Switch}

\subsubsection{Primer}
Unlike traditional switches, the programmable switches
allow users to install specific actions on the switch data path.
thereby enable line-rate packet processing.
The core of programmable switch is Reconfigurable Match Table (RMT),
pioneered by a seminal SIGCOMM'13 paper~\cite{RMT-SIGCOMM13}.
RMT was first proposed to enhance OpenSDN deployment.
Since then, the programmable switches have seen great success in both industry and academic. 
P4~\cite{p4-paper}, a young domain-specific language specifically designed for packet processing, is the de-facto programming language for programmable switches. P4 greatly simplifies
packet manipulation and has helped the wide adoption of programmable switches.

The arise of programmable switch shifted the network computation paradigm:
it breaks the common belief held by distributed system designers
and opens doors to improve, redesign, or create distributed systems in unimaginable ways.
In essence, programmable switch is a centralized computation point
that can mitigate synchronization and consistency issues,
acting as cache front end, or simply be a network function offloading unit. 
Recent systems demonstrate performance improvements in domains like
caching for KV~\cite{netcache-sosp17, incbricks-asplos17},
caching for load-balancing~\cite{distcache-fast19},
in-network coherence directories~\cite{pegasus-osdi20},
congestion control~\cite{hpcc-sigcomm19},
distributed lock management~\cite{netlock-sigcomm20},
databases~\cite{cheetah-sigmod20},
scheduling~\cite{racksched-osdi20},
and network function processing~\cite{tea-sigcomm20}.
%consensus,
%machine learning,

Most recently, researchers started using programmable switches
to consolidate computation resource.
Wang et al.~\cite{wang-hotcloud20} observes programmable switches are
heavily under-utilized, hence use a set of
compile/run-time techniques to deploy multiple p4 programs
onto one programmable switch, thereby enabling multi-tenancy and consolidation.
TEA~\cite{tea-sigcomm20} consolidates NFs at rack-scale,
providing NF-as-a-service to the servers under the ToR switch.
Das et al.~\cite{active-hotnets20} takes a fresh look at
active networking and uses p4 to turn a programmable switch
into a physical computing device akin to a virtual machine.

\subsubsection{Feasibility}

Programmable switch has limited number of ports,
e.g., 64 ports for Intel Tofino2, hence not able to
accommodate the exploded number of disaggregated devices.
Further, with the diminishing of Dennard Scaling and Moore's Law,
the merchant chip is not likely to see dramatic computation power increase,
which in turn limits the number of ports a certain chip can support.

Most of commodity programmable switches are Ethernet-based.
To communicate with such switches,
an endpoint requires an Ethernet gear
(could be any physical form: a chip, a device, or integrated IPs)
with at least L1 and L2 functionalities (i.e., a PHY and a MAC). 
Since all disaggregated devices are directly attached to the network,
it is reasonable to assume they have such Ethernet gear equipped.
As for regular servers, they already have NIC installed.
In all, programmable switches are able to support heterogeneous devices.
Note that both the switch and the endpoints are free to
use other physical and link layer protocols (e.g., Infiniband),
there is nothing fundamental about using Ethernet except it is
already widely used so its beneficial continue using it.

As we mentioned earlier, several work~\cite{tea-sigcomm20,active-hotnets20,wang-hotcloud20}
have tried to consolidate NFs and applications onto programmable switches.
No work has tried to build transport on it, though we believe it is doable.
However, the multi-tenancy support is still at its infancy.
Most of the existing work leverage compile-time tricks to
overcome hardware limitations, which result in some inevitable cost.
With enough momentum, the vendors might wight in and develop certain
virtualization features on programmable switches.
It would be interesting to explore what those features might be.

\subsubsection{Summary}

For both regular server's network disaggregation and consolidation,
programmable switches can partially meet their goals.
Servers can offload their transport processing,
network functions, and advanced application-specific functions to
the programmable switches -  this is already possible now.
However, current commodity programmable switches
are not able to meet the goals for disaggregated datacenter.
Specifically, it is not able to solve the exploded port counts
without a significant increase in cost.

%What about line cards.

\subsection{Circuit Switch}

\subsubsection{Primer}

Circuit switch establishes a dedicated channel between
endpoints connected to it. It guarantees the full bandwidth
of the channel and remains connected for the duration of a
certain session. It creates an illusion as if endpoints are
directly and physically connected.

Circuit switch operates at the physical layer with no buffers,
no arbitration, and no packet inspection mechanisms.
Thus, they are cheaper and more power-efficient than
traditional electrical packet switch. As a result,
circuit switches could support hundreds or even thousands of
ports with lower CAPEX and OPEX than equivalent packet switches,
making it a good candidate to interconnect disaggregated devices in a rack.

Circuit switch has seen great improvement over the previous decade.
Around 2010, Helios~\cite{helios-sigcomm10} first proposes to integrate circuit switches into
the datacenter network and uses a hybrid packet and circuit switching mode.
Mordia~\cite{mordia-sigcomm13} improves the switching time from
tens of milliseconds to microsecond level. In response, Mordia proposes
a proactive scheduling mechanism instead of a reactive one.
REACToR~\cite{reactor-nsdi14} leverages Mordia's fast switching and builds a hybrid
ToR using both packet and circuit switches, enjoy the benefits of both.
But REACToR is sensitive to the traffic pattern.
Until then, circuit switch solutions were tightly coupling
their data plane with the control plane. The control plane
reconfigures the switches in response to traffic demands.
Such solutions cannot scale well.
Hence, RotorNet~\cite{rotornet-sigcomm17} proposes a fully decentralized
control plane solution using specialized hardware and a round-robin policy.
In addition, RotorNet can scale to 1000s of ports with 10 us switching delay.
The latest work, Sirius~\cite{sirius-sigcomm20} achieves nanosecond-level
switching time (3.84 ns for end-to-end reconfiguration).
Overall, the state-of-the-art circuit switch is able to
achieve fast nanosecond-level switching, works with decentralized control plane,
while still able to provide high port counts and consumes very little energy.
Given the foreseeable limitations of electrical packet switch,
circuit switch is gradually making its way into datacenters (e.g., Shoal~\cite{shoal-nsdi19} and dRedBox~\cite{dRedBox-DATE}).

\subsubsection{Feasibility}

Clearly, circuit switch is a good candidate to deploy networks in disaggregated datacenters.
As it supports high port counts hence able to accommodate the exploded number of devices.
Further, circuit switch is able to overcome the over-subscription problem while operating
with very low energy consumption compared to traditional packet switch.

To use circuit switch,
endpoints need to use specialized physical and link layer protocols,
with companion upper layer software~\cite{alex-thesis2020}.
This is relatively easier to achieve in servers with regular NICs
than the heterogeneous devices.
Past work has built an FPGA-based NIC~\cite{alex-thesis2020} for this purpose.
Similar to the programmable switch case,
we anticipate devices using circuit switch would incorporate customized network gear (e.g., a device, a chip, or IPs).

However, circuit switch is not able to
run any computation other than the scheduling algorithm (if any).
Unlike electrical packet switch which encapsulates the control complexity
within the switch, circuit switch \textit{exposes} the control complexity
to the rest of the network~\cite{rotornet-sigcomm17}.
Hence, circuit switch is not able to meet any other goals requiring computation.

\subsubsection{Summary}

Circuit switch has several appealing traits
such as high port counts, power efficient, and excellent scalability.
These make it a good candidate to \textit{build} disaggregated datacenters,
but not necessary for network consolidation.
Its lack of computation power is the key limitation.
If future circuit switch technologies are able to incorporate
any form of computation, then it would become one of the best choices
to build network resource pool.

\subsection{Coherent Fabrics}
\subsubsection{Primer}

In recent years, there are several industry proposals to build new interconnect
fabrics across endpoints in a server or in a rack.
They include Gen-Z~\cite{GenZ}, CXL~\cite{CXL}, OpenCAPI~\cite{OpenCAPI}, and CCIX~\cite{CCIX}.
These fabrics usually provide a universal memory interface and hardware-level memory/cache coherence across different endpoints.

Gen-Z~\cite{GenZ} is a new datacenter-scale fabric providing low latency
and high bandwidth accesses to remote resources.
It supports byte-addressable memory access, block memory access, and
accelerator-specific messaging interface.
Gen-Z is a full-stack solution, it specifies the physical layer,
link layer, network layer, transport layer, and above virtual memory interfaces.
It also defines its own routers and switches.
Each Gen-Z compliant device has a Gen-Z controller attached.
This controller translates user requests into Gen-Z requests
and sends to remote.

OpenCAPI~\cite{OpenCAPI}, CCIX~\cite{CCIX}, and CXL~\cite{CXL}
are all intra-server memory coherent interconnects.
OpenCAPI attaches CPUs to accelerators and I/O devices with minimal overhead.
It provides coherent memory interface across CPU and various devices.
CCIX provides a similar set of features.
On top of those capabilities, CXL further exposes a window directly
into the processor caching hierarchy.
All of them use the widely available PCIe physical and link layer to transmit data within the chassis.

Recently, CXL and Gen-Z consortium announced that
they will bridge their protocols and improve compatibility.
It is likely that in the near future,
CXL could be extended out beyond server boundary
and have coherent access to remote memory (or accelerators) via Gen-Z.
%Likewise, IBM has OpenCAPI has been extended to access disaggregated memory.
Those emerging coherent fabric protocols are gradually
making their ways into datacenters and being used for disaggregation purpose.

\subsubsection{Feasibility}

We will focus on Gen-Z as it is the only datacenter-scale fabric for now.
Unfortunately, there are no commercial Gen-Z products available, so we will
draw our discussion purely based on its latest specification~\cite{GenZ}.

In theory, Gen-Z's topology is able to support high port count.
It uses a combination of routers and switches, and they can be
customized for high port count.
However, they run at link layer or network layer,
thereby lacking any other computation power.
Hence, they are not able to support packet processing offload, nor resource consolidation.

As Gen-Z attaches its own controller to each device,
it is able to support any types of heterogeneous devices.
Also, it supports per-device customization.

\subsubsection{Summary}

Most of the emerging coherent fabrics are still under heavy development
(except OpenCAPI, which is already used in IBM Power series), no commercial
products are available. All of them have great potentials but with high uncertainties.

The main obstacle in adopting these fabrics is the requirement to
replace existing network infrastructure with new switches and new hardware network controllers (one at each endpoint).
These controllers cannot be easily managed or reconfigured, and they are not programmable.

Although these emerging coherent fabrics
are beneficial for traditional disaggregation on compute or memory,
they cannot satisfy the requirements for network disaggregation.

\subsection{Middleboxes and NFV}
\subsubsection{Primer}

Middleboxes, also known as hardware-based network appliances, originally
resided in specialized hardware boxes from various vendors.
They provide network functionalities such as firewall, packet filtering,
NAT, load balancing, and so on.
They are mostly black boxes for network operators.

In the early 2000s, people were still championing middleboxes~\cite{walfish-osdi04}.
But starting form early 2010s, as the workloads were rapidly changing,
people started questioning middleboxes' black box nature
and proposed software-centric middlebox deployment,
which resulted in consolidated middlebox architecture~\cite{comb-nsdi12}.
Around the same time, APLOMB~\cite{aplomb-sigcomm20} took a step further
by outsouring enterprise middleboxes processing all together to the cloud.
Despite middleboxes' usefulness and ubiquitousness, they come with
a set of problems, many of which arise from the fact that they are
hardware-based: they are costly, difficult to manage, and their
functionality is hard or impossible to change.

In response, also in the early 2010s,
the Network Function Virtualization (NFV) concept was proposed.
NFV advocates moving traditional network functions out of
proprietary middleboxes into virtualized software applications
that can be run on commodity, general purpose processors.

The past decade was a golden age for NFV.
Along the timeline, it is very clear what researchers were
focusing on at their time.
In the beginning, single-machine solutions such as ClickOS~\cite{clickos-nsdi14} arise just to enable virtualized
NF development. Not soon after, E2~\cite{e2} was proposed
to help distributed NF deployment. E2 deals with a set of typical
distributed system issues such as failure handling and scaling.
However, during 2016, despite all the promised benefits of NFV,
there has been little progress towards large-scale deployment.
One of the reasons is performance degradation due to virtualization.
Hence, NetBricks~\cite{netbricks} uses safe language Rust to avoid that.
In the same vein, Metron~\cite{metron-nsdi18} and ResQ~\cite{resq-nsdi18}
also propose low-level processor hacks to improve single machine efficiency.

It was clear that CPU will not be able to keep up with the
fast growing network speed.
As a result, there was a renewed interest in moving NFV
back to specialized hardware.
Notably, early work ClickNP~\cite{clickos-nsdi14} deploys NF to
FPGA-based programmable NICs.
FPGA provides massive cheap parallelism and is an ideal medium to run NFs.
Many work followed~\cite{flowblaze-nsdi19,panic-osdi20} and Microsoft
has deployed FPGA-based NF platform in their Azure cloud~\cite{azure-nsdi18}.

Over the years, the whole space moved from specialized middleboxes
to consolidated software-based NFs, and finally find their
way back to using specialized hardware.
But unlike original closed middleboxes, these new hardware (e.g., programmable switch or programmable NIC) are open, programmable,
and supported by the community at large.

\subsubsection{Feasibility}

Both middleboxes and NFV are not able to provide
high number of ports. As the former being a specialized box
and the latter mostly runs on commodity hardware.
Likewise, both of them cannot support heterogeneous endpoints
, nor can they support offloaded transports.
By design, both of them are able to run offloaded NFs. 

Similarly, both of them could support resource consolidation.
Prior work has tried to consolidate middleboxes~\cite{comb-nsdi12}.
As for NFV consolidation, TEA~\cite{tea-sigcomm20} accomplish that using programmable switches,
PANIC~\cite{panic-osdi20} uses programmable NIC,
and SNF~\cite{snf-socc20} uses a serverless framework.
For manageability, middleboxes' closed system nature makes it hard to manage and scale.
On the contrary, NFV is relatively easier to manage and has much more mature systems.

\subsubsection{Summary}

Overall, middlebox is no longer considered a good
solution for future datacenter development,
as its black box nature cannot fit in.
NFV systems are more diverse and open,
in which both hardware and software have open standard and
backed by the community.
For regular server datacenters,
NFV has already been consolidated and provided as a service~\cite{tea-sigcomm20,snf-socc20,panic-osdi20}.
However, none of these systems is able
to do so for disaggregated datacenters.


\subsection{Multi-host NIC}
\subsubsection{Primer}
Multi-host NIC~\cite{Intel-RedRockCanyon,Mellanox-Multihost}, as its name suggests, is a physical NIC shared by multiple hosts.
It connects to hosts via extended PCIe cables.
It appears as independent NIC to each host.
Internally, it can partition its uplink bandwidth among connected hosts,
follow a certain policy (e.g., fair partition).
To host, it is no different than using a normal exclusive NIC,
hence each host runs its own network stack.
Multi-host NIC is proposed to consolidate network resources
in virtualized environment, but it has never been widely deployed.

\subsubsection{Feasibility}

Multi-host NICs reduces the number of ports ToR switch needs,
and it consolidates traditional NIC functionalities.
However, it still requires each end host to run its transport and network
functions in software.
Also, multi-host NICs do not support programmability or rapid on-field upgradability.

\subsubsection{Summary}

Multi-host NIC can meet the port count goal.
Since it is using a more general PCIe interface, it could
potentially support heterogeneous devices.
Although it is designed to consolidate resources,
it only does so for physical and link layer resources.
Higher level protocols such as transports and network functions
cannot be offloaded to the multi-host NIC.

\subsection{Summary of Network Device Review}

We have now reviewed programmable switch,
circuit switch, coherent fabrics, middleboxes, NFV, and multi-host NIC.
Table~\ref{tabel-related-work} summarizes whether each system can meet
the goals for network disaggregation and consolidation.
Unfortunately, none of them can make the cut.
Programmable switch and NFV are the closest solution for regular-server
datacenter, but they are not able to solve exploded port count and
to accommodate heterogeneous devices. Coherent fabrics and multi-host NIC
are able to meet specific goals for disaggregated datacenter, but
lack the support for computation offload and consolidation.

We find that programmable switch, circuit switch, and NFV
all made their first appearance in the early 2010s.
The past decade has witnessed their rapid growth:
programmable switch and NFV are widely deployed in
datacenters~\cite{hpcc-sigcomm19,azure-nsdi18};
though circuit switch has received less adoption,
it is gaining its momentum~\cite{dRedBox-DATE,sirius-sigcomm20}.

\textit{But are they the right devices to build next decade's datacenter network?}
Our answer is no.
We think the next-generation datacenter,
including regular-server and disaggregated,
should use a disaggregated and consolidated network,
for the reasons in Section~\ref{sec:motivation}.
But none of the above device is able to meet our goals.
As a result, we propose a new device called \sysname,
which meets all the goals in Table~\ref{tabel-related-work}
\fi
\section{Conclusion}
\label{sec:snic:conclude}

We propose network disaggregation and consolidation by building SuperNIC, a new networking device specifically for a disaggregated datacenter.
Our FPGA prototype demonstrates the performance and cost benefits of \snic.
Our experience also reveals many new challenges in a new networking design space that could guide future researchers.

\section{Acknowledgments}
Chapter 5, in part, has been submitted for publication of the material as it may appear in Yizhou Shan, Will Lin, Ryan Kosta, Arvind Krishnamurthy, Yiying Zhang, ``Disaggregating and Consolidating Network Functionalities with SuperNIC'', \textit{arXiv, 2022}. The dissertation author was the primary investigator and author of this paper.
\fi
\chapter{LegoOS}


\section{Disaggregate Hardware Resource}
\label{sec:lego:motivation}

{    
\begin{figure}[h]
\begin{subfigure}{3in}
    \begin{center}
    \centerline{\includegraphics[width=3in]{lego/Figures/g_plot_google_util.pdf}}
    \caption[Google Cluster.]{Google Cluster.}
    \label{fig-googleutil}    
    \end{center}
\end{subfigure}
\begin{subfigure}{3in}
    \begin{center}    
    \centerline{\includegraphics[width=3in]{lego/Figures/g_plot_ali_util.pdf}}    
    \caption[Alibaba Cluster.]{Alibaba Cluster.}
    \label{fig-aliutil}
    \end{center}    
\end{subfigure}
\caption[Data center resource utilization.]{Data center resource utilization.}
\label{fig-resource-anal}
\end{figure}
}
{
\begin{figure*}[t]
\begin{subfigure}{1.7in}
\begin{center}
\centerline{\includegraphics[width=1.7in]{lego/Figures/monolithic-arch.pdf}}
\caption[Monolithic OS.]{OSes Designed for Monolithic Servers.}
\label{fig-monolithic}
\end{center}
\end{subfigure}
\begin{minipage}{0.05in}
\hspace{0.05in}
\end{minipage}
\begin{subfigure}{1.8in}
\begin{center}
\centerline{\includegraphics[width=1.8in]{lego/Figures/multikernel-arch.pdf}}
\caption[Multikernel Architecture.]{Multi-kernel Architecture. \small{P-NIC: programmable NIC.}}
\label{fig-multikernel}
\end{center}
\end{subfigure}
\begin{minipage}{0.05in}
\hspace{0.05in}
\end{minipage}
\begin{subfigure}{2.5in}
\begin{center}
\centerline{\includegraphics[width=2.6in]{lego/Figures/lego-arch.pdf}}
\caption[Splitkernel Architecture.]{Splitkernel Architecture.}
\label{fig-splitkernel}
\end{center}
\end{subfigure}
\caption[Operating System Architecture.]{Operating System Architecture.}
\end{figure*}
}

This section
motivates the hardware resource disaggregation architecture
and discusses the challenges in managing disaggregated hardware.

\subsection{Limitations of Monolithic Servers}
\label{sec:lego:monolimit}
A monolithic server has been the unit of deployment and operation in datacenters for decades.
This long-standing {\em server-centric} architecture has several key limitations.

\noindent{\textit{\uline{Inefficient resource utilization.}}}
With a server being the physical boundary of resource allocation, 
it is difficult to fully utilize all resources in a datacenter~\cite{Barroso-COMPUTER,Quasar-ASPLOS,PowerNap}.
We analyzed two production cluster traces: a 29-day Google one~\cite{GoogleTrace}
and a 12-hour Alibaba one~\cite{AliTrace}.
Figure~\ref{fig-resource-anal} plots the aggregated CPU and memory utilization in the two clusters.
For both clusters, only around half of the CPU and memory are utilized.
Interestingly,
a significant amount of jobs are being evicted at the same time in these traces
(\eg, evicting low-priority jobs to make room for high-priority ones~\cite{Borg}).
One of the main reasons for resource under-utilization in these production clusters is 
the constraint that CPU and memory for a job have to be allocated from 
the same physical machine.

\noindent{\textit{\uline{Poor hardware elasticity.}}}
It is difficult to add, move, remove, or reconfigure hardware components
after they have been installed in a monolithic server~\cite{FB-Wedge100}. %, and
Because of this rigidity, datacenter owners have to plan out server configurations in advance.
However, with today's speed of change in application requirements, such plans have to be adjusted frequently,
and when changes happen, it often comes with waste in existing server hardware.

\noindent{\textit{\uline{Coarse failure domain.}}}
The failure unit of monolithic servers is coarse.
When a hardware component within a server fails, %(\eg, processor, memory chip, RAID controller), 
the whole server is often unusable and applications running on it can all crash.
Previous analysis~\cite{Failure-Disk-FAST07} found that motherboard, memory, CPU, power supply failures account for 
50\% to 82\% of hardware failures in a server.
Unfortunately, monolithic servers cannot continue to operate when any of these devices fail.

\noindent{\textit{\uline{Bad support for heterogeneity.}}}
Driven by application needs, new hardware technologies are finding their ways into modern datacenters~\cite{sigarch-dc}.
Datacenters no longer host only commodity servers with CPU, DRAM, and hard disks. 
They include non-traditional and specialized hardware like GPGPU~\cite{GPU-google,GPU-aws}, 
TPU~\cite{TPU}, 
DPU~\cite{DPU},
FPGA~\cite{Putnam14-FPGA,Amazon-F1}, %,SmartNIC-nsdi18},
non-volatile memory~\cite{Intel3DXpoint}, %,facebook-eurosys18},
and NVMe-based SSDs~\cite{everspin}.
The monolithic server model tightly couples hardware devices with each other and with a motherboard.
As a result, making new hardware devices work with existing servers is a painful and lengthy process~\cite{Putnam14-FPGA}.
%The current practice of making new hardware work is not only slow but also expensive.
Mover, datacenters often need to purchase new servers to host certain hardware.
Other parts of the new servers can go underutilized 
and old servers need to retire to make room for new ones.

\subsection{Hardware Resource Disaggregation}
The server-centric architecture is a bad fit for the fast-changing datacenter hardware, software, and cost needs.
There is an emerging interest in utilizing resources beyond a local machine~\cite{Gao16-OSDI},
such as distributed memory~\cite{Dragojevic14-FaRM,Nelson15-ATC,Aguilera17-SOCC,Novakovic16-SOCC} and network swapping~\cite{GU17-NSDI}. 
These solutions improve resource utilization over traditional systems.
However, they cannot solve all the issues of monolithic servers (\eg, the last three issues in \S\ref{sec:lego:monolimit}), 
since their hardware model is still a monolithic one.
To fully support the growing heterogeneity in hardware and to provide elasticity and flexibility at the hardware level, 
we should {\em break the monolithic server model.}% into flexible resource components.

We envision a {\em hardware resource disaggregation} architecture 
where hardware resources in traditional servers are disseminated into network-attached {\em hardware components}.
Each component has a controller and a network interface,
can operate on its own,
and is an {\em independent, failure-isolated} entity.

The disaggregated approach largely increases the flexibility of a datacenter.
Applications can freely use resources from any hardware component,
which makes resource allocation easy and efficient.
Different types of hardware resources can {\em scale independently}.
It is easy to add, remove, or reconfigure components.
New types of hardware components can easily be deployed in a datacenter ---
by simply enabling the hardware to talk to the network and adding a new network link to connect it.
Finally, hardware resource disaggregation enables fine-grain failure isolation, % because of decomposed hardware resources.
since one component failure will not affect the rest of a cluster.

Three hardware trends are making resource disaggregation feasible in datacenters.
First, network speed has grown by more than an order of magnitude and has become more scalable in the past decade % faster both in bandwidth and latency
with new technologies like Remote Direct Memory Access ({\it RDMA})~\cite{ibverbs} 
and new topologies and switches~\cite{FireBox-FASTKeynote,costa15-r2c2,Costa-WRSC14},
enabling fast accesses of hardware components that are disaggregated across the network.
InfiniBand will soon reach 200Gbps and sub-600 nanosecond speed~\cite{Mellanox-ConnectX6-IB},
being only 2\x\ to 4\x\ slower than main memory bus in bandwidth.
With main memory bus facing a bandwidth wall~\cite{BW-Wall-ISCA09},
future network bandwidth (at line rate) is even projected to exceed local DRAM bandwidth~\cite{CacheCloud-hotcloud18}.

Second, network interfaces are moving closer to hardware components,
with technologies like Intel OmniPath~\cite{OmniPath},
RDMA~\cite{ibverbs},
and NVMe over Fabrics~\cite{NVMe-fabrics-Inteltalk,NVMe-fabrics}.
As a result, hardware devices will be able to access network directly 
without the need to attach any processors. 

Finally, hardware devices are incorporating more processing power~\cite{Ahn15-PIM,Bojnordi12,Mellanox-SmartNIC,Mellanox-SmartNIC2,Agilio-SmartNIC,Junwhan-ISCA17},
allowing application and OS logics to be offloaded to hardware~\cite{Willow,Kaufmann16-ASPLOS}.
On-device processing power will enable system software to manage disaggregated hardware components locally.

With these hardware trends and the limitations of monolithic servers,
we believe that future datacenters will be able to largely benefit from hardware resource disaggregation.
In fact, there have already been several initial hardware proposals in resource disaggregation~\cite{OCP},
including disaggregated memory~\cite{Lim09-disaggregate,Scaleout-numa,Nitu18-EUROSYS}, 
disaggregated flash~\cite{FlashDisaggregation,ReFlex},
%new power state for disaggregated memory~\cite{Nitu18-EUROSYS},
Intel Rack-Scale System~\cite{IntelRackScale}, 
HP ``The Machine''~\cite{HP-TheMachine,HP-MemoryOS}, 
IBM Composable System~\cite{IBM-Composable},
and Berkeley Firebox~\cite{FireBox-FASTKeynote}.

\subsection{OSes for Resource Disaggregation}
Despite various benefits hardware resource disaggregation promises, 
it is still unclear how to manage or utilize disaggregated hardware in a datacenter.
Unfortunately, existing OSes and distributed systems cannot work well with this new architecture.
Single-node OSes like Linux view a server as the unit of management and assume all hardware components are local (Figure~\ref{fig-monolithic}).
A potential approach is to run these OSes on processors
and access memory, storage, and other hardware resources remotely.
Recent disaggregated systems like soNUMA~\cite{Scaleout-numa} take this approach.
However, this approach incurs high network latency and bandwidth consumption with remote device management,
misses the opportunity of exploiting device-local computation power,
and makes processors the single point of failure.

Multi-kernel solutions~\cite{Baumann-SOSP09,Barrelfish-DC,Helios-SOSP,fos-SOCC,Hive-SOSP} (Figure~\ref{fig-multikernel}) 
view different cores, processors, or programmable devices within a server separately 
by running a kernel on each core/device and using message passing to communicate across kernels.
These kernels still run in a single server and all access some common hardware resources in the server like memory and the network interface.
Moreover, they do not manage distributed resources or handle failures in a disaggregated cluster. 

There have been various distributed OS proposals,
most of which date decades back~\cite{Amoeba-Experience,Sprite,MOSIX}. %,V-System,Accent-SOSP,DEMOS-SOSP,Charlotte}.
Most of these distributed OSes manage a set of monolithic servers
instead of hardware components.

Hardware resource disaggregation is fundamentally different from the traditional monolithic server model.
A complete disaggregation of processor, memory, and storage 
means that when managing one of them, there will be no local accesses to the other two.
For example, processors will have no local memory or storage to store user or kernel data.
%Memory and storage components will only have limited processing power. %not have no local memory to serve as cache.
An OS also needs to manage distributed hardware resource and handle hardware component failure.
We summarize the following key challenges in building an OS for resource disaggregation,
some of which have previously been identified~\cite{HP-MemoryOS}.

\begin{itemize}
\item How to deliver good performance when application execution involves the access of network-partitioned disaggregated hardware
and current network is still slower than local buses?

\item How to locally manage individual hardware components with limited hardware resources?

%\item How to communicate across components?

\item How to manage distributed hardware resources?

\item How to handle a component failure without affecting other components or running applications?

\item What abstraction should be exposed to users and how to support existing datacenter applications?

\end{itemize}

Instead of retrofitting existing OSes to confront these challenges,
we take the approach of designing a new OS architecture from the ground up for hardware resource disaggregation.


\chapter{Clio: A Hardware-Software Co-Designed Disaggregated Memory}


\section{Introduction}
\label{sec:clio:introduction}

Modern datacenter applications like graph computing, data analytics, and deep learning have an increasing demand for access to large amounts of memory~\cite{FastSwap}.
Unfortunately, servers are facing {\em memory capacity walls} because of pin, space, and power limitations~\cite{HP-MemoryEvol,ITRS14,MemoryWall95}.
Going forward, it is imperative for datacenters to seek solutions that can go beyond what a (local) machine can offer, \ie, using remote memory.
At the same time, datacenters are seeing the needs from management and resource utilization perspectives
to {\em disaggregate} resources~\cite{Ali-SinglesDay,SnowFlake-NSDI20,FB1}\textemdash separating hardware resources into different network-attached pools 
that can be scaled and managed independently.
These real needs have pushed the idea of memory disaggregation ({\em \md} for short):
organizing computation and memory resources as two separate network-attached
pools, one with compute nodes ({\em CN}s) and one with memory nodes ({\em MN}s).

So far, \md\ researches have all taken one of two approaches: 
building/emulating \MN{}s using regular servers~\cite{AIFM,InfiniSwap,FastSwap,Shan18-OSDI,zombieland} 
or using raw memory devices with no processing power~\cite{Tsai20-ATC,Lim09-disaggregate,Lim12-HPCA,HP-TheMachine}.
%,ATC21Paper
%\textit{Is it possible to build \MN{}s without a server}, as promised by the original proposal of \md~\cite{Lim09-disaggregate,Shan18-OSDI}?
The fundamental issues of server-based approaches such as RDMA-based systems are the monetary and energy cost of a host server and the inherent performance and scalability limitations caused by the way NICs interact with the host server's virtual memory system.
Raw-device-based solutions have low costs.
However, they introduce performance, security, and management problems
because when \MN{}s have no processing power, all the data and control planes have to be handled at \CN{}s~\cite{Tsai20-ATC}.
%\MN{}s become too ``dumb'' and low level when removing its processing power altogether.
%overhead of root cause why RDMA is unfit for \md\ is that it is designed to work around a host server,
%and traditional servers' virtual memory system is not designed for \md.
%Contrary to server-based \md\ solutions, \pdm\ is another extreme where there is no processing at all at \MN{}s.
%The root cause of \pdm's various performance, security, and management problems is exactly this: 
%\MN{}s are too ``dumb'' and low level.

%From the above discussion, we find that both server-based \md\ and raw physical \md\ have their limitations.
Server-based \MN{}s and \MN{}s with no processing power are two extreme approaches of building \MN{}s.
We seek a sweet spot in the middle by proposing a hardware-based \md\ solution that has the right amount of processing power at \MN{}s.
%for the right type of \MN{} management system.
%To mitigate RDMA's various issues for \md, one could try to improve RDMA's hardware design or add a software layer to work around RDMA's problems.
%This paper takes a different approach by starting
Furthermore, we take a clean-slate approach by starting from the requirements of \md\
and designing a \md-native system.
%LegoOS is the only existing work that also proposes a .
%Unfortunately, LegoOS still uses server software and the RDMA network to emulate its hardware design, leaving all the research questions we ask in this paper unanswered.


We built {\em \sys}\footnote{Clio is the daughter of Mnemosyne, the Greek goddess of memory.}, a hardware-based disaggregated memory system.
%focusing on the hardware-based virtual memory system and network stack.
\sys\ includes a \CN-side user-space library called {\em \syslib}
and a new hardware-based \MN\ device called {\em \sysboard}.
Multiple application processes running on different \CN{}s can allocate memory from the same \sysboard, with each process having its own {\em remote virtual memory address space}.
Furthermore, one remote virtual memory address space can span multiple \sysboard{}s.
Applications can perform byte-granularity remote memory read/write and use \sys's synchronization primitives for synchronizing concurrent accesses to shared remote memory .

A key research question in designing \sys\ is \textit{\textbf{how to use limited hardware resources to achieve 100\Gbps, microsecond-level average and tail latency for TBs of memory and thousands of concurrent clients?}}
These goals are important and unique for \md.
A good \md\ solution should reduce the total CapEx and OpEx costs compared to traditional non-disaggregated systems and thus cannot afford to use large amounts of hardware resources at \MN{}s.
Meanwhile, remote memory accesses should have high throughput and low average and tail latency, because even after caching data at \CN-local memory, there can still be fairly frequent accesses to \MN{}s and the overall application performance can be impacted if they are slow~\cite{disagg-osdi16}.
Finally, unlike traditional single-server memory, a disaggregated \MN\ should allow many \CN{}s to store large amounts of data so that we only need a few of them to reduce costs and connection points in a cluster.
%cost, performance, and scalability goals 
How to achieve each of the above cost, performance, and scalability goals {\em individually} is relatively well understood.
%: using hardware to achieve high performance, using more hardware to support larger scales, and using fewer hardware to reduce costs.
However, achieving all these seemingly conflicting goals {\em simultaneously} is hard and previously unexplored.

Our main idea is to \textbf{\textit{eliminate state from the \MN\ hardware}}.
Here, we overload the term ``state elimination'' with two meanings: 1) the \MN\ can treat each of its incoming requests in isolation even if requests that the client issues can sometimes be inter-dependent, 
and 2) the \MN\ hardware does not store metadata or deals with it.
Without remembering previous requests or storing metadata, an \MN\ would only need a tiny amount of on-chip memory that does not grow with more clients, thereby {\em saving monetary and energy cost} and achieving {\em great scalability}.
Moreover, without state, the hardware pipeline can be made {\em smooth} and {\em performance deterministic}.
A smooth pipeline means that the pipeline does not stall, which is only possible if requests do not need to wait for each other.
It can then take one incoming data unit from the network every fixed number of cycles (1 cycle in our implementation), achieving constantly {\em high throughput}.
A performance-deterministic pipeline means that the hardware processing does not need to wait for any slower metadata operations and thus has {\em bounded tail latency}.
%When and only when complete elimination is impossible, we . 

Effective as it is, can we really eliminate state from \MN\ hardware? 
First, as with any memory systems, users of a disaggregate memory system expect it to deliver certain reliability and consistency guarantees (\eg, a successful write should have all its data written to remote memory, a read should not see the intermediate state of a write, etc.). 
Implementing these guarantees requires proper ordering among requests and involves state even on a single server. 
The network separation of disaggregated memory would only make matters more complicated.
Second, quite a few memory operations involve metadata, and they too need to be supported by disaggregated memory.
Finally, many memory and network functionalities are traditionally associated with a client process and involve per-process/client metadata (\eg, one page table per process, one connection per client, etc.). 
Overcoming these challenges require the re-design of traditional memory and network systems.
%How can we overcome these challenges to eliminate state from \MN\ hardware?
%The network separation of \MN{}s and \CN{}s means that packets could potentially be dropped or reordered. 
%Traditional 
%\textbf{\textit{How can we eliminate state from \MN\ hardware when remote memory operations are stateful by nature?}}
%For example, memory allocation needs to check the state of space availability; reliable memory operations over the network requires state for ordering and re-transmission.

Our first approach is to separate the metadata/control plane and the data plane, with the former running as software on a low-power ARM-based SoC at \MN\ and the latter in hardware at \MN. 
Metadata operations like memory allocation usually need more memory but are rarer (thus not as performance critical) compared to data operations.
A low-power SoC's computation speed and its local DRAM are sufficient for metadata operations.
On the other hand, data operations (\ie, all memory accesses) should be fast and are best handled purely in hardware. 
Even though the separation of data and control plane is a common technique that has been applied in many areas~\cite{4d-sdn,netvirt,arrakis}, a separation of memory system control and data planes has not been explored before and is not easy, as we will show in this paper.

Our second approach is to re-design the memory and networking data plane so that most state can be managed only at the \CN\ side.
Our observation here is that the \MN{} only {\em responds} to memory requests but never {\em initiates} any.
This \CN-request-\MN-respond model allows us to use a custom, connection-less reliable transport protocol that implements almost all transport-layer services and state at \CN{}s, allowing \MN{}s to be free from traditional transport-layer processing.
%optimized for transferring memory requests and responses between CNs and MNs, which is much simpler and more scalable than RDMA, thanks to its stateless operation at MNs.  
Specifically, our transport protocol manages request IDs, transport logic, retransmission buffer, congestion, and incast control all at \CN{}s. It provides 
reliability by ordering and retrying an entire memory request at the \CN\ side.
%This \CN-request-\MN-respond model allows us to use a connection-less, RPC-based network protocol between \CN\ and \MN\ and build the \MN\ with only the physical and link network layers (\ie, no transport layer or above).
%a ``transport-less'' \MN\ network design.
%We then manage request IDs, transport logic, retransmission buffer, congestion and incast control all at \CN{}s.
%Furthermore, we provide reliability by ordering and retrying an entire memory request at the \CN\ side.
As a result, the \MN{} does not need to worry about per-request state or inter-request ordering and only needs a tiny amount of hardware resources which do not grow with the number of clients.
 
With the above two approaches, the hardware can be largely simplified and thus cheaper, faster, and more scalable.
However, we found that \textit{\textbf{complete state elimination at \MN{}s is neither feasible nor ideal}}. To ensure correctness, the \MN\ has to maintain some state (\eg, to deal with non-idempotent operations). To ensure good data-plane performance, not every operation that involves state should be moved to the low-power SoC or to \CN{}s.
Thus, our approach is to eliminate as much state as we can without affecting performance or correctness and to carefully design the remaining state so that it causes small and bounded space and performance overhead.

For example, we perform paging-based virtual-to-physical memory address mapping and access permission checking at the \MN\ hardware pipeline, as these operations are needed for every data access.
Page table is a kind of state that could potentially cause performance and scalability issues but has to be accessed in the data path.
We propose a new overflow-free, hash-based page table design where 1) all page table lookups have bounded and low latency (at most one DRAM access time in our implementation), and 2) the total size of all page table entries does not grow with the number of client processes.
As a result, even though we cannot eliminate page table from the \MN\ hardware, we can still meet our cost, performance, or scalability requirements.

Another data-plane operation that involves metadata is page fault handling, which is a relatively common operation because we allocate physical memory on demand.
Today's page fault handling process is slow and involves metadata for physical memory allocation.
We propose a new mechanism to handle page faults in hardware and finish all the handling within bounded hardware cycles.
We make page fault handling performance deterministic by moving  physical memory allocation operations to software running at the SoC.
We further move these allocation operations off the performance-critical path by pre-generating free physical pages to a fix-sized buffer that the hardware pipeline can pull when handling page faults.

We prototyped \sysboard\ with a small set of Xilinx ZCU106 MPSoC FPGA boards~\cite{ZCU106} and built three applications using \sys:
a FaaS-style image compression utility, a radix-tree index, and a key-value store.
%We prototyped \sys's memory device with FPGA (\sysboard).
%\sys\ achieves high throughput (100\Gbps\ with FPGA prototype), low (tail) latency (XXX\mus\ end-to-end MTU round trip latency), 
%low cost (XXX\x\ energy saving), %and great extendibility,
%and \sys\ scales well with XXX. %eliminates {\em all} scalability bottlenecks in both its memory and network systems.
We compared \sys\ with native RDMA, two RDMA-based disaggregated/remote memory systems~\cite{Tsai20-ATC,Kalia14-RDMAKV}, 
a software emulation of hardware-based disaggregated memory~\cite{Shan18-OSDI},
%an FPGA-based RDMA implementation~\cite{StRoM}, 
and a software-based SmartNIC~\cite{BlueField}.
\sys\ scales much better and has orders of magnitude lower tail latency than RDMA, 
while achieving similar throughput and median latency as RDMA (even with the slower FPGA frequency in our prototype).
\sys\ has 1.1\x\ to 3.4\x\ energy saving compared to CPU-based and SmartNIC-based disaggregated memory systems 
and is 2.7\x\ faster than SmartNIC solutions. 
%From our efforts of building \sys, we found that complete state elimination at \MN{}s is neither feasible nor necessary. Instead, we eliminate as much state as we can without affecting performance or correctness and carefully design the remaining state so that it causes small and bounded space and performance overhead.

\sys\ is publicly available at \url{https://github.com/WukLab/Clio}.
%We will make \sys\ publicly available upon this paper's publication.



\if 0
A \sysboard\ consists of three main components: 1) a hardware chip that integrates a thin network stack and a virtual memory system to handle data requests (the {\em fast path}), 2) an ARM processor that runs software to handle metadata requests and background tasks (the {\em slow path}), and 3) an FPGA that hosts application computation offloading (the {\em extend path}).

In building \sys, we explore new requirements, challenges, and benefits of \md.
Specifically, we answer three important research questions.

First, \textbf{how does the design and implementation of a
dedicated hardware \MN\ differ from server and programmable NIC
designs?}
%To protect the accesses from different applications to the memory in an \MN\ and to make such accesses flexible, 
%we should use a virtual memory system to check and map virtual memory addresses to physical addresses.
Current \md\ solutions rely on a host server (its OS and MMU) to provide a virtual memory system so that accesses to the memory are protected and flexible. 
Using a whole server just for the virtual memory system is overkill and unnecessarily adds monetary and energy costs to \md. %consumes too much power.
Another possibility is to use a low-power processor (\eg, ARM) in a SmartNIC to run the virtual memory system~\cite{iPipe}. 
However, doing so has high performance impact mainly because the virtual memory system is on a separate chip from the NIC.
%However, doing so has performance overhead, both because the virtual memory system is on a separate chip from the NIC
%and because the virtual memory system is not designed for \md.
%To date, there has been no attempt
%\yizhou{I think here we should emphasize that: MNs emulated using host server has cost and power waste while the SmartNIC-based ones have perf issue. In all, there is no ideal solution for building real server-less MNs. Hence we took a ...}
Overall, server-based approaches have cost overheads while SmartNIC solutions have performance overheads.
We took a clean-slate approach by building a hardware-based virtual memory system that is integrated with a customized hardware network stack, 
both of which are designed specifically for handling virtual memory requests sent over the network.

Second, \textbf{how can a low-cost \MN\ host TBs of memory and support thousands of concurrent application processes?}
Different from traditional (local) memory, an \MN\ is intended to be shared by many applications running at different \CN{}s,
and the more applications it can support, the more efficiently its memory can be utilized.
Thus, we aim to have each \MN\ host TBs of memory for thousands of concurrent applications processes.
However, a hardware design is constrained by the limited resources in a hardware chip such as on-chip memory.
Compared to traditional software-based virtual memory systems, 
how can an \MN\ use orders of magnitude less resources while achieving orders of magnitude higher scalability?
%which raises the problem of how to manage the scalability we target in hardware.
%is especially accute in an \hdm\ because of its low cost requirements.
Current solutions like RDMA NICs swap metadata between NIC's on-chip memory and host server memory,
which comes with performance overhead (4\x\ compared with when metadata is in the NIC memory~\cite{Pythia}).
%RDMA improves scalability by adding more hardware resources to cache more metadata/states), which comes with increasing monetary and energy costs.

Our clean-slate approach is to carefully examine each virtual-memory and networking task 
and to redesign them to 1) eliminate states and metadata whenever possible (\eg, by minimizing indirection),
2) move complex but non-performance-critical states, metadata, and tasks to the software slow path,
3) shift functionalities to the \CN\ (\syslib) to reduce \MN's complexity 
(\eg, our network transport runs at \syslib, and \MN\ is ``transport-less''),
and 4) design bounded-size, inherently scalable data structures.
%For the remaining functionalities that still need caching, we redesign them to guarantee good cache-miss performance.
As a result, each \MN\ (\sysboard) could support TBs of memory and thousands of application processes with only 1.5\MB\ on-chip memory.
%\yizhou{On top of network scalability, I think its also worth explaining how we are able to support TBs of memory. Because this reminds me of O(1) memory issues. With TBs of memory, the memory related metadata and OPs increased as well. Our pros: our fast path use huge page + hashtable-based pgtable, bounded latency. Our cons: our buddy allocator running in ARM is still prone to this huge memory issue.}
%For certain functionalities, we avoid states and metadata alltogether by re-designing the functionalities.
%For functionalities that could be and move the ones that can be moved to the \CN\ side.
%and not relying on cache for good performance.

Third, \textbf{how to minimize tail latency in a \md\ system}?
%\textbf{is it possible to minimize tail latency by processing all read/write requests deterministically?} 
Tail latency is important in datacenters especially for workloads that have large fanouts (\eg, Spark jobs).
Although much effort has focused on improving the network and core scheduling for low tail latency~\cite{nanoPU,Shenango,Shinjuku,ZygOS,RPCValet},
the memory system has largely been overlooked.
However, the (remote) memory system is what contributes to extreme long tails in a \md\ system.
For example, RDMA's round-trip latency is around 1--2\mus\ in the common case,
but its tail could be as long as 16.8\ms\ (Figure~\ref{fig-miss-hit} and \S\ref{sec:clio:rdma}). % (when there is a page fault).
%The main reason for RDMA's various tails is its reliance on the host virtual memory system, which is not designed for \md\ or for low tail latency.

We reexamine traditional memory system
%from the tail latency and perspectives 
and propose a set of novel mechanisms to bound \sys's tail latency.
Our core idea is to include {\em all} the functionalities that are needed to fulfill all types of data requests in one hardware pipeline
and to make this hardware pipeline {\em performance deterministic}.
This pipeline takes one incoming data unit every cycle (\ie, no pipeline stalls) and completes every request in a fixed number of cycles,
which yields 100\Gbps\ throughput, 2.5\mus\ at median and 3.2\mus\ at 99-percentile end-to-end latency (Figure~\ref{fig-tail-latency}).
Two major technical hurdles in achieving this performance are to perform page table lookups and to handle page faults in a bounded, short time period.  
For the former, we propose a new {\em overflow-free} hash-based page table that bounds all page table lookups to {\em at most one DRAM access} (instead of the long page table walk in a traditional CPU architecture).
For the latter, we propose a new mechanism to handle page faults in hardware with bounded cycles (instead of the costly process of interrupting and handling page faults in the OS).

\fi

%These mechanisms include 1) handling page faults in hardware,
%2) d
%RDMA’s way is to pre-reserve memory and to use large on-device cache. But cache can’t always work, and reserving memory results in inefficient memory utilization.
%Our solution is to allocate physical memory on-demand and to make the cache miss path fast by handling page faults in hardware and by designing a new hash-based page table. We also carefully designed our hardware pipeline to have deterministic, bounded latency.




\if 0
%Unlike traditional computer
%modern computers that manage and use memory at the same place,
% where memory is managed and used at the same place, % at the server CPU and MMU,
Unlike traditional computers that manage and use memory at the same place,
there are many ways of building a disaggregated memory system when memory is far from computation.
For example, should memory be managed at \CN{}s or \MN{}s?
Should \MN{}s be full server boxes, raw DRAM chips, or something in between?
Should computation only happen at \CN{}s or can some of it happen at \MN{}s as well?

Memory disaggregation also poses new requirements in performance, scalability, cost, and deployment.
Like today's storage disaggregation solutions~\cite{AMAZON-S3,SnowFlake-NSDI20,Pangu}, 
the disaggregated memory pool is intended to host large amounts of memory
that can be shared by many clients.
At the same time, to be an appealing solution that can be adopted by today's datacenters, 
memory disaggregation should improve performance per dollar and be easy to deploy and manage.
%and have an easy way 
%At the same time, it is desirable to maintain or decrease data-center costs
%when introducing memory disaggregation.
%Thus, each \MN\ 
%Memory disaggregation presents new requirements in performance, cost
%Existing approaches in building disaggregated-memory systems take three main forms,
%and they have different choices of the above questions and their own limitations.

Existing memory disaggregation solutions have taken two main approaches,
neither of which can meet all the above requirements.
The first type treats \MN{}s as raw, physical memory 
and manage it at \CN{}s~\cite{Genz-citation,CXL-citation,Tsai20-ATC}.
Although \MN{}s are cheap to build, using and managing them at \CN{}s
is slow because of the network round trips needed to access \MN{}s.
%which causes high communication overhead and security concerns, 

The second type accesses \MN{}s over a virtual memory interface
and relies on a host server to virtualize and manage memory at \MN{}s. 
RDMA-based solutions~\cite{FaRM,FastSwap,InfiniSwap,StRoM,Kalia14-RDMAKV,Aguilera18-ATC} all fall into this category.
%The main issue with this approach is the involvement of host 
Using a whole server to manage memory is overkill and unnecessarily consumes too much power.
Moreover, with today's server architecture, a NIC needs to either suffer from a slow crossover to the host memory and/or CPU for 
fetching virtual memory metadata and executing virtual memory tasks like page fault handling~\cite{NICPageFaultArchPaper},
or use limited on-NIC memory to cache metadata, which results in serious scalability issues~\cite{FaRM,Tsai17-SOSP}.
%or rely on a host server to run a virtual memory system (RDMA- and messaging-based),
%which has tail latency, scalability, and cost issues.

We argue that memory disaggregation can and should achieve all the performance, scalability, cost, and deployment requirements.
The solution is to virtualize and manage disaggregated memory in hardware and build \MN\ as a stand-alone device,
as doing so avoids the energy cost of a power-hungry server CPU and the performance overhead to cross multiple devices in a server.
%can achieve better scalability and performance with hardware pipelining and parallelism.
We propose \textit{\sys} (\textit{Disaggregated Virtual Memory Access}), 
a cross-layer, software/hardware-codesign solution that %centers around 
%allows both applications running at \CN{}s and computation offloads running at \MN\ hardware
allows applications to access disaggregated memory through a virtual memory interface
and at the same time, allows \MN{}s to be built without a server box.
%\sys\ includes a memory system, a network system, and a framework for computation offloading,
%all of which are catered towards the hardware-based disaggregated virtual memory model and memory disaggregation's unique requirements.
The \sys\ \MN\ is a new hardware device (\textit{\sysboard}) that we designed from scratch and prototyped with FPGA. %a Xilinx FPGA-based MPSoC board.
The rest of \sys\ uses common hardware and network infrastructure available in today's datacenters with a user-space library (\textit{\syslib}) at each \CN,
making \sys\ easy to be adopted.
The key piece of \sys\ is a hardware-based virtual memory system that runs entirely in \sysboard.
It achieves the same functionality as traditional, software-based virtual memory system,
but with very different approaches --- new designs that fit the hardware model and memory disaggregation.
\sys\ also includes a new network system that we co-design with \sys's memory system
by exploiting memory disaggregation's {\em asymmetric} nature.
Finally, to mitigate the network communication costs between \CN{}s and \MN{}s,
\sys\ creates a framework for applications to safely and easily offload their computation.
% support computation offloading (from \CN{}s to \MN{}s),
%both catered towards the hardware-based virtual disaggregated memory model.

\sys\ offers three ways of using disaggregated memory.
First, applications or a system layer (\eg, a swap system~\cite{InfiniSwap,FastSwap}) running at \CN{}s can allocate (remote) virtual memory spaces and read/write data in them.
In addition, \sys\ offers a set of extended APIs like synchronization primitives and pointer chasing, % for applications running at \CN{}s to use,
and users can also write and deploy new extended APIs at \MN{}s.
Second, applications can offload some of their computation to \MN{}s.
\sys\ provides such offloads with the same virtual memory address space and interface as their \CN\ counterparts.
Finally, users can build \textit{memory services} that run entirely at \MN{} hardware (\eg, a key-value store).
\sys\ offers each memory service its own virtual memory address space.

\sys\ achieves 
\textbf{1) \textit{low tail latency and high throughput}} --- \sys\ achieves network line rate (100\Gbps\ with our FPGA prototype)
and low average and tail latency (\eg, 2.7\mus\ avg and 3.2\mus\ 99-percentile for 1000 clients concurrently reading 16\,B), 
\textbf{2) \textit{excellent scalability}} --- one \MN{} can %freely\footnote{as long as sustainable by port link bandwidth and total memory size}
scale perfectly to thousands of concurrent clients and TBs of memory, % with the number of \CN{}s, client processes, and disjoint memory regions,
\textbf{3) \textit{low cost}} --- \sys\ completely removes the need for a server box and power-hungry CPUs at \MN{}s,
%and \sysboard\ can be built with XXX\MB\ on-chip memory and simple logic,
\textbf{4) \textit{safe and flexible computation offloading}} --- 
applications can safely offload their computation and enjoy a unified virtual memory system support,
and \textbf{5) \textit{extendible}} --- existing application-level~\cite{AIFM} and system-level~\cite{InfiniSwap,FastSwap,Semeru} 
disaggregated memory solutions can easily use \sys\ as the low-level platform,
and \sys\ itself can be easily extended and reconfigured.
Achieving these goals requires novel design and careful engineering in almost every part of the system.

The first major challenge is to build a full-fledged hardware-based virtual memory system that can deliver bounded, low latency for any type of data accesses.
Directly porting today's software-based virtual memory system to hardware will not only require large hardware resources but also results in long tails for tasks like page fault handling.
Our idea is to implement a {\em deterministic} hardware pipeline that bounds the latency of {\em all} memory access operations %including page fault handling 
and to keep non-deterministic and/or complex operations in software (running at few low-power cores in \sysboard).
We use a novel, asynchronous approach to efficiently interact between hardware and software to achieve good foreground performance.
We further propose a new conflict-free hash-based page table design that bounds address mapping to take at most one memory access.

{
\begin{figure}[th]
\begin{center}
\centerline{\includegraphics[width=\textwidth]{hotpot/Figures/architecture.pdf}}
\caption[\hotpot\ Architecture.]{\hotpot\ Architecture.}
\label{fig-architecture}
\end{center}
\end{figure}
}


Another big challenge is to achieve scalability with only minimal hardware resources at \sysboard.
Our general idea is to avoid maintaining states or data structures that could grow with clients or \CN{}s.
For example, for \sys's memory system, we design the page table to have a total size proportional to the physical memory size on an \MN, 
not to the number of client processes using the \MN.
Similarly, we avoid maintaining any states that could grow with network flows or clients at \sysboard.
To achieve this while delivering end-to-end reliability, 
\sys\ 1) uses a connection-less, RPC-like interface, % on top of a standard Ethernet link layer,
2) treats network errors as \sys\ request failure and re-executes the entire request, 
3) shifts stateful tasks like re-execution, packet ordering, and congestion control to the \CN\ side, % (in \syslib\ software),
and 4) removes ordering guarantees from the network and provides memory operation ordering at \syslib.

The rest of the paper will dive deep into \sys\ design and our FPGA prototype implementation.
We built five applications on top of \sys:
an image compression utility, a binary-tree index, a key-value store, a multi-version object store, and a simple data analytics service.
%We prototyped \sys's memory device with FPGA (\sysboard).
%\sys\ achieves high throughput (100\Gbps\ with FPGA prototype), low (tail) latency (XXX\mus\ end-to-end MTU round trip latency), 
%low cost (XXX\x\ energy saving), %and great extendibility,
%and \sys\ scales well with XXX. %eliminates {\em all} scalability bottlenecks in both its memory and network systems.
We compared \sys\ with vanilla RDMA, two RDMA-based disaggregated/remote memory systems~\cite{Tsai20-ATC,Kalia14-RDMAKV}, 
%an FPGA-based RDMA implementation~\cite{StRoM}, 
and a software-based SmartNIC~\cite{BlueField}.
\sys\ scales much better and has orders of magnitude lower tail latency than RDMA, 
while achieving similar throughput and min latency as RDMA (even with the slower FPGA frequency in our prototype).
\sys\ has 1.1\x\ to 3.4\x\ energy saving compared to CPU-based and SmartNIC-based disaggregated memory systems 
and is 2.7\x\ faster than SmartNIC solutions. 

\fi

\section{Goals and Related Works}
\label{sec:clio:motivation}

Resource disaggregation 
separates different types of resources into different pools,
each of which can be independently managed and scaled.
Applications can allocate resources from any node in a resource pool, resulting in tight resource packing. %regardless of where other types of resources sit.
Because of these benefits, % its manageability, independent scaling, and efficient resource utilization benefits,
many datacenters have adopted the idea of disaggregation, often at the storage 
layer~\cite{FACEBOOK-BRYCECANYON,FB1,SnowFlake-NSDI20,AMAZON-S3,AMAZON-EBS,Pangu,FC-SAN-book}.
With the success of disaggregated storage,
researchers in academia and industry have also sought ways to disaggregate memory
(and persistent memory)
\cite{Lim09-disaggregate,FireBox-FASTKeynote,IntelRackScale,Lim12-HPCA,Shan18-OSDI,hotpot-socc17,RAMCloud,Tsai20-ATC,AIFM,FastSwap,InfiniSwap,Semeru,Nitu18-EUROSYS}.
Different from storage disaggregation,
\md\ needs to achieve at least an order of magnitude higher performance and it should offer a byte-addressable interface.
Thus, \md\ poses new challenges and requires new designs.
This section discusses the requirements of \md\ and why existing solutions cannot fully meet them.

\subsection{\md\ Design Goals}
\label{sec:clio:requirements}
In general, \md\ has the following features, some of which are hard requirements while others are desired goals.

\stepcounter{reqs}
\boldpara{R\arabic{reqs}: Hosting large amounts of memory with high utilization.}
To keep the number of memory devices and total cost of a cluster low,
each \MN\ should host hundreds GBs to a few TBs of memory that is expected to be close to fully utilized.
To most efficiently use the disaggregated memory, we should allow applications to create and access {\em disjoint} memory regions of arbitrary sizes at \MN.

\stepcounter{reqs}
\boldpara{R\arabic{reqs}: Supporting a huge number of concurrent clients.}
To ensure tight and efficient resource packing,
%To fully exploit the resource-utilization benefit of disaggregation, 
we should allow many (\eg, thousands of) client processes running on tens of \CN{}s to access and share an \MN.
This scenario is especially important for new data-center trends like serverless computing and microservices where applications run as large amounts of small units.
%Thus, disaggregated memory should scale with client servers, client processes, memory size, and memory regions.
%each client process should be able to access large amounts of memory and many disjoint memory regions 
%scale with client servers, client processes, memory size, memory regions

\stepcounter{reqs}
\boldpara{R\arabic{reqs}: Low-latency and high-throughput.}
We envision future systems to have a new memory hierarchy, where disaggregated memory is larger and slower than local memory but still faster than storage.
Since \md\ is network-based, a reasonable performance target of it is to match the state-of-the-art network speed,
\ie, 100\Gbps\ throughput (for bigger requests) and sub-2\mus\ median end-to-end latency (for smaller requests).
%When used as dynamically allocated memory, disaggregated memory should deliver low latency so as not to slow down application execution.
%When it is used as (cache of) data store, throughput is more important, 
%and accesses to disaggregated memory should reach network line rate.
 
\stepcounter{reqs}
\boldpara{R\arabic{reqs}: Low tail latency.}
%Many workloads in datacenters have tight SLOs. 
Maintaining a low tail latency is important in meeting service-level objectives (SLOs) in data centers.
%Even though accesses to \MN{}s are much rarer than local memory accesses at \CN{}s (\eg, because \CN{}s cache remote data locally), %Although \CN{}s could cache data at their local memory to reduce the amount of remote memory accesses,
Long tails like RDMA's 16.8\ms\ remote memory access can be detrimental to applications that are short running (\eg, serverless computing workloads) or have large fan-outs or big DAGs
(because they need to wait for the slowest step to finish)~\cite{taillatency}.
%Within a rack, the network wire delay is short (~XXX\mus) and stable.
%Other factors such as memory access delays, software delays, and network queueing delays are the major factors in the tail.
%One of the obstacles to adopting disaggregated memory is the fear for unpreditable performance and much worse tail latency than local memory. %, \eg, during network congestion
%In addition to network unstability, we found that memory systems could also contribute to long tails with today's RDMA-based systems (\S\ref{sec:}).
%Having predictable, bounded performance and low tail latency would be one key to the successful adoption of memory disaggregation in data centers,
%where SLAs are important.

\stepcounter{reqs}
\boldpara{R\arabic{reqs}: Protected memory accesses.}
%When deployed in datacenters, it is important to protect disaggregated memory from undesired or malicious accesses.
As an \MN{} can be shared by multi-tenant applications running at \CN{}s, % and applications from different tenants over the network,
we should properly isolate memory spaces used by them.
Moreover, to prevent buggy or malicious clients from reading/writing arbitrary memory at \MN{}s, we should not allow the direct access of \MN{}s' physical memory from the network and \MN{}s should check the access permission.
%we should properly protect and isolate memory accesses to \MN{}s
%(\eg, by not allowing \CN{}s to directly access \MN{} physical memory over the network).
%and should not allow \CN{}s to directly access physical memory at \MN{}s.
%they should have at least the same level of safety guarantees as today's normal (local) memory
%and potentially also guard against new security threasts such as side channel attacks over the network~\cite{Tsai19-Security}.
%when shared across side channel attacks~\cite{Meltdown,Spectre}

\stepcounter{reqs}
\boldpara{R\arabic{reqs}: Low cost.}
A major goal and benefit of resource disaggregation is cost reduction.
A good \md\ system should have low {\em overall} CapEx and OpEx costs.
Such a system thus should not 1) use expensive hardware to build \MN{}s, 
2) consume huge energy at \MN{}s,
and 3) add more costs at \CN{}s than the costs saved at \MN{}s.
%1) means that server-based \MN\ designs are not ideal, and a hardware-based \MN\ design should make careful choice to keep the hardware resource consumption low;
%2) implies that CPU-based \MN\ solutions are likely to be too power hunger, and low-power solutions that run too slow cannot work either;
%and 3) implies that 
%Previous work~\cite{Shan18-OSDI} has shown that \md\ could improve memory resource utilization by around 50\% 
%(\ie, a \md\ cluster only needs to host half of the memory compared to a non-disaggregated cluster).
%This means that %1) a \md\ system should aim to have close-to-full utilization of its memory and have minimal memory waste, and 2) 
%building and running an \MN\ should not double the cost of hosting memory, as such a \md\ system would cost even more than no disaggregation.
%Using a server to build an \MN\ is thus not a good option, 
%since a server box
%and its CPU
%costs more than the DRAM it hosts.
%Another cost-related goal of \md\ is to have as much memory used by applications as possible (\ie, minimal memory wastes).
%However, if adding an \MN{} to host a certain amount of memory doubles the CapEx and OpEx cost, there will be no cost saving.
%This indicates that 
%Although disaggregated memory pool can be built with regular servers, 
%(in fact, most of today's disaggregated memory solutions~\cite{AIFM,FarMem,InfiniSwap,Semeru} are server-based),
%it is much cheaper to build and run standalone disaggregated memory devices without a server box or a CPU.


\stepcounter{reqs}
\boldpara{R\arabic{reqs}: Flexible.}
With the fast development of datacenter applications, hardware, and network, a sustainable \md\ solution should be flexible and extendable,
for example, to support high-level APIs like pointer chasing~\cite{AIFM,Aguilera-FarMemory},
to offload some application logic to memory devices~\cite{AIFM,StRoM},
or to incorporate different network transports~\cite{Homa,NDP,TONIC} and congestion control algorithms~\cite{swift-sigcomm,1RMA,hpcc-sigcomm19}.

\subsection{Server-Based Disaggregated Memory}
%RDMA-Based and Messaging-Based Disaggregated Memory}
\label{sec:clio:rdma}

\md\ research so far has mainly taken a server-based approach by using regular servers as \MN{}s~\cite{InfiniSwap,FastSwap,Semeru,Shan18-OSDI,AIFM,zombieland,FaRM},
usually on top of RDMA.
%which are connected to \CN{}s with RDMA or TCP~\cite{}.
%The second approach, taken by most disaggregated and remote memory solutions,
%is using a virtual memory abstraction provided by host server at \MN{}s
%and build their own layer on top of RDMA~\cite{InfiniSwap,FastSwap,Semeru} or TCP~\cite{AIFM}.
%many server-based disaggregated memory systems,
%uses the abstraction provided by a network layer 
%They use RDMA~\cite{InfiniSwap,FastSwap,Semeru} or TCP~\cite{AIFM} as the communication layer.
%As discussed in \S\ref{sec:intro}, 
The common limitation of these systems is their reliance on a host server and the resulting CPU energy costs, both of which violate \textbf{R6}.
%RDMA and TCP's symmetric architecture and connection-based, reliable transports are ill fit for memory disaggregation.
%Each of them also has their own problems.
%\md\ solutions that use TCP or similar transports incur high performance overhead because of costly network stack and memory copy, violating \textbf{R3}.
%other messaging-based disaggregated memory systems first copy application objects to messages at \CN{}s
%and then from messages to memory locations at \MN{}s.
%These systems not only incur the performance overhead of memory copies 
%but also require intensive CPU cycles to run the transport layer,
%making them unfit for memory disaggregation.
%These memory copies not only add performance overhead but also 
%RDMA is a high-speed, zero-copy, low CPU-utilization network technology 
%that has been adopted by several major datacenters~\cite{Microsoft,Alibaba}.

\ulinebfpara{RDMA} is what most server-based \md\ solutions are based on, with some using RDMA for swapping memory between \CN{}s and \MN{}s~\cite{InfiniSwap,FastSwap,Semeru} and some using RDMA for explicitly accessing \MN{}s~\cite{AIFM,zombieland,FaRM}.
%is more efficient than TCP-like transports, % avoids memory copying and can bypass CPU and OS kernel for most operations.
%but it still relies on the host server to run a virtual memory system,
Although RDMA has low average latency and high throughput, it has a set of scalability and tail-latency problems.
% is not a scalable or low-cost way to build disaggregated memory systems
%because of its reliance on the host CPU, MMU, and OS to run a virtual memory system.
%a host server box (CPU, MMU, and OS).
%RDMA's main issue when used for memory disaggregation is its reliance on host CPU, MMU, and OS
%and its limited scalability.
%RDMA relies on a host server's virtual memory system to manage the memory in this server.
%(\eg, for virtual and physical memory allocation, address translation, and page fault handling).
%Thus, \MN{}s can only be operated with a host server box.

A process ($P_M$) running at an \MN\ needs to allocate memory in its virtual memory address space 
and {\em register} the allocated memory (called a memory region, or MR) with the RDMA NIC (RNIC).
The host OS and MMU set up and manage the page table that maps $P_M$'s virtual addresses ({\em VA}s) to physical memory addresses ({\em PA}s).
%Client applications use $P_M$'s virtual memory address ($P_M{\text -}VA$) and MR information to perform RDMA read/write.
%On the \CN-side, there needs to be an extra level of indirection to first translate application abstraction to the $P_M{\text -}VA$ and MR combination.
%On the \MN-side, RNICs rely on host OS and MMU to manage page tables that map $P_M{\text -}VA$ to $PA$ (physical memory address).
To avoid always accessing host memory for address mapping, RNICs cache page table entries (PTEs),
but when more PTEs are accessed than what this cache can hold, RDMA performance degrades significantly (Figure~\ref{fig-pte-mr} and \cite{FaRM,Tsai17-SOSP}).
Similarly, RNICs cache MR metadata and incur degraded performance when the cache is full. 
Thus, RDMA has serious performance issues with either large memory (PTEs) or many disjoint memory regions (MRs), violating \textbf{R1}.
Moreover, RDMA uses a slow way to support on-demand allocation: the RNIC interrupts the host OS for handling page faults.
From our experiments, a faulting RDMA access is 14100\x\ slower than a no-fault access (violating \textbf{R4}).
%Page faults happen at the initial accesses to allocated virtual memory addresses, 
%causing long tails (violating \textbf{R4}).
%Slow initial accesses have a big impact on applications like serverless computing, which run short but very frequently.
%to handle page faults, which is extremely slow (14100\x\ slower than a no-fault access from our experiments).

To mitigate the above performance and scalability issues, most RDMA-based systems today~\cite{FaRM,Tsai17-SOSP} 
preallocate a big MR with huge pages and pin it in physical memory.
This results in inefficient memory space utilization and violates \textbf{R1}.
Even with this approach, there can still be a scalability issue (\textbf{R2}),
as RDMA needs to create at least one MR for each protection domain (\ie, each client).

In addition to problems caused by RDMA's memory system design, reliable RDMA, the mode used by most \md\ solutions, suffers from a connection queue pair (QP) scalability issue, also violating \textbf{R2}.
Finally, today's RNICs violate \textbf{R7} because of their rigid one-sided RDMA interface and the close-sourced, hardware-based transport implementation.
Solutions like 1RMA~\cite{1RMA} and IRN~\cite{IRN} mitigate the above issues by either onloading part of the transport back to software or proposing a new hardware design.

\ulinebfpara{LegoOS}~\cite{Shan18-OSDI}, our own previous work, is a distributed operating system designed for resource disaggregation.
Its \MN{} includes a virtual memory system that maps VAs of application processes running at \CN{}s to \MN\ PAs. \sys's \MN{} performs the same type of address translation.
%also advocates for managing memory at where memory is (called mComponent in LegoOS).
% is the only disaggregated memory system that adopts this abstraction and manages memory at \MN{}s.
However, LegoOS emulates \MN\ devices using regular servers and we built its virtual memory system in software,
which has a stark difference from a hardware-based virtual memory system. 
%In fact, we started our virtual memory design from LegoOS's but ended up finding 
%that none of its design or implementation fit a hardware environment. 
For example, LegoOS uses a thread pool that handles incoming memory requests by looking up a hash table for address translation and permission checking.
This software approach is the major performance bottleneck in LegoOS (\S\ref{sec:clio:results}),
violating \textbf{R3}.
%have to redesign every factor of it.
%still relies on host CPU and cannot be easily adapted to a hardware-based virtual memory system.
Moreover, LegoOS %has no computation offloading support, 
%and it 
uses RDMA for its network communication hence inheriting its limitations.

\subsection{Physical Disaggregated Memory}
\label{sec:clio:pdm}

One way to build \md\ without a host server is to treat it as raw, physical memory,
%\ie, client servers send read/write request with physical memory address, and the memory node directly read/write to that address without any address mapping.
a model we call {\em \pdm}.
The \pdm\ model has been adopted by a set of coherent interconnect proposals~\cite{Genz-citation,CXL-citation},
HPE's Memory-Driven Computing project~\cite{HP-TheMachine,THEMACHINE-HOTOS,HP-MODC-POSTER,THEMACHINE-WEB}.
A recent disaggregated hashing system~\cite{race-atc21} and our own recent work on disaggregated key-value systems~\cite{Tsai20-ATC} also adopt the \pdm\ model and emulate remote memory with regular servers.
%two RDMA-based software systems~\cite{Tsai17-SOSP,Tsai20-ATC} 
%(these systems use the OS kernel to register physical memory directly, which is a special RDMA configuration).
To prevent applications from accessing raw physical memory,
these solutions add an indirection layer at \CN{}s in hardware~\cite{Genz-citation,CXL-citation} or software~\cite{Tsai20-ATC,race-atc21}
to map client process VAs or keys
to \MN\ PAs. 

There are several common problems with all the \pdm\ solutions.
First, because \MN{}s in \pdm\ are raw memory, \CN{}s need multiple network round trips to access an \MN\ 
for complex operations like pointer chasing and concurrent operations that need synchronization~\cite{Tsai20-ATC}, violating \textbf{R3} and \textbf{R7}.
Second, \pdm\ requires the client side to manage disaggregated memory.
For example, \CN{}s need to coordinate with each other or use a global server~\cite{Tsai20-ATC} to perform tasks like memory allocation.
Non-\MN-side processing is much harder, performs worse compared to memory-side management (violating \textbf{R3}), and could even result in higher overall costs because of the high computation added at \CN{}s (violating \textbf{R6}).
%and when some memory needs to be migrated from one \MN\ to another (\eg, for load balancing), 
%all the \CN{}s that have mapped this memory need to update their mappings.
%Such complex client-side memory management defeats the purpose and benefits of disaggregation (\textbf{R5}).
Third, exposing physical memory makes it hard to provide security guarantees (\textbf{R5}),
as \MN{}s have to authenticate that every access is to a legit physical memory address belonging to the application.
%\MN{}s have to trust that \CN{}s will never access beyond their allocated physical memory regions. 
%coordination across client servers (\eg, through a global controller~\cite{Tsai20-ATC} or a distributed consensus system).
%Fourth, it is difficult to build compute offloads on physical memory and unsafe to run them without memory protection.
Finally, all existing \pdm\ solutions require physical memory pinning at \MN{}s, causing memory wastes and violating \textbf{R1}.
%it is also unclear how page faults will be handled.

In addition to the above problems, none of the coherent interconnects or HPE's Memory-Driven Computing have been fully built.
When they do, they will require new hardware at all endpoints and new switches. 
Moreover, the interconnects automatically make caches at different endpoints coherent, which could cause performance overhead that is not always necessary (violating \textbf{R3}).

Besides the above \pdm\ works, there are also proposals to include some processing power in between the disaggregated memory layer and the computation layer.
soNUMA~\cite{soNUMA} is a hardware-based solution that scales out NUMA nodes by extending each NUMA node with a hardware unit that services remote memory accesses.
Unlike \sys\ which physically separates \MN{}s from \CN{}s across generic data-center networks, soNUMA still bundles memory and CPU cores, and it is a single-server solution.
Thus, soNUMA works only on a limited scale (violating \textbf{R2}) and is not flexible (violating \textbf{R7}).
MIND~\cite{mind:sosp21}, a concurrent work with \sys, proposes to use a programmable switch for managing coherence directories and memory address mappings between compute nodes and memory nodes.
Unlike \sys\ which adds processing power to every \MN, MIND's single programmable switch has limited hardware resources and could be the bottleneck for both performance and scalability.

{
\begin{figure}[t]
\begin{center}
\footnotesize
\lstinputlisting[
language=C
]{clio/code-eg.cpp}
\mycaption{fig-clio-code-eg}{Sample code using \sys.}
{

}
\end{center}
\end{figure}
}

\section{\sys\ Overview}
\label{sec:hdm}

\sys\ co-designs software with hardware, \CN{}s with \MN{}s, and network stack with virtual memory system, 
so that at the \MN{}, the entire data path is handled in hardware with high throughput, low (tail) latency, and minimal hardware resources. 
This section gives an overview of \sys's interface and architecture (Figure~\ref{fig-arch}).

%The distributed \md\ platform that we built allows an application to run on one or more \CN{}s and access memory located in one or more \MN{}s through a virtualized memory interface. % (to be discussed in \S\ref{sec:abstraction}).
%The way that we manage distributed \MN{}s is similar to LegoOS's two-level management~\cite{Shan18-OSDI}, and we omit it in this paper to focus on the single-\MN\ \sys\ design.
%\sys\ co-designs software with hardware, \CN{}s with \MN{}s, and network stack with virtual memory system, 
%so that at the \MN{}, the entire data path is handled in hardware with high throughput, low (tail) latency, and minimal hardware resources. 
%This section gives an overview of \sys's interface and architecture (Figure~\ref{fig-arch}).
%Our current evaluation of \sys\ stays within a rack (which is the typical deployment scale of \md). \sys's design could potentially be extended to work at the data-center scale,
%\eg, by \fixme{XXX}.
%we (and many others~\cite{XXX}) envision the typical deployment of \MN{}s to be within the same rack as \CN{}s.



\subsection{\sys\ Interface}
\label{sec:abstraction}


Similar to recent \md\ proposals~\cite{AIFM,sebastian-hotcloud20}, our current implementation adopts a non-transparent interface where
applications (running at \CN{}s) allocate and access disaggregated memory via explicit API calls. Doing so gives users opportunities to perform application-specific performance optimizations. 
By design, \sys’s APIs can also be called by a runtime like the AIFM runtime~\cite{AIFM} or by the kernel/hardware at \CN\ like LegoOS' pComponent~\cite{Shan18-OSDI} to support a transparent interface and allow the use of unmodified user applications.
We leave such extension to future work.
%For example, for the last case, the \CN\ kernel or hardware captures misses in \CN’s local memory and then calls \sys’s APIs to fulfill the misses.

%\footnote{\sysboard\ could potentially work with a transparent \md\ solution if \CN{}s can directly intercept memory instructions and issue remote requests using application VAs (\eg, with LegoOS's processor architecture).}
Apart from the regular (local) virtual memory address space, each process has a separate {\em \textbf{R}emote virtual memory \textbf{A}ddress \textbf{S}pace} ({\em \rspace} for short).
Each application process has a unique global {\em PID} across all \CN{}s which is assigned by \sys\ when the application starts.
Overall, programming in \rspace\ is similar to traditional multi-threaded programming except that memory read and write are explicit and that processes running on different \CN{}s can share memory in the same \rspace.
Figure~\ref{fig-code-eg} illustrates the usage of \sys\ with a simple example.


An application process can perform a set of virtual memory operations in its \rspace,
including \alloc, \sysfree, \Cliosysread, \Cliosyswrite, 
and a set of atomic and synchronization primitives (\eg, \syslock, \sysunlock, \fence).
\alloc\ works like \texttt{malloc} and returns a VA in \rspace. \Cliosysread\ and \Cliosyswrite\ can then be issued to any allocated VAs.
As with the traditional virtual memory interface, allocation and access in \rspace\ are in byte granularity.
We offer {\em synchronous} and {\em asynchronous} options for \alloc, \sysfree, \Cliosysread, and \Cliosyswrite.
%, with which users can choose between performance and consistency levels.
%A synchronous API blocks until the result is ready.
%An asynchronous API is non-blocking, and the application calls \poll\ to get its result.




\if 0
\sys\ exposes an isolated virtual memory address space to each ``{\em collection}''.
A collection can be an application process running at a \CN, a middle layer like JVM running at a \CN, a computation offload running at \MN, or any combination of them.
Within a collection, there can be multiple {\em threads}. 
One thread can only be at one node, but a collection can have threads on both a \CN\ and an \MN.
\sys\ offers basic virtual memory APIs like \alloc, \Cliosysread, and \Cliosyswrite, 
a set of atomic and synchronization primitives (\tas, \cas, \fence), 
and extended APIs like array indexing and pointer chasing.
We choose this virtual memory abstraction instead of alternatives like a key-value interface,
because its versatility, generality, and backward compatibility with today's single-machine virtual memory abstraction.
\fi



\ulinebfpara{Intra-thread request ordering.}
Within a thread, synchronous APIs follow strict ordering.
An application thread that calls a synchronous API blocks until it gets the result.
Asynchronous APIs are non-blocking. A calling thread proceeds after calling an asynchronous API and later calls \poll\ to get the result. 
Asynchronous APIs follow a release order.
%By default, all the APIs execute in an asynchronous fashion (\ie, there can be multiple outstanding operations),
%and we follow a release ordering of memory operations within each thread.
Specifically, asynchronous APIs may be executed out of order as long as
1) all asynchronous operations before a \release\ complete before the \release\ returns,
and 2) \release\ operations are strictly ordered.
On top of this release order, 
we guarantee that there is no concurrent asynchronous operations with dependencies (Write-After-Read, Read-After-Write, Write-After-Write) and target the same page.
The resulting memory consistency level is the same as architecture like ARMv8~\cite{ARMv8}.
In addition, we also ensure consistency between metadata and data operations, by ensuring that potentially conflicting operations execute synchronously in the program order. For example, if there is an ongoing \sysfree\ request to a VA, no read or write to it can start until the \sysfree\ finishes.
Finally, failed or unresponsive requests are transparently retried, and they follow the same ordering guarantees.
%We chose this consistency level as our default because it is enough for most programs and 
%it allows \sys\ to use a lightweight network layer that can tolerate reordering (\S\ref{sec:network}).
%In addition to our default mode, we also provide a synchronous version of each API:
%there can be only one outstanding synchronous operation within a thread.  
%the relaxed consistency makes it possible 




{
\begin{figure}[th]
\begin{center}
\centerline{\includegraphics[width=\textwidth]{hotpot/Figures/architecture.pdf}}
\caption[\hotpot\ Architecture.]{\hotpot\ Architecture.}
\label{fig-architecture}
\end{center}
\end{figure}
}


\ulinebfpara{Thread synchronization and data coherence.}
Threads and processes can share data even when they are not on the same \CN.
Similar to traditional concurrent programming, \sys\ threads can use synchronization primitives to build critical sections (\eg, with \syslock) 
and other semantics (\eg, flushing all requests with \fence).

An application can choose to cache data read from \Cliosysread\ at the \CN\ (\eg, by maintaining \texttt{local\_rbuf} in the code example).
Different processes sharing data in a \rspace\ can have their own cached copies at different \CN{}s.
Similar to ~\cite{Shan18-OSDI}, \sys\ does not make these cached copies coherent automatically and lets applications choose their own coherence
protocols.
%mechanisms and policies.
%Note that maintaining coherence when caching shared data is not handled by DVMA automatically. 
We made this deliberate decision because automatic cache coherence on every read/write would incur  high performance overhead with commodity Ethernet infrastructure
and application semantics could reduce this overhead.
%(\eg, by infrequent messaging).
%to let users decide on their coherence mechanism on top of \sys\ and to prevent the potential high network communication overhead.
%if threads cache shared data, there can be multiple locations of caches. 
%We do not make these caches coherent automatically, as it could cause high network communication overhead.
%Users can manage coherence on their own.
%Different users can share memory and can use \sys's synchronization primitives to achieve inter-user synchronization (\eg, using \tas\ to define critical section). 


\subsection{\sys\ Architecture}

In \sys\ (Figure~\ref{fig-arch}), \CN{}s are regular servers each equipped with a regular Ethernet NIC and connected to a top-of-rack (ToR) switch.
\MN{}s are our customized devices directly connected to a ToR switch.
%
Applications run at \CN{}s on top of our user-space library called {\em \syslib}.
%\syslib\ handles application requests in the user space.
It is in charge of request ordering, request retry, congestion, and incast control. 
%\syslib\ issues raw Ethernet requests directly to the NIC (bypassing kernel with zero memory copy, similar to DPDK~\cite{DPDK}). 
%Similar to DPDK~\cite{DPDK}, \syslib\ bypasses kernel and has zero memory copy capability.

By design, an \MN\ in \sys\ is a \sysboard\ consisting of an ASIC which runs the hardware logic for all data accesses (we call it the {\em fast path} and prototyped it with FPGA),
an ARM processor which runs software for handling metadata and control operations (\ie, the {\em slow path}),
and an FPGA which hosts application computation offloading (\ie, the {\em extend path}).
An incoming request arrives at the ASIC and travels through standard Ethernet physical and MAC layers 
and a Match-and-Action-Table (MAT) that decides which of the three paths the request should go to based on the request type.
If the request is a data access (fast path), it stays in the ASIC and goes through a hardware-based virtual memory system
that performs three tasks in the same pipeline: address translation, permission checking, and page fault handling (if any).
Afterward, the actual memory access is performed through the memory controller, and the response is formed and sent out through the network stack.
Metadata operations such as memory allocation are sent to the slow path. % and handled in software that runs on Linux at the ARM processor.
Finally, customized requests with %customized, high-level operations such as pointer chasing and 
offloaded computation are handled in the extend path.
%which executes the corresponding application offload.
%The offload itself could internally issue data and metadata operations to the fast and the slow paths.
%In the rest of the paper, we focus on the design of the fast and the slow paths and how they interact with each other.






\if 0
\subsection{Paper Scope And Potential Extensions}
\label{sec:scope}
%We made it real by building an open-source, distributed smart hardware-based disaggregated memory framework.
This paper focuses on the low-level systems problem of how to most efficiently build an \MN\ hardware device and to support it with \CN-side software. 
In real deployment, a single \sysboard\ is usually insufficient.
A distributed set of \sysboard\ would allow applications to allocate and access memory from multiple \MN{}s using a unified virtual memory interface. 
Systems like LegoOS~\cite{Shan18-OSDI} and Clover~\cite{Tsai20-ATC} have demonstrated how to build a distributed \MN\ platform.
For example, LegoOS uses a “two-level” approach, where a global controller manages the entire memory space at coarse granularity and each \MN\ manages its own memory at fine granularity (like how \sysboard\ does). Each \MN\ can be over-committed (\ie, allocating more virtual memory than its physical memory size), and when an \MN\ is under memory pressure, it migrates data to another \MN\ (coordinated by the global controller).
In addition, LegoOS leaves the handling of \MN\ failure to applications, since most data-center applications already have their own reliability mechanisms.
A similar mechanism could be used to extend \sys\ to a distributed system, and we leave it for future work.

Another potential extension to our current implementation of \sys\ is a transparent user interface.
By design, \sys’s APIs can be called by different layers: directly by applications as what we show in this paper, by a runtime like the AIFM runtime~\cite{AIFM}, or by the kernel/hardware at \CN\ like LegoOS' pComponent~\cite{Shan18-OSDI}. \sysboard\ doesn’t need any change to support any of these usages.
The latter two usages of \sysboard\ would result in a transparent interface and allow the use of unmodified user applications.
For example, for the last case, the \CN\ kernel or hardware captures misses in \CN’s local memory and then calls \sys’s APIs to fulfill the misses.
%(after porting \syslib\ to the kernel space or to hardware).

\fi


%\yizhou{Clio architecture shares a lot similarities with disaggregated storage systems such as fiber channel storage area network (FC SAN). The key difference is that Clio targets microsecond-level disaggregated memory system which poses stricter requirements on the system design.}


%\yizhou{
%1) I think we should explicitly say "By design, MN consists of ASIC, FPGA, and ARM parts.. In our impl, the ASIC part is using FPGA too." I get the motive behind it. The way it is described makes readers think this is the real stuff. 2) Similarly, the customized extend path does not really have to be FPGA as well. It could be ASIC too. In all, I suggest differentiate "by design" and "our impl".}

\if 0

\section{Hardware-Based Virtual Disaggregated Memory}
\label{sec:vdm}

We advocate for a new approach for memory disaggregation:
a hardware-based ``smart'' virtual disaggregated memory system.
Specifically, this approach encorporates the following design principles.
%We believe that disaggregated memory should be managed at the memory side behind a per-client virtual memory abstraction, 
%but without the reliance on a host server and with the support for computation offloading.
%With a per-client virtual memory abstraction, 
%application-level programs and low-level systems like databases and JVM can all sit on \sys\ with each of them properly isolated,
%and the physical location of disaggregated memory can be transparent and thus non-contiguous and moved around.
%By building a hardware-based ``smart'' virtual disaggregated memory system,
%we can achieve low-latency performance, scalability, efficient memory utilization, low cost, and ease of management,
%as what the rest of this paper will show.
%Below are the principles that guide the design of \sys.

\boldpara{Managing memory at \MN{}s and exposing a per-client virtual memory abstraction.}
We manage disaggregated memory entirely within a disaggregated memory pool by building a virtual memory system at each \MN.
By encapsulating management in the disaggregated memory pool and allowing client applications to
access it as a black box,
we can achieve independent, transparent resource pool management,
which is a key reason behind industry's wide adoption of storage disaggregation~\cite{FACEBOOK-BRYCECANYON,FB1,SnowFlake-NSDI20,Ali-SinglesDay}. %why storage disaggregation has become the recent industry norm;
It also avoids the network communication overhead incurred in solutions that handle some or all management tasks at \CN{}s.

We choose to expose a {\em per-client virtual memory} abstraction,
where each client has an isolated space that it can access with virtual memory addresses at byte granularity.
%\textbf{Q1)} \textit{what abstraction disaggregated memory exposes}?
%Disaggregated memory should expose an abstraction that is versatile enough to support many different uses.
%One such abstraction is {\em per-client virtual memory},
Just like the classical virtual memory abstraction, this abstraction is low-level and generic enough to support many applications
and high-level enough to protect and hide raw memory. 
%This abstraction inherits the versatility and transparency benefits of the classical virtual memory abstraction. %process address spaces:
%1) application-level programs and low-level systems like databases and JVM can all sit on the same abstraction
%with each of them properly isolated,
%and 2) the physical location of disaggregated memory can be transparent and thus non-contiguous and moved around.
%The abstraction we believe to be the best fit for memory disaggregation is a virtual memory interface that is facing client processes running at \CN{}s.
%With this abstraction, disaggregated memory can be managed at \MN{}s, which is more efficient and flexible than at \CN{}s (\S\ref{sec:intro} and \S\ref{sec:pdm}).
%It is worth noticing that %\sys\ is a versatile system that can either 
For example, memory disaggregation solutions that operate below the application level,
\eg, language runtime~\cite{Semeru}, data-structure library~\cite{AIFM}, swap system~\cite{InfiniSwap,FastSwap}, %\fixme{anything else?},
can sit on top of \sys. % and are orthogonal to our work. 
They can be ported to \sys\ by replacing their RDMA-/messaging-based abstraction to \sys's virtual memory abstraction. % (Figure~\ref{fig-usage} U3).

\boldpara{Building a virtual memory system in hardware.}
We demonstrate that \MN{}s can be built as standalone hardware devices without a server box,
as its benefits outweigh the complexity of hardware development.
Building \MN\ as a single hardware device avoids the monetary cost of a whole server box and the energy cost of a power-hungry CPU.
It also avoids the performance overhead of NIC talking to the host server for handling virtual memory tasks like page fault handling.
Moreover, a hardware implementation could allow greater parallelism and customized pipelines
that is crucial in supporting disaggregate memory's scalability goals (TB-level memory, thousands of clients)
and in meeting today's and future high-speed network line rate~\cite{TONIC}.
%Today's disaggregated memory systems unfortunately have a strong reliance on host CPU, MMU, and OS
%for running a virtual memory system, even though these systems themselves do not necessarily need to run anything at the host server.
%The solution is obvious but never carried out before: building the virtual memory system in \MN\ hardware.

\boldpara{Designing a network layer that exploits memory disaggregation's unique features.}
We improve network communication performance and reduce its costs by 
exploiting the unique nature of memory disaggregation. % to co-design a new network layer.
%Unfortunately, existing disaggregated memory systems lose this optimization opportunity by sitting on RDMA or TCP.
Unlike general-purpose network solutions such as TCP and RDMA that have the same design for all endpoints (\ie, symmetric),
the network system for disaggregated memory can be {\em asymmetric}, as \CN{}s are always the request initiator and \MN{}s only respond to requests.
%RDMA and TCP have two fundamental features that make them unfit for memory disaggregation:
%1) they have the same design for all endpoints (\ie, symmetric),
%while memory disaggregation are asymmetric in nature (\CN{}s are always the request initiator and \MN{}s only respond to requests);
Moreover, not all memory operations require strict ordering.
Thus, we can relax network layer's reliability guarantees (\eg, allowing packet reordering)
and enforce (weaker) orderings at the memory operation level.
%2) their reliable transports are connection-based and follow strict packet ordering.
%These features result in scalability bottlenecks and added hardware complexity.
%However, as we will show in \S\ref{sec:network}, neither are required if we can customize the network layer for the client-facing virtual memory model.

\boldpara{Supporting computation offloading with a unified virtual memory view.}
The network communication between \CN{}s and \MN{}s is the major cause of disaggregated memory's performance overhead.
To reduce this overhead, applications should be able to offload their less computation-intensive tasks to \MN{}s.
\sys\ offers a unified virtual memory view for application computation at both \CN\ and \MN.
%and offloads running at \MN{}s cannot have a coherent memory view with computation running at \CN{}s.
Furthermore, running offloaded computation in hardware is a desired option, %reduces \MN{}'s energy cost 
as it could achieve more parallelism and performance customization while avoiding CPU energy cost.
\sys\ allows offloads that run in hardware to use the \sys\ virtual memory interface in the same way as how software runs on \sys.
%However, %a major obstacle of hardware-based offloading currently is 
%it is hard to offload computation to \MN\ hardware today,
%largely because there is no good system support for offloads to use (virtualized) memory
%We build an offloading framework that offers a unified virtual memory view for application computation at both \%CN\ and \MN.
%applications should be able to easily and safely offload their less computation-intensive tasks to \MN{}s.




\section{Hardware Active Disaggregated Memory}
\label{sec:phdm}

%This section gives an overview of the \phdm\ model, 
%its benefits, and its use cases.

%\subsection{\phdm\ Overview}
%\label{sec:phdmoverview}

\phdm\ has a server-based compute pool that runs applications
and a separate memory pool that consists of network-attached hardware memory devices.
%\phdm\ has a distributed, network-attached memory pool 
%that is separated from the compute pool.
%\phdm\ manages memory at the remote memory pool.
%and runs applications at the compute pool.
An application process running at a \CN{} can use memory from any \MN{},
and an \MN{} can host data for many applications running on different \CN{}s.
%With this design of disaggregated and distributed memory services,
Thus, \phdm\ delivers independent management and scaling of memory and compute pools 
and efficient memory resource utilization.
%Doing so reduces network RTTs and makes management tasks such as configuration, upgrade, and failure handling easy
%with no or little disturbance to \CN{}s.

An \phdm\ system manages and virtualizes physical memory at \MN{}s
and offers one or more {\em distributed memory services}.
\phdm\ provides a protected, client-centric virtual memory abstraction 
and potentially more higher-level memory service interfaces.
Applications running at \CN{}s can directly and safely use these interfaces to access remote memory. % and be protected from each other.
%These services have a client-centric view, providing interfaces that can directly and easily be used by clients 
%and offering proper protection across different applications.
%It provides high-level memory services and
% can support new types of applications that require large memory space, 
%inter-node memory sharing, or fast data storage.
%Section~\ref{sec:usecases} gives several examples of \phdm\ memory services.

\phdm\ executes all performance-critical tasks on hardware at \MN{}s,
%As explained in \S\ref{sec:disaggregation}, 
%With fast increasing datacenter network bandwidth, \MN{}s need to be able to serve large amounts of 
%requests concurrently to deliver {\em network line-rate} performance.
%Running software in many-core CPU could potentially meet the performance requirement but comes with high monetary and energy cost,
%while running software in low-energy ``wimpy'' cores cannot keep up with high network line rate.
%as handling memory requests in hardware is critical in meeting 
as doing so can meet the performance, scalability, and cost goals of remote memory systems.
With varying application needs in today's datacenters,
disaggregated memory systems should be flexible enough to provide different services and interfaces.
Thus, in our proposed \phdm\ model, the remote memory layer is {\em reconfigurable}.
%We further propose to implement \phdm's memory services on programmable hardware
%to meet our target usage and deployment models:
Specifically, an \MN{} can be configured to offer different (sets of) memory services,
but once a cluster of \MN{}s are configured they are only reconfigured when 
there is a need to change, patch, or upgrade services.
This model is in line with Microsoft's FPGA deployment~\cite{Catapult}
and what we view as a good use of reconfigurable hardware in datacenters.

\if 0
Finally, we view \phdm\ as a general model that can have different hardware choices for \MN{}s,
as long as they incorporate some programmable hardware for performance-critical and non-fixed tasks in memory services
(\eg, a single FPGA, FPGA+SoC, ASIC+FPGA, or ASIC+FPGA+SoC).
Non-performance-critical tasks can run in software, and fixed functionalities can run in non-programmable hardware.
\fi


%\subsection{Use Cases}
%\label{sec:usecases}
Many types of applications can make use of \phdm.
%In some cases, applications could explicitly call APIs an \phdm\ service provides
%(and be aware of the disaggregated nature of their in-memory data).
%In other cases, they could transparently access an \phdm\ service without knowing the disaggregated or remote nature.
Below we give some examples. % on what services an \phdm\ system can provide and how applications can use them.
We implemented an instance of the first three types in this paper,
leaving the rest for future work.

\boldpara{Extended (semantic-rich) virtual memory.}
A basic service \phdm\ can provide is a remote virtual memory space that lets applications
store in-memory data (\eg, as extended, slower heaps).
%Similar to traditional process address spaces, each application process on each \CN{}
%can have their own remote address space that \phdm\ protects from other processes'.
In addition to simple, hardware-like virtual memory APIs such as reading and writing to a memory address, 
\phdm\ could provide higher-level APIs like synchronization primitives, pointer manipulation, 
vector and scatter-gather operations~\cite{Aguilera-FarMemory}.
Applications and language libraries can then build complex data structures like vectors 
and trees with these APIs.
%We implemented this extended virtual memory service in \sys.

\boldpara{In-memory and ephemeral storage.}
\phdm\ could offer in-memory storage services such as distributed key-value stores, databases, and file systems.
With \phdm, many storage operations (\eg, key-value pair lookup, SQL select) 
could be implemented in hardware at where the data is, offering enhanced performance. 
%In addition, built-in distributed support in \phdm\ could be used to offer high availability and failure resilience
%to the in-memory storage services.
\phdm\ is also a good fit for building ephemeral storage and storage caching that do not require failure resilience~\cite{SnowFlake-NSDI20,Pocket,fitzpatrick2004distributed}.
%We implemented a distributed key-value store in \sys\ with optional replication support.

\boldpara{Data sharing.}
Since multiple \CN{}s can access the same \MN{}s in \phdm,
\phdm\ could be used for data sharing and communication across \CN{}s.
This is especially useful for new datacenter services like serverless computing~\cite{Berkeley-Serverless},
which currently has no or poor support for managing states and inter-function communication.
With \phdm, serverless functions can run on \CN{}s and store states or communication messages in the disaggregated memory layer.
Similarly, \phdm\ can also be used for storing global states such as the parameter server in distributed machine learning systems.
%We implemented a multi-version data sharing service in \sys\
%that can be accessed by different \CN{}s concurrently.

\boldpara{Offloading data processing.}
\phdm\ is a good candidate for offloading data processing and data analytics. 
Data-intensive applications can offload computation that frequently access in-memory data together with 
these data to \MN{}s.
One such example is disaggregated Spark shuffle~\cite{Stuedi-ATC19}, where the shuffle
operation could be implemented in programmable hardware and the shuffle data could be 
stored in \MN{}s of \phdm.

\boldpara{Remote swap and remote disk.}
Legacy applications and libraries can also benefit from \phdm\ in a transparent way.
Two such examples are remote memory-based disk and remote swap~\cite{InfiniSwap}.
The OS at the \CN{}s can add a memory-based block device that sits in \phdm\
in a similar way as building the {\em ramdisk} module.
Applications can directly use this new device or use it as a swap space.

\boldpara{Disaggregated OS.}
Recently, there have been proposals to completely disaggregate memory from compute.
\lego~\cite{Shan18-OSDI} is such a proposal that organizes compute nodes as processors with no memory 
and \MN{}s as memory devices with no computation.
%While \lego\ currently only works on emulated hardware, it can build on \phdm\ to have a 
%real hardware solution.
\lego\ can build on \phdm\ by configuring \CN{}s as its compute nodes %(called {\em pComponent}s) 
and \MN{}s as its memory nodes. %(called {\em mComponent}s).


\fi

\if 0


\if 0
For all the non-synchronization APIs, we offer two versions: synchronous and asynchronous.
While synchronous APIs are always strictly ordered within a user (and thus slower),
we relax the ordering of asynchronous APIs to a release consistency where operations can be executed out of order as long as 
1) read/write dependencies (WAR, RAW, WAW) are followed,
2) operations before a \fence\ must all complete before the \fence\ can return,
and 3) \fence{} operations are strictly ordered.
This release consistency is the same as architectures like ARMv8~\cite{ARMv8} and makes it possible for \sys\ to use a connectionless network layer with possible reordering (\S\ref{sec:network}).
Different users can share memory and can use \sys's synchronization primitives to achieve inter-user synchronization (\eg, using \tas\ to define critical section). 
\fi

\if 0
\sys\ is versatile in that there are many ways to use \sys's virtual memory view (we call it {\em \sys\ address space}).
Below and in Figure~\ref{fig-usage}, we list five typical ways to use \sys.
We implemented five applications with U2, U4, and U5;
U1 and U3 require building new hardware and/or OS/low-level systems, which are beyond this paper's scope.
%where client application processes can directly access disaggregated memory with their virtual memory addresses,
%no matter whether they run on \CN{}s or on \MN{}s.
%\sys\ only performs one address mapping: application virtual address to remote physical address (done at memory node),
%ordering

\stepcounter{case}
\boldpara{{\bf U\arabic{case}}: Entire virtual memory controlled by OS/hardware.}
A completely disaggregated solution like LegoOS~\cite{Shan18-OSDI}, where \CN{}s are compute devices with only CPU cache but no memory
can use \sys\ as the memory layer by sending \sys\ read/write requests to fulfill 
last-level-cache misses. % (called ExCache in LegoOS) misses.
With this usage, a \sys\ address space becomes the entire virtual memory address space
of a process, and \sys\ is completely transparent to application processes.

%usages
\stepcounter{case}
\boldpara{{\bf U\arabic{case}}: Extended virtual memory controlled by applications.}
%Applications run purely at \CN{}s and exploit \MN{}s for larger, dynamically allocated memory space.
%Note that the application ``virtual memory addresses'' \sys\ use (we call them \textit{\sys\ virtual memory address}es) 
%may not necessarily come from normal process memory address space.
%In fact, there are three ways to use \sys, and they have their own interpretation of what \sys\ virtual memory addresses mean,
%as illustrated in Figure~\ref{fig-arch}.
Without changing existing server hardware or OS at \CN{}s, 
an application process can explicitly call \sys\ APIs by linking \syslib\ which creates an extra \textit{remote virtual memory address space} (\ie, a \sys\ address space)
that is separate from the process' normal (local) virtual memory address space.
Applications have precise control over what and when to use disaggregated memory and can use or implement high-level APIs like pointer chasing.

%The returned addresses of \alloc\ and input addresses of \sysread/\syswrite requests will all be virtual addresses in this space ({\em dvma-vaddr}).

\stepcounter{case}
\boldpara{{\bf U\arabic{case}}: Remote memory space controlled by a middle layer.}
A system layer like a remote-memory swap system~\cite{InfiniSwap,FastSwap} or a language runtime~\cite{Semeru}
can sit on top of \sys\ and use a \sys\ address space as its own managed space (\eg, a swap partition, a JVM heap).
Applications on top of such a layer can transparently access larger memory backed by \sys.
%Second, applications can sit on top a remote-memory swap system like FastSwap~\cite{FastSwap} and InfiniSwap~\cite{InfiniSwap},
%which uses \sys\ virtual memory addresses as swap partition address ({\em swap-vaddr}) and performs \sys\ read/write to swap in/out disaggregated memory.

%Using client process virtual memory addresses also has the benefit that disaggregated memory can potentially be 
%integrated into client machine's memory micro-architecture~\cite{Lim09-disaggregate}.

\stepcounter{case}
\boldpara{{\bf U\arabic{case}}: Memory services running at \MN{}s.}
Users can deploy {\em memory services} that run entirely at \MN{}s (\eg, 
an in-memory key-value store or object data store) 
and expose their own interface (\eg, key-value get/set) 
to the clients of these services that run at \CN{}s.
These services can be deployed to \sysboard's programmable hardware and/or software platforms.
Each memory service has its own isolated \sys\ address space and can use \sys's APIs, 
which makes hardware implementation easier and execution safer.

\stepcounter{case}
\boldpara{{\bf U\arabic{case}}: Partial computation offloading.}
While U1, U2, and U3 perform computation completely at \CN{}s and U4 performs computation entirely at \MN{}s,
applications can also split their computation across \CN{} and \MN{}.
%An application can offload some of its computation tasks to an \MN{} (run in hardware or software).
\sys\ offers a {\em unified} address space for an application's \CN{} and \MN{} parts.
These parts are treated as different \sys\ threads. 
As explained earlier, if they cache shared data locally, they need to have their own mechanism to make the cache coherent, if desired.
\sys\ offers synchronization primitives to assist them work with their shared data.
%with each being treated as a traditional thread (\eg, \CN{} and \MN{} threads can be synchronized using \tas\ and other \sys\ synchronization primitives).
%\stepcounter{principle}
%\ulinebfpara{Principle \arabic{principle}:}
%\textit{The disaggregated memory abstraction should face client processes}.

\fi

\section{\sys\ Design}
\label{sec:clio:design}

This section presents the design challenges of building a hardware-based \md\ system and our solutions.

\subsection{Design Challenges and Principles}
Building a hardware-based \md\ platform is a previously unexplored area and introduces new challenges mainly because of restrictions of hardware and the unique requirements of \md.
%Below, we discuss these challenges and our design principles.


\boldpara{Challenge 1: The hardware should avoid maintaining or processing complex data structures}, because unlike software, hardware has limited resources such as on-chip memory and logic cells.
For example, Linux and many other software systems use trees (\eg, the vma tree) for allocation.
Maintaining and searching a big tree data structure in hardware, however, would require huge on-chip memory and many logic cells to perform the look up operation (or alternatively use fewer resources but suffer from performance loss).


\boldpara{Challenge 2: Data buffers and metadata that the hardware uses should be minimal and have bounded sizes}, so that they can be statically planned and fit into the on-chip memory.
Unfortunately, traditional software approaches 
% (usually unawaringly) 
involve various data buffers and metadata that are large and grow with increasing scale.
%; they thus cannot meet our goal.
For example, today's reliable network transports maintain per-connection sequence numbers and buffer unacknowledged packets for packet ordering and retransmission, and they grow with the number of connections.
Although swapping between on-chip and off-chip memory is possible, doing so would increase both tail latency and hardware logic complexity, especially under large scale.
%Thus, it is desirable to resort as little as possible to swapping. 
%Achieving the bounded buffer/state goal is even harder when we simultaneously need to meet our scalability goals.


\boldpara{Challenge 3: The hardware pipeline should be deterministic and smooth},
\ie, it uses a bounded, known number of cycles to process a data unit, and for each cycle, the pipeline can take in one new data unit (from the network).
The former would ensure low tail latency, while the latter would guarantee a throughput that could match network line rate.
Another subtle benefit of a deterministic pipeline is that we can know the maximum time a data unit stays at \MN,
which could help bound the size of certain buffers (\eg, \S\ref{sec:clio:ordering}).
However, many traditional hardware solutions are not designed to be deterministic or smooth, and we cannot directly adapt their approaches.
For example, traditional CPU pipelines could have stalls because of data hazards and have non-deterministic latency to handle memory instructions.

To confront these challenges, we took a clean-slate approach by designing \sys's virtual memory system and network system with the following principles that all aim to eliminate state in hardware or bound their performance and space overhead.

\boldpara{Principle 1: Avoid state whenever possible.}
Not all state in server-based solutions is necessary if we could redesign the hardware.
For example, we get rid of RDMA's MR indirection and its metadata altogether
by directly mapping application process' \rspace\ VAs to PAs (instead of to MRs then to PAs).
%Another type of states that we get rid of is network connections.

\boldpara{Principle 2: Moving non-critical operations and state to software and making the hardware fast path deterministic.}
If an operation is non-critical and it involves complex processing logic and/or metadata, our idea is to move it to the software slow path running in an ARM processor.
For example, VA allocation (\alloc) is expected to be a rare operation
because applications know the disaggregated nature and would typically have only a few large allocations during the execution.
Handling \alloc, however, would involve dealing with complex allocation trees.
We thus handle \alloc\ and \sysfree\ in the software slow path.
Furthermore, in order to make the fast path performance deterministic, we {\em decouple} all slow-path tasks from the performance-critical path by {\em asynchronously} performing them in the background.
%Note that page fault is a relatively critical operation, as all first accesses to allocated virtual pages will cause a fault, and applications like serverless computing could access large amounts of (new) memory in a short period of time.



\boldpara{Principle 3: Shifting functionalities and state to \CN{}s.}
While hardware resources are scarce at \MN{}s, \CN{}s have sufficient memory and processing power, and it is faster to develop functionalities in \CN\ software.
A viable solution is to shift state and functionalities from \MN{}s to \CN{}s.
The key question here is how much and what to shift.
%; shifting too much would make \sys\ similar to \pdm\ and suffer from various performance and security issues of \pdm.
Our strategy is to shift functionalities to \CN{}s only if doing so 1) could largely reduce hardware resource consumption at \MN{}s, 2) does not slow down common-case foreground data operations, 3) does not sacrifice security guarantees, and 4) adds bounded memory space and CPU cycle overheads to \CN{}s.
As a tradeoff, the shift may result in certain uncommon operations (\eg, handling a failed request) being slower.
%Our insight here is that we can if a shift results in performance tradeoff only for uncommon cases
%For example, traditional reliable network stacks require maintaining 

\boldpara{Principle 4: Making off-chip data structures efficient and scalable.}
Principles 1 to 3 allow us to reduce \MN\ hardware to only the most essential functionalities and state. 
We store the remaining state in off-chip memory and cache a fixed amount of them in on-chip memory.
Different from most caching solutions, our focus is to make the access to off-chip data structure fast and scalable,
\ie, all cache misses have bounded latency regardless of the number of client processes accessing an \MN\ or the amount of physical memory the \MN\ hosts.

\boldpara{Principle 5: Making the hardware fast path smooth by treating each data unit independently at \MN.}
If data units have dependencies (\eg, must be executed in a certain order), then the fast path cannot always execute a data unit when receiving it.
To handle one data unit per cycle and reach network line rate, we make each data unit independent by including all the information needed to process a unit in it and by allowing \MN{}s to execute data units in any order that they arrive.
%there is no dependency check across data units at \MN{}s.
To deliver our consistency guarantees, we opt for enforcing request ordering at \CN{}s before sending them out.

The rest of this section presents how we follow these principles to design \sys's three main functionalities: memory address translation and protection, page fault handling, and networking. We also briefly discuss our offloading support.

\subsection{Scalable, Fast Address Translation}
\label{sec:clio:addr-trans}
%Similar to traditional virtual memory addressing, we use fix-size pages as VA/PA allocation and address translation unit, while data accesses are in the unit of byte.
Similar to traditional virtual memory systems, we use fix-size pages as address allocation and translation unit, while data accesses are in the granularity of byte.
Despite the similarity in the goal of address translation,
%(\ie, translating a virtual page to a physical page number),
the radix-tree-style, per-address space page table design used by all current architectures~\cite{ecuckoo-asplos20} does not fit \md\ for two reasons.
First, each request from the network could be from a different client process. If each process has its own page table, \MN\ would need to cache and look up many page table roots, causing additional overhead.
Second, a multi-level page table design requires multiple DRAM accesses when there is a translation lookaside buffer (TLB) miss~\cite{hashpgtable-sigmetrics16}.
TLB misses will be much more common in a \md\ environment, since with more applications sharing an \MN, the total working set size is much bigger than that in a single-server setting, while the TLB size in an \MN\ will be similar or even smaller than a single server's TLB (for cost concerns). To make matters worse, each DRAM access is more costly for systems like RDMA NIC which has to cross the PCIe bus to access the page table in main memory~\cite{Pythia,pcie-sigcomm}.

\ulinebfpara{Flat, single page table design (Principle 4).}~~
We propose a new {\em overflow-free} hash-based page table design that sets the total page table size according to the physical memory size 
and bounds \textit{address translation to at most one DRAM access}.
Specifically, we store {\em all} page table entries (PTEs) from {\em all} processes in a single hash table 
whose size is proportional to the physical memory size of an \MN. 
The location of this page table is fixed in the off-chip DRAM and is known by the fast path address translation unit, thus avoiding any lookups.
%We create the page table to always have enough entries to cover the entire physical memory.
As we anticipate applications to allocate big chunks of VAs in their \rspace, we use huge pages and support a configurable set of page sizes.
%Each PTE is 8 bytes, 
With the default 4\MB\ page size, the hash table consumes only 0.4\%\ of the physical memory.
%For example, for 1\TB\ physical memory and 4\MB\ page size, the whole page table is only 4\MB\ (each PTE is 8 bytes).

{
\begin{figure}[t]
\begin{center}
\centerline{\includegraphics[width=\columnwidth]{Figures/CoreMem.pdf}}
%\vspace{-0.1in}
\mycaption{fig-coremem}{\sysboard\ Design.}
{
%Thick lines represent data payload. Thin lines represent header information.
%Double line represents metadata from ARM.
%Dashed line represents internal operations.
Green, yellow, and red areas are anticipated to be built with 
ASIC, FPGA, and low-power cores.
}
\end{center}
%\end{minipage}
%\vspace{-0.2in}
\end{figure}
}


The hash value of a VA and its PID is used as the index to determine which hash bucket the corresponding PTE goes to.
Each hash bucket has a fixed number of ($K$) slots.
To access the page table, we always fetch
the entire bucket including all $K$ slots in a single DRAM access.
%Normally, a hash table with fixed slots will have an overflow problem because of hash conflicts (\eg, when more than $K$ VA+PID combinations have the same hash value).

A well-known problem with hash-based page table design is hash collisions that could overflow a bucket.
Existing hash-based page table designs rely on collision chaining~\cite{TransCache-isca10} or open addressing~\cite{hashpgtable-sigmetrics16} to handle overflows, both require multiple DRAM accesses or even costly software intervention.
In order to bound address translation to at most one DRAM access, we use a novel technique to avoid hash overflows at \textit{VA allocation time}.

\ulinebfpara{VA allocation (Principle 2).}~~
The slow path software handles \alloc\ requests and allocates VA.
The software allocator maintains a per-process VA allocation tree that records allocated VA ranges and permissions, similar to the Linux vma tree~\cite{linux-rb-vma}.
To allocate size $k$ of VAs, it first finds an available address range of size $k$ in the tree.
It then calculates the hash values of the virtual pages in this address range
and checks if inserting them to the page table would cause any hash overflow. 
If so, it %marks the failed virtual pages in the tree as ``unusable'' and 
does another search for available VAs.
These steps repeat until it finds a valid VA range that does not cause hash overflow.
%It then send these VA page numbers (VPN) to the hardware fast path, which inserts the corresponding PTEs (invalid PTEs without PAs) to the page table. The last step is crucial for fast page fault handling in the hardware.
%It then send these VA page numbers and their permissions to the hardware fast path, which establishes the corresponding PTEs (with invalid state, \ie, with permission set up but no PAs). %The last step is crucial for fast page fault handling in the hardware as it enables in-line VA permission checking.

Our design trades potential retry overhead at allocation time (at the slow path) for better run-time performance and simpler hardware design (at the fast path).
% core-memory implementation.
This overhead is manageable because
1) each retry takes only a few microseconds with our implementation (\S\ref{sec:clio:impl}),
%is fast even when running at ARM processor (\mus\ level), 
2) we employ huge pages, which means fewer pages need to be allocated, 
3) we choose a hash function that has very low collision rate~\cite{lookup3-wiki},
and 4) we set the page table to have extra slots (2\x\ by default) which absorbs most overflows.
%and 5) the \sys\ 64-bit virtual address space is huge.
%In fact, none of our evaluated application's \alloc\ requests ever require retry, even when the \alloc\ size is as huge as 1\TB.
We find no conflicts when memory is below half utilized and has only up to 60 retries when memory is close to full (Figure~\ref{fig-alloc-conflict}).

\ulinebfpara{TLB.}~~
%and its consistency with the page table.}
\sys\ implements a TLB in a fix-sized on-chip memory area and looks it up using content-addressable-memory in the fast path.
%and use LRU for replacement, 
On a TLB miss, the fast path fetches the PTE from off-chip memory and inserts it to the TLB by replacing an existing TLB entry with the LRU policy.
When updating a PTE, the fast path also updates the TLB, in a way that ensures the consistency of inflight operations.
%
%As with traditional TLB, \sys\ also faces a consistency problem between TLB and the page table.
%Traditionally, the OS needs to shootdown TLBs after modifying PTEs~\cite{tlbshootdown-eurosys20}.
%\sys's software slow path is the party that modifies a PTE, \eg, deleting a PTE when handling \sysfree.
%With our separate fast- and slow-path design, there could be potential %consistency problems when both paths try to access/modify the page table.
%The first consistency problem happens when the slow path needs to update the %page table (\eg, when handling a \sysfree).
%between slow-path generated PTE changes (update or delete) and fast-path TLB.
%Instead of letting the slow path change the page table in DRAM and shootdown the TLB in the on-chip memory,
%we let the hardware fast path pipeline handle all page table {\em and} TLB modifications.
%With the latter, it is easier to ensure the consistency of inflight fast-path operations, as they are in the same pipeline as the TLB/page-table modification logic.
%\zhiyuan{Is the above correct?}
%If it directly applies the change to the page table in DRAM, then the TLB will be inconsistent.
%Today's computer relies on the OS to perform costly~\cite{XXX} TLB shootdowns for TLB consistency.
%Following our design of only having the fast path managing TLB,
%To solve this problem, we adopt a simple principle: the fast path is the only unit accessing and changing TLB {\em and} the page table.
%We use a different approach where
%Thus, the slow path hands over its PTE change requests to the fast path, 
%which performs both the TLB change and the PTE change in its pipeline in a way that is consistent for inflight fast-path operations.

\ulinebfpara{Limitation.}~~
A downside of our overflow-free VA allocation design is that it cannot guarantee that a specific VA can be inserted into the page table. This is not a problem for regular VA allocation but could be problematic for allocations that require a fixed VA (\eg, \texttt{mmap(MAP\_FIXED})). 
Currently, \sys\ finds a new VA range if the user-specified range cannot be inserted into the page table. Applications that must map at fixed VAs (\eg, libraries) will need to use \CN-local memory.


\subsection{Low-Tail-Latency Page Fault Handling}

A key reason to disaggregate memory is to consolidate memory usages on less DRAM so that memory utilization is higher and the total monetary cost is lower (\textbf{R1}). Thus, remote memory space is desired to run close to full capacity, and we allow memory over-commitment at an \MN, necessitating page fault handling. Meanwhile, applications like JVM-based ones allocate a large heap memory space at the startup time and then slowly use it to allocate smaller objects. Similarly, many existing far-memory systems~\cite{Tsai20-ATC,AIFM,FaRM} allocate a big chunk of remote memory and then use different parts of it for smaller objects to avoid frequently triggering the slow remote allocation operation.
In these cases, it is desirable for a \md\ system to delay the allocation of physical memory to when the memory is actually used (\ie, {\em on-demand} allocation) or to ``reshape'' memory~\cite{cliquemap-sigcomm21} during runtime, necessitating page fault handling.

Page faults are traditionally signaled by the hardware and handled by the OS. %, and they can happen when a PTE is invalid (VA created, PA not allocated)
%or when there is a permission violation. 
%A common case of page faults are {\em first-time accesses}, which requires on-demand physical memory allocation.
%While the latter is uncommon, the former happens at every initial access to a VA and could be common 
%First-time accesses can be common in applications like serverless computing and microservices which frequently start many short running processes and incur many initial-access page faults.
%Unfortunately, today's page fault handling mechanism is slow because of the costly interrupt and trap-to-kernel process.
This is a slow process because of the costly interrupt and kernel-trapping flow.
For example, a remote page fault via RDMA costs 16.8\ms\ from our experiments using Mellanox ConnectX-4.
To avoid page faults, most RDMA-based systems pre-allocate big chunks of physical memory and pin them physically.
However, doing so results in memory wastes and makes it hard for an \MN\ to pack more applications, violating \textbf{R1} and \textbf{R2}.

%\boldpara{Decoupling PA allocation from page fault handling.}
%To meet \textbf{R2}, \textbf{R3}, and \textbf{R6}, 
We propose to {\em handle page faults in hardware and with bounded latency}\textemdash a {\em constant three cycles} to be more specific with our implementation of \sysboard.
%Achieving this performance is not easy.
%While handling permission-violation faults in hardware is easy (just by sending an error message as the request response),
%Particularly, 
Handling initial-access faults in hardware is challenging, as initial accesses require PA allocation, which is a slow operation that involves manipulating complex data structures.
Thus, we handle PA allocation in the slow path (\textbf{Challenge 1}).
However, if the fast-path page fault handler has to wait for the slow path to generate a PA for each page fault,
it will slow down the data plane.
%As allocation is performed by ARM, fetching the allocation results via the slow path between FPGA and ARM would hugely affect foreground performance.

To solve this problem, we propose an asynchronous design to shift PA allocation off the performance-critical path (\textbf{Principle 2}).
Specifically, we maintain a set of {\em free physical page numbers} in an {\em async buffer},
which the ARM continuously fulfills by finding free physical page addresses and reserving them without actually using the pages. % actual allocation.
During a page fault, the page fault handler simply fetches a pre-allocated physical page address. % of the corresponding page size.
%This asynchronous design enables us to avoid the long wait for ARM to do an allocation on the fly.
Note that even though a single PA allocation operation has a non-trivial delay, 
the throughput of generating PAs and filling the async buffer is higher than network line rate.
%the throughput of generating PAs and filling the async buffer is higher than page fault rate.
Thus, the fast path can always find free PAs in the async buffer in time.
After getting a PA from the async buffer and establishing a valid PTE, %(promoted from the specially marked invalid state), 
the page fault handler performs three tasks in parallel: 
writing the PTE to the off-chip page table, inserting the PTE to the TLB,
and continuing the original faulting request.
This parallel design hides the performance overhead of the first two tasks, allowing foreground requests to proceed immediately.

A recent work~\cite{lee-isca20} also handles page faults in hardware. 
%Its goal, however, is to accelerate data fetching from storage and focuses 
Its focus is on the complex interaction with kernel and storage devices, and it is a simulation-only work. \sys\ uses a different design for handling page faults in hardware with the goal of low tail latency, and we built it in FPGA.

\ulinebfpara{Putting the virtual memory system together.}~~
We illustrate how \sysboard{}'s virtual memory system works using a simple example of allocating some memory and writing to it.
The first step (\alloc) is handled by the slow path, which allocates a VA range by finding an available set of slots in the hash page table.
The slow path forwards the new PTEs to the fast path, which inserts them to the page table.
At this point, the PTEs are invalid.
This VA range is returned to the client.
When the client performs the first write, the request goes to the fast path.
There will be a TLB miss, followed by a fetch of the PTE.
Since the PTE is invalid, the page fault handler will be triggered,
which fetches a free PA from the async buffer and establishes the valid PTE.
It will then execute the write, update the page table, and insert the PTE to TLB.

\subsection{Asymmetric Network Tailored for \md}
\label{sec:clio:network}
With large amounts of research and development efforts, today's data-center network systems are highly optimized in their performance.
Our goal of \sys's network system is unique and fits \md's requirements\textemdash minimizing the network stack's hardware resource consumption at \MN{}s and achieving great scalability while maintaining similar performance as today's fast network.
Traditional software-based reliable transports like Linux TCP incurs high performance overhead.
Today's hardware-based reliable transports like RDMA are fast, but they require a fair amount of on-chip memory to maintain state, \eg, per-connection sequence numbers, congestion state~\cite{TONIC}, and bitmaps~\cite{IRN,MELO-APNet}, not meeting our low-cost goal.
%and buffers (\eg, the unacknowledged packet buffer, packet reorder buffer)
%Maintaining them in on-chip memory is not a viable solution because of \textbf{Challenge 2} (cost). 

Our insight is that different from general-purpose network communication where each endpoint can be both the sender (requester) and the receiver (responder) that exchange general-purpose messages,
\MN{}s only respond to requests sent by \CN{}s (except for memory migration from one \MN\ to another \MN\ (\S\ref{sec:clio:dist}), in which case we use another simple protocol to achieve the similar goal).
Moreover, these requests are all memory-related operations that have their specific properties.
With these insights, we design a new network system with two main ideas.
%and does so without the need to change any existing data-center network infrastructure.
Our first idea is to maintain transport logic, state, and data buffers only at \CN{}s,
essentially making \MN{}s ``transportless'' (\textbf{Principle 3}). 
%\footnote{1RMA~\cite{1RMA}, a recent server-based remote memory system, onloads most of its retransmission and congestion logic from the NIC to the host CPU. As a result, 1RMA's NIC is simple.
%and connectionless. 
%However, 1RMA relies on a companion host to function, violating the ``server-less'' goal of \MN{}s.}.
Our second idea is to relax the reliability of the transport and instead enforce ordering and loss recovery at the memory request level, so that \MN{}s' hardware pipeline can process data units as soon as they arrive (\textbf{Principle 5}).

With these ideas, we implemented a transport in \syslib\ at \CN{}s. \syslib\ bypasses the kernel to directly issue raw Ethernet requests to an Ethernet NIC.
\CN{}s use regular, commodity Ethernet NICs and regular Ethernet switches to connect to \MN{}s.
\MN{}s include only standard Ethernet physical, link, and network layers and a slim layer for handling corner-case requests (\S\ref{sec:clio:ordering}).
%\yizhou{We use lossless Ethernet enabled by Priority Flow Control (PFC) as it reduces packet loss and retry rate. PFC comes with a set of issues~\cite{DCQCN-sigcomm15,hpcc-sigcomm19}. Hence we design our protocol to avoid triggering it as much as possible.}
We now describe our detailed design.

%Both \CN{}s and \MN{}s only need standard Ethernet physical and link layers in the hardware.
%Thus, \CN{} servers can continue using regular Ethernet-based NICs, and \MN{}s can be built with low cost.

\ulinebfpara{Removing connections with request-response semantics.}
Connections (\ie, QPs) are a major scalability issue with RDMA.
Similar to recent works~\cite{Homa,1RMA}, we make our network system connection-less using request-response pairs.
Applications running at \CN{}s directly initiate \sys\ APIs to an \MN\ without any connections.
%Since all \sys\ operations are in the RPC style initiated by client servers, \ie, sending read/write requests and getting data/response back,
%Similar to Homa~\cite{Homa}, 
\syslib\ assigns a unique request ID to each request. The \MN\ attaches the same request ID when sending the response back. \syslib\ uses responses as ACKs and matches a response with an outstanding request using the request ID.
Neither \CN{}s nor \MN{}s send ACKs.
%the response of each \sys\ request as the ACK and matches it to the request using a request ID.

\ulinebfpara{Lifting reliability to the memory request level.}
Instead of triggering a retransmission protocol for every lost/corrupted packet at the transport layer, 
\syslib\ retries the entire memory request if any packet is lost or corrupted in the sending or the receiving direction.
On the receiving path, \MN{}'s network stack only checks a packet's integrity at the link layer. If a packet is corrupted, the \MN\ immediately sends a NACK to the sender \CN.
%, otherwise the packet is delivered to the address translation module.
\syslib\ retries a memory request if one of three situations happens: a NACK is received, the response from \MN\ is corrupted, or no response is received within a \texttt{TIMEOUT} period.
%dynamic retransmission timeout (RTO) period. The RTO is computed %on a per-path basis 
%using the moving average of prior end-to-end RTTs.
%We also avoid complex logic/states to detect lost packets and simply determine the failure of a request if no response is received in a \texttt{TIMEOUT} period.
%We rely on standard link-layer mechanisms to correct packet corruptions and report uncorrectable ones~\cite{FEC}.
In addition to lifting retransmission from transport to the request level, we also lift ordering to the memory request level
and allow out-of-order packet delivery (see details in \S\ref{sec:clio:ordering}).

%\boldpara{\CN-controlled congestion and incast.}
\ulinebfpara{\CN-managed congestion and incast control.}
Our goal of controlling congestion in the network and handling incast that can happen both at a \CN\ and an \MN\ is to minimize state at \MN.
To this end, we build the entire congestion and incast control at the \CN\ in the \syslib.
%\sys\ performs congestion control (CC) at \syslib\ to 1) keep \MN{}s stateless, and 2) make it easy for users to change the CC policy.
%Our current CC policy exploits the fact that \CN{}s know the sizes of both requests and expected responses to control congestion on both the outgoing and incoming directions at \CN{}s.
To control congestion, \syslib\ adopts a simple delay-based, reactive policy that uses end-to-end RTT delay as the congestion signal, similar to recent sender-managed, delay-based mechanisms~\cite{mittal2015timely,swift-sigcomm,1RMA}.
%We onload CC and implement it using software to keep it flexible and easier to adopt new policies. To this end, we use a simple delay-based, reactive CC mechanism at \syslib.
%End-to-end RTT delay is our primary congestion signal.
%Delay has been shown to be an effective and robust signal in the heterogeneous datacenter environments~\cite{mittal2015timely,swift-sigcomm,1RMA}.
%It requires no special switch features such as ECN~\cite{DCQCN} or INT~\cite{HPCC}.
Each \CN\ maintains one congestion window, \textit{cwnd}, per \MN\ 
%which is shared by all processes accessing the \MN.
%It 
that controls the maximum number of outstanding requests that can be made to the \MN\ from this \CN.
We adjust \textit{cwnd} based on measured delay using a standard Additive Increase Multiplicative Decrease (AIMD) algorithm.

To handle incast to a \CN, we exploit the fact that the \CN{} knows the sizes of expected responses for the requests that it sends out and that responses are the major incoming traffic to it.
Each \syslib\ maintains one incast window, \textit{iwnd}, which controls the maximum bytes of expected responses. \syslib\ sends a request only when both \textit{cwnd} and \textit{iwnd} have room.

Handling incast to an \MN\ is more challenging, as we cannot throttle incoming traffic at the \MN\ side or would otherwise maintain state at \MN{}s.
To have \CN{}s handle incast to \MN{}s, we draw inspiration from Swift~\cite{swift-sigcomm} by allowing \textit{cwnd} to fall below one packet when long delay is observed at a \CN. For example, a \textit{cwnd} of 0.1 means that the \CN\ can only send a packet within 10 RTTs.
Essentially, this situation happens when the network between a \CN\ and an \MN\ is really congested, and the only way is to slow the sending speed.

\subsection{Request Ordering and Data Consistency}
\label{sec:clio:ordering}

As explained in \S\ref{sec:clio:abstraction}, \sys\ supports both synchronous and asynchronous remote memory APIs, with the former following a sequential, one-at-a-time order in a thread and the latter following a release order in a thread.
Furthermore, \sys\ provides synchronization primitives for inter-thread consistency.
We now discuss how \sys\ achieves these correctness guarantees by presenting our mechanisms for handling intra-request intra-thread ordering, inter-request intra-thread ordering, inter-thread consistency, and retries.
At the end, we will provide the rationales behind our design.

One difficulty in designing the request ordering and consistency mechanisms is our relaxed network ordering guarantees, 
which we adopt to minimize the hardware resource consumption for the network layer at \MN{}s (\S\ref{sec:clio:network}).
On an asynchronous network, it is generally hard to guarantee any type of request ordering when there can be multiple outstanding requests (either multiple threads accessing shared memory or a single thread issuing multiple asynchronous APIs). It is even harder for \sys\ because we aim to make \MN\ stateless as much as possible.
%Allowing memory requests to be asynchronous Ensuring even a weaker consistency level like release consistency is hard when the network can reorder and drop packets and when we aim to minimize states at \MN{}s.
Our general approaches are 1) using \CN{}s to ensure that no two concurrently outstanding requests are dependent on each other, and 2) using \MN{}s to ensure that every user request is only executed once even in the event of retries.

%\zhiyuan{We design our consistency model based on the requirements of disaggregation and the efficiency of hardware design.
%Most desegregation use cases (e.g. with local caches) rarely send out memory access requests with dependency, so strict ordering can be an overkill; This allows us to apply released consistency model and simplify hardware design} 

%\zhiyuan{We apply an released consistency similar to ARMv8, which allows 
%which dependent requests in Clio only includes memory fences and requests that operates on the same address. We also provide synchronization primitives, \syslock and \fence between threads. The execution order of other requests and request from different threads. can be reordered at \CN{}, network or \MN{}.}

%\zhiyuan{We apply two rules to implement our consistency model. First, for a client, all committed but not completed requests must not have dependency; Second, At MN side, all request must be executed and only executed once. The rules ensures the correctness of the implementation.}

\ulinebfpara{Allowing intra-request packet re-ordering (T1).}
%\zhiyuan{We use client side libraries to ensure first rule.}
%Packet reordering can happen in the network, \eg, due to data-center multipath routing~\cite{ECMP}. %which can happen at the link layer.
A request or a response in \sys\ can contain multiple link-layer packets. 
Enforcing packet ordering above the link layer normally requires maintaining state (\eg, packet sequence ID) at both the sender and the receiver.
To avoid maintaining such state at \MN{}s,
our approach is to deal with packet reordering only at \CN{}s in \syslib\ (\textbf{Principle 3}).
Specifically, \syslib\ splits a request that is bigger than link-layer maximum transmission unit (MTU) into several link-layer packets
and attaches a \sys\ header to each packet, which includes sender-receiver addresses, a request ID, and request type.
This enables the \MN{} to treat each packet independently (\textbf{Principle 5}).
It executes packets as soon as they arrive, even if they are not in the sending order.
This out-of-order data placement semantic is in line with RDMA specification~\cite{IRN}. 
Note that only write requests will be bigger than MTU, and the order of data writing within a write request does not affect correctness as long as proper {\em inter-request} ordering is followed.
When a \CN\ receives multiple link-layer packets belonging to the same request response, 
\syslib\ reassembles them before delivering them to the application.


\ulinebfpara{Enforcing intra-thread inter-request ordering at \CN\ (T2).}
Since only one synchronous request can be outstanding in a thread, there cannot be any inter-request reordering problem.
%Synchronous requests follow strict program order with only at most one outstanding request.
%and we do not need to take any special measurement for synchronous request ordering.
On the other hand, there can be multiple outstanding asynchronous requests.
Our provided consistency level disallows concurrent asynchronous requests that are dependent on each other (WAW, RAW, or WAR).
In addition, all requests must complete before \release.



We enforce these ordering requirements at \CN{}s in \syslib\ instead of at \MN{}s (\textbf{Principle 3}) for two reasons.
First, enforcing ordering at \MN{}s requires more on-chip memory and complex logic in hardware.
Second, even if we enforce ordering at \MN{}s, network reordering would still break end-to-end ordering guarantees.
%To enforce ordering for asynchronous operations at the client side, 

Specifically, \syslib\ keeps track of all inflight requests and matches every new request's virtual page number (VPN) to the inflight ones'. 
If a WAR, RAW, or WAW dependency is detected, \syslib\ blocks the new request until the conflicting request finishes.
When \syslib\ sees a \release\ operation, it waits until all inflight requests return or time out.
We currently track dependencies at the page granularity mainly to reduce tracking complexity and metadata overhead. The downside is that false dependencies could happen (\eg, two accesses to the same page but different addresses).
False dependencies could be reduced by dynamically adapting the tracking granularity if application access patterns are tracked\textemdash we leave this improvement for future work.
%tailored for application needs.
%In reality, this is not a problem, as with a low-latency system like \sys, the amount of outstanding requests is small and the chance of two outstanding requests accessing the same page is extremely rare.
%, which largely depends on application usage.
%Intuitively, it is problematic for data structure systems~\cite{AIFM} with small granularity access, but works well for others with larger accesses. We leave this optimization for future work.
%For synchronous operations, \syslib\ only returns when a request gets a response, effectively achieving strict ordering.

\ulinebfpara{Inter-thread/process consistency (T3).}
%\zhiyuan{At the \MN{} side, we ensures the semantic of execute once.}
%\zhiyuan{Move the retry to earlier position?}
Multi-threaded or multi-process concurrent programming on \sys\ could use the synchronization primitives \sys\ provides to ensure data consistency (\S\ref{sec:clio:abstraction}).
%, \eg, by protecting a critical section using \syslock.
We implemented all synchronization primitives like \syslock\ and \fence\ at \MN,
because they need to work across threads and processes that possibly reside on different \CN{}s.
Before a request enters either the fast or the slow paths, 
\MN\ checks if it is a synchronization primitive.
%an atomic operation or a \fence.
For primitives like \syslock\ that internally is implemented using atomic operations like \texttt{TAS}, \MN\ blocks future atomic operations until the current one completes.
For \fence, \MN\ blocks all future requests until all inflight ones complete.
Synchronization primitives are one of the only two cases where \MN\ needs to maintain state.
%As these operations operation executes in bounded time, the hardware resources for states are bounded.
As these operations are infrequent and each of these operations executes in bounded time, the hardware resources for maintaining their state are minimal and bounded.

\ulinebfpara{Handling retries (T4).}
\syslib\ retries a request after a \texttt{TIMEOUT} period without receiving any response. Potential consistency problems could happen as \sysboard\ could execute a retried write after the data is written by another write request thus undoing this other request's write. Such situations could happen when the original request's response is lost or delayed and/or when the network reorders packets. 
%Such situation would violate \sys's consistency guarantees.
%Such situations could happen because the network could reorder packets and because we support multiple concurrent processes sharing the same data.
%
We use two techniques to solve this problem.

First, \syslib\ attaches a new request ID to each retry, essentially making it a new request with its own matching response. Together with \syslib's ordering enforcement, it ensures that there is only one outstanding request (or a retry) at any time.
Second, we maintain a small buffer at \MN\ to record the request IDs of recently executed writes and atomic APIs and the results of the atomic APIs. A retry attaches its own request ID and the ID of the failed request. If \MN\ finds a match of the latter in the buffer, it will not execute the request. For atomic APIs, it sends the cached result as the response. We set this buffer's size to be 3$\times$\texttt{TIMEOUT}$\times$\textit{bandwidth}, which is 30\KB\ in our setting. It is one of the only two types of state \MN\ maintains and does not affect the scalability of \MN, since its size is statically associated with the link bandwidth and the \texttt{TIMEOUT} value.
With this size, the \MN\ can ``remember'' an operation long enough for two retries from the \CN. Only when both retries and the original request all fail, the \MN\ will fail to properly handle a future retry. This case is extremely rare~\cite{Homa}, and we report the error to the application, similar to \cite{Kalia14-RDMAKV,1RMA}.

\ulinebfpara{Why T1 to T4?}
We now briefly discuss the rationale behind why we need all T1 to T4 to properly deliver our consistency guarantees. 
First, assume that there is no packet loss or corruption (\ie, no retry) but the network can reorder packets. 
In this case, using T1 and T2 alone is enough to guarantee the proper ordering of \sys\ memory operations, since they guarantee that network reordering will only affect either packets within the same request or requests that are not dependent on each other.
T3 guarantees the correctness of synchronization primitives since the \MN\ is the serialization point and is where these primitives are executed.
Now, consider the case where there are retries.
Because of the asynchronous network, a timed-out request could just be slow and still reach the \MN, either before or after the execution of the retried request. If another request is executed in between the original and the retried requests, inconsistency could happen (\eg, losing the data of this other request if it is a write). The root cause of this problem is that one request can be executed twice when it is retried.
T4 solves this problem by ensuring that the \MN\ only executes a request once even if it is retried.


\subsection{Extension and Offloading Support}
\label{sec:clio:extended}
To avoid network round trips when working with complex data structures and/or performing data-intensive operations,
we extend the core \MN\ to support application computation offloading in the extend path.
%which includes an FPGA chip and the ARM processor.
%We only have space to give a high-level overview of the extend path, leaving details to a follow-on paper.
Users can write and deploy application offloads both in FPGA and in software (run in the ARM).
%An offload can either be the handler of a high-level API (\eg, pointer chasing) or an entire function (\eg, data filtering).
To ease the development of offloads, \sys\ offers the same virtual memory interface as the one to applications running at \CN{}s.
Each offload has its own PID and virtual memory address space, and they use the same virtual memory APIs (\S\ref{sec:clio:abstraction}) to access on-board memory. It could also share data with processes running at \CN{}s in the same way that two \CN\ processes share memory.
%Developing offloads is thus closer to traditional multi-threaded programming (in terms of memory accesses).
Finally, an offload’s data and control paths could be split to FPGA and ARM and use the same async-buffer mechanism for communication between them. 
These unique designs made developing computation offloads easier and closer to traditional multi-threaded software programming.


\subsection{Distributed \MN{}s}
\label{sec:clio:dist}
Our discussion so far focused on a single \MN\ (\sysboard).
To more efficiently use remote memory space and to allow one application to use more memory than what one \sysboard\ can offer, we extend the single-\MN\ design to a distributed one with multiple \MN{}s.
Specifically, an application process' \rspace\ can span multiple \MN{}s, and one \MN\ can host multiple \rspace{}s.
We adopt LegoOS' two-level distributed virtual memory management approach to manage distributed \MN{}s in \sys.
A global controller manages \rspace{}s in coarse granularity (assigning 1\GB\ virtual memory regions to different \MN{}s).
Each \MN\ then manages the assigned regions at fine granularity.

The main difference between LegoOS and \sys's distributed memory system is that in \sys, each \MN\ can be over-committed (\ie, allocating more virtual memory than its physical memory size), and when an \MN\ is under memory pressure, it migrates data to another \MN\ that is less pressured (coordinated by the global controller).
The traditional way of providing memory over-commitment is through memory swapping, which could be potentially implemented by swapping memory between \MN{}s. 
However, swapping would cause performance impact on the data path and add complexity to the hardware implementation.
Instead of swapping, we \textit{proactively} migrate a rarely accessed memory region to another \MN\ when an \MN\ is under memory pressure (its free physical memory space is below a threshold).
During migration, we pause all client requests to the region being migrated.
With our 10\Gbps\ experimental board, migrating a 1\GB\ region takes 1.3 second.
Migration happens rarely and, unlike swapping, happens in the background.
Thus, it has little disturbance to foreground application performance.



\section{\sys\ Implementation}
\label{sec:impl}

Apart from challenges discussed in \S\ref{sec:design}, our implementation of \sys\ also needs to overcome several practical challenges, for example, how can different hardware components most efficiently work together in \sysboard, how to minimize software overhead in \syslib. 
This section describes how we implemented \sysboard\ and \syslib, focusing on the new techniques we designed to overcome these challenges.
Currently, \sys\ consists of 24.6K SLOC (excluding computation offloads and third-party IPs).
They include 5.6K SLOC in SpinalHDL~\cite{SpinalHDL} and 2K in C HLS for FPGA hardware, and 17K in C for \syslib\ and ARM software.
We use vendor-supplied interconnect and DDR IPs, and an open-source MAC and PHY network stack~\cite{Corundum-FCCM20}.

\ulinebfpara{\sysboard\ Prototyping.}~~
%\label{sec:impl}
\if 0
By design (Figure~\ref{fig-coremem}), a \sysboard\ consists of an data path ASIC, a small FPGA, and a few low-power cores,
a network interface (at least one port of 100\Gbps\ or higher), and an array of off-chip DRAMs of at least hundreds of GBs.
All components except the DRAMs are expected to be integrated into a single chip.
The ASIC is dedicated to fixed logics including \sys's virtual memory fast path and network system (L1+L2 MAC).
The software cores run the virtual memory and offloads' slow paths.
The FPGA runs offloads' fast paths.
\fi
We prototyped \sysboard\ with a low-cost (\$2495 retail price) Xilinx MPSoC board~\cite{ZCU106} and build the hardware fast path (which is anticipated to be built in ASIC) with FPGA.
All \sys's FPGA modules run at 250\,MHz clock frequency and 512-bit data width.
They all %(network, pre-processor, core memory) are able to 
achieve an {\em Initiation Interval} ({\em II}) of one
(II is the number of clock cycles between the start time
of consecutive loop iterations, and it decides the maximum
achievable bandwidth). Achieving II of one is not easy and
requires careful pipeline design in all the modules. With II one, our data path can
achieve a maximum of 128\Gbps\ throughput even with just the slower FPGA clock frequency and would be higher with real ASIC implementation.

Our prototyping board consists of a small FPGA with 504K logic cells (LUTs) and 4.75\MB\ FPGA memory (BRAM),
a quad-core ARM Cortex-A53 processor,
two 10\Gbps\ SFP+ ports connected to the FPGA, 
and 2\GB\ of off-chip on-board memory.
This board has several differences from our anticipated real \sysboard:
its network port bandwidth and on-board memory size are both much lower than our target,
and like all FPGA prototypes, its clock frequency is much lower than real ASIC.
Unfortunately, no board on the market offers the combination of small FPGA/ARM (required for low cost) 
and large memory and high-speed network ports. %high-speed network and/or large amounts of on-board DRAM with small processing units,
%but building one only requires board-level changes.

%A \sys\ memory board includes a small FPGA, some ARM cores, and some DRAM chips.
%Figure~\ref{fig-board} shows an overview of \sys's board design.
Nonetheless, certain features of this board are likely to exist in a real \sysboard,
and these features guide our implementation.
Its ARM processor and the FPGA connect through an interconnect that has high bandwidth (90\GB/s) but high delay (40\mus).
Although better interconnects could be built, crossing ARM and FPGA would inevitably incur non-trivial latency.
%that allows the FPGA to perform DMA operations to 
%ARM's local memory (the ARM has \fixme{XXX} local DRAM). % (\fixme{XXX} RTT and \fixme{XXX} bandwidth).
With this board, the ARM's access to on-board DRAM is much slower than the FPGA's access because the ARM has to first physically cross the FPGA then to the DRAM.
%We envision each board to host 100\GB{}s of DRAM.
%the physical connection between software cores and ASIC/FPGA is likely to continue having high bandwidth but non-trivial delay,
A better design would connect the ARM directly to the DRAM, 
but it will still be slower for the ARM to access on-board DRAM than its local on-chip memory.

%We use our prototype board's FPGA to implement both the ASIC and the FPGA in our design.
%(\ie,  clock cycle down the pipeline).
%(II is the number of clock cycles between the start time of consecutive loop iterations,
%and it decides the maximum achievable throughput).
%Achieving an II of one is not easy and requires careful pipeline design in all the modules.
%which we omit because of space constraint. %do not have paper space to cover.
%As a result, our FPGA path can achieve a maximum of 128\Gbps\ throughput. 

%Below, we pick some techniques used in our prototyping implementation that will still be applicable in a real \sysboard.
%Physical links between ARM and FPGA and between ARM and on-board DRAM are slow both in latency.
%First, thanks to our overall design of a performance-deterministic and low-tail-latency fast path,
%we can carefully allocate small, bounded buffers in the pipeline to 
%handle slower operations like PTE fetch (which takes one DRAM access time).
%

To mitigate the problem of slow accesses to on-board DRAM from ARM,
we maintain shadow copies of metadata at ARM's local DRAM.
%to avoid the much higher (79\x\ in our experiment) cost of going to the on-board DRAM.
For example, we store a {\em shadow} version of the page table in ARM's local memory,
so that the control path can read page table content faster.
When the control path needs to perform a virtual memory space allocation, it reads the shadow page table to test if an address would cause an overflow (\S\ref{sec:addr-trans}).
We keep the shadow page table in sync with the real page table by updating both tables when adding, removing, or updating the page table entries.
%
%first, we shift operations involving ARM off the performance-critical path.

In addition to maintaining shadow metadata, we employ an efficient polling mechanism for ARM/FPGA communication.
We dedicate one ARM core to busy poll an RX ring buffer between ARM and FPGA,
where the FPGA posts tasks for ARM.
This polling thread hands over tasks to other worker threads for task handling %that perform the tasks
and post responses to a TX ring buffer.
%We use DMA to implement the ring buffers, 
%as DMA is the fastest communication methods we found among all the available ones.

%matching future high-bandwidth networks.
\sysboard's network stack builds on top of standard, vendor-supplied Ethernet physical and link-layer IPs, with just an additional thin checksum-verify and ack-generation layer on top.
This layer uses much fewer resources compared to a normal RDMA-like stack (\S\ref{sec:results-cost}).
%
We use lossless Ethernet with Priority Flow Control (PFC) for less packet loss and retransmission. Since PFC has issues like head-of-line blocking~\cite{DCQCN-sigcomm15,hpcc-sigcomm19,alibaba-rdma-nsdi21,IRN}, we rely on our congestion and incast control to avoid triggering PFC as much as possible.

Finally, to assist \sys\ users in building their applications, we implemented a simple software simulator
of \sysboard\ which works with \syslib\ for developers to test their code without the need to run an actual \sysboard.

%\if 0
\ulinebfpara{\syslib\ Implementation.}~~
Even though we optimize the performance of \sysboard, the end-to-end application performance can still be hugely impacted if the host software component (\syslib) is not as fast.
Thus, our \syslib\ implementation aims to provide low-latency performance by adopting several ideas (e.g., data inlining, doorbell batching) from recent low-latency I/O solutions~\cite{ERPC,Kalia14-RDMAKV,Kalia16-ATC,Tsai17-SOSP,Shinjuku,Shenango,demikernel-sosp21}.
We implemented \syslib\ in the user space. 
It has three parts: a user-facing request ordering layer that performs dependency check and ordering of address-conflicting requests,
a transport layer that performs congestion/incast control and request-level retransmission, 
and a low-level device driver layer that interacts with the NIC (similar to DPDK~\cite{DPDK} but simpler).
\syslib\ bypasses kernel and directly issues raw Ethernet requests to the NIC with zero memory copy.
%The transport layer implements our core logic. The shim layer is similar to DPDK but much simplified and customized for our own usage. 
%We use per-thread inline polling.
For synchronous APIs, we let the requesting thread poll the NIC for receiving the response right after each request.
For asynchronous APIs, the application thread proceeds with other computations after issuing the request and only busy polls when the program calls \poll.
%\fi

\if 0
Requests coming into the board first go through a thin network stack followed by a command pre-processor. 
The pre-processor serves as a coordinator across different components.
It uses a match-and-action table (MAT) to decide which component to route the request to.
It then detects conflicts among operations with different destination components 
and properly sequences them to deliver a specific synchronization guarantee.
%The action part of the MAT specifies what operations would cause a conflict and how to sequence them.
That is, we use the pre-processor to guarantee inter-component synchronization
and leave intra-component synchronization to each individual component.
For example, to ensure the consistency between virtual memory metadata operations (handled by ARM)
and data operations (handled by an FPGA component) in a session,
the pre-processor blocks the session's metadata (data) operations when the session has an in-flight data (metadata) operation
to the same virtual page.
%But to maximize performance, the pre-processor sends out all non-blocked requests as soon as they arrive.

After the pre-processor, metadata requests go to ARM and data requests go to the {\em core-memory} FPGA component.
When these components finish processing the requests, they send the results in reply messages
back to the clients (via the network TX stack).
%the reply to the post-processor
%(co-located with the pre-processor in the same component),
%which will form replies and forward them to the network stack's TX (sending) stack.


\subsection{Virtual Memory System Data Plane}
\label{sec:dataplane}
%emphasize difference from virt mem sys in software
We implemented the virtual memory data plane in a {\em core memory} module on FPGA (Figure~\ref{fig-coremem} and Appendix).
It performs two main functions: virtual-to-physical address translation and DRAM data access.
Although at a high level, these functionalities are similar to traditional software-based virtual memory systems,
implementing them in hardware with our cost and performance goals (\textbf{R1}, \textbf{R2}, \textbf{R3}) is challenging.
%we face unique hardware challenges in working with small FPGA area, clock frequency budgets,
%and the goals of reaching 100\Gbps\ network line rates and single-digit microsecond latency.
To achieve these goals, we design the data-plane pipeline to have {\em deterministic performance} %smooth with minimal stalls 
by moving slower operations off the performance-critical path and by bounding the length of them. 
As a result, each virtual memory read/write request requires \textbf{at most two DRAM accesses} 
on the performance-critical path (one being the data access itself), 
and the whole pipeline with page fault handling takes only \textbf{three FPGA cycles}. 
%two of which are on the performance critical path).
To achieve the cost goal, we store all data and large metadata in off-chip DRAM, but minimize the performance impact
of accessing DRAM by either making these operations asynchronous or infrequent (through on-chip caching).

We propose a new overflow-free hash-based page table design that bounds address translation to at most one DRAM access.
%Bounding address translation to at most one DRAM access 
This design not only delivers excellent performance 
but also largely reduces the complexity (and thus FPGA area cost) of our core memory pipeline. % (because \zhiyuan{XXX}). 
We store the entire page table of a process in a DRAM hash table whose hash buckets each has a fixed number of slots (\eg, 8 slots per bucket).
To look up a virtual memory address, we compute its hash value (using the {\em lookup3}~\cite{lookup3-wiki} hash function) 
and fetch the entire bucket (with all its slots) in a single DRAM read.
Normally, a hash table with fixed slots will have an overflow problem because of hash conflicts (\eg, in a Xilinx FPGA key-value store~\cite{FPGA-KV}).
We use a novel technique to {\em proactively avoid} hash overflow at virtual address allocation time (see \S\ref{sec:metadataplane}).
To further improve address translation performance, we cache hot page table entries (PTEs) on chip (in FPGA BRAM),
use CAM (content-addressable-memory) to look up the cache,
and use LRU for replacement, similar to traditional TLB design.
%Under our current implementation, the PTE cache is shared across application processes.
%But it is fairly easy to change it to provide isolation between processes (\eg, for performance or security).

When a request arrives from the XBar, its header (\pid, virtual address, size) goes to the {\em address translation pipeline},
and its data goes to the {\em data access pipeline}.
The address translation pipeline first looks up the virtual address in the on-chip PTE cache.
If there is a hit, it sends the translation result to a result-buffer unit
and updates the statistics of the PTE cache through a {\em PTE cache manager} (for replacement policy).
Otherwise, it forwards the request to the {\em PTE fetch unit} which reads the corresponding hash bucket from DRAM
and checks if there is a matching PTE.
If so, we send the result to the result-buffer unit.
Otherwise, we forward the request and the (fault) result to the next unit,
a {\em page fault handler}.

The page fault handler deals with two cases.
If the PTE lookup result is a permission violation, it sends an error message to the result-buffer unit.
If the lookup result is an unallocated physical page, we need to do an on-demand physical memory allocation.
As allocation is performed by ARM, fetching the allocation results via the slow path between FPGA and ARM would hugely affect foreground performance.
We propose an asynchronous design to avoid this performance overhead.
We maintain a set of {\em free physical page lists} (of different page sizes),
which ARM continuously fills by allocating physical pages. % actual allocation.
During an on-demand page fault, the page fault handler simply fetches a pre-allocated physical page address 
from the corresponding free page list. % of the corresponding page size.
%This asynchronous design enables us to avoid the long wait for ARM to do an allocation on the fly.

The page fault handler then performs three tasks in parallel: 
writing the new PTE to the DRAM page table, sending the new PTE to the PTE cache manager (which then inserts the PTE to the PTE cache),
and sending the new PTE to the translate result unit.
This early result forwarding avoids the performance overhead of one DRAM write,
but requires additional measures to guarantee consistency.
The inconsistent case would happen when another in-flight request to the same virtual address 
finishes the PTE fetching step before the new PTE is written to DRAM,
in which case the already fetched PTE would be invalid causing another page fault.
To avoid this case, %creating another PTE for this other request, 
we temporarily store the previously created PTE in a small buffer at the page fault handler before writing it to DRAM.
For all the requests coming into the page fault handler, we lookup this new PTE buffer 
and bypass the page fault handling logics for a match.
After a new PTE has been written to DRAM, we send a signal back to the pipeline (PTE updated dash line)
to remove the corresponding entry in the new PTE buffer.
Thus, PTEs live in the buffer only for the duration of the DRAM write, and the buffer could be kept small 
(in the rare case when the buffer is full, we stall the pipeline).
%With the above performance-optimization techniques, the whole page fault handler takes only {\em three cycles} in total.

The translation result buffer unit gathers resulting physical addresses and sends them to the data access pipeline in order.
We follow request arrival order here (\ie, an in-order pipeline) to provide stronger consistency guarantees
and to make our implementation simpler. 
%An out-of-order pipeline can be built on top of our current design to allow further performance improvement.

The data access pipeline runs in parallel with the address translation pipeline.
It first buffers request headers and request data in two FIFO queues.
After receiving a translation result from the address translation pipeline, 
it takes the next header and data from the two queues and forms a physical memory access request.
Because all these queues are in the same order as the request arrival order, 
we do not need to do any re-ordering in the data access pipeline.
When the physical memory access completes, 
the data access pipeline forms a response request which is sent back to the client (via XBar and network stack).

\subsection{Virtual Memory System Metadata Plane}
\label{sec:metadataplane}

The ARM processor handles all the metadata and control operations.
Physical links between ARM and FPGA and between ARM and on-board DRAM are slow both in latency and in bandwidth.
To mitigate this performance problem, 
first, we shift operations involving ARM off the performance-critical path.
Second, we maintain shadow copies of metadata at ARM's local DRAM 
to avoid the much higher (79\x\ in our experiment) cost of going to the on-board DRAM.
Third, we employ an efficient polling mechanism for ARM-FPGA communication.
We dedicate one ARM core to busy poll an (RX) ring buffer between ARM and FPGA,
where the FPGA posts tasks for ARM.
This polling thread hands over tasks to other worker threads for task handling %that perform the tasks
and post responses to a TX ring buffer.
We use DMA to implement the ring buffers, 
as DMA is the fastest communication methods we found among all the available ones.

The major metadata tasks in \sys\ are virtual and physical memory allocation and free.
Virtual memory allocation (free) happens when applications call \alloc\ (\free).
The \sys\ library sends the slice ID(s) the \alloc\ is designated to 
together with the size to be allocated and the \pid\ (see \S~\ref{sec:dist-virtmem}).
The FPGA command pre-processor forwards these requests to ARM.
ARM maintains a VMA (virtual-memory-address) tree for each slice of a registered process,
similar to Linux VMA trees.
It also maintains a {\em shadow} version of hash page tables in its local memory for fast accesses.

We adapted Linux' VMA-tree-based virtual memory allocation algorithm to accommodate our fix-slot hash page table design as follows.
After finding an available address range in the VMA tree, we calculate the hash values of the
virtual pages and check if inserting them to the shadow page table would cause hash overflow. 
If so, we mark the failed virtual pages in the VMA tree as ``unusable'' and do another VMA-tree search
%We also insert special ``unusable'' PTEs of the failed pages to the shadow page table to be used for future \free.
%by linking them outside the fix-slot bucket
%These steps repeat 
until we find a valid virtual address range or run out of virtual addresses.
When a valid virtual address range is found, we insert corresponding PTEs to both the shadow page table in ARM memory
and the real page table in the on-board DRAM (through ARM's DDR interface).
These PTEs have no physical addresses and are set to invalid.
To handle an \free\ request, ARM sends a PTE invalidation request to FPGA,
which invalidates the PTE in its cache and in the main DRAM page table.
ARM also removes the corresponding PTE in its shadow page table.
If there are other previously marked ``unusable'' PTEs that fall to the same hash 
bucket, we mark them usable again, as the freed PTE creates an empty slot.
%one in the same hash bucket, we remove them from the shadow page table as well and mark them as usable again.

ARM also manages physical memory allocation and uses a traditional buddy allocation algorithm, which takes $\sim100\ns$ per allocation from our experiments. 
We decouple ARM's generation of 
new physical pages from FPGA's consumption of them.
ARM keeps filling the free page lists
with newly allocated physical page addresses until the lists are full.
FPGA consumes a free physical page at the page fault handling time and 
notifies ARM about the consumption. % (through another ring buffer).
%For the allocation algorithm, we simply use the traditional buddy allocator.

\subsection{Network Layer}
\label{sec:network}

We built our network system on top of an open-source 10\Gbps\ FPGA IP/UDP stack~\cite{Corundum-FCCM20}.

\fi

\section{Building Applications on \sys}
\label{sec:app}

We built five applications on top of \sys, one that uses the basic \sys\ APIs, one that implements and uses a high-level, extended API, and two that offload data processing tasks to \MN{}s, and one that splits computation across \CN{}s and \MN{}s.

\ulinebfpara{Image compression.}
We build a simple image compression/decompression utility that runs purely at \CN.
Each client of the utility (\eg, a Facebook user) has its own collection of photos, 
stored in two arrays at \MN{}s, one for compressed and one for original, both allocated with \alloc.
Because clients' photos need to be protected from each other, we use one process
per client to run the utility.
The utility simply reads a photo from \MN\ using \Cliosysread, compresses/decompresses it,
and writes it back to the other array using \Cliosyswrite.
Note that we use compression and decompression as an example of image processing.
These operations could potentially be offloaded to \MN{}s.
However, in reality, there can be many other types of image processing that are more complex and are hard and costly to implement in hardware, necessitating software processing at \CN{}s.
We implemented this utility with 1K C code in 3 developer days.

\ulinebfpara{Radix tree.}
To demonstrate how to build a data structure on \sys\
using \sys's extended API, we built a radix tree with linked lists and pointers.
Data-structure-level systems like AIFM~\cite{AIFM} could follow this example to make simple changes in their libraries to run on \sys.
%where all nodes in a layer is linked in a list and each node points a link list
%of its children.
We first built an extended pointer-chasing functionality in FPGA at the \MN\ which follows pointers in a linked list and performs a value comparison
at each traversed list node. It returns either the node value when there is a match or null when the next pointer becomes null. 
We then expose this functionality to \CN{}s as an extended API.
The software running at \CN\ allocates a big contiguous remote memory space using \alloc\ and uses this space to store radix tree nodes. Nodes in each layer are linked to a list.
To search a radix tree, the \CN\ software goes through each layer of the tree and calls the pointer chasing API until a match is found.
We implemented the radix tree with 300 C code at \CN\ and 150 SpinalHDL code at \sysboard\ in less than one developer day.


{
\begin{figure*}[th]
\begin{center}
\centerline{\includegraphics[width=0.5\textwidth]{clio/Figures/g_plot_scalability_conn.pdf}}
\mycaption{fig-conn}{Process (Connection) Scalability.}
{
}
\end{center}
\end{figure*}
}
{
\begin{figure*}[h]
\begin{center}
\centerline{\includegraphics[width=0.5\textwidth]{clio/Figures/g_plot_scalability_pte.pdf}}
\mycaption{fig-pte-mr}{PTE and MR Scalability.}
{
RDMA fails beyond $2^{18}$ MRs. 
}
\end{center}
\end{figure*}
}
{
\begin{figure*}[h]
\begin{center}
\centerline{\includegraphics[width=0.5\textwidth]{clio/Figures/g_plot_latency_comparison.pdf}}
\mycaption{fig-miss-hit}{Comparison of TLB Miss and page fault.}
{
\sys-ASIC are projected values of TLB hit.
}
\end{center}
\end{figure*}
}
{
\begin{figure*}[h]
\begin{center}
\centerline{\includegraphics[width=0.5\textwidth]{clio/Figures/clio_rdma_lat_cdf.pdf}}
\mycaption{fig-tail-latency}{Latency CDF.}
{
}
\end{center}
\end{figure*}
}
{
\begin{figure*}[th]
\begin{center}
\centerline{\includegraphics[width=0.5\textwidth]{clio/Figures/g_plot_throughput.pdf}}
\mycaption{fig-read-write-throughput}{End-to-End Goodput.}
{
1\KB\ requests. % between 1 \CN\ and 1 \MN.
}
\end{center}
\end{figure*}
}
{
\begin{figure*}[h]
\begin{center}
\centerline{\includegraphics[width=0.5\textwidth]{clio/Figures/g_plot_onboard_throughput.pdf}}
\mycaption{fig-onboard-throughput}{On-board Goodput.}
{
FPGA test module generates requests at maximum speed.
}
\end{center}
\end{figure*}
}
{
\begin{figure*}[h]
\begin{center}
\centerline{\includegraphics[width=0.5\textwidth]{clio/Figures/g_plot_read_latency.pdf}}
\mycaption{fig-read-lat}{Read Latency.}
{
HERD-BF: HERD running on BlueField. %SmartNIC.
}
\end{center}
\end{figure*}
}
{
\begin{figure*}[h]
\begin{center}
\centerline{\includegraphics[width=0.5\textwidth]{clio/Figures/g_plot_write_latency.pdf}}
\mycaption{fig-write-lat}{Write Latency.}
{
Clover requires $\ge$ 2 RTTs for write.
}
\end{center}
\end{figure*}
}


\ulinebfpara{Key-value store.}
We built {\em \syskv}, a key-value store that supports concurrent 
create/update/read/delete key-value entries
with atomic write and read committed consistency.
\syskv\ runs at an \MN\ as a computation offloading module.
Users can access it through a key-value interface from multiple \CN{}s.
The \syskv\ module has its own virtual memory address space and uses \sys\ virtual memory APIs to access it.
\syskv\ uses a chained hash table in its virtual memory space for managing the metadata of key-value pairs, and it stores the actual key values at separate locations in the space.
Each hash bucket has a chain of slots. Each slot
contains the virtual addresses of seven key-value pairs.
It also stores a fingerprint for each key-value pair.

To create a new key-value pair, \syskv\ allocates space for
the key-value data with an \alloc\ call and writes the
data with an \Cliosyswrite. It then calculates the hash and the fingerprint of the key. 
Afterward, it fetches the last hash slot in the corresponding hash bucket using the hash value. If that
slot is full, \syskv\ allocates another slot using \alloc; otherwise, it just uses the fetched last slot. 
It then inserts the virtual address and fingerprint of the data into the last/new slot.
Finally, it links the current last slot to the new slot if a new one is created.

To perform a read, \syskv\ locates the hash bucket (with the key's hash value) and fetches one slot in the bucket
chain at a time using \Cliosysread. It then compares the fingerprint of the key to the seven entries in the slot. If there is no match, it
fetches the next slot in the bucket. Otherwise, with a matched
entry, it reads the key-value pair using the address stored in that
entry with an \Cliosysread. It then compares the full key and returns
the value if it is a match. Otherwise, it keeps searching the bucket.

The above describes a single-\MN\ \syskv\ system. Another \CN-side load balancer is used to partition key-value pairs into different \MN{}s.
Since all \CN{}s requests of the same partition go to the same \MN\ and \sys\ APIs within an \MN\ are properly ordered, it is fairly easy for \syskv\ to guarantee the atomic-write, read-committed consistency level.

We implemented \syskv\ with 772 SpinalHDL code in 6 developer days.
To evaluate \sys's virtual memory API overhead at \sysboard, we also implemented a key-value store with the same design as \syskv\ but with raw physical memory interface.
This physical-memory-based implementation takes more time to develop and only yields 4\%–12\% latency improvement and 1\%–5\% throughput improvement over \syskv. %\yiying{Zhiyuan, do you still remember or can estimate how much longer does the physical memory one take than our final virtual memory one?}
%\zhiyuan{basically we only replace the DMA module with physical DMA and legomem-VA DMA. So the effort should remains the same. The extra work for PA is handling the memory management, but that part is very simple for the current impl. I'll get details later}
%\zhiyuan{Also easy change from physical DMA to virtual DMA can be seens as easy adoption to Clio? 27 lines added on top of phys compared to 809 lines of code}
%\yiying{What is 809 LOC? 809 for the original phys impl? and to change it to virt, we replace 809 lines with 27 lines?} \zhiyuan{we add 27 line on top of 908 to adopt the original phys code to use Clio virtual memory}



\ulinebfpara{Multi-version object store.}
We built a multi-version object store ({\em \sysmv}) which lets users on \CN{}s create an object, append a new version to an object, 
read a specific version  or the latest version of an object, and delete an object.
Similar to \syskv, \sysmv\ has its own address space.
In the address space, it uses an array to store versions of data for each object, a map to store the mapping from object IDs to the per-object array addresses, and a list to store free object IDs. 
When a new object is created, \sysmv\ allocates a new array (with \alloc) 
and writes the virtual memory address of the array into the object ID map.
Appending a new version to an object simply increases the latest version number
and uses that as an index to the object array for writing the value.
Reading a version simply reads the corresponding element of the array.

\sysmv\ allows concurrent accesses from \CN{}s to an object and guarantees sequential consistency for each object.
Each \sysmv\ user request involves at least two internal
\sys\ operations, some of which include both metadata and
data operations. This compound request pattern makes it tricky
to deal with synchronization problems, as \sysmv\ needs to
ensure that no internal \sys\ operation of a later \sysmv\ request could affect the correctness of an earlier \sysmv\ request. 
% Fortunately, both \sys's fast path and slow path guarantee sequential delivery of \sys\ operations. Since \sysmv\ only issues one Clio operation per
% clock cycle, the ordering that \sys\ modules guarantee is sufficient to deliver \sysmv’s consistency guarantees. 
We implemented \sysmv\ with 1680 lines of C HLS code in 15 developer days.
 
\ulinebfpara{Simple data analytics.}
Our final example is a simple DataFrame-like data processing application ({\em \sysdf}),
which splits its computation between \CN\ and \MN.
We implement \texttt{select} and \texttt{aggregate} at \MN\ as two offloads,
as offloading them can reduce the amount of data sent over the network.
We keep other operations like \texttt{shuffle} and \texttt{histogram} at \CN.
For the same user, all these modules share the same address space regardless of whether they are at \CN\ or \MN.
Thanks to \sys's support of computation offloading sharing the same address space as computations running at host, \sysdf's implementation is largely simplified and its performance is improved by avoiding data serialization/deserialization.
We implemented \sysdf\ with 202 lines of SpinalHDL code and 170 lines of C interface code in 7 developer days.



{
\begin{figure*}[th]
\begin{center}
\centerline{\includegraphics[width=0.5\textwidth]{clio/Figures/g_plot_scalability_conn.pdf}}
\mycaption{fig-conn}{Process (Connection) Scalability.}
{
}
\end{center}
\end{figure*}
}
{
\begin{figure*}[h]
\begin{center}
\centerline{\includegraphics[width=0.5\textwidth]{clio/Figures/g_plot_scalability_pte.pdf}}
\mycaption{fig-pte-mr}{PTE and MR Scalability.}
{
RDMA fails beyond $2^{18}$ MRs. 
}
\end{center}
\end{figure*}
}
{
\begin{figure*}[h]
\begin{center}
\centerline{\includegraphics[width=0.5\textwidth]{clio/Figures/g_plot_latency_comparison.pdf}}
\mycaption{fig-miss-hit}{Comparison of TLB Miss and page fault.}
{
\sys-ASIC are projected values of TLB hit.
}
\end{center}
\end{figure*}
}
{
\begin{figure*}[h]
\begin{center}
\centerline{\includegraphics[width=0.5\textwidth]{clio/Figures/clio_rdma_lat_cdf.pdf}}
\mycaption{fig-tail-latency}{Latency CDF.}
{
}
\end{center}
\end{figure*}
}
{
\begin{figure*}[th]
\begin{center}
\centerline{\includegraphics[width=0.5\textwidth]{clio/Figures/g_plot_throughput.pdf}}
\mycaption{fig-read-write-throughput}{End-to-End Goodput.}
{
1\KB\ requests. % between 1 \CN\ and 1 \MN.
}
\end{center}
\end{figure*}
}
{
\begin{figure*}[h]
\begin{center}
\centerline{\includegraphics[width=0.5\textwidth]{clio/Figures/g_plot_onboard_throughput.pdf}}
\mycaption{fig-onboard-throughput}{On-board Goodput.}
{
FPGA test module generates requests at maximum speed.
}
\end{center}
\end{figure*}
}
{
\begin{figure*}[h]
\begin{center}
\centerline{\includegraphics[width=0.5\textwidth]{clio/Figures/g_plot_read_latency.pdf}}
\mycaption{fig-read-lat}{Read Latency.}
{
HERD-BF: HERD running on BlueField. %SmartNIC.
}
\end{center}
\end{figure*}
}
{
\begin{figure*}[h]
\begin{center}
\centerline{\includegraphics[width=0.5\textwidth]{clio/Figures/g_plot_write_latency.pdf}}
\mycaption{fig-write-lat}{Write Latency.}
{
Clover requires $\ge$ 2 RTTs for write.
}
\end{center}
\end{figure*}
}

\section{Evaluation}
\label{sec:clio:results}


Our evaluation reveals the scalability, throughput, median and tail latency, energy and resource consumption of \sys.
%, and how it compares with state-of-the-art systems. 
We compare \sys's end-to-end performance with industry-grade NICs (ASIC) and well-tuned RDMA-based software systems.
All \sys's results are FPGA-based, which would be improved with ASIC implementation.
%Nonetheless, \sys\ significantly outperforms RDMA on scalability and tail latency, while being similar on other measurements.

\ulinebfpara{Environment.}
We evaluated \sys\ in our local cluster of four \CN{}s and four \MN{}s (Xilinx ZCU106 boards),
%\footnote{Unfortunately, our process of purchasing and setting up a bigger cluster was significantly delayed because of COVID-19},
all connected to an Nvidia 40\Gbps\ VPI switch.
Each \CN\ is a Dell PowerEdge R740 server equipped with a Xeon Gold 5128 CPU and a 40\Gbps\ Nvidia ConnectX-3 NIC,
with two of them also having an Nvidia BlueField SmartNIC~\cite{BlueField}.
We also include results from CloudLab~\cite{CloudLab} with the Nvidia ConnectX-5 NIC.


\subsection{Basic Microbenchmark Performance}

{
\begin{figure*}[th]
\begin{minipage}{\figWidthSix}
\begin{center}
\centerline{\includegraphics[width=\columnwidth]{Figures/g_plot_alloc_free.pdf}}
\vspace{-0.1in}
\captionsetup{width=.9\columnwidth}
\mycaption{fig-alloc-free}{Alloc/Free Latency.}
{
ODP means On-Demand-Paging mode
}
\end{center}
\end{minipage}
\begin{minipage}{\figWidthSix}
\begin{center}
\centerline{\includegraphics[width=\columnwidth]{Figures/g_plot_alloc_conflict.pdf}}
\vspace{-0.1in}
\captionsetup{width=.9\columnwidth}
\mycaption{fig-alloc-conflict}{Alloc Retry Rate.}
{
%Alloc's number of retries when vary physical memory utilization.
}
\end{center}
\end{minipage}
\begin{minipage}{\figWidthSix}
\begin{center}
\centerline{\includegraphics[width=\columnwidth]{Figures/g_plot_latency_breakdown.pdf}}
\vspace{-0.1in}
\captionsetup{width=.9\columnwidth}
\mycaption{fig-lat-break}{Latency Breakdown.}
{
Breakdown of time spent at \sysboard.
}
\end{center}
\end{minipage}
\begin{minipage}{\figWidthSix}
\begin{center}
\centerline{\includegraphics[width=\columnwidth]{Figures/g_plot_ycsb_mn.pdf}}
\vspace{-0.1in}
\captionsetup{width=.9\columnwidth}
\mycaption{fig-ycsb-mn}{\syskv\ Scalability against \MN{}s.}
{
}
\end{center}
\end{minipage}
\vspace{-0.15in}
\end{figure*}
}

{
\begin{figure*}[th]
\begin{minipage}{\figWidthSix}
\begin{center}
\centerline{\includegraphics[width=\columnwidth]{Figures/g_plot_image_compression.pdf}}
\vspace{-0.1in}
\captionsetup{width=.9\columnwidth}
\mycaption{fig-photo}{Image Compression.}
{
}
\end{center}
\end{minipage}
\begin{minipage}{\figWidthSix}
\begin{center}
\centerline{\includegraphics[width=\columnwidth]{Figures/g_plot_radix_tree.pdf}}
\vspace{-0.1in}
\captionsetup{width=.9\columnwidth}
\mycaption{fig-radix}{Radix Tree Search Latency.}
{
}
\end{center}
\end{minipage}
\begin{minipage}{\figWidthSix}
\begin{center}
\centerline{\includegraphics[width=\columnwidth]{Figures/g_plot_ycsb_cn.pdf}}
\vspace{-0.1in}
\captionsetup{width=.9\columnwidth}
\mycaption{fig-kvstore}{Key-Value Store YCSB Latency.}
{
}
\end{center}
\end{minipage}
%\if 0 g_plot_ycsb_mn
%\fi
\begin{minipage}{\figWidthSix}
\begin{center}
\centerline{\includegraphics[width=\columnwidth]{Figures/g_plot_mvstore.pdf}}
\vspace{-0.1in}
\captionsetup{width=.9\columnwidth}
\mycaption{fig-mvstore}{\sysmv\ Object Read/Write Latency.}
{
}
\end{center}
\end{minipage}
%\vspace{-0.15in}
\end{figure*}
}


\ulinebfpara{Scalability.}
We first compare the scalability of \sys\ and RDMA.
Figure~\ref{fig-conn} measures the latency of \sys\ and RDMA as the number of client processes increases.
For RDMA, each process uses its own QP.
Since \sys\ is connectionless, it scales perfectly with the number of processes.
RDMA scales poorly with its QP, and the problem persists with newer generations of RNIC,
which is also confirmed by our previous works~\cite{Pythia,Storm}.

Figure~\ref{fig-pte-mr} evaluates the scalability with respect to PTEs and memory regions.
For the memory region test, we register multiple MRs using the same physical memory for RDMA.
For \sys, we map a large range of VAs (up to 4\TB) to a small physical memory space, as our testbed only has 2\GB\ physical memory.
However, the number of PTEs and the amount of processing needed are the same for \sysboard\ as if it had a real 4\TB\ physical memory.
Thus, this workload stress tests \sysboard's scalability.
%For \sys\ (which gets rid of the MR concept), we use multiple processes to share the same memory,
%resulting in one PTE per process.
RDMA's performance starts to degrade when there are more than $2^8$ (local cluster) or $2^{12}$ (CloudLab),
and the scalability wrt MR is worse than wrt PTE.
In fact, RDMA fails to run beyond $2^{18}$ MRs.
In contrast, \sys\ scales well and never fails (at least up to 4\TB\ memory).
It has two levels of latency that are both stable: a lower latency below $2^4$ for TLB hit and a higher latency above $2^4$ for TLB miss (which always involves one DRAM access).
A \sysboard\ could use a larger TLB if optimal performance is desired.

These experiments confirm that \textbf{\sys\ can handle thousands of concurrent clients and TBs of memory}.



\ulinebfpara{Latency variation.}
Figure~\ref{fig-miss-hit} plots the latency of reading/writing 16\,B data 
when the operation results in a TLB hit, a TLB miss, a first-access page fault, and MR miss (for RDMA only, when the MR metadata is not in RNIC).
RDMA's performance degrades significantly with misses.
Its page fault handling is extremely slow (16.8\ms).
We confirm the same effect on CloudLab with the newer ConnectX-5 NICs.
\sys\ only incurs a small TLB miss cost and \textbf{no additional cost of page fault handling}.

We also include a projection of \sys's latency if it was to be implemented using a real ASIC-based \sysboard.
Specifically, we collect the latency breakdown of time spent on the network wire and at \CN, time spent on third-party FPGA IPs,
number of cycles on FPGA, and time on accessing on-board DRAM.
We maintain the first two parts, scale the FPGA part to ASIC's frequency (2\,GHz), use DDR access time collected on our server to replace the access time to on-board DRAM (which 
goes through a slow board memory controller).
This estimation is conservative, as a real ASIC implementation of the third-party IPs would make the total latency lower.
Our estimated read latency is better than RDMA, while write latency is worse.
We suspect the reason being Nvidia RNIC's optimization of replying a write before it is fully written to DRAM, which \sys\ could also potentially adopt.

Figure~\ref{fig-tail-latency} plots the request latency CDF of continuously running read/write 16\,B data while not triggering page faults.
Even without page faults, \sys\ has much less latency variation and a much shorter tail than RDMA.
%Thanks to our bounded address translation and deterministic hardware design, \sys\ has much less latency variation and a much shorter tail than RDMA.

{
\begin{figure*}[th]
\begin{center}
\centerline{\includegraphics[width=0.5\textwidth]{clio/Figures/g_plot_dp.pdf}}
\mycaption{fig-dataframe}{Select-Aggregate-Shuffle.}
{
Y axis starts at 4 sec. 
CN represents computation done at \CN.
%, as histogram  time is the same.
}
\end{center}
\end{figure*}
}
{
\begin{figure*}[h]
\begin{center}
\centerline{\includegraphics[width=0.5\textwidth]{clio/Figures/g_plot_ycsb_energy.pdf}}
\mycaption{fig-energy}{Energy Comparison.}
{
Darker/lighter shades represent energy spent at \MN{}s and \CN{}s.
}
\end{center}
\end{figure*}
}
{
\begin{table}\small
\begin{center}
\begin{center}
\begin{tabular}{ p{1.2in} | p{0.5in} |p{0.6in} }
 & \textbf{Logic} & \textbf{Memory} \\
\textbf{System/Module} & \textbf{(LUT)} & \textbf{(BRAM)} \\
\hline
\hline
StRoM-RoCEv2 & 39\% & 76\% \\
Tonic-SACK & 48\% & 40\% \\
\hline
\sys\ (Total) & 31\% & 31\% \\
VirtMem & 5.5\% & 3\% \\
NetStack & 2.3\% & 1.7\% \\
\hline
Go-Back-N & 5.8\% & 2.6\% \\
\end{tabular}
\end{center}
\mycaption{fig-fpga-resource}{Clio FPGA Utilization.}
{
}
\end{center}
\end{table}
}

\ulinebfpara{Read/write throughput.}
We measure \sys's throughput by varying the number of concurrent client threads (Figure~\ref{fig-read-write-throughput}).
\sys's default asynchronous APIs quickly reach the line rate of our testbed (9.4\Gbps\ maximum throughput).
Its synchronous APIs could also reach line rate fairly quickly.

Figure~\ref{fig-onboard-throughput} measures the maximum throughput of \sys's FPGA implementation without the bottleneck of the board's 10\Gbps\ port, by generating traffic on board.
Both read and write can reach more than 110\Gbps\ when request size is large.
Read throughput is lower than write when request size is smaller.
We found the throughput bottleneck to be the third-party non-pipelined DMA IP
(which could potentially be improved).

\ulinebfpara{Comparison with other systems.}
We compare \sys\ with native one-sided RDMA, Clover~\cite{Tsai20-ATC}, HERD~\cite{Kalia14-RDMAKV}, and LegoOS~\cite{Shan18-OSDI}.
We ran HERD on both CPU and BlueField (HERD-BF).
%Native RDMA can be considered as a baseline (optimal performance but low-level, restrictive interface).
Clover is a passive disaggregated persistent memory system which we adapted as a passive disaggregated memory (PDM) system.
HERD is an RDMA-based system that supports a key-value interface with an RPC-like architecture.
LegoOS builds its virtual memory system in software at \MN.
%It uses one RDMA read for its read and one RDMA write plus one 
%it can be considered as a software-based active disaggregated memory system. 

\sys's performance is similar to HERD and close to native RDMA.
%\sys's write performance is better than Clover and similar to HERD. %but has a constant overhead over native RDMA.
Clover's write is the worst because it uses at least 2 RTTs for writes to deliver its consistency guarantees without any processing power at \MN{}s.
HERD-BF's latency is much higher than when HERD runs on CPU
due to the slow communication between BlueField's ConnectX-5 chip and ARM processor chip.
LegoOS's latency is almost two times higher than \sys's when request size is small.
In addition, from our experiment, LegoOS can only reach a peak throughput of 77\Gbps, while \sys\ can reach 110\Gbps.
LegoOS' performance overhead comes from its software approach, demonstrating the necessity of a hardware-based solution like \sys.
%due to the slow communication between BlueField's Connect-X5 chip and ARM processor %chip..
%\sys's write overhead can be attributed to \fixme{XXX}.

\ulinebfpara{Allocation performance.}
Figure~\ref{fig-alloc-free} shows \sys's VA and PA allocation and RDMA's MR registration performance.
%Physical memory allocation includes the time to perform an allocation with the buddy algorithm and to insert the allocated address into the free page list.
%It is very fast, indicating that our asynchronous free physical page generation could keep up with most workloads' page fault speed.
%Virtual memory allocation and free (measured from client on \CN) are slower,
\sys's PA allocation takes less than 20\mus, and the VA allocation is much faster than RDMA MR registration,
although both get slower with larger allocation/registration size.
%since these operations involve the costly crossing between FPGA and ARM.
%They are also slower with larger sizes, as searching the VMA tree for a big free region takes more time.
Figure~\ref{fig-alloc-conflict} shows the number of retries at allocation time with three allocation sizes as the physical memory fills up.
%running at on-board ARM processor.
%The hash-based page table is proportional to the physical memory size. 
%Hence higher its utilization, higher the overflow probability therefore higher number of retries.
%The page table has 2\x\ extra slots by default.
There is no retry when memory is below half utilized. Even when memory is close to full, there are at most 60 retries per allocation request, with roughly 0.5\ms\ per retry. This confirms that our design of avoiding hash overflows at allocation time is practical.
%co-design of overflow-free hash-based page table and allocation retry scheme is practical.


\ulinebfpara{Close look at \sysboard{} components.}
To further understand \sys's performance, % and to determine the reason for worse large-read performance,
we profile different parts of \sys's processing for read and write of 4\,B to 1\KB.
\syslib\ adds a very small overhead (250\ns\ in total), 
thanks to our efficient threading model and network stack implementation.
Figure~\ref{fig-lat-break} shows the latency breakdown at \sysboard.
Time to fetch data from DRAM (DDRAccess) and to transfer it over the wire (WireDelay) are the main 
contributor to read latency, especially with large read size.
Both could be largely improved in a real \sysboard\ with better memory controller and higher frequency.
TLB miss (which takes one DRAM read) is the other main part of the latencies.


\subsection{Application Performance}

\ulinebfpara{Image Compression.}
We run a workload where each client 
compresses and decompresses 1000 256*256-pixel images with increasing number of concurrently running clients.
Figure~\ref{fig-photo} shows the total runtime per client.
We compare \sys\ with RDMA, with both performing computation at the \CN\ side and the RDMA using one-sided operations instead of \sys\ APIs to read/write images in remote memory.
\sys's performance stays the same as the number of clients increase.
RDMA's performance does not scale because it requires each client to register a different MR to have protected memory accesses.
With more MRs, RDMA runs into the case where the RNIC cannot hold all the MR metadata and many accesses would involve a slow read to host main memory.

\ulinebfpara{Radix Tree.}
Figure~\ref{fig-radix} shows the latency of searching a key in pre-populated radix trees when varying the tree size. 
We again compare with RDMA which uses one-sided read operations to perform the tree traversal task.
RDMA's performance is worse than \sys,
because it requires multiple RTTs to traverse the tree,
while \sys\ only needs one RTT for each pointer chasing (each tree level).
In addition, RDMA also scales worse than \sys.

\ulinebfpara{Key-value store.}
Figure~\ref{fig-kvstore} evaluates \syskv\ using the YCSB benchmark~\cite{YCSB} and compares it to Clover, HERD, and HERD-BF.
We run two \CN{}s and 8 threads per \CN.
We use 100K key-value entries and run 100K operations per test,
with YCSB's default key-value size of 1\KB. %where the key size is 8 bytes and the value size is 1\KB.
The accesses to keys follow the Zipf distribution ($\theta=0.99$).
We use three YCSB workloads with different {\em get-set} ratios: 
100\% {\em get} (workload C), 5\% {\em set} (B), and 50\% {\em set} (A).
\syskv\ performs the best.
HERD running on BlueField performs the worst, mainly because BlueField's slower crossing between its NIC chip and ARM chip.




Figures~\ref{fig-ycsb-mn} shows the throughput of \syskv\ when varying the number of MNs. Similar to our
\sys\ scalability results, \syskv\ can reach a CN’s maximum
throughput and can handle concurrent get/set requests even
under contention. These results are similar to or better than
previous FPGA-based and RDMA-based key-value stores that
are fine-tuned for just key-value workloads (Table 3 in \cite{KVDIRECT}),
while we got our results without any performance tuning.

\ulinebfpara{Multi-version data store.}
%\subsubsection{Multi-Version Data Store}
We evaluate \sysmv\ by varying the number of \CN{}s that concurrently access data objects (of 16\,B) on an \MN\ using workloads of 50\% read (of different versions) and 50\% write under uniform and Zipf distribution of objects (Figure~\ref{fig-mvstore}). 
\sysmv's read and write have the same performance, and reading any version has the 
same performance, since we use an array-based version design. 
%Running multiple \MN{}s have similar performance and we omit for space.




\ulinebfpara{Data analytics.}
We run a simple workload which first \texttt{select} rows in a table whose field-A matches a value (\eg, gender is female)
and calculate \texttt{avg} of field-B (\eg, final score) of all the rows.
Finally, it calculates the histogram of the selected rows (\eg, score distribution), which can be presented to the user together with the avg value. %(\eg, how female students' scores compare to the whole class).
\sys\ executes the first two steps at \MN\ offloads and the final step at \CN,
while RDMA always reads rows to \CN\ and then does each operation.
Figure~\ref{fig-dataframe} plots the total run time as the select ratio decreases (fewer rows selected).
% When the select ratio is high, \sys\ and RDMA send a similar amount of data across the network,
% and as the CPU computation is faster than our FPGA implementation for these operations, \sys's overall performance is worse than RDMA.
When the select ratio is low, \sys\ transfers much less data than RDMA, resulting in its better performance.




%To put \sys\ in respective with other existing RDMA-based and FPGA-based key-value stores that we couldn't directly compare with (\eg, close-sourced), we compare 
%\syskv's latency results with reported latencies in ~\cite{KVDIRECT}. 
%\syskv\ has {\bf lower end-to-end latency than all these existing systems}.

%then sends the data to \CN, which shuffles the data and sends the shuffled 
%data back to \MN\ for aggregation.

\subsection{CapEx, Energy, and FPGA Utilization}
\label{sec:clio:results-cost}


We estimate the cost of server and \sysboard\ using market prices of different hardware units. When using 1\TB\ DRAM, a server-based \MN\ costs 1.1-1.5\x\ and consumes 1.9-2.7\x\ power compared to \sysboard. These numbers become 1.4-2.5\x\ and 5.1-8.6\x\ with OptaneDimm~\cite{optane-dcpm}, which we expect to be the more likely remote memory media in future systems.


We measure the total energy used for running YCSB workloads
by collecting the total CPU (or FPGA) cycles and the Watt of a CPU core~\cite{gold5128}, ARM processor~\cite{armpower}, and FPGA (measured).
We omit the energy used by DRAM and NICs in all the calculations. 
Clover, a system that centers its design around low cost, has slightly higher energy than \sys.
Even though there is no processing at \MN{}s for Clover, its \CN{}s use more cycles to process and manage memory.
HERD consumes 1.6\x\ to 3\x\ more energy than \sys, mainly because of its CPU overhead at \MN{}s.
Surprisingly, HERD-BF consumes the most energy, even though it is a low-power ARM-based SmartNIC.
This is because of its worse performance and longer total runtime.

Figure~\ref{fig-fpga-resource} compares the FPGA utilization among Clio, StRoM's RoCEv2~\cite{StRoM}, and Tonic's selective ack stack~\cite{TONIC}.
%With our design that is tailored to save resources, 
%\sys\ consumes roughly one third of the total resources.
Both StRoM and Tonic include only a network stack but they consume more resources than \sys.
Within \sys, the virtual memory (VirtMem) and
the network stack (NetStack) consume a small fraction of the total resources,
with the rest being vendor IPs (PHY, MAC, DDR4, and interconnect).
%To put things in perspective, we implement a Go-back-N network stack which supports 1K connections. It uses 2.5\x\ more logic than what our current network stack consumes. 
Overall, our efficient hardware implementation leaves most FPGA resources available for application offloads.

%\vspace{-0.05in}
\section{Discussion and Conclusion}
\label{sec:discussion}
%\vspace{-0.05in}

%We presented a new hardware-based active disaggregated memory model
%and \sys, a real system built with this model.
We presented \sys, a new hardware-based disaggregated memory system.
Our FPGA prototype demonstrates that \sys\ achieves great performance, scalability, and cost-saving.
This work not only guides the future development of \md\ solutions
but also demonstrates how to implement a core OS subsystem in hardware and co-design it with the network.
We now present our concluding thoughts.
%\sys\ outperforms passive disaggregated memory system and 
%achieves comparable performance as CPU-based active disaggregated memory,
%while significantly reduces the monetary cost of server-based disaggregated memory systems.


%\ulinebfpara{Data caching.}

\ulinebfpara{Security and performance isolation.}
\sys’s protection domain is a user process, which is the same as the traditional single-server process-address-space-based protection. The difference is that \sys\ performs permission checks at MNs: it restricts a process’ access to only its (remote) memory address space and does this check based on the global PID. Thus, the safety of \sys\ relies on PIDs to be authentic (\eg, by letting a trusted CN OS or trusted CN hardware attach process IDs to each \sys\ request). There have been researches on attacking RDMA systems by forging requests~\cite{ReDMArk-security21} and on adding security features to RDMA~\cite{1RMA,sRDMA-ATC20}. How these and other existing security works relate and could be extended in a memory disaggregation setting is an open problem, and we leave this for future work.
%We believe we could draw inspiration from these research works to add security features to Clio, which we leave to future work.

There are also designs in our current implementation that could be improved to provide more protection against side-channel and DoS attacks.
For example, currently, the TLB is shared across application processes,
and there is no network bandwidth limit for an individual connection.
Adding more isolation to these components would potentially increase the cost of \sysboard\ or reduce its performance.
We leave exploring such tradeoffs to future work.
%It is fairly easy to change them to provide more isolation and SLA guarantees,
%and we leave it for future work. % isolation between processes.

\ulinebfpara{Failure handling.}
Although memory systems are usually assumed to be volatile, % (\ie, in-memory data can be lost),
there are still situations that require proper failure handling (\eg, for high availability or to use memory for storing data).
As there can be many ways to build memory services on \sys\ 
and many such services are already or would benefit from handling failure on their own,
we choose not to have any built-in failure handling mechanism in \sys.
Instead, \sys\ should offer primitives like replicated writes for users to build their own services.
We leave adding such API extensions to \sys\ as future work.
%The usage of \sys\ ranges from temporarily hosting cached or ephemeral data to permanently storing data.

\ulinebfpara{\CN-side stack.}
An interesting finding we have is that \CN-side systems
could become a performance bottleneck after we made the remote memory layer very fast.
Surprisingly, most of our performance tuning efforts are spent on the \CN\ side (\eg, thread model, network stack implementation).
Nonetheless, software implementation is inevitably slower than customized hardware implementation.
Future works could potentially improve \sys's \CN\ side performance by offloading the software stack to a customized hardware NIC.
%More future research could go into finding the best way for client applications to fully exploit \sys's remote memory performance.

%\vspace{-0.02in}
\if 0
%\vspace{-0.05in}
\section{Conclusion}
\label{sec:conclude}
%\vspace{-0.05in}

We presented \sys, a new hardware-based disaggregated memory system.
Our FPGA prototype demonstrates that \sys\ achieves great performance, scalability, and cost saving.
This work not only guides the future development of \md\ solutions
but also demonstrates how to implement a core OS subsystem in hardware and co-design it with the network.
%and \sys, a real system built with this model.
%\sys\ outperforms passive disaggregated memory system and 
%achieves comparable performance as CPU-based active disaggregated memory,
%while significantly reducing the monetary cost of server-based disaggregated memory systems.
%
\fi

\section*{Acknowledgement}

We would like to thank the anonymous reviewers and our shepherd Mark Silberstein
for their tremendous feedback and comments, which have
substantially improved the content and presentation of this paper.
We are also thankful to Geoff Voelker, Harry Xu, Steven Swanson, Alex Forencich for their valuable feedback on our work.

This material is based upon work supported by the National
Science Foundation under the following grant: NSF 2022675, and gifts from VMware.
Any opinions, findings, and conclusions or recommendations
expressed in this material are those of the authors and do not 
necessarily reflect the views of NSF or other institutions.



%{
\begin{table*}[th]\footnotesize
\begin{center}
\begin{tabular}{ p{1.0in} | p{0.3in} |p{0.35in} }

 & \textbf{Logic} & \textbf{Memory} \\
\textbf{System/Module} & \textbf{(LUT)} & \textbf{(BRAM)} \\
\hline
\hline
StRoM-RoCEv2-10G & 39\% & 76\% \\
Tonic-SACK-100G & 48\% & 40\% \\
\hline
\sys\ (Total) & 30\% & 31\% \\
VirtMem & 4.8\% & 3\% \\
NetStack & 2.3\% & 1.7\% \\
\hline
Go-Back-N & 5.8\% & 2.6\% \\

\end{tabular}
\end{center}
\mycaption{tbl-resource-util}{XXX}
{
XXXX
}
%\vspace{-0.1in}
\end{table*}
}
\fi

\chapter{Disaggregating and Consolidating Network Functionalities with SuperNIC}

\if 0
%\NOTE{Check defs.tex for comment marcos.}

\begin{abstract}

Resource disaggregation has gained huge popularity in recent years.  %in both academia and industry. 
Existing works demonstrate how to disaggregate compute, memory, and storage resources. We, for the first time, demonstrate how to disaggregate {\em network resources} by proposing a network resource pool that consists of a new hardware-based network device called {\em SuperNIC}. 
Each SuperNIC consolidates network functionalities from multiple endpoints by fairly sharing limited hardware resources, and it achieves its performance goals by an auto-scaled, highly parallel data plane and a scalable control plane.
We prototyped SuperNIC with FPGA and demonstrate its performance and cost benefits with real network functions and customized disaggregated applications.

\if 0
For decades, the unit of deployment, operation, and failure in datacenters has been a monolithic server,
one that contains all the hardware resources needed to run a user program.
This decade-old server architecture is seeing its limits in the face of today’s datacenter needs.
Hardware resource disaggregation is a solution that breaks full-blown,
general-purpose servers into segregated, network-attached hardware resource pools,
each of which can be built, managed, and scaled independently.

While increasing amounts of effort go into disaggregating compute, memory, and storage,
the fourth major resource in computing, \textit{network},
has been completely left out.
No work has attempted to disaggregate the network.
At first glance, network cannot be disaggregated from either
a traditional monolithic server or a disaggregated device,
as they both need to be attached to the network and
each endpoint is provisioned with its own network interface and the associated network stack.

In this research exam,
I will first present our preliminary study
to motivate network disaggregation and consolidation.
Then I will present a detailed survey on all possible solutions and show why they fall short.
Those solutions include programmable switch, circuit switch, coherent fabrics, middlebox, NFV, and multi-host NIC.
Finally, I will present our initial work, an FPGA-based new network device that meets
all our goals for network disaggregation and consolidation.
\fi

\end{abstract}

\section{Introduction}
\label{sec:snic:intro}

{\em Hardware resource disaggregation} is a solution that decomposes full-blown, general-purpose servers into segregated, network-attached hardware resource pools, each of which can be built, managed, and scaled independently. With disaggregation, different resources can be allocated from any device in their corresponding pools, exposing vast amounts of resources to applications and at the same time improving resource utilization. Disaggregation also allows data-center providers to independently deploy, manage, and scale different types of resources.
Because of these benefits, disaggregation has gained significant traction from both academia~\cite{LegoOS,FireBox-FASTKeynote,ATC20-pDPM,Nitu18-EUROSYS,DDC-hotcloud20,aifm-osdi20,Semeru,kona,InfiniSwap,FastSwap} and industry~\cite{HP-TheMachine,IntelRackScale,alibaba-polardb,facebook-disaggregation,SnowFlake-NSDI20}.

While increasing amounts of effort go into disaggregating compute~\cite{LegoOS,disagg-gpu}, memory (or persistent memory)~\cite{LegoOS,HP-TheMachine,Lim09-disaggregate,remote-region-atc18,ATC20-pDPM,Semeru,InfiniSwap,FastSwap,hotpot-socc17}, and storage~\cite{PolarFS-VLDB18,SnowFlake-NSDI20,hailstorm-asplos20,ana-eurosys16,gimbal}, the fourth major resource, \textit{network}, has been completely left out.
At first glance, ``network'' cannot be disaggregated from either a traditional monolithic server or a disaggregated device (in this paper collectively called {\em endpoints}), as they both need to be attached to the network.        
%To answer this question, we explore the minimal network functionalities an endpoint needs to have for its connectivity.
%\bolditpara{Proposal: what can be disaggregated?}
However, we observe that even though endpoints need basic connectivity, it is not necessary to run {\em network-related tasks} at the endpoints.
These network tasks, or {\em \nt}s, include the transport layer and all high-level layers such as network virtualization, packet filtering and encryption, and application-specific functions.
%everything including and above the transport layer can 
%each endpoint only needs to manage the connectivity and reliability of the {\em last hop} --- between the endpoint to its direct connection point, and thus only needs a link layer that can handle problems happening within the last hop.
%\noteys{the above reasoning does not make sense to me. we don't have enough context to setup "last hop".}
%Everything else can be disaggregated, including a transport layer for reliable end-to-end delivery, network functions like packet filtering and network virtualization, and application-specific functionalities such as data caching. We collectively call all these ``detachable'' functionalities {\em network tasks}, or {\em \nt}s.

This paper, for the first time, proposes the concept of {\em network disaggregation} and builds a real disaggregated network system to segregate \nt{}s from endpoints.
%systematically answers a set of key questions in network disaggregation.

%\bolditpara{Proposal: disaggregated network resource pool.}
At the core of our network-disaggregation proposal is the concept of a rack-scale disaggregated {\em network resource pool}, which consists of a set of hardware devices that can execute \nt{}s and collectively provide ``network'' as a service (Figure~\ref{fig-snic-topology}), similar to how today's disaggregated storage pool provides data storage service to compute nodes. 
Endpoints can offload (\ie, disaggregate) part or all of their \nt{}s to the network resource pool.
After \nt{}s are disaggregated, we further propose to {\em consolidate} them by aggregating a rack's endpoint \nt{}s onto a small set of network devices.
%\notearvind{might need to generalize to a network pool}
%, thereby reducing the total number of network .

We foresee two architectures of the network resource pool within a rack. The first architecture inserts a network pool between endpoints and the ToR switch by attaching a small set of endpoints to one network device, which is then connected to the ToR switch (Figure~\ref{fig-snic-topology} (a)). The second architecture attaches the pool of network devices to the ToR switch, which then connects to all the endpoints (Figure~\ref{fig-snic-topology} (b)). 

%\bolditpara{Motivating: what are the potential benefits of disaggregating and consolidating \nt{}s?}
%Same as disaggregating other resources like storage, 
Network disaggregation and consolidation have several key benefits.
(1) Disaggregating \nt{}s into a separate pool allows data center providers to build and manage network functionalities only at one place instead of at each endpoint. 
This is especially helpful for heterogeneous disaggregated clusters where a full network stack would otherwise need to be developed and customized for each type of endpoint.
(2) Disaggregating \nt{}s into a separate pool allows the {\em independent scaling} of hardware resources used for network functionalities without the need to change endpoints.
(3) Each endpoint can use more network resources than what can traditionally fit in a single NIC. 
(4) With \nt\ consolidation, the total number of network devices can be reduced, allowing a rack to host more endpoints.
%The final and important benefit comes from consolidation.
(5) The network pool only needs to provision hardware resources for the peak \textit{aggregated} bandwidth in a rack instead of each endpoint provisioning for its own peak, reducing the overall CapEx cost.

Before these benefits can be exploited in a real data center, network disaggregation needs to first meet several goals, which no existing solutions fully support (see \S\ref{sec:snic:related}).

{
\begin{figure}
\begin{center}
\centerline{\includegraphics[width=\textwidth]{snic/Figures/fig-topology.pdf}}
\mycaption{fig-snic-topology}{Overall Architectures of \sysname.}
{
Two ways of connecting \snic{}s to form a disaggregated network resource pool. In (a), dashed lines represent links that are optional.
}
\end{center}
\end{figure}
}

%\bolditpara{Building: what are the key requirements of network disaggregation and consolidation?}
First, each disaggregated network device should meet endpoints' original performance goals even when handling a much larger (aggregated) load than what each endpoint traditionally handles.
The aggregated load will also likely require many different \nt{}s, ranging from transports to application-specific functionalities.
Moreover, after aggregating traffic, there are likely more load spikes (each coming from a different endpoint) that the device needs to handle.

Second, using a disaggregated network pool should reduce the total cost of a cluster. This means that each disaggregated network device should provision the right amount of hardware resources (CapEx) and use as little of them as needed at run time (OpEx). At the same time, the remaining part of a rack (\eg, endpoints, ToR switch, cables) needs to be carefully designed to be low cost.

Third, as we are consolidating \nt{}s from multiple endpoints, in a multi-tenant environment, there would be more entities that need to be isolated. We should ensure that they fairly and safely share various hardware resources in a disaggregated network pool. 

Finally, network devices in a pool need to work together so that lightly loaded devices can handle traffic for other devices that are overloaded.
This load balancing would allow each device to provision less hardware resources as long as the entire pool can handle the peak aggregated load of the rack.

%key challenges consolidation
%sharing, autoscaling, dist
%control plane scalability

Meeting these requirements together is not easy as they imply that the disaggregated network devices need to use minimal and properly isolated hardware resources to handle large loads with high variation, while achieving application performance as if there is no disaggregation.

To tackle these challenges and to demonstrate the feasibility of network disaggregation, we built \textit{\textbf{SuperNIC}} (or \textit{\snic} for short), a new hardware-based programmable network device designed for network disaggregation.
%why new hardware-based sNIC. functions like transport need high speed parallel processing, and software is too slow for that. however, traditional NIC hardware or hardware-based SmartNIC does not offer the autoscaling or fair sharing feature we need for consolidation.
An \snic\ device consists of an ASIC for fixed systems logic, FPGA for running and reconfiguring \nt{}s, and software cores for executing the control plane.
We further built a distributed \snic\ platform that serves as a disaggregated network pool.
Users can deploy a single \nt\ written for FPGA or a directed acyclic graph (DAG) execution plan of \nt{}s to the pool.

To tightly \textbf{consolidate} \nt{}s within an \snic, we support three types of resource sharing: (1) splitting an \snic's hardware resources across different \nt{}s ({\em space sharing}), (2) allowing multiple applications to use the same \nt{} at different times ({\em time sharing}), and (3) configuring the same hardware resources to run different \nt{}s at different times ({\em time sharing with context switching}).
For space sharing, we partition the FPGA space into {\em region}s, with each hosting one or more \nt{}s.
Each region could be individually {\em reconfigured} (via FPGA partial reconfiguration, or {\em PR}) for starting new \nt{}s or to context switch \nt{}s.
Different from traditional software systems, hardware context switching with PR is orders of magnitude slower, which could potentially impact application performance significantly.
To solve this unique challenge, we propose a set of policies and mechanisms to reduce the need to perform PR or to move it off the performance-critical path, \eg, by keeping de-scheduled \nt{}s around like a traditional victim cache, by not over-reacting to load spikes, and by utilizing other \snic{}s when one \snic\ is overloaded.
%\notearvind{Might be worth saying that we also rely on other sNICs' resources if a local sNIC is overloaded.}

To achieve high \textbf{performance} under large, varying load with minimal cost, we automatically scale (auto-scale) an \nt{} by adding/removing instances of it and sending different flows in an application across these instances.
%\noteyiying{@Yizhou, do we send different flows to different instances or it's packet level? --- YS: We use flows. We cannot do individual packet LB, because there are states associated with each flow.}
We further launch different \nt{}s belonging to the same application in parallel and send forked packets to them in parallel for faster processing.
%We achieve high throughput using two levels of parallelism:
%{\em \nt{} parallelism} where a packet goes through multiple \nt{}s in parallel and {\em instance parallelism} where we launch multiple instances of the same \nt{} to handle different packets in an application.
%Apart from the above data-plane designs, we build a scalable control plane.
%To achieve low scheduling latency and scalability, we 
%propose a scheduler that centers around a new notion, {\em \nt\ chaining}.
%The idea is to 
To achieve low scheduling latency and improve scalability, we group \nt{}s that are likely to be executed in a sequence into a chain.
% and to have our central scheduler schedule packets only once for the entire chain. 
Our scheduler reserves credits for the entire chain as much as possible so that packets execute the chain as a whole without involving the scheduler in between.
%Doing so improves both packet-processing latency and scheduler scalability.

To provide \textbf{fairness}, we adopt a fine-grained approach that treats each internal hardware resource separately, \eg, ingress/egress bandwidth, internal bandwidth of each shared \nt, payload buffer space, and on-board memory, as doing so allows a higher degree of consolidation.
%Our context is unique in that the packet processing system itself requires multi-dimensional resource sharing. 
We adopt Dominant Resource Fairness (DRF)~\cite{DRF} for this multi-dimensional resource sharing.
%For the first time in networking systems, we consider multi-dimensional resource sharing and provide Dominant Resource Fairness (DRF)~\cite{DRF}. 
Instead of user-supplied, static per-resource demands as in traditional DRF systems, we monitor the actual load demands at run time and use them as the target in the DRF algorithm.
Furthermore, we propose to use ingress bandwidth throttling to control the allocation of other types of resources.
We also build a simple virtual memory system to \textbf{isolate and protect} accesses to on-board memory. %All \nt's memory accesses use 

Finally, for \textbf{distributed \snic{}s}, we automatically scale out \nt{}s beyond a single \snic\ when load increases and support different mechanisms for balancing loads across \snic{}s depending on the network pool architectures.
For example, with the switch-attached pool architecture, we use the ToR switch to balance all traffic across \snic{}s.
With the intermediate pool architecture, we further support a peer-to-peer, \snic-initiated load migration when one \snic\ is overloaded.

We prototype \snic\ with FPGA using two 100\Gbps, multi-port HiTech Global HTG-9200 boards~\cite{htg9200}.
%The data plane runs on FPGA directly, while the control plane runs in software cores deployed on FPGA.
We build three types of \nt{}s to run on \snic:
reliable transport, traditional network functions, and application-specific tasks, and port two end-to-end use cases to \snic.
The first use case is a key-value store we built on top of real disaggregated memory devices~\cite{Clio}.
We explore using \snic{}s for traditional \nt{}s like the transport layer and customized \nt{}s like key-value data replication and caching.
%For the latter, the client only needs to send one copy to the \snic, which will send copies of the data to multiple memory devices.
%customized network abstraction for disaggregated memory device: a key-value store interface (rather than the standard messaging interface).
%Furthermore, we 
The second use case is a Virtual Private Cloud application we built on top of regular servers by connecting \snic{}s at both the sender and the receiver side.
We disaggregate \nt{}s like encapsulation, firewall, and encryption to the \snic{}s.
%a go-back-N reliable transport; a set of network functions including firewall, AES encryption, and VPN Gateway; and a set of application-specific functions including key-value store replication and caching.
We evaluate \snic\ and the ported applications with micro- and macro-benchmarks and compare \snic\ with no network disaggregation and disaggregation using alternative solutions such as multi-host NICs and a recent multi-tenant SmartNIC~\cite{panic-osdi20}.
Overall, \snic\ achieves 52\% to 56\% CapEx and OpEx cost savings with only 4\% performance overhead compared to a traditional non-disaggregated per-endpoint SmartNIC scenario.
%Our results running a Facebook key-value trace~\cite{Atikoglu12-SIGMETRICS} show that \snic's consolidation of four endhosts and two \nt{}s saves 64\% costs compared to no consolidation, with only 1.3\% performance overhead.
Furthermore, the customized key-value store caching and replication functionalities on \snic\ improves throughput by 1.31\x\ to 3.88\x\ and latency by 1.21\x\ to 1.37\x\ when compared to today's remote memory systems with no \snic.
\section{Motivation and Related Works}
\label{sec:motivation}

\subsection{Benefits of Network Disaggregation}
\label{sec:benefits}
As discussed in \S\ref{sec:intro}, disaggregating network functionalities into a separate pool has several key benefits for data centers, some of which are especially acute for future disaggregated, heterogeneous data centers~\cite{LegoOS,last-cpu-hotos, FratOS-eurosys}.

\bolditpara{Flexible management and low development cost.}
Modern data centers are deploying an increasing variety of network tasks to endpoints, usually in different forms (\eg, software running on a CPU, fixed-function tasks built as ASIC in a NIC, software and programmable hardware deployed in a SmartNIC). 
It requires significant efforts to build and deploy them to different network devices on regular servers and to different types of disaggregated hardware devices.
After deployment, configuring, monitoring, and managing them on all the endpoints is also hard.
%, even in a homogeneous cluster.
In contrast, developing, deploying, and managing network tasks in a disaggregated network pool with homogeneous devices is easy and flexible.

\bolditpara{Independent scaling.}
It is easy to increase/decrease network hardware resources in a disaggregated network pool by adding/removing network devices in the pool. Without disaggregation, changing network resources would involve changing endpoints (\eg, upgrading or adding NICs).

\bolditpara{Access to large network resources.}
With disaggregation, an endpoint can potentially use the entire network pool's resources, far beyond what a single NIC or server can offer.
This is especially useful when there are occasional huge traffic spikes or peak usages of many \nt{}s.
Without network disaggregation, the endpoint would need to install a large NIC/SmartNIC that is bloated with features and resources not commonly used~\cite{SmartNIC-nsdi18,Caulfield-2018}.

Beside the above benefits, a major benefit of network disaggregation is cost savings. A consolidated network pool only needs to collectively provision for the peak aggregated traffic and the maximum total \nt{}s used by the whole rack at any single time.
In contrast, today's non-disaggregated network systems require each endpoint to provision for its individual peak traffic and maximum \nt{}s.
To understand how significant this cost is in the real world, we analyze a set of traces from both traditional server-based production data centers and disaggregated clusters.

\bolditpara{Server-based data center traffic analysis.}
To understand network behavior in server-based data centers, we analyze two sets of traces: a Facebook trace that consists of Web, Cache, and Hadoop workloads~\cite{facebook-sigcomm15}, and an Alibaba trace that hosts latency-critical and batch jobs together~\cite{alibaba-trace}. 

We first perform a consolidation analysis where we calculate the sum of peaks in each individual endhost's traffic (sum of peak) and the peak of aggregated traffic within a rack and across the entire data center (peak of sum). 
These calculations model the cases of no disaggregation, disaggregation and consolidation at the rack level and at the data-center level.
%The former informs us about the total amount of network resources that needs to be provisioned if each endhost provisioned for its own peak, and the latter corresponds to the amount of resources a disaggregated network pool needs to provision.
Figure~\ref{fig-fb-alibaba} shows this result for the two data centers. 
For both of them, a rack-level consolidation consumes one magnitude fewer resources than no consolidation.

We then analyze the load spikes in these traces
%Our first finding is that \fixme{XXX}\% and \fixme{XXX}\% spikes are at least one and two seconds long.
by comparing different endhosts' spikes and analyzing whether they spike at similar or different times, which will imply how much chance there is for efficient consolidation.
Specifically, we count how much time in the entire 1-day trace $X$ number of endhosts spike together.
Figure~\ref{fig-spike-var} shows that 55\% of the time only one or two servers spike together, and only 14\% of the time four or more servers spike together.
This result shows that servers mostly spike at different times, confirming the large potential of consolidation benefits.
%\ryan{We analyze the portion of time N servers are spiking for a given time period. Figure~\ref{fig-spike-var} shows that 55\% of the time 1 to 2 servers are peaking. And only 14\% of the time do 4 servers peak concurrently.
%Overall, it confirms that although individual server's traffic is spiky, the correlation among servers is low and the number of servers concurrently peaking is small.}
%
%Both data centers exhibit high burstiness, and bursts happen at different times for different end hosts. % 
%Other prior work has also reported similar findings~\cite{netkernel-atc20,Gao16-OSDI}.

\bolditpara{Disaggregated cluster traffic analysis.}
Resource disaggregation introduces new types of network traffic that used to be within a server, \eg, a CPU device accesses data in a remote memory device. 
If not handled properly, such traffic could add a huge burden to the data-center network~\cite{sirius-sigcomm20}.
To understand this type of traffic, we analyzed a set of disaggregated-memory network traces collected by Gao et al. using five endhosts~\cite{Gao16-OSDI}.
Figure~\ref{fig-OSDI16NetTrace} plots the CDF and the timeline of network traffic from four workloads.
These workloads all exhibit fluctuating loads, with some having clear patterns of high and low load periods.
We further perform a similar analysis using sum of peaks vs. peak of aggregated traffic as our server-based trace analysis.
Consolidating just five endhosts already results in 1.1\x\ to 2.4\x\ savings with these traces.





\subsection{Limitations of Alternative Solutions}
\label{sec:related}

The above analysis makes a case for disaggregating and consolidating network tasks from individual servers and devices.
A question that follows is \textit{where} to host these \nt{}s and whether existing solutions could achieve the goals of network disaggregation and consolidation.
%\notearvind{I will refine this a bit more and see whether Gimbal should be incorporated here or somewhere else.}

%\noteys{different from the p4 switch, the aruba switch seems another type of programmable switch that maybe able to support more complex ops}
%
The first possibility is to host them at a \textbf{programmable ToR switch}. Programmable switches allow for configurable data planes, but they typically support only a small amount of computation at high line rates. SmartNICs, on the other hand, handle more stateful and complex computations but at lower rates. Transport protocol processing and encrypted communications are examples of complex network tasks better supported by a SmartNIC than a programmable switch. Moreover, existing programmable switches lack proper multi-tenancy and consolidation support~\cite{Wang-HotCloud20}. As a consequence, most data center designs require the use of SmartNICs even in the presence of programmable switches, and our proposal simply disaggregates SmartNIC-like capabilities into a consolidated tier.


Another possibility is upcoming \emph{multi-host SmartNICs} (e.g., Mellanox BlueField3) that are enhancements of today's \textbf{multi-host NICs}~\cite{ocp-nic,Intel-RedRockCanyon}. These NICs connect to multiple servers via PCIe connections and provide general-purpose programmable cores. Our work identifies three key extensions to such devices. (1) Our approach enables a greater degree of aggregation as we enable coordinated management of a distributed pool of network devices. (2) Moreover, in the case of these multi-host SmartNICs, \nt{}s that cannot be mapped to the NIC's fixed functions have to be offloaded as software. In contrast, \snic\ allows the acceleration of \nt{}s in hardware, enabling higher performance while tackling issues related to runtime reconfigurability of hardware. (3) Our approach provides adaptive mechanisms for adjusting to workloads and providing fair allocation of resources across applications. It is also worth noting that (1) and (3) can by themselves be used in conjunction with commercial multi-host SmartNICs to achieve a different software-based instantiation of our vision.
%\noteys{"multi-host SmartNICs support NT offloading only in software", not quite true. those smartNICs have ASIC NFs and FPGA (like IPU). its better to mention MH-NIC are generally used to offload software NTs.}\notearvind{Reworded it a bit. Specified the possibility of fixed functions. Also narrowly defined it as SmartNICs with programmable cores. BTW, I don't think I have seen a MH-Innova, which would be the closest to our system.}

\textbf{Middleboxes} are a traditional way of running network functions inside the network either through hardware black-box devices that cannot be changed after deployment~\cite{aplomb-sigcomm20,comb-nsdi12,walfish-osdi04} or through server-based Network Function Virtualization (NFV) that enables flexible software-based middleboxes~\cite{clickos-nsdi14,e2-sosp15,metron-nsdi18,NFP-sigcomm17,parabox-sosr17}, but at the cost of lower performance~\cite{netbricks,netvm-nsdi14}.  Our deployment setting differs from traditional datacenter middleboxes: we target "nearby" disaggregation, as in the endhost or SmartNIC tasks are disaggregated to a nearby entity typically located on the same rack. Consequently, our mechanisms are co-designed to take advantage of this locality (e.g., we use simple hardware mechanisms for flow control between the end-host and the disaggregated networking unit). Further, we target network functionality that is expected either at the endhost itself or at the edge of the network, such as endhost transport protocols, applying network virtualization, enhancing security, which all require nearby disaggregation and are also not typically targeted by middleboxes. We do note that our dynamic resource allocation mechanisms are inspired by related NFV techniques, but we apply them in the context of reconfigurable hardware devices.


%\yiying{Also mention our distributed pooling idea which is not in any of the above existing solutions (maybe need to mention works that use distributed programmable switches/SmartNICs if any)}


%Apart from our \snic\ proposal, there are several options.

% The first possibility is to host them at the \textbf{ToR switch}.
% This approach lets endpoints go through only one hop to the ToR switch, compared to two hops with approaches that use a middle layer like \snic.
% However, it requires ToR switches to be programmable and have more ports to connect to disaggregated devices, both of which add monetary costs and requires changes to the data-center network infrastructure~\cite{zhao-nsdi19,zhang-nsdi19}.
% Moreover, existing programmable switches lack proper multi-tenancy and consolidation support~\cite{Wang-HotCloud20}.  %such as flexible space and time sharing
%\yizhou{maybe build the relation to in-network computing (INC)? traditional pswitch work like netchain/netkv/ATP offload certain app functionlaties to the network. it is a specific case of network-disaggregation-and-consolidation. And we are proposing a larger-scope and more generic scheme.}



% \textbf{Middleboxes} are a traditional way of running network functions inside the network.
% Traditional hardware middleboxes are specialized black-box devices that cannot be changed after deployment~\cite{aplomb-sigcomm20,comb-nsdi12,walfish-osdi04}. Network Function Virtualization uses regular servers to build flexible software-based middleboxes~\cite{clickos-nsdi14,e2-sosp15,metron-nsdi18,NFP-sigcomm17,parabox-sosr17}, but at the cost of running at lower performance~\cite{netbricks,netvm-nsdi14}. 
% \snic\ has the benefits of both: it is flexible as it supports offloading a wide range of \nt{}s and can be reconfigured at run time, and it also achieves high-bandwidth line rate processing.

%Increasing amount of data centers attach \textbf{SmartNICs} and \textbf{specialized ASIC/FPGA devices} like Intel IPU~\cite{intel-ipu}, Amazon Nitro~\cite{aws-nitro}, and Microsoft Catapult~\cite{Catapult-v2} to single servers.
%These devices can host offloaded network functions and other customized tasks. However, they do not support the consolidation of multiple endpoints.

\if 0
Optical circuit switch is an emerging solution to build disaggregated datacenters as it offers high port count and consumes much less energy than traditional electrical packet switches~\cite{shoal-nsdi19,helios-sigcomm10,sirius-sigcomm20, sipml-sigcomm21}. Circuit switch reconfigures and reconnects physical connections among ports. As a result, it has no computation on its data path, thus not able to consolidate any network functions.
\fi

Finally, there are emerging \textbf{interconnections designed for disaggregated devices} such as Gen-Z~\cite{GenZ} and CXL~\cite{CXL}.
These solutions mainly target the coherence problem where the same data is cached at different disaggregated devices.
The transparent coherence these systems provide requires new hardware units at every device, in addition to a centralized manager.
\snic\ supports the disaggregation and consolidation of all types of network tasks and does not require specialized units at endpoints.

%In summary, although different existing network solutions provide some features of network disaggregation and consolidation, none of them meet all our target goals, thus necessitating the design of a new network solution.
%{
\begin{table*}[th]\smallsize
\begin{center}
\begin{tabular}{ p{1.7in} | p{0.5in} | p{0.85in} | p{0.6in} | p{0.55in} | p{0.8in}  | p{0.7in} }

\textbf{Solution} & \textbf{Port Count \textasteriskcentered} & \textbf{Heterogeneous End-Points \textasteriskcentered} & \textbf{Offloaded Transport} & \textbf{Network Function} & \textbf{Manageability} & \textbf{Consolidated Resources} \\
\hline
\hline
Programmable Switch~\cite{RMT-SIGCOMM13,netcache-sosp17} & \xmark & \cmark & \cmark & \cmark  & $\bigcirc$  & $\bigcirc$ \\
\hline
Circuit Switch~\cite{sirius-sigcomm20,shoal-nsdi19,dRedBox-DATE} & \cmark & \cmark & \xmark & \xmark  & \xmark  &\xmark \\
\hline
Coherent Fabrics~\cite{GenZ,CXL,CCIX} & $\bigcirc$ & \cmark & \xmark & \xmark & \xmark & \xmark\\
\hline
Middleboxes~\cite{walfish-osdi04,comb-nsdi12,aplomb-sigcomm20} & \xmark  & \xmark  & \xmark & \cmark & \xmark  & \cmark  \\
\hline
NFV~\cite{clickos-nsdi14,e2,netbricks} & \xmark  & \xmark  & \xmark & \cmark & \cmark & \cmark \\
\hline
Multi-Host NIC~\cite{Intel-RedRockCanyon,Mellanox-Multihost} & \cmark &  $\bigcirc$ & \xmark & \xmark & \xmark  & $\bigcirc$ \\
\hline
\hline
\textbf{\sysname}     & \cmark & \cmark &  \cmark & \cmark & \cmark & \cmark \\
\hline
\end{tabular}
\end{center}
\mycaption{tabel-related-work}{Comparison of Network Solutions.}
{
\textasteriskcentered\ features only applicable to disaggregated datacenters.
$\bigcirc$ partially support.
}
\end{table*}
}








\if 0
\subsection{for Server-Based Datacenters}
\label{sec:motivation-server}
%scale, computation tax, underutilization etc.

%Today, each server runs a full network stack (either in the host CPU or in a NIC), and many start to execute complex network functions and application-specific tasks~\cite{flexnic-asplos16,snap-sosp19}.
%(\eg, FlexNIC~\cite{flexnic} demonstrates the benefits of application-specific network handling and iPipe~\cite{iPipe} factors out a distributed application into a collection of actor-based NFs). 
%Network stacks in today's data centers could consume 30\%-40\% host CPU cycles with the presence of a high-speed network~\cite{tonic-nsdi20}.
%However, network communication only happens for a small amount of time during application execution, and not all network functions are always invoked. 
%
To understand network behavior in real, server-based data centers, we analyze two sets of traces: a Facebook trace that consists of Web, Cache, and Hadoop workloads~\cite{facebook-sigcomm15}, and an Alibaba trace that hosts latency-critical and batch jobs together~\cite{alibaba-trace}. 
Both data centers exhibit high burstiness, and bursts happen at different times for different end hosts. Other prior work has also reported similar findings~\cite{netkernel-atc20,Gao16-OSDI}. We omit the CDF and timeline plots due to space constraints. 
We perform a similar consolidation analysis as the disaggregated memory traces, as shown in Figure~\ref{fig-fb-alibaba}. 
%Specifically, we first measure the peak load of every end host and then sum all these peaks across the whole data center (or in Facebook's case, an entire workload). This case uses no consolidation (\ie, today's scheme) and the sum is the total amount of resources that a data center needs to provision.
In addition to the sum of individual-endhost peaks and the peak of aggregated traffic across the entire data center (or, in Facebook's case, an entire workload),
%we model the case where each rack's network traffic could be handled in a consolidated way (\eg, with \snic{}s). We first
we sum the traffic under a rack %for each time point and then measure the peak of the summed traffic. Afterwards, we 
and sum all the per-rack peaks.
%Finally, we model the unrealistic case of consolidating the entire data-center's traffic by summing the traffic of the whole data center at each time point and then measuring the peak of the sums.
For both Facebook and Alibaba, rack-level consolidation consumes one to two orders of magnitude fewer resources than no consolidation.


\if 0
We measure the duration of high-intensity communications or bursts at 25\mus\ granularity.
%We say that a switch's egress link is \textit{hot} if, for the measurement period, its utilization exceeds 50\%. An unbroken sequence of hot samples indicates a \textit{burst}. 
Figure~\ref{fig-burst_duration} presents the CDF of burst durations across three workloads.  We observe that a significant fraction of these bursts are only one sampling period long. The 90th percentile duration is less than 200\mus\ for all three rack types.
%, with Web racks having the lowest 90th percentile burst duration at 50us (two sampling periods). Hadoop racks have the longest tail of the three, but even then, almost all bursts concluded within 0.5\,ms.
The results indicate that bursts not only exist, almost all high utilization at the edge of the data center network is part of a burst, and that the durations of these high-intensity communications is short.
\fi

In addition to the bursty traffic patterns that are conducive to the consolidation benefits of \sysname,
the under-utilization of network functions in today's network devices is another major motivation for \sysname. 
For example, existing NICs are bloated with features that are not utilized in the common case~\cite{SmartNIC-nsdi18,Caulfield-2018}. 
Wang et al.~\cite{Wang-HotCloud20} reported that state-of-the-art programmable switches have low resource utilization (common NFs consume less than 1\%), 
and even a complex application consumes only a small fraction of resources (\eg, NetChain~\cite{netchain-nsdi18} uses 3\% MAT resources). 
Over time, vendors keep adding more features into their network device products to meet the diverse requirements from different customers, resulting in significant resource waste that could otherwise be avoided by \sysname's consolidation solution. 

% Crucially, for both cases above, the consolidation enables statistical multiplexing of resources that allows us to move away from provisioning for peak utilization on a per-node basis to provisioning for the expected peak utilization on a rack basis.
Finally, consolidating network tasks into a separate pool makes it easy for datacenter operators to manage them (\eg, monitoring, reconfiguring, and upgrading).

%The offloading of the network stack and host-side network functions (\eg, Open vSwitch) to the \snic{} has associated cost savings as we can substitute the use of expensive x86 cores with cheaper circuitry at the \snic.  Moreover, in all of these cases, network consolidation allows us to provision less networking associated compute resources at the \snic{} when compared to the traditional non-consolidated deployment. 


\subsection{for Disaggregated Datacenters}

Resource disaggregation is a data-center architecture that organizes different hardware resources into separate, network-attached pools.
While today's data centers use regular servers to form these pools~\cite{alibaba-polardb,SnowFlake-NSDI20,Borg-eurosys20}, 
future data centers could benefit from using specialized {\em devices} to build such pools, \eg, a memory pool consisting of network-attached memory~\cite{clio-arxiv} or Optane boards~\cite{HP-TheMachine,ATC20-pDPM}.
Three practical and key technical hurdles need to be solved before data centers can readily deploy such device-based disaggregated resource pools. Network disaggregation solves all of them.


%More data centers are migrating to a disaggregated architecture where different resources are managed as individual, network-attached pools~\cite{Alibaba,Facebook,snowflake-nsdi20,google-paper,XXX}.
%Nodes in a resource pool can both be a regular server (\eg, a server dedicated to provide data storage~\cite{snowflake-nsdi20,XXX}) or a network-attached device (\eg, a persistent-memory device board~\cite{HP-TheMachine,XXX}).
%There is a growing demand for deploying specialized devices (such as memory devices, network-attached NVMes, and key-value storage devices) inside the data center. 

%Network disaggregation is especially useful in building and deploying such disaggregated devices by solving three key problems at the same time.
First, when a monolithic server is replaced with multiple network-attached disaggregated devices (\eg, one CPU processor, one memory device, one storage device to replace a server), the number of network endpoints could increase to hundreds per rack~\cite{shoal-nsdi19}. % and an order of magnitude more than what a ToR switch could handle).
If all these devices directly connect to a ToR switch, the rack needs to use an expensive, high-port-count ToR switch or multiple low-port-count ToR switches (and the resulting increased scale of the entire switch hierarchy~\cite{zhang-nsdi19,zhao-nsdi19}).
%, and all the network infrastructure above ToR switches may also need to be upgraded.
With our proposed architecture, devices in a rack connect to a small set of \snic{}s that then can be accommodated by today's data-center ToR switch.
%, thus requiring no network infrastructure changes in existing data centers.

Second, building different types of disaggregated devices involves adding standard networking functionalities like a reliable transport layer to each of them, with many of them also desiring various customized network functions.
Disaggregated devices will come in many forms, some with a software processor~\cite{LegoOS}, some with only hardware units~\cite{ATC20-pDPM}, and some with both software and hardware~\cite{clio-arxiv}. Designing and implementing network tasks in each type of disaggregated device will be a daunting job. Moreover, it likely will involve adding additional hardware units to the devices.
By building \nt{}s once for a single type of hardware (\ie, \snic), we could significantly save development and CapEx costs.
%that would otherwise be required at every disaggregated device.
%three types of costs in a disaggregated device:
%{\em development cost} to build network stacks in heterogeneous hardware (\eg, ASIC, FPGA, general-purpose processor),
%{\em CapEx cost} for additional hardware network units in each device,
%and {\em energy cost} to run the network stacks there.
%
%Finally, when each device hosts its own network functionalities (and in heterogeneous hardware), 
%managing them will be difficult for data center providers.
%For example, to change the configuration of a network policy or to add a new network
%function to a set of disaggregated devices, each of them needs to change their network
%stack, which is hard and sometimes impossible as devices are often 
%manufactured by different vendors and have their (locked-in) hardware 
%implementations. 
%Network disaggregation makes it easy for operators to manage network functionalities for heterogeneous devices.

Finally, resource disaggregation introduces new types of network traffic that used to be within a server, \eg, a CPU device accesses a remote memory device to read/write data. 
If not handled properly, such traffic could add a huge burden to the data-center network~\cite{sirius-sigcomm20}.
To understand this type of traffic, we analyzed a set of disaggregated-memory network traces collected by Gao et al. using five endhosts~\cite{Gao16-OSDI}.
Figure~\ref{fig-OSDI16NetTrace} plots the CDF and the timeline of network traffic from four workloads.
These workloads all exhibit fluctuating loads, with some having clear patterns of high and low load periods.
We further compare the case where we provision network resources for the peak load of every workload and the case where we could consolidate and only provision for the aggregated network demand (\ie, sum of peaks vs. peak of aggregated traffic).
Consolidating just five endhosts already results in 1.1\x\ to 2.4\x\ savings with these traces.
%Overall, network consolidation allows us to provision for the {\em aggregated peak}, saving both CapEx and OpEx costs.
%The traffic is collected every five seconds.

\fi


\if 0
\section{Network Disaggregation and Consolidation}

%\NOTE{
%This section is similar to the sec2.2 in the proposal.
%1) Define the idea of "Network Disaggregation and Consolidation".
%Define a set of Key Goals: perf, cost, programmbility, consolidation and so on.
%2) present related fields.
%}
We propose to disaggregate and consolidate network functionalities.
The core of \sysname\ is a disaggregated network pool that sits in between endpoints and a ToR switch.
This pool consists of a set of \textit{SuperNICs} (\textit{\snic}s), each of which connects to the ToR switch (up link) and a small set of end hosts (down links).
In addition, all the \snic{}s are connected together through a ring or a torus.
We aim to support three broad types of end hosts:
regular servers, \textit{passive} disaggregated devices which only receive and handle network requests (\eg, a storage device that handles file I/O requests), 
and \textit{active} disaggregated devices which could both initiate and receive network requests (\eg, a memory device that handles memory read/write requests and swaps or flushes its memory data to a storage device).

\textbf{Architecture.}

\textbf{Goals.} A general set of goals, which we should show
that related work cannot meet.

\textbf{Related Work}
\TODO{Optional. We could describe
one system here as an example, leave the whole thing
to Related Work Section.}

\TODO{
Network disaggregation and consolidation is a concept.
SuperNIC is just one implementation choice.
So, before we dive into SuperNIC, we must explain
why SuperNIC, why other solutions cannot work.
Specifically, why not use pSwitch to realize this idea?
Why not use multihost NIC to do this?
We don't need a lengthy exploration here.
}

\fi

\if 0
\subsection{Resource Disaggregation}

A monolithic server has been the unit of deployment and operation
in datacenters for decades. This long-standing server-centric architecture
has several key limitations:
\textit{a) Inefficient resource utilization}. With a server being the physical
boundary of resource allocation, it is difficult to fully utilize all resources
in a datacenter. Reports show Google and Alibaba's datacenter usually only
have 40\%-60\% utilization~\cite{legoOS,Borg-eurosys20}. One of the main
reasons is the constraint that CPU and memory for a job have to be allocated
from the same physical machine.
\textit{b) Poor hardware elasticity}.
It is difficult to add, move, remove, and reconfigure
hardware components after they have been installed in
a monolithic server~\cite{FB-Wedge100}.
However, such plans have to be adjusted frequently because of
today's speed of change in application requirements.
\textit{c) Coarse failure domain}.
The failure unit of monolithic servers is coarse.
When a hardware component within a server fails, the whole
server is often unusable.
\textit{d) Bad support for heterogeneity}.
Driven by application needs, new hardware technologies
are finding their ways into datacenters (e.g., FPGA, GPU, and TPU).
However, datacenters often need to purchase
new servers to host certain hardware.
Other parts of the new servers can go underutilized~\cite{legoOS}.

Resource Disaggregation breaks the monolithic server model,
in which hardware resources in traditional servers are disseminated
into network-attached hardware devices.
Each device has a controller and a network interface,
can operate on its own, is an independent, failure-isolated entity.
The disaggregated approach largely increases the flexibility of a datacenter.
Applications can freely use resources from any hardware device,
which makes resource allocation easy and efficient.
Different types of hardware resources can scale independently.
It is easy to add, remove, or reconfigure devices.
New types of hardware devices can easily be deployed in a datacenter
— by simply enabling the hardware to talk to the network and adding a
new network link to connect it.
Finally, hardware resource disaggregation enables fine-grain failure isolation,
since one device failure will not affect the rest of a cluster

Disaggregation has been a very active research area for the past decade.
It was first proposed in early 2010s for memory disaggregation~\cite{Lim09-disaggregate}.
Over the years, researchers have further explored
disaggregation's impact on
operating system~\cite{legoOS}, storage~\cite{snowflake-nsdi20},
persistent memory~\cite{ATC20-pDPM},
data structure~\cite{aifm-osdi20},
failure model~\cite{disaggregation-hotcloud20}, and many more.
Overall, prior research focuses on the \textit{infrastructure},
as in integrating disaggregation with various existing systems,
thereby lays a solid system-level foundation.
Researchers now gradually shift their focus onto higher level components,
such as programming framework, security, and the focus of this paper, network.
Ultimately, to prepare a complete ecosystem for future disaggregated datacenters.

\subsubsection{Challenges for Network Design}

One of the base designs for disaggregation is that each device has an attached network interface.
However, the fourth major computation resource in datacenter, network, has been completely left out in the process of disaggregation.
The network is facing unprecedented challenges and we think
the disaggregation idea would not be practical if those challenges are not addressed properly

First, when splitting a monolithic server into multiple network-attached
devices, the number of network endpoints will dramatically increase
(potentially hundreds or even thousands of devices per rack~\cite{shoal-nsdi19}).
Attaching those devices directly to Top-of-Rack (ToR) switches are not feasible,
because it would increase the number of ToRs and subsequently spine and core switches dramatically in a Clos-based topology.
This poses a huge impact on existing datacenter networks, in terms of cost, power, and physical upgrade capability,
collectively called the \textit{network scale tax}~\cite{sirius-sigcomm20}.

Second, with the increasing degree of heterogeneity,
it is hard to deploy and run a consistent network stack across all those
heterogeneous disaggregated devices (e.g., ASIC, FPGA, and general-purpose CPUs).
Each device would need a network stack to have reliable connection with others.
Normally the network stack runs on top of a CPU or a programmable NIC,
but this is often not possible for heterogeneous devices.
In addition, driven by application demand, datacenters have the need to update their network
protocols especially congestion control algorithms very frequently~\cite{swift-sigcomm20}.
Thus, the network stack used by disaggregated devices should be universal, consistent
across devices, and easy to upgrade on the field.

Third, the network must provide low latency and high bandwidth for disaggregated devices and their applications.
With emerging applications like distributed ML training running directly on top of disaggregated devices,
network traffic is generated and consumed by hardware directly and hence, is expected
to grow even faster then the current trend (doubling every year~\cite{sirius-sigcomm20}).
Prior work~\cite{legoOS} shows that the network latency should be as low as 5 us to have reasonable performance (i.e., around 20\% slowdown).

\subsection{Current Datacenter Network Limitations}

The network in traditional server clusters are facing numerous
issues and challenges as well. 

As datacenters are moving to 100 Gbps network,
the CPU utilization of software network stacks becomes increasingly prohibitive.
Despite numerous efforts to improve their performance, software network stacks
tend to consume 30-40\% of CPU cycles~\cite{tonic-nsdi20}.

Another major hurdle and motivation for consolidating network resources
is the difficulty of provisioning the right amount of network resources
for individual end hosts (for both regular servers and disaggregated devices).
We will present out preliminary study on this part.
For regular-server clusters, we use traces we have collected from different
workloads (namely Web, Cache, and Hadoop) running inside Facebook datacenters.
We measure the duration of high-intensity communications or bursts at 25 us granularity.
Figure~\ref{fig-burst_duration} presents the CDF of burst duration across three
workloads. We observe that a significant fraction of these bursts are only one
sampling period long.
The 90th percentile duration is less than 200 us for all three rack types.
The results indicate that bursts not only exist, almost all high utilization
at the edge of the datacenter network is part of a burst, and that the duration
of these high-intensity communications is short.
Other prior work has also reported similar findings~\cite{NetKernel-ATC20}.

Disaggregated cluster introduces new types of network traffic,
where devices access other types of devices over the network
(e.g., compute devices accessing remote memory devices).
We use a set of remote memory-swap traces collected by Gao et al.~\cite{Gao16-OSDI}
to model traffic in disaggregated cluster (CDF and time in Figure~\ref{fig-OSDI16NetTrace}).
We observe similar patterns as regular-server clusters, i.e., burst but under-utilized.

In addition to the traffic patterns, under-utilization of network
functionalities in today's network devices is another major issue.
For example, existing NICs are bloated with features are not utilized~\cite{Caulfield-2018,firestone-nsdi18}.
Wang et al.~\cite{wang-hotcloud20} reports that state-of-the-art programmable switches have
low resource utilization, and even a complex application would consume only a small
fraction of resources. In general, vendors keep adding more features into
their network device products to meet the diverse requirements from different
customers, resulting in huge resource waste that could otherwise avoided
by network disaggregation and consolidation.

%{
\begin{table*}[th]\smallsize
\begin{center}
\begin{tabular}{ p{1.7in} | p{0.5in} | p{0.85in} | p{0.6in} | p{0.55in} | p{0.8in}  | p{0.7in} }

\textbf{Solution} & \textbf{Port Count \textasteriskcentered} & \textbf{Heterogeneous End-Points \textasteriskcentered} & \textbf{Offloaded Transport} & \textbf{Network Function} & \textbf{Manageability} & \textbf{Consolidated Resources} \\
\hline
\hline
Programmable Switch~\cite{RMT-SIGCOMM13,netcache-sosp17} & \xmark & \cmark & \cmark & \cmark  & $\bigcirc$  & $\bigcirc$ \\
\hline
Circuit Switch~\cite{sirius-sigcomm20,shoal-nsdi19,dRedBox-DATE} & \cmark & \cmark & \xmark & \xmark  & \xmark  &\xmark \\
\hline
Coherent Fabrics~\cite{GenZ,CXL,CCIX} & $\bigcirc$ & \cmark & \xmark & \xmark & \xmark & \xmark\\
\hline
Middleboxes~\cite{walfish-osdi04,comb-nsdi12,aplomb-sigcomm20} & \xmark  & \xmark  & \xmark & \cmark & \xmark  & \cmark  \\
\hline
NFV~\cite{clickos-nsdi14,e2,netbricks} & \xmark  & \xmark  & \xmark & \cmark & \cmark & \cmark \\
\hline
Multi-Host NIC~\cite{Intel-RedRockCanyon,Mellanox-Multihost} & \cmark &  $\bigcirc$ & \xmark & \xmark & \xmark  & $\bigcirc$ \\
\hline
\hline
\textbf{\sysname}     & \cmark & \cmark &  \cmark & \cmark & \cmark & \cmark \\
\hline
\end{tabular}
\end{center}
\mycaption{tabel-related-work}{Comparison of Network Solutions.}
{
\textasteriskcentered\ features only applicable to disaggregated datacenters.
$\bigcirc$ partially support.
}
\end{table*}
}


\subsection{Disaggregate and Consolidate Network}

To tackle the new network design challenges posed by resource disaggregation
and the limitations of current networks, we propose to disaggregate
and then consolidate the network computation resource into a \textit{network resource pool}.

In this paper, we focus one three types of network functionalities:
1) packet processing logic in NIC hardware,
2) software network stack running at processing units,
3) advanced application-specific network functions.

For disaggregated devices, network disaggregation removes
those network functionalities from the device and replaces a complex NIC
with a simpler one. This step \textit{decouples}
the core device from network and hence, allow them to change and scale independently.

After disaggregation, the network functionalities are further consolidated
into a datacenter-wide network resource pool. This pool provides services
for both disaggregated devices and regular servers.
Essentially, the network resource pool provides \textit{network-as-a-service}.

To the best of our knowledge, no work has attempted
to disaggregate the network before.
At first glance, the network cannot be disaggregated from
either a traditional server or a disaggregated device,
as they both need to be attached to the network and each
endpoint is provisioned with its own network interface and associated
network stacks. So can we disaggregate and then consolidate network?

To answer this question, we need to find out whether
a) it is possible to decouple the network functionalities,
and b) how to build the network resource pool.
In this paper, we focus on the second question.
We review emerging network devices and evaluate whether they
could meet the goals of consolidation and whether they are good
candidates for implementing the network resource pool.

% \subsection{Transports}
% \subsection{Congestion Control}
% \subsection{Network Functions}
% \subsection{Application-Specific Computing}

\fi
\section{SuperNIC Overview}
\label{sec:overview}

%As discussed in \S\ref{sec:related}, although different existing network solutions provide some features of network disaggregation and consolidation, none of them meet all our target goals, thus necessitating the design of a new network solution.
This section gives a high-level overview of the overall architecture of the \snic\ platform and how to use it. %We defer the detailed description of \snic\ design to \S\ref{sec:design} and \S\ref{sec:dist}.

\bolditpara{Overall Architectures.}~~
We support two ways of attaching an \snic\ pool in a rack (Figure~\ref{fig-topology}).
In the first architecture, the \snic\ pool is an intermediate layer between endpoints (servers or devices) and the ToR switch.
Each \snic\ uses one port to connect to the ToR switch.
Optionally, all the \snic{}s can be directly connected to each other, \eg, with a ring topology.
All remaining ports in the \snic\ connect endpoints.
We expect each of these endpoint-connecting links to have high bandwidth (\eg, 100\Gbps) and the uplink to the switch to have the same or slightly higher bandwidth (\eg, 100\Gbps\ or 200\Gbps). 
%Differently, today's multi-host NICs break one link into several sub-links each with a fixed portion of the original link's bandwidth.\notearvind{This is the thing that the reviewer complained about.  Apparently, BF3 has a pcie switch at the ingress that removes this problem. We could skip making this point - but I think that reviewer is not on the SIGCOMM PC!}
%The sum of the link bandwidth at each endpoint that connects to an \snic\ can and should exceed the link bandwidth between the \snic\ and the ToR switch. 
%This is because different endpoints' loads peak at different times (\S\ref{sec:motivation-server}), and after \snic's consolidation, the aggregated traffic would mostly fit the \snic's uplink, as shown in Figure~\ref{fig-fb-alibaba}.
%
The second architecture attaches \snic{}s to the ToR switch, and endpoints directly attach to the ToR switch.
In this architecture, the ToR switch re-directs incoming or outgoing traffic to one or more \snic{}s. 
%and balances load when doing the redirection.
Note that for both architectures, the actual implementation could either package the network pool with the ToR switch to form a new ``switch box'' or be separated out as an pluggable pool. 


\if 0
\snic\ is a data-center-scale solution. %Any endpoint in a data center can be the sender and/or the receiver, and any 
An endpoint could either connect to an \snic\ or directly to a ToR switch.
A given \snic\ can connect different types of endpoints.
However, there is a potential benefit in connecting similar endpoints to an \snic.
Doing so offers more opportunity for resource consolidation, as similar endpoints (\eg, memory devices) are likely to use the same set of \nt{}s (\eg, encryption).
%On the other hand, the same type of endpoints are more likely to receive similar workloads (\eg, a replicated write sent to two memory devices) that could result in synchronized traffic peak and burden the \snic.

When an \snic\ fails, or its link to the ToR switch fails, if other links and the basic switching functionalities are still alive, the \snic\ would turn into a passthrough device, forwarding \nt{}s to other \snic{}s for processing.
When the entire \snic\ fails, the endpoints connected to it will be disconnected to the rest of the data center.
This failure could be viewed as equivalent to traditional ToR switch failure but with a smaller failure domain (only the endpoints under the failed \snic\ instead of the whole rack).
Data centers that desire stronger reliability~\cite{pangu-nsdi21} could use a multi-homed solution by connecting each endpoint to two \snic{}s.
\fi

\bolditpara{Requirements for endpoints and the last hop.}~~
For basic connectivity, an endpoint needs to have the equivalence of physical and link layers.
For reliable transmission, the link layer needs to provide basic reliability functionality if the reliable transport is offloaded to \snic.
This is because packets could still be corrupted or dropped during the point-to-point transmission between an endpoint and its connected \snic/switch (the last hop).
Thus, the endpoint's link layer should be able to detect corrupted or dropped packets. It will either correct the corruption or treat it as a lost packet and retransmit it.
%Since the connection is point-to-point, the reliable link layer only needs one logical flow and requires a small retransmission buffer.
%Our implementation uses only \fixme{XXX} more resource than an unreliable link layer.
%Since the connection is point-to-point, t
The link layer also requires a simple flow control to slow down packet sending when the \snic\ pool is overloaded or the application's fair share is exceeded.
%In addition, an endpoint should perform simple flow control of the last hop (\eg, by slowing down the transmission when receiving back pressure or using PFC).
%This addition is the only change to today's endpoints that use an unreliable link layer, and it is only needed when the reliable transport is offloaded to \snic.
%We choose 64\KB\ buffer size, which is more than sufficient in the worst case.
%Overall, our reliable link layer only uses 37\% more resources than an unreliable link layer.

%The above requirements are all that is needed for disaggregating network functionalities, and any endpoints that meets these requirements can work with \snic.
Any interconnect fabric that meets the above requirements can be used as the last-hop link.
PCIe is one such example, as it supports reliable data transfer and flow control.
%Our \snic\ prototype uses Ethernet and extends standard non-reliable link layer to handle the reliability and rely on Priority Flow Control (PFC) for flow control.
Our \snic\ prototype uses Ethernet as it is more flexible.
We use Priority Flow Control (PFC) for the one-hop flow control and add simple retransmission support.
%extend the unreliable Ethernet link layer with a small one-hop reliable retransmission.
%By design, \snic\ can work with different types of physical links between endpoints and the \snic. Our prototype uses regular Ethernet. Future extensions could use faster/tighter links like PCIe to further reduce latency overhead. 
Unlike a traditional reliable link layer, our {\em point-to-point} reliable link layer is lightweight, %(only 37\% more resources than an unreliable link layer with our implementation), 
as it only needs to maintain one logical flow and a small retransmission buffer for the small Bandwidth-Delay Product (BDP) of the last hop (64\KB\ in our prototype).

\if 0
%\fixme{TODO: Need to revisit the following three paragraphs depending on how much is implemented and evaluated. Also this is a place to shorten if we need more space.}
We envision three types of endpoints and different network features for them.
The first type is regular servers.
Servers could choose to offload a transport protocol, network functions, and/or application-specific tasks to \snic\ to save CPU cycles and/or to accelerate performance.
Since servers have plenty of memory (larger than or similar to what an \snic\ has), they are more fit to store data than \snic{}s.
One interesting architecture we explore is to offload a reliable transport to \snic\ but to have the server still buffer un-acknowledged packets until receiving an ACK from the receiver (which we refer to as an {\em end-to-end buffer}).
%In this case, \snic\ will discard packets after they leave the \snic, saving its memory for other tasks.

The second type is disaggregated devices that only serve as a request handler (\eg, a memory device that accepts memory alloc/read/write operations~\cite{clio-arxiv,ATC20-pDPM}). 
Such a device often has limited processing power and would offload most tasks such as a transport protocol, network functions like encryption, and device-specific functionalities like replication to \snic. 
Since it never serves as a request originator, there is no need to maintain any packet store, and a failure could be handled by having the client retry the entire request~\cite{clio-arxiv,homa-sigcomm18,1RMA-sigcomm20}.

The final type is disaggregated devices that could serve as request initiators (\eg, a disaggregated CPU or GPU device) but have little memory (because memory is disaggregated to memory devices~\cite{LegoOS}).
When offloading \nt{}s to \snic{}s, they do not have significant memory to maintain end-to-end buffers like regular servers.
On the other hand, if they do not buffer packets at all and rely on \snic\ to buffer un-acknowledged packets, errors can still happen when packets are not successfully delivered to the \snic. 
We propose a {\em one-hop buffer}---buffering a packet at the device only until the next hop (\ie, the \snic) acknowledges.
Doing so reduces the amount of time each packet is maintained and the overall memory consumption.
\fi

\bolditpara{Using SuperNIC.}~~
To use the \snic\ platform, users first write and deploy \nt{}s.
%The user can use any endpoints in the data center as the sender and the receiver (even if the endpoint is not connected to an \snic\ and connects directly to a ToR switch).
They specify which \snic\ (sender side or receiver side) to deploy an \nt.
Users also specify whether an \nt\ needs to access the packet payload and whether it needs to use on-board memory.
For the latter, we provide a virtual memory interface that gives each \nt\ its own virtual address space.
Optionally, users can specify which applications share the same \nt{}(s).
Currently, our FPGA prototype only supports \nt{}s written on FPGA (deployed as netlists).
Future implementation could extend \snic{}s to support p4 programs running on RMT pipelines~\cite{p4fpga-sosr17} and generic software programs running on a processor.

After all the \nt{}s that a user desires have been deployed, the user specifies one or multiple user-written or compiler-generated~\cite{clicknp-sigcomm16,NFP-sigcomm17} DAGs of the execution order of deployed \nt{}s. Users could also add more DAGs at run time. Compared to existing works which let users specify an NF DAG when deploying NFs~\cite{e2-sosp15,flowtags-nsdi14,clicknp-sigcomm16}, we allow more flexible usages and sharing of deployed \nt{}s. %Different from traditional NF execution flows that execute NFs in sequence, we allow multiple \nt{}s to execute in parallel. 
The \snic\ stores user-specified DAGs in its memory and assigns a unique identifier (UID) to each DAG.
At run time, each packet carries a UID, which \snic\ uses to fetch the DAG.


\section{SuperNIC Board Design}
\label{sec:snic:design}

{
\begin{figure*}[th]
\begin{minipage}{1.1\columnwidth}
\begin{center}
\centerline{\includegraphics[width=0.95\textwidth]{Figures/board.pdf}}
\vspace{-0.1in}
\mycaption{fig-board}{\snic\ On-Board Design.}
{
RL: Rate Limiter. PT: Page Table
}
\end{center}
\end{minipage}
\begin{minipage}{0.05in}
\hspace{0.05in}
\end{minipage}
\begin{minipage}{0.9\columnwidth}
%\begin{center}
\hspace{0.2in} \includegraphics[width=0.9\textwidth]{Figures/scheduler.pdf}
\vspace{-0.1in}
\mycaption{fig-sched}{\snic\ Packet Scheduler and \nt\ Region Design.}
{
Double arrows, single arrows, and thick arrows represent packet headers, credits, and packet payload.
}
%\end{center}
\end{minipage}
\vspace{-0.1in}
\end{figure*}
}


Traditional server SmartNICs have plenty of hardware resources when hosting network functions for applications running on the local server~\cite{SmartNIC-nsdi18,Caulfield-2018}.
In contrast, \snic{} is anticipated to often be fully occupied or even over-committed, as it needs to host \nt{}s from more tenants with limited hardware resources to save costs.
Thus, a key and unique challenge in designing \snic{}s is space- and performance-efficient consolidation in a multi-tenant environment.
Moreover, \snic\ faces a more dynamic environment where not only the load of an application but also applications themselves could change from time to time.
Thus, unlike traditional SmartNICs that focus on packet processing and packet scheduling, \snic\ also needs to schedule \nt{}s efficiently.
%, and both types of scheduling needs to accommodate to a dynamic, multi-tenant environment.
%be more scalable and more flexible.
This section first goes over the high-level architecture of \snic, then discusses our mechanisms for efficient packet and \nt\ scheduling, followed by the discussion of our scheduling and fairness policies, and ends with a description of \snic's virtual memory system.
%We handle the former with SoftCores and the latter with hardware.
%An \snic\ has three major tasks: packet processing, packet scheduling, and \nt\ scheduling.
%overall approach \fixme{TODO if have space}  

\subsection{Board Architecture and Packet Flow}

We design the \snic\ board to simultaneously achieve several critical goals:
\textbf{G1)} parsing/de-parsing and scheduling packets at line rate;
\textbf{G2)} high-throughput, low-latency execution of \nt\ DAGs;
\textbf{G3)} safe and fair sharing of all on-board resources;
\textbf{G4)} quick adaptation to traffic load and workload changes;
\textbf{G5)} good scalability to handle many concurrent workloads and \nt{}s;
\textbf{G6)} flexible configuration and adjustment of control-plane policies;
and \textbf{G7)} efficient usage of on-board hardware resources.
%using most of the on-board hardware resources for application \nt{}s.
Figure~\ref{fig-snic-board} illustrates the high-level architecture of the \snic\ board.

\snic's data plane consists of reconfigurable hardware (\eg, FPGA) for running user \nt{}s (blue parts in Figure~\ref{fig-snic-board})
and a small amount of non-reconfigurable hardware (ASIC) for non-user functionalities, similar to the ``shell'' or ``OS'' concept~\cite{Catapult-v2,Amazon-F1,amorphos-osdi18,coyote-osdi20}.
We choose a hardware-based data-plane design because \nt{}s like transports demand high-speed, parallel processing, and a fully reconfigurable hardware allows the maximum flexibility in \nt\ hardware designs.
Many of our design ideas can potentially be applied to other types of hardware and software \nt\ implementations, such as PISA pipelines and ARM cores.
%In our prototype, 90\% space of the \snic\ chip is dedicated for \nt{}s (\textbf{G7}).

We divide the \nt\ area into {\em region}s, each of which could be individually reprogrammed to run different \nt{}s.
Different \nt\ regions can be executed in parallel.

The control plane runs as software on a small set of general-purpose cores (SoftCores for short) (\eg, a small ARM-based SoC). 
To achieve the performance that the data plane requires and the flexibility that the control plane needs, we cleanly separate these two planes.
The data plane handles all packet processing on ASIC and FPGA (\textbf{G1}).
The control plane is responsible for setting up policies and scheduling \nt{}s and is handled by the SoftCores (\textbf{G6}).
In our prototype, we built everything on FPGA. 


When a packet arrives at an RX port, it goes through a standard physical and reliable link layer.
Then our parser parses the packet's header and uses a Match-and-Action Table (MAT) to decide where to route the packet next.
The parser also performs rate limiting for multi-tenancy fairness (\S\ref{sec:snic:policy}).
The parser creates a packet descriptor for each packet and attaches it to its header. The descriptor contains fields for storing metadata, such as an \nt\ DAG UID and the address of the payload in the packet store. 
The SoftCores determine and install rules in the MAT, which include three cases for routing packets to the next step.
%There are three cases.
First, if a packet specifies no \nt\ information or is supposed to be handled by another \snic\ (\S\ref{sec:snic:dist}), the \snic\ will only perform simple switching functionality and send it to the corresponding TX port (red line).
Second, if a packet specifies the operation type \texttt{CTRL}, it will be routed to the SoftCores (orange line). These packets are for control tasks like adding or removing \nt{}s, adding \nt\ DAGs (\S\ref{sec:snic:ntsched}), and control messages sent from other \snic{}s (\S\ref{sec:snic:dist}).

Finally, all the remaining packets need to be processed on the \snic, which is the common case.
Their payloads are sent to the {\em packet store}, and their headers go to a central scheduler (black arrows). 
The scheduler determines when and which \nt\ chain(s) will serve a packet and sends the packet to the corresponding region(s) for execution (blue arrows).
If an \nt\ needs the payload for processing, the payload is fetched from the packet store and sent to the \nt.
During the execution, an \nt\ could access the on-board memory through a virtual memory interface, in addition to accessing on-chip memory.
After an \nt\ chain finishes, if there are more \nt{}s to be executed, the packet is sent back to the scheduler to begin another round of scheduling and execution.
When all \nt{}s are done, the packet is sent to the corresponding TX port.
%, which uses an arbiter to guarantee bandwidth fairness across applications.


\subsection{Packet Scheduling Mechanism}
\label{sec:snic:packetsched}

We now discuss the design of \snic's packet scheduling mechanism. Figure~\ref{fig-sched} illustrates the overall flow of \snic's packet scheduling and execution.
%To efficiently execute such complex \nt\ DAGs, we first propose the concept of {\em \nt\ chain}, which allows one packet to be processed by a chain of \nt{}s without the need to go through the scheduler multiple times.
%Second, we build a flexible run-time system that supports both {\em \nt-level parallelism} (running different \nt{}s in parallel) and {\em instance-level parallelism} (running multiple instances of the same \nt{}s in parallel) in hardware.
%Existing hardware-based network function systems like Click~\cite{clicknp-sigcomm16}, E2~\cite{e2-sosp15}, and PANIC~\cite{panic-osdi20} focus on scheduling a single NF or a simple sequence of NFs with instance-level parallelism.
%\snic\ provides all three types of scheduling: \nt\ chaining, \nt-level parallelism, and instance-level parallelism, and in a scalable, efficient way.
%Doing so achieves high-throughput, low-latency \nt\ DAG execution (\textbf{G2}), quick adaptation to traffic load changes (\textbf{G4}), and fast and scalable scheduling (\textbf{G5}).

\bolditpara{\nt-chain-based FPGA architecture and scheduling.}
%Today's network systems are seeing increasing amounts of network functions that can be executed as a DAG (\eg, previous work found 53.8\% NF pairs can run in parallel~\cite{NFP-sigcomm17}).
As \snic\ enables more types of endpoints and workloads to offload their network tasks, the number of \nt{}s and their execution flows will become even more complex, which could impact both the complexity of board design and the performance of packet scheduling.
Our idea to confront these challenges is to execute as many \nt{}s as possible in one go, by chaining \nt{}s together.
%There can be different ways to split an \nt\ DAG can 
We put chained \nt{}s (called an {\em \nt\ chain}) in one \nt\ region (\eg, \nt{}1$\xrightarrow[]{}$\nt{}3 and \nt{}2$\xrightarrow[]{}$\nt{}4 in Figure~\ref{fig-snic-board}).
%A packet that uses a chain goes through all the \nt{}s in the chain without the need to involve the scheduler in between.
Instead of connecting each \nt\ to a central scheduler (as what PANIC~\cite{panic-osdi20} does), we connect each region to the scheduler.
Doing so allows us to use a much smaller crossbar between the \nt\ regions and the central scheduler, thereby reducing hardware complexity and area cost (\textbf{G7}).

{
\begin{figure*}
\begin{center}
\centerline{\includegraphics[width=\textwidth]{snic/Figures/nt-example.pdf}}
\mycaption{fig-nt-example}{An Example of \nt\ chaining and scheduling.}
{
Top: user1 and user2's \nt\ DAGs and \snic's generated bitstreams for them.
Bottom: timeline of \nt\ bandwidth allocation change.
Dark grey and light grey represent user1 and user2's load.
The launched chains are \nt{}1$\xrightarrow[]{}$\nt{}2 and \nt{}3$\xrightarrow[]{}$\nt{}4,
with \nt{}2 and \nt{}4 being shared by the two users.
The maximum throughput of NT1, NT2, and NT4 are 10 units each, and NT3's is 7 units.
NT2 is the dominant resource for user1, and NT4 is the dominant for user2.
}
\end{center}
\end{figure*}
}


Furthermore, we leverage \nt\ chains to reduce the scheduling overhead and improve the scalability of the central scheduler.
Our idea is to {\em reserve} credits for an {\em entire} \nt\ chain as much as possible and then execute the chain as a whole; only when that is not possible, we fall back to a mechanism that may involve the scheduler in the middle of a chain. 
Doing so reduces the need for a packet to go through the scheduler after every \nt, thereby improving both the packet's processing latency and the central scheduler's scalability (\textbf{G5}).

On top of the fixed chain design, we propose an optimization to enable efficient \nt\ time sharing across multiple users and to accommodate cases where some packets of an application only access a part of a chain (\textbf{G4}, \textbf{G6}).
Our idea is to support the {\em skipping} of arbitrary \nt(s) in a chain.
For example, a user can access \nt{}1 and \nt{}4 by first skipping \nt{}3 in Region-1 and then skipping \nt{}2 in Region-2 in Figure~\ref{fig-sched}.


\noindent{\ul{\textbf{Scheduling packets with \nt-level and instance-level parallelism.}}}~~
At an \snic, different regions run in parallel.
We exploit two types of parallelism by controlling what \nt{}s to put in parallel regions.
The first type concurrently executes {\em different packets} at multiple instances of the {\em same \nt\ chain} (what we call {\em instance-level parallelism}).
We automatically create more/less instances of an \nt\ chain based on load and send different packets in a round-robin way to the parallel instances.
We will discuss our \nt\ autoscaling policy in \ref{sec:snic:policy}.


The second type concurrently executes the {\em same packet} at multiple {\em different \nt{}s} (what we call {\em \nt-level parallelism}).
We infer what \nt{}s can run in parallel in an \nt\ DAG (\eg, in Figure~\ref{fig-nt-example}, \nt{}1 and \nt{}2 can run in parallel with \nt{}3 for user1).
We expect a fair amount of opportunities to explore \nt-level parallelism, as previous work found that 53.8\% NF pairs can run in parallel~\cite{NFP-sigcomm17}.
To execute a packet at several \nt{}s concurrently, the scheduler makes copies of the packet header and sends them to these \nt{}s concurrently. To obey the order of \nt{}s that users specify, we maintain a {\em synchronization buffer} to store packet headers after they return from an \nt{}'s execution and before they could go to the next stage of \nt{}s (Figure~\ref{fig-sched}).

\subsection{\nt\ (De-)Launching Mechanism}
\label{sec:snic:ntsched}
\snic's SoftCore handles \nt\ deployment, launch, and scheduling tasks, as a part of the control path.
A new challenge specific to \snic\ comes from the need to do more frequent \nt\ reconfigurations than traditional programmable network devices.
To enable more consolidation, we allow multiple \nt{}s to {\em time share} an FPGA space, and we auto-scale \nt{}s.
Both these cases involve the slow FPGA PR process.
We propose a set of new mechanisms and policies to reduce or hide the PR costs.
We now discuss the mechanisms and defer the policy discussions to \S\ref{sec:snic:policy}.

\bolditpara{\nt\ deployment.}~~
Users deploy \nt{}s to the \snic\ platform ahead of time as FPGA netlists (which can be thought of as Intermediate Representations in software).
When receiving newly deployed \nt{} netlists for an application, we first generate a set of FPGA bitstreams (which can be thought of as executable binaries in software).
We enumerate all possible combinations of \nt{}s under user-specified \nt\ DAG ordering requirements when generating bitstreams. 
This is because bitstream generation is a slow process that usually takes a few hours or even longer. 
Generating more bitstreams at deployment time gives the \snic\ more flexibility to choose different \nt\ combinations at the run time. 
Figure~\ref{fig-nt-example} shows an example of generated bitstreams based on two DAGs of two users.
%\noteyiying{delete the following sentence if need more space}
%We do not generate bitstreams for three-\nt{} chains in this example, as that exceeds what a region can hold.
%For example, with three \nt{}s where the user specifies the first two to fan in to the third, we generate bitstreams for \nt{}1, \nt{}2, \nt{}3, \nt{}1$\xrightarrow{}$\nt{}2, \nt{}2$\xrightarrow{}$\nt{}1, \nt{}1$\xrightarrow{}$\nt{}3, \nt{}2$\xrightarrow{}$\nt{}3, \nt{}1$\xrightarrow{}$\nt{}2$\xrightarrow{}$\nt{}3, and \nt{}2$\xrightarrow{}$\nt{}1$\xrightarrow{}$\nt{}3.
We store pre-generated bitstreams in the \snic{}'s on-board memory; each bitstream is small, normally less than 5\MB.

When generating bitstreams, we attach a small \snic\ wrapper to each \nt\ (Figure~\ref{fig-sched}).
%This module monitors the load of each \nt\ and \yiying{Will, what else is in the module?}
This wrapper is essential: it enables skipping an \nt\ in a chain (\S\ref{sec:snic:packetsched}), monitors the runtime load of the \nt\ (\S\ref{sec:snic:policy}), ensures signal integrity during PR, and provides a set of virtual interfaces for \nt{}s to access other board resources like on-board memory (\S\ref{sec:snic:memory}).

\bolditpara{\nt\ chain launching.}~~
We start a new \nt\ chain when an application is deployed (pre-launch), when the chain is first accessed by a packet in an application (on-demand), or when we scale out an existing \nt\ chain. For the first and third cases, we start the new \nt\ only when there is a free region (see \S\ref{sec:snic:policy} for detail).
%The SoftCore picks a free region to launch the chain. 
For the on-demand launching case, when all regions are full, we still need to launch the new chain to be able to serve the application. In this case, we need to de-schedule a current \nt\ chain to launch the new chain (see \S\ref{sec:snic:policy} for how we pick the region).

The \snic\ SoftCore handles this context switching with a {\em stop-and-launch} process.
Specifically, the SoftCore sends a signal to the current \nt{}s to let them ``stop''.
These \nt{}s then store their states in on-board memory to prepare for the stop.
At the same time, the SoftCore informs the scheduler to stop accepting new packets. 
The scheduler will buffer packets received after this point.
After the above {\em stop step}s finish, the SoftCore reads the new bitstream from the on-board memory via DMA and starts the FPGA PR process ({\em launch}).
This process is the slowest step, as the maximum reported achievable PR throughput is around 800\,MB/s~\cite{coyote-osdi20}, or about 5\,\ms\ for our default region size.
Afterwards, the newly launched chain can start serving packets, and it will first serve previously buffered packets, if any.

%\boldpara{Swap and victim region.}~~
%A unique and new challenge in scheduling \nt{}s in FPGA is that unlike software, the \nt\ region reconfiguration process as described above is slow. 
%We tackle this challenge from both the mechanism and policy perspectives.
%We now discuss our ideas in mechanisms.

To reduce the overhead caused by \nt\ reconfiguration, we use a technique similar to the traditional victim cache design. We keep a de-scheduled \nt\ chain in a region around for a while unless the region is needed to launch a new chain. If the de-scheduled \nt\ chain is accessed again during this time, we can directly use it in that region, reducing the need to do PR at that time.

%we always leave one or few regions empty; we call them {\em swap region}s.
%To start a new chain, the SoftCore directly launches it in one swap region and start processing new packets with it.
%In the background, the SoftCore de-schedule an \nt\ chain by performing the {\em stop} step as illustrated above to create one more swap region.
%This swap region approach allows us to reduce the reconfiguration time to only the {\em launch} time. 

%Our second idea is inspired by victim cache in the traditional CPU architecture. 
%When de-scheduling an \nt\ chain, instead of clear the region, we leave it as is and tag it as a special {\em victim swap region}.
%If a swap region is needed to launch a new chain, this region can be picked.
%Otherwise, before it's picked, if packets access this chain again, they can directly use the region.
%If the chain's load is increased to be more than another chain's load, this other chain's region will be marked as the victim region, and the original victim region becomes a normal region.
%Our victim region approach avoids the overhead of de-scheduling and launching an \nt\ chain when there's only a short period of low load.

\subsection{Packet and \nt\ Scheduling Policy}
\label{sec:snic:policy}

We now discuss our packet and \nt\ scheduling policies.
Figure~\ref{fig-nt-example} shows an example of how an \snic\ with three regions evolves as load changes.

\bolditpara{Overall \nt\ scheduling strategy.}~~
Our overall strategy is to avoid FPGA PR as much as possible and treat \nt\ context switching (\ie, replacing a current \nt\ chain with a new one through FPGA PR) as a last resort, since context switching prevents the old \nt\ from running altogether and could result in thrashing in the worst case. 

For on-demand \nt\ launching, we first check if the \nt\ is the same as any existing \nt{} on the \snic.
If so, and if the existing \nt{} still has available bandwidth, we time share the existing \nt\ with the new application. %(\circled{1}).
In this case, new traffic can be served immediately.
Otherwise, we check if there is a free region.
If so, we launch the \nt\ at this region, and new traffic can be served after FPGA PR finishes.
Otherwise, we reach out to the distributed \snic\ platform and check if any other \snic\ has the same \nt\ with available bandwidth or a free region. 
If so, we route traffic to that \snic\ (to be discussed in \S\ref{sec:snic:dist}).
Otherwise, we run the \nt\ at the endpoint if users provide the alternative way of executing it there.
%user an alternative way of launching the \nt, \eg, running in hardware or software at the endpoint.
If all of the above fail, we resort to context switching by picking the region that is least loaded and using stop-and-launch to start the \nt.

We also try to hide PR latency behind the performance-critical path as much as possible.
%This is because it is slow to switch contexts (involving reconfiguring \nt{}s) and the 
Specifically, when a new application is deployed, we check if any of its \nt{}s is missing on an \snic. If there are any and if there are free regions on the \snic\ for these \nt{}s, we {\em pre-launch} them at the free regions, instead of launching them {\em on-demand} when the first packet accesses the \nt{}s, as the latter would require waiting for the slow PR and slow down the first few packets. 
%These pre-launched \nt{}s are the first batch of victims we choose to de-schedule if free regions are needed for other \nt{}s.

\bolditpara{\nt\ auto-scaling.}~~
To adapt to load changes, \snic\ automatically scales out/down instances of the same \nt\ (instance-level parallelism) (\textbf{G2, G4}).
Specifically, we use our per-\nt\ monitoring module to identify bottleneck \nt{}s and the load that they are expected to handle.
If there are free regions, we add more instances of these \nt{}s by performing PR on the free regions.
When the load to an \nt\ reduces to what can be handled with one instance less, we stop one of its instances and migrate the traffic of the stopped instance to other running instances.
Since PR is slow, we should scale out/down an \nt\ only if there is a persistent load increase/decrease instead of just occasional load spikes.
To do so, we only scale out/down an \nt\ if the past \texttt{MONITOR\_PERIOD} time has overloaded/underloaded the \nt.
\texttt{MONITOR\_PERIOD} should be at least longer than the PR latency to avoid thrashing. Since our measured PR latency is 5\ms, we set \texttt{MONITOR\_PERIOD} to be 5\ms\ by default. 
After experimenting other length, we find this length to work the best with most real-world traffic~\cite{facebook-sigcomm15,Atikoglu12-SIGMETRICS}.
%\noteyiying{Update this after getting Will's new monitoring sensitivity results.}
%\texttt{HIGH\_LOAD} is a heuristic depending on workloads.
%Previous works have reported different traffic peak lengths~\cite{facebook-sigcomm15,facebook-sigmetrics12}, \eg, millisecond-level in the 2015 Facebook trace~\cite{facebook-sigcomm15}, for which we could set \texttt{HIGH\_LOAD} to \fixme{XXX}.

\bolditpara{Scheduling with fairness.}~~
As we target a multi-tenant environment, \snic\ needs to fairly allocate its resources to different applications (\textbf{G3}).
%On top of the above strategy, we seek fairness across applications.
Different from existing fairness solutions, we treat every \nt\ as a separate type of resource, in addition to ingress bandwidth, egress bandwidth, packet store, and on-board memory space.
This is because we support the time sharing of an \nt, and different \nt{}s can be shared by different sets of users. Our fairness policy follows Dominant Resource Fairness (DRF)~\cite{DRF}, where we identify the {\em dominant} resource type for each application and seek a fair share for each application's dominant type. We also support weighted DRF~\cite{DRF,beyond-DRF} for users with different priorities.

Instead of a user-supplied static resource demand vector used in traditional DRF systems, we use {\em dynamically monitored resource demands} as the target in the DRF algorithm.
Specifically, at each {\em epoch}, we use the ingress parser, egress de-parser, the central scheduler, and our virtual memory system to monitor the actual load demand before requests are dispatched to a particular type of resource.
For example, for each user, the central scheduler measures the rate of packets that should be sent next to an \nt\ before assigning credits; \ie, even if there is no credit for the \nt, we still capture the intended load it should handle.
Based on the measured load at every type of resource for an application, we determine the dominant type of resource and use DRF to allocate resources after considering all applications' measured load vectors.
At the end of every epoch, our DRF algorithm outputs a new vector of resource allocation for each application, which the next epoch will use.
Compared to static resource demand vectors, our run-time monitoring and dynamic resource vectors can promptly adapt to load changes to maximize resource utilization. % with a higher degree of consolidation.

%We rerun the DRF algorithm right after scaling out/down an \nt{}, since scaling essentially changes the ``cap'' of the \nt's resource amount.

Another novelty is in how we achieve the assigned allocation.
Instead of throttling an application's packets at each \nt\ and every type of resource to match the DRF allocation, we only control the application's ingress bandwidth allocation.
Our observation is that since each \nt's throughput for an application, its packet buffer space consumption, and egress bandwidth are all proportional to its ingress bandwidth, we could effectively control these allocations through the ingress bandwidth allocation.
Doing so avoids the complexity of throttling management at every type of resource.
Moreover, throttling traffic early on at the ingress ports helps reduce the load going to the central scheduler and the amount of payload going to the packet store.
Our current implementation controls ingress bandwidth through rate limiting.
Future work could also use other mechanisms like Weighted Fair Queuing.
The only resource that is not proportional to ingress bandwidth is on-board memory space.
We control it through our virtual memory system (\S\ref{sec:snic:memory}).

Finally, the length of an epoch, \texttt{EPOCH\_LEN}, is a configurable parameter.
At every epoch, we need to run the DRF algorithm and possibly change the bandwidth and memory allocation.
Thus, \texttt{EPOCH\_LEN} should be longer than the time taken to perform these operations (around 3\mus\ with our implementation).
%Our measured time is 3\mus\ for running the DRF algorithm, negligible for changing bandwidth, and 15-20\mus\ for swapping out a 2\MB\ page.
%Note that swapping out memory can be done in a lazy fashion and does not complete in an epoch.
Meanwhile, it is desirable to set a short \texttt{EPOCH\_LEN} to quickly adapt to load variations and to update rate allocations approximately once per average RTT~\cite{xcp-sigcomm02, rcp-sigcomm06}.
Thus, we set the default value of \texttt{EPOCH\_LEN} to 20\mus.

\subsection{Virtual Memory System}
\label{sec:snic:memory}
\snic's allow \nt{}s to use off-chip, on-board memory.
To isolate different applications' memory spaces and to allow the over-subscription of physical memory space in an \snic, we build a simple page-based virtual memory system.
\nt{}s access on-board memory via a virtual memory interface,
where each \nt\ has its own virtual address space.
Our virtual memory system translates virtual memory addresses into physical ones and checks access permissions with a single-level page table.
We use huge pages (2\MB\ size) to reduce the amount of on-chip memory to store the page table.
Physical pages are allocated on demand; when a virtual page is first accessed, \snic\ allocates a physical page from a free list.

We further support the over-subscription of an \snic's on-board memory, \ie, an \snic\ can allocate more virtual memory space than its physical memory space.
When the physical memory is full, adding more \nt\ would require shrinking memory already assigned to existing applications (\S\ref{sec:snic:ntsched}).
In this case, we reduce already assigned memory spaces by migrating memory pages to a remote \snic, \ie, swapping out pages.
To decide what pages to swap out, we first use the DRF algorithm to identify what \nt{}(s) should shrink their memory space. 
Within such an \nt, we pick the least recently accessed physical page to swap out.
Our virtual memory system tracks virtual memory accesses to capture per-page access frequency. 
It also transparently swaps in a page when it is accessed.
If no other \snic{} has free memory space when the \snic\ needs to grow its virtual memory space, we reject requests to add new \nt{}s or to enlarge existing \nt{}'s memory.
\section{Distributed SuperNIC}
\label{sec:dist}

The design discussion so far focused on a single \snic. To enable better consolidation and network as a service, we 
%When we use a single \snic\ to consolidate \nt{}s of its connected endpoints, the \snic\ needs to be provisioned with the aggregated peak load of these endpoints.
%To further reduce cost, we 
build a rack-scale distributed \snic\ platform that enables one \snic\ to use other \snic{}s' resources.
With this platform, a rack's \snic{}s can collectively provision for the maximum aggregated load of all the endpoints in the rack.

As discussed in \S\ref{sec:overview}, we support two types of \snic\ pool topology.
For the switch-attached topology, the ToR switch serves as the load balancer across different \snic{}s.
It also decides which \snic\ to launch a new instance of an \nt\ with the goal of balancing traffic and efficiently utilizing \snic\ hardware resources.
%Specifically, it chooses the \snic\ that has enough free regions and is lightly loaded to launch new instances of \nt\ chains.
%Afterwards, the switch simply directs incoming flows to the \snic{}s that contain their target \nt{}s.
Supporting the intermediate-pool topology where the ToR switch cannot perform the above tasks is more complex. Below we discuss our design for it.

%\boldpara{Distributed Control Plane.}~~
SoftCores on the \snic{}s in the intermediate pool form a distributed control plane. 
They communicate with each other to exchange metadata and cooperate in performing distributed tasks. % like \nt\ migration and memory swapping.
We choose this peer-to-peer design instead of a centralized one, because the latter requires another global manager and adds complexity and cost to the rack architecture. %\zac{I am not sure about this argument -- I have heard that other places (e.g. google) have used the centralized manager architecture because it's easier to build and deploy than P2P ones. I personally don't buy this argument either. Do you have stronger support?}
Every \snic\ collects its FPGA space, on-board memory, and port bandwidth consumption, and it periodically sends this information to all the other \snic{}s in the rack.
Each \snic\ thus has a global view of the rack and can redirect traffic to other \snic{}s if it is overloaded.
%make decisions like \nt\ migration independently.
To redirect traffic, the \snic's SoftCore sets a rule in the parser MAT to forward certain packets (\eg, ones to be processed by an \nt\ chain on another \snic) to the remote \snic.

%\notearvind{We could also talk about something more basic - if the same NT is loaded on many snics, we can balance the load across all instances and achieve good consolidation. Maybe that would be easier for reviewers to accept before we talk about NT migration.}

%\boldpara{\nt\ Migration.}~~
If an \snic\ is overloaded and no other \snic{}s currently have the \nt\ chain that needs to be launched, the \snic\ tries to launch the chain at another \snic.
Specifically, the \snic's SoftCore first identifies the set of \snic{}s in the same rack that have available resources to host the \nt\ chain.
Among them, it picks one that is closest in distance to it (\ie, fewest hops).
The \snic's SoftCore then sends the bitstreams of the \nt{} chain to this picked remote \snic, which launches the chain in one of its own free regions.
When the original \snic\ has a free region, it moves back the migrated \nt\ chain. 
%It does so by first launching the \nt\ chain locally, then removing the MAT tunneling rule, and finally instructing the remote \snic\ to remove its \nt\ chain.
If the \nt\ chain is stateful, then the SoftCore manages a state migration process after launching the \nt\ chain locally, by first pausing new traffic, then migrating the \nt's states (if any) from the remote \snic\ to the local \snic. %and finally removing the MAT rule.


\section{Case Studies}
\label{sec:snic:application}

We now present two use cases of \snic\ that we implemented, one for disaggregated memory and one for regular servers.

\subsection{Disaggregated Key-Value Store}
\label{sec:snic:kvstore}
We first demonstrate the usage of \snic\ in a disaggregated environment by adapting a
recent open-source FPGA-based disaggregated memory device called {\em Clio}~\cite{Clio}.
The original Clio device hosts standard physical and link layers, a Go-Back-N reliable transport, and a system that maps keys to physical addresses of the corresponding values.
Clients running at regular servers send key-value load/store/delete requests to Clio devices over the network.
When porting to \snic, we do not change the client-side or Clio's core key-value mapping functionality.

\bolditpara{Disaggregating transport.}~~
The Go-Back-N transport consumes a fair amount of on-chip resources %ß(5.8\% LUTs and 2.6\% BRAM of the Clio device and 
(roughly the same amount as Clio's core key-value functionality~\cite{clio-arxiv}).
%(mainly on-chip memory used to store states for retransmission). 
We move the Go-Back-N stack from multiple Clio devices to an \snic\ and consolidate them by handling the aggregated load.
After moving the Go-Back-N stack, we extend each Clio device's link layer to a reliable one (\S\ref{sec:snic:overview}).

\bolditpara{Disaggregating KV-store-specific functionalities.}~~
A unique opportunity that \snic\ offers is its centralized position when connecting a set of endpoints, which users could potentially use to more efficiently coordinate the endpoints.
We explore this opportunity by building a replication service and a caching service as two \nt{}s in the \snic.
%that connects the Clio devices.

For \textbf{replication}, the client sends a replicated write request with a replication degree $K$, which the \snic\ handles by replicating the data and sending them to $K$ Clio devices. 
In comparison, the original Clio client needs to send $K$ copies of data to $K$ Clio devices or send one copy to a primary device, which then sends copies to the secondary device(s).
The former increases the bandwidth consumption at the client side, and the latter increases end-to-end latency.

For \textbf{caching}, the \snic\ maintains recently written/read key-value pairs in a small buffer. It checks this cache on every read request. If there is a cache hit, the \snic\ directly returns the value to the client, avoiding the cost of accessing Clio devices. Our current implementation that uses simple FIFO replacement already yields good results. Future improvements like LRU could perform even better.

\subsection{Virtual Private Cloud}
\label{sec:snic:vpc}

Cloud vendors offer Virtual Private Cloud (VPC) for customers to have an isolated network environment where their traffic is not affected by others and where they can deploy their own network functions such as firewall, network address translation (NAT), and encryption.
Today's VPC functionalities are implemented either in software~\cite{andromeda-google-nsdi18,ovs-nsdi15,ovs-sigcomm21} or offloaded to specialized hardware at the server~\cite{vfp-nsdi17,SmartNIC-nsdi18,aws-nitro}.
As cloud workloads experience dynamic loads and do not always use all the network functions (\S\ref{sec:snic:motivation}), VPC functionalities are a good fit for offloading to \snic.
Our baseline here is regular servers running Open vSwitch (OVS) with three NFs, firewall, NAT, and AES encryption/decryption. %Both senders and receivers servers employ the same setting and are connected to a physical switch directly.
We connect \snic{}s to both sender and receiver servers and then offload these three NFs to each \snic\ as one \nt\ chain. 

{
\begin{table}\small
\begin{center}
\begin{tabular}{ p{1.2in} | p{1in} |p{1in} }    
 & \textbf{Logic} & \textbf{Memory} \\    
\textbf{Module} & \textbf{(LUT)} & \textbf{(BRAM)} \\    
\hline    
\hline    
%Firewall     & 2.8\% & 0.5\% \\    
%AES-256       & 0.4\% & 0 \\    
%Transport    & 1.3\% & 0.42\% \\    
%\hline
%\hline 
\snic{} Core & 4.36\%   & 4.74\% \\ 
Packet Store & 0.91\%   & 9.17\% \\
PHY+MAC      & 0.72\%   & 0.35\% \\
DDR4Controller         & 1.57\%   & 0.29\% \\
MicroBlaze   & 0.25\%   & 1.81\% \\
Misc         & 1.52\%   & 0.75\% \\
%\textbf{Total (w/o \nt{})}        & \textbf{9.33\%}   & \textbf{17.11\%} \\
\hline
\textbf{Total}        & \textbf{9.33\%}   & \textbf{17.11\%} \\    
    
\end{tabular}    
\mycaption{fig-fpga-resource}{SuperNIC FPGA Utilization.}    
{    
Shell cost in an FPGA. Most resources left for running \nt{}s.    
}
\end{center}
\end{table}
}
{
\begin{figure*}[t]
\begin{center}
\centerline{\includegraphics[width=0.5\textwidth]{snic/Figures/fig-single-rack-capex-perDevCost.pdf}}
\mycaption{fig-rack-capex}{Per-Endpoint CapEx.}
{
A rack's network cost divided by endpoint count. 
}
\end{center}
\end{figure*}
}
{
\begin{figure*}[h]
\begin{center}
\centerline{\includegraphics[width=0.5\textwidth]{snic/Figures/g_plot_credit.pdf}}
\mycaption{fig-credit}{Throughput with different credits.}
{
}
\end{center}
\end{figure*}
}
{
\begin{figure*}[h]
\begin{center}
\centerline{\includegraphics[width=0.5\textwidth]{snic/Figures/fig-dist-nic-load-increase.pdf}}
\mycaption{fig-sim-dist-nic}{Distributed \snic{}.}
{
}
\end{center}
\end{figure*}
}
{
\begin{figure*}[h]
\begin{center}
\centerline{\includegraphics[width=0.5\textwidth]{snic/Figures/fig-dist-nic-latency.pdf}}
\mycaption{fig-topology-cmp}{Topology Comparison.}
{
}
\end{center}
\end{figure*}
}


\section{Evaluation Results}
\label{sec:results}

%\boldpara{Implementation.}~~
We implemented \snic\ on the HiTech Global HTG-9200 board~\cite{htg9200}. 
Each board has nine 100\Gbps\ ports, 10\GB\ on-board memory, and a Xilinx VU9P chip with 2,586K LUTs and 43\MB\ BRAM.
We implemented most of \snic's data path in SpinalHDL~\cite{spinalhdl} and \snic's control path in C (running in a MicroBlaze SoftCore~\cite{microblaze-xilinx} on the FPGA).
Most data path modules run at 250 MHz.
In total, \snic\ consists of 8.5K SLOC (excluding any \nt\ code).
%Figure~\ref{fig-fpga-resource} shows the FPGA resource consumption of different modules in \snic.
The core \snic\ modules consume less than 5\% resources of the Xilinx VU9P chip, leaving most of it for \nt{}s (see Appendix for full report).
To put it in perspective, the Go-back-N reliable transport we implement consumes 1.3\% LUTs and 0.4\% BRAM.

\bolditpara{Environment.}~~ 
We perform both cycle-accurate simulation (with Verilator~\cite{verilator-site}) and real end-to-end deployment.
Our deployment testbed is a rack with a 32-port 100\Gbps\ Ethernet switch, two HTG-9200 boards acting as two \snic{}s, eight Dell PowerEdge
R740 servers, each equipped with a Xeon Gold 5128 CPU and an NVidia 100\Gbps\ ConnectX-4 NIC, and two Xilinx 10\Gbps\ ZCU106 boards running as Clio~\cite{clio-arxiv} disaggregated memory devices.
Each \snic\ uses one port to connect to the ToR switch and one port to connect to the other \snic.
Depending on different evaluation settings, an \snic{}'s downlinks connect to two servers or two Clio devices.

\subsection{Overall Performance and Costs}
\label{sec:eval-overview}

\bolditpara{CapEx cost saving.}~~
We compare the CapEx cost of \snic's two architectures with no network disaggregation, traditional multi-host NIC, and traditional NIC-enhanced switches.
All calculations are based on a single rack and include 1) endpoint NICs, 2) cables between endpoints, \snic{}s, and the ToR switch, 3) the ToR switch, and 4) cost of \snic{}s or multi-host NICs.
We use market price when calculating the price of regular endpoint NICs and cables.
With \snic, the endpoint NICs can be shrunken down to physical and link layers (based on our prototype estimation, it is roughly 20\% of the original NIC cost of \$500), and the cables connecting endpoints and \snic{}s in the middle-layer-pool architecture can be shortened (roughly 60\% of the original cable cost of \$100~\cite{RAIL-NSDI}).
We estimate the ToR-switch cost based on the number of ports it needs to support and a constant per-port cost of \$250~\cite{fs-64port-switch}.

To estimate the cost of an \snic, we separate the non-\nt\ parts and the \nt\ regions. The former has a constant hardware cost, while the latter can be provisioned to the peak of aggregated traffic and \nt\ usages, both of which are workload dependent. We use the peak-of-sum to sum-of-peak ratios obtained from the Facebook traces (\S\ref{sec:benefits}). Today's multi-host NIC and NIC-enhanced switches do not consolidate traffic, and we assume that they will be provisioned for the sum-of-peak. See Appendix for detailed calculation. 

Figure~\ref{fig-rack-capex} plots the per-endpoint dollar CapEx cost. Overall, \snic\ achieves \textbf{52\% and 18\% CapEx savings} for the middle-layer and switch-attached pool architecture compared to no disaggregation.
Multi-host NIC and NIC-enhanced switches both have higher CapEx costs than \snic, because of its high provisioning without auto-scaling. The NIC-enhanced switches are even more costly than traditional racks mainly because of the added switch ports and NICs.

\bolditpara{OpEx saving and single-\snic\ performance.}~~
We compare a single \snic\ connecting four endpoints with the baseline of no disaggregation when these endpoints each run its \nt{}s on its own SmartNIC.
We generate workloads for each endpoint based on the Facebook memcached dataset distributions~\cite{Atikoglu12-SIGMETRICS}.
For the per-endpoint SmartNIC, we statically allocate the hardware resources that can cover the peak load.
Overall, we found that \snic\ achieves \textbf{56\% OpEx saving}, because \snic\ dynamically scales the right amount of hardware resources for the aggregated load.
\snic\ only adds \textbf{only 4\% performance overhead} over the optimal performance that the baseline gets with maximum allocation.

%\bolditpara{Single \snic\ throughput.}~~
We then test the throughput a real \snic\ board can achieve with a dummy \nt.
These packets go through every functional module of the \snic, including the central scheduler and the packet store. 
We change the number of initial credits and packet size to evaluate their effect on throughput, as shown in Figure~\ref{fig-credit}.
These results demonstrate that our FPGA prototype of \snic\ could reach more than 100\Gbps\ throughput. 
With higher frequency, future ASIC implementation could reach even higher throughput.
%Similar to PANIC~\cite{panic-osdi20}, we find that having more initial credits achieves higher throughput, and 8 credits are enough for 100\Gbps\ network.

\if 0
\bolditpara{System scalability.}~~
We evaluate \snic{}'s aggregated throughput by varying the number of physical ports. We use the same dummy NT and let packets go through all modules.
Figure~\ref{fig-scalability} shows that \snic{} scales linearly when ports increased. It confirms \snic{} on-board design is scalable and able to support a lot of endpoints.
\fi

%\boldpara{Single \snic\ latency.}~~
Next, we evaluate the latency overhead a real \snic\ board adds.  
It takes 1.3\mus\ for a packet to go through the entire sNIC data path. % (PHY, MAC, sNIC core, MAC, and PHY). 
Most of the latency is introduced by the third-party PHY and MAC modules, which could potentially be improved with real ASIC implementation and/or a PCIe link. 
The \snic\ core only takes 196\,ns.
Our scheduler achieves a small, fixed delay of 16 cycles, or 64\,ns with the FPGA frequency. 
%The synchronization buffer has an overhead of 4 cycles, or 16\,ns.
To put things into perspective, commodity switch's latency is around 0.8 to 1\mus.  

{
\begin{figure*}[th]
\begin{center}
\centerline{\includegraphics[width=0.5\textwidth]{snic/Figures/fig-conslid-overview-new.pdf}}
\mycaption{fig-sim-overview}{Performance and OpEx Overview.}
{
Lower is better for both axis.
}
\end{center}
\end{figure*}
}
{
\begin{figure*}[h]
\begin{center}
\centerline{\includegraphics[width=0.5\textwidth]{snic/Figures/fig-single-snic-low-corr.pdf}}
\mycaption{fig-sim-single-snic}{Single \snic{} Sensitivity.}
{
Each lines is running a different set of experiments.
}
\end{center}
\end{figure*}
}
{
\begin{figure*}[h]
\begin{center}
\centerline{\includegraphics[width=0.5\textwidth]{snic/Figures/g_plot_nt_chain.pdf}}
\mycaption{fig-nt-chain}{\nt{} Chain.}
{
}
\end{center}
\end{figure*}
}
{
\begin{figure*}[h]
\begin{center}
\centerline{\includegraphics[width=0.5\textwidth]{snic/Figures/g_plot_nt_parallel.pdf}}
\mycaption{fig-nt-parallel}{\nt{} Parallelism.}
{
}
\end{center}
\end{figure*}
}

\bolditpara{Distributed \snic{}s.}~~
To understand the effect of distributed \snic\ pool, we compare the two pool topology with a single \snic\ (no distributed support).
Figure~\ref{fig-sim-dist-nic} shows the performance and OpEx cost.
Here, we use the workload generated from the Facebook distribution as the foreground traffic and vary the load of the background traffic.
As background load increases, a single \snic\ gets more over-committed and its performance becomes worse.
With distributed \snic, we use an alternative \snic{} to handle the load, thus not impacting the foreground workload's performance. Note that the background workload's performance is not impacted either, as long as the entire \snic\ pool can handle both workloads' traffic. Furthermore, these workloads are throughput oriented, and we achieve max throughput with distributed \snic{}s.
As for OpEx, the intermediate-pool topology uses one \snic\ to redirect traffic to the other \snic.
As the passthrough traffic also consumes energy, its total OpEx increases as the total traffic increases.
The switch-attached topology does not have the redirecting overhead.
The single \snic\ also sees a higher OpEx as load increases because the workload runs longer and consumers more energy.

We then compare the two topologies of \snic\ pool. % and with no disaggregation. 
Figure~\ref{fig-topology-cmp} shows the latency comparison.
%As expected, no disaggregation achieves the best latency, as it does not need any additional network hops. 
The intermediate-pool topology where we connect the \snic{}s using a ring has a range of latency depending on whether the traffic only goes through the immediately connected \snic\ (single-\snic) or needs to be redirected to another \snic\ when the immediate \snic\ is overloaded. 
Because of the ring connection, this other \snic's distance to the immediately connected \snic\ determines the additional latency incurred (intermediate-best and worst).
In contrast, the switch-attached topology has a constant latency, even when one or multiple \snic{}s are overloaded. This is because the traffic always goes through the switch which directs it to the right \snic.

\subsection{Deep Dive into \snic\ Designs}
\label{sec:deepdive}

We now perform a set of experiments to understand the implications of \snic's various designs in terms of performance and OpEx cost, also with the Facebook distribution.
%We calculate OpEx cost as the amount of and duration of hardware resources used. 
%For these experiments, we generate workloads also from the same Facebook distributions.

\bolditpara{Effect of auto-scaling.}~~
We compare our auto-scaled implementation of \snic\ with two types of static allocations (\ie, no load-based scaling): allocating for the highest load needs ({\em static-max}) and allocating for the average load needs ({\em static-avg}), and an unrealistic auto-scaled scheme which instantly scales the right amount of \nt\ instances as load changes and incurs zero context switching overhead ({\em as-instant}).
We generate two workloads using the Facebook distributions, one where different endpoints spike at similar time (high correlation) and one where they spike at different times (low correlation).
Figure~\ref{fig-sim-overview} shows the performance slowdown (compared to no network disaggregation) and OpEx costs (compared to static-max).

\snic\ is at the Pareto frontier compared to the two static allocation schemes. Static-max has the best performance but the worst OpEx cost, as it pays for the peak hardware resources for the entire duration. In contrast, static-avg has the worst performance but best OpEx cost, since it only allocates the resource for the average usage for the entire duration.
Compared to \snic, as-instant only achieves slightly better performance with slightly more OpEx spending, as it tightly matches resources to the load which is unrealistic.

Comparing the two workloads, the low-correlation one always has better performance and more OpEx spending than the high-correlation one (except for as-instant which always yields best performance and static-max which always yields best performance and worst OpEx).
This is because with low correlation, the aggregated traffic is more flattened out, which gives \snic\ better chance to properly handle. As a result, \snic\ scales the right amount of resources to satisfy the load's performance needs.
When correlation is high (which is unlikely from our trace analysis in \S\ref{sec:benefits}), there will be fewer but more intensive spikes. When \snic\ is not fast or powerful enough to handle some of them, less resources is used but the performance goes down.

\bolditpara{Effect of victim cache.}~~
To evaluate the effect of our victim-cache design, we set the baseline to be disabling victim cache (blue dot in Figure~\ref{fig-sim-single-snic}).
We then change how often a de-scheduled \nt\ can be kept around as a victim instead of being completely deleted (shown as percentage on the green line). 
This configuration models how often an \snic's area is free to host victim \nt{}s.
As expected, the more de-scheduled \nt{}s we can keep around, the better performance we achieve, with no victim cache (baseline) having the worst performance.
The OpEx implication is less intuitive.
Here, we only count the time and amount of \nt\ regions that are actually accessed, as only those will cause the dynamic power (when idle, FPGA has a static power consumption regardless of how it is programmed).
With fewer de-scheduled \nt{}s kept around, more \nt{}s need to be re-launched (through FPGA PR) when the workload demands them. 
These re-launching overhead causes the OpEx to also go up.

%In Figure~\ref{fig-sim-single-snic}, we run low-correlation workload with an average peak length of 15 ms. The default monitor period is 5 ms.
%We use a victim cache with just one NT region and manually control its availability (i.e., whether a scale-down NT area can be inserted into the victim cache. Once in the victim cache, the NT region will be kept warm until next scale-out event). A larger victim cache availability rate yields better performance and lower OpEx.
%This is because the NT region in the victim cache is not counted towards the total user-perceivable OpEx and also absorbs traffic spikes otherwise will be missed by long PR latency.

\bolditpara{Effect of area over-commitment.}~~
We change the degree of area over-commitment by limiting how much hardware resources (\ie, NT regions) the workload can use compared to the ideal amount of resources needed to fully execute it.
%which is the amount of resources actually used over the total amount of hardware resources needed to execute the workload without context switching.
As Figure~\ref{fig-sim-single-snic} shows, as we increase the area over-commitment rate, we see worse performance and less resources (OpEx) used. 
Thus, our design uses distributed \snic{}s to avoid the over-commitment of a single \snic.
%However, even with 50\% overcommitment, the performance is only \fixme{XXX}\% worse than the baseline of no overcommitment. 
%there will be fewer NT regions left for scaling out. As as result, we see lower OpEx but at the cost of worse performance.

% we should just remove this monitor period sensitivty i think
%\bolditpara{Effect of monitoring period length.}~~
%Figure~\ref{fig-sim-single-snic} also shows monitor period's impact. For this low-correlation workload, a longer

\if 0
\TODO{revisit}
\bolditpara{Effect of average spike length.}~~
Using the same low-correlation workload, Figure~\ref{fig-sim-single-snic} shows average peak length's impact on performance and cost.
Surprisingly, we saw almost no performance loss with a short peak length (i.e., 5 and 10 ms). A deep investigation reveals that \snic's deep buffers served as a cushion and absorb the spiky traffic during PR. As the peak length increases, \snic's buffer will overflow and performance gradually worsens.
Since the performance loss originated from missed spikes during our PR and monitor period, the loss is actually capped. As we keep increasing the peak length (just pass the 20 ms mark, which is the sum of monitor period' 15 ms and PR's 5 ms), the performance gets better. Though it trades for a higher NT region utilization hence higher OpEx cost.
\fi

%\subsubsection{\nt\ Chaining and \nt-Level Parallelism}

\bolditpara{\nt\ chaining.}~~
To evaluate the effect of \snic's \nt-chaining technique, we change the length of \nt\ sequence from 2 to 7 (as prior work found real NFs are usually less than 7 in sequence~\cite{NFP-sigcomm17}).
In comparison, we implemented PANIC's scheduling mechanism on our platform, so that everything else is the same as \snic.
We also evaluate the case where \snic\ splits the chain into two sub-chains.
Figure~\ref{fig-nt-chain} shows the total latency of running the \nt\ sequence with these schemes.
\snic\ outperforms PANIC because it avoids going through the scheduler during the sequence for single-chain and only goes through the scheduler once for half-chain.

{
\begin{figure*}[t]
\begin{center}
\centerline{\includegraphics[width=0.5\textwidth]{snic/Figures/g_plot_ycsb.pdf}}
\mycaption{fig-snic-ycsb}{\small YCSB Latency.}
{
}
\end{center}
\end{figure*}
}
{
\begin{figure*}[h]
\begin{center}
\centerline{\includegraphics[width=0.5\textwidth]{snic/Figures/g_plot_ycsb_throughput.pdf}}
\mycaption{fig-snic-ycsb-tput}{\small YCSB Throughput.}
{
}
\end{center}
\end{figure*}
}
{
\begin{figure*}[h]
\begin{center}
\centerline{\includegraphics[width=0.5\textwidth]{snic/Figures/g_plot_ycsb_replication.pdf}}
\mycaption{fig-ycsb-replication}{\small Replicated YCSB.}
{
}
\end{center}
\end{figure*}
}
{
\begin{figure*}[h]
\begin{center}
\centerline{\includegraphics[width=0.5\textwidth]{snic/Figures/g_plot_ovs.pdf}}
\mycaption{fig-ovs}{\small VPC Performance.}
{
}
\end{center}
\end{figure*}
}

%{
\begin{figure*}[th]
\begin{minipage}{\figWidthSix}
\begin{center}
\centerline{\includegraphics[width=\columnwidth]{Figures/g_plot_conslid_perf.pdf}}
\vspace{-0.1in}
\mycaption{fig-kv-consolid}{Consolidation Performance w/ FB Key-Value.}
{
}
\end{center}
\end{minipage}
\begin{minipage}{\figWidthSix}
\begin{center}
\centerline{\includegraphics[width=\columnwidth]{Figures/g_plot_conslid_cost.pdf}}
\vspace{-0.1in}
\mycaption{fig-kv-cost}{Consolidation Resource Usage w/ FB KV.}
{
}
\end{center}
\end{minipage}
\vspace{-0.1in}
\end{figure*}
}
{
\begin{figure}[th]
\begin{center}
\centerline{\includegraphics[width=\textwidth]{snic/Figures/drf.pdf}}
\mycaption{fig-drf}{\nt{} Scheduling.}
{
SO: Scaling Out.
}
\end{center}
\end{figure}
}

\bolditpara{\nt-level parallelism.}~~
%As we mentioned earlier, users or compilers identify opportunities to parallelize, \snic{} follows the execution DAG. \snic{}-Parallel employs \nt-level parallelism and has lower latency compared to the basic \snic{} model.
We then evaluate the effect of \snic's \nt-level parallelism by increasing the number of \nt{}s that could run in parallel.
We compare with PANIC (on our platform), which does not support \nt-level parallelism.
We also show a case where we split \nt{}s into two groups and run these groups as two parallel \nt-chains (half-parallel).
Figure~\ref{fig-nt-parallel} shows the total latency of these schemes.
As expected, running all \nt{}s in parallel achieves the best performance.
The tradeoff is more \nt\ region consumption.
Half-parallel only uses two regions and still outperforms the baseline. % no parallelism PANIC scheme.

\bolditpara{DRF Fairness.}~~
To evaluate the effectiveness of our scheduling policy, we ran the synthetic workloads as described in Figure~\ref{fig-nt-example} and use the default \texttt{EPOCH\_LEN} of 20\mus\ and \texttt{MONITOR\_PERIOD} of 10\ms.
%Here, we set \texttt{EPOCH\_LEN} to 20\mus\ and \texttt{MONITOR\_PERIOD} to 10 epochs or 200us.
Figure~\ref{fig-drf} shows the resulting throughput timeline for different \nt{}s of the two users.
In between epoch 1 and 2, the loads of user2 increased to the second step.
At the next epoch, we run DRF and adjust the allocation. After the DRF algorithm finishes (in around 3\mus), user2 gets a higher (and fairer) allocation of \nt{}2 and \nt{}4, while user1's allocation decreases.
After observing \nt{}2 being overloaded for 10\ms, the \snic\ scales out \nt{}2 by adding one more instance of it at time epoch-503.
After PR is done (in 5\ms), both user1 and user2's throughput increase.



\subsection{End-to-End Application Performance}

We now present our end-to-end application performance conducted on our rack testbed. Because of space constraint, \textit{we move the consolidation experiments of these applications to the Appendix}.

\subsubsection{Disaggregated Key-Value Store}

In this set of experiments, we use one client server and two Clio devices. The Clio devices connect to one \snic\ which connects to the ToR switch. We run YCSB's workloads A (50\% set, 50\% get), B (5\% set, 95\% get), and C (100\% get)~\cite{ycsb-socc10} for these experiments. We use 100K key-value entries and run 100K operations per test, with YCSB's default key-value size of 1\KB\ and Zipf accesses ($\theta=0.99$). 
%The accesses to keys follow the Zipf distribution ($\theta=0.99$).

\bolditpara{Non-replicated YCSB performance and caching.}~~
%Our disaggregated key-value store experiments build on top of the Clio disaggregated memory platform~\cite{clio-arxiv}, where the client side is a regular server and the memory side is two Clio boards.
%Our baseline is the original Clio, which runs a Go-back-N transport on the Clio boards.
%With \snic, we first move the Go-back-N transport as an \nt\ to the \snic\ which connects the two Clio boards to the ToR switch.
%We then add a caching \nt\ to the \snic.
We first evaluate the performance of running YCSB without replication using one client server and one Clio memory device.
Figure~\ref{fig-ycsb} and \ref{fig-ycsb-tput} plot the average end-to-end latency and throughput of running the YCSB workloads with (1) the original Clio, (2) Clio's Go-Back-N transport offloaded to \snic\ (Clio-sNIC), (3) adding caching on top of Clio-sNIC (Clio-sNIC-\$), (4) Clover~\cite{ATC20-pDPM}, a {\em passive} disaggregated memory system where all processing happens at the client side and a global metadata server, (5) HERD~\cite{Kalia14-RDMAKV}, a two-sided RDMA system where both the client and memory sides are regular servers, and (6) HERD running on the NVidia BlueField SmartNIC~\cite{bluefield} (HERD-BF).
\snic's performance is on par with Clio, Clover, and HERD, as it only adds a small overhead to the baseline Clio.
With caching \nt, \snic\ achieves the best performance among all systems, esp. on throughput. 
This is because all links in our testbed are 100\Gbps\ except for the link to the 10\Gbps\ Clio boards. When there is a cache hit at the \snic, we avoid going to the 10\Gbps\ Clio boards.
HERD-BF performs the worst because of the slow link between its NIC and the ARM processor.

\bolditpara{Replicated YCSB performance.}~~
%As many key-value store users desire strong reliability, replication at the memory nodes is usually required.
We then test Clio, Clover, and Clio with \snic\ with replicated write to two Clio devices. HERD does not support replication, and we do not include it here.
Clover performs replicated write in a similar way as the baseline Clio, but with a more complex protocol.
Figure~\ref{fig-ycsb-replication} plots the average end-to-end latency with and without replicated writes using the YCSB A and B workloads.
With \snic's replication \nt, the overhead that replication adds is negligible,
while both Clio and Clover incur significant overheads when performing replication.

\subsubsection{Virtual Private Cloud}
We use one sender server and one receiver server, both running Open vSwitch (OVS)~\cite{ovs-nsdi15}, to evaluate VPC.
Our baseline is the default Open vSwitch that runs firewall, NAT, and AES.
We further improve the baseline by running DPDK to bypass the kernel.
In the \snic\ setup, we connect the sender to an \snic\ and the receiver to another \snic.
Each \snic\ runs the three NFs as a chain.
Figure~\ref{fig-ovs} shows the throughput results.
Overall, we find OVS to be a major performance bottleneck in all the settings. Using DPDK improves OVS performance to some extent.
Compare to running \nt{}s at servers, offloading them to the \snic\ improves throughput, but is still bounded by the OVS running at the endhosts.
%Offloading the \nt{}s to \snic\ offers a further performance improvement, mainly because of the hardware implementation.
%Offloading the \nt{}s to \snic{} achieves nearly identical performance to running them at the server side.

\if 0
\subsubsection{Consolidation across Multiple Endhosts}
To evaluate the benefit and tradeoff of consolidation, we deploy a testbed with four sender and four receiving servers with four setups:
each endhost connects to a ToR switch with 100\Gbps\ or 40\Gbps\ link (baseline, no consolidation), and four endhosts connect to an \snic, each with 100\Gbps\ or 40\Gbps\ link, and the \snic\ connects to the ToR switch with a 100\Gbps\ or 40\Gbps\ link (\snic\ consolidation).
%, and 3) four endhosts connect to an emulated multi-host NIC, each with a 25\Gbps\ link (\S\ref{sec:related}), and the multi-host NIC connects to the ToR switch with a 100\Gbps\ link. %(multi-host NIC, statically partitioned link bandwidth).
For both settings, we execute two \nt{}s, firewall and NAT, in FPGA. 
For the baseline, each endhost has its own set of \nt{}s, while %the multi-host NIC uses one set of \nt{}s in total and 
\snic\ autoscales \nt{}s as described in \S\ref{sec:policy}.
On each server, we generate traffic to follow inter-arrival and size distribution reported in the Facebook 2012 key-value store trace~\cite{Atikoglu12-SIGMETRICS}.
%the Hadoop load distribution reported in the 2015 Facebook workloads~\cite{facebook-sigcomm15}.
%Since there is no reported inter-arrival time for these workloads, we use the inter-arrival time reported by the 2012 Facebook workloads~\cite{Atikoglu12-SIGMETRICS}.
%We measure the application throughput (IOPS) every 10\ms\ time unit to evaluate the throughput changes over time.

Figure~\ref{fig-kv-consolid} reports the throughput comparison of \snic\ and the baseline.
%average IOPS and 95-percentile IOPS across all time units for the three settings. 
\snic\ only adds 1.3\% performance overhead to the baseline under 100\Gbps\ network and 18\% overhead under 40\Gbps\ network. 
We further analyze the workload and found its median and 95-percentile loads to be 24\Gbps\ and 32\Gbps.
With four senders/receivers, the aggregated load is mostly under 100\Gbps\ but often exceeds 40\Gbps.
Note that a multi-host NIC would not be able to achieve \snic's performance, as it subdivides the 100\Gbps\ or 40\Gbps\ into four 25\Gbps\ or 10\Gbps\ sub-links, which would result in each endhost exceeding its sub-link capacity.


We then calculate the amount of FPGA used for running the \nt{}s multiplied by the duration they are used for, to capture the run-time resource consumption with \snic's autoscaling mechanism. The baseline has one set of \nt{}s per endhost for the whole duration.
Figure~\ref{fig-kv-cost} shows this comparison when consolidating two and four endhosts to an \snic\ and using \nt{}s of different performance metrics.
For a slower \nt{} (\eg, one that can only sustain 20\Gbps\ max load), the \snic\ auto-scales more instances of it, resulting in less cost saving.
Our implementation of firewall \nt{} reaches 100\Gbps, while the AES \nt\ is 30\Gbps, resulting in a 64\% cost saving when deploying both of them.



%On the other hand, multi-host NIC incurs higher performance overhead, especially for the tail.
%Whenever any endhost exceeds 25\Gbps\ load, multi-host NIC will have a bottleneck link.
%On the other hand, \snic\ can sustain the peak of aggregated traffic, which is mostly under 100\Gbps, demonstrating the benefit of run-time, dynamic consolidation.


\subsubsection{Distributed \snic{}}
To run an \nt\ at a remote \snic,
an \snic{}'s SoftCore first sends a control message to the remote \snic{} to launch the \nt{} and then installs forwarding rules to its parser. This process takes 2.3\mus\ in our testbed.
Afterwards, packets are forwarded to the remote \snic. We observe an addition of 1.3\mus\ latency when packets go through the remote \snic. %\zac{Maybe put these numbers in a table and include std. dev or some kind of latency CDF (depending on how much space you have available)?}
%An \snic{} can "borrow" at most \texttt{100 Gbps x T} bandwidth, where \texttt{T} is the number of peer links.

\fi






\if 0
\subsection{Microbenchmark Results}
%\yizhou{mention this is simulation? Without phy/mac's limitation, internal link can run with their maximum BW which is 128Gbps.}
Our microbenchmark tests evaluate various \snic\ techniques in a controlled manner. 
Similar to PANIC~\cite{panic-osdi20}, for these tests, we implement a packet generator to generate traffic on the FPGA
and a delay unit to emulate \nt{}s and other tasks like PR by delaying packets in a controlled way.


\subsubsection{\snic\ System Module Performance}

\fi


\if 0

\begin{enumerate}
    \item Figure~\ref{fig-4}. A single snic and a single end-host. Run a set of NFs at either snic or end-host, and show their latency/bandwidth difference. This experiment showcase how much overhead snic puts on the data path. (NFs: transport, firewall, AES, SHA). This figure can be a bar graph, each NF has two bars. (Instead of single NF, we should run chains as well.)
    \item Figure~\ref{fig-1}. The above fig shows basic NF perf when offloaded to snic. This figure shows how snic impact end-to-end app performance. One snic (Transport), one clio (KVS-virt), and one client (YCSB). Client issues YCSB requests (workloads A, B, C), using different value size (e.g., 64, 128, 256, 512, 1024). We show the end-to-end latency. And we could compare to Clio, Clover, HERD. So in total 4 lines.
    \item if have time, compare to Bluefield (HERD?).
    \item Figure~\ref{fig-2}. Building on top of Figure~\ref{fig-1}. We want to show app can offload their functionalities to snic to accelerate certain operations (e.g., coordination, replication, etc). In this test, we use 1 client, 1 snic with 2 clio boards connected to it. We add a KVS Replication NF to snic. A KV write request will specify whether the data should be replicated. If so, the Replciatio NF will proactively duplicate the request and send to both Clio boards. We should compare with replicated Clio/Clover. Show throughput or total app runtime?
    \item Virtual Private Cloud. VPC. 1 snic, 1 server first. Encap/Decap, Firewall, NAT NFs.
    \item For the final figure (optional, if we have time), compose KVS and VPC. 1 clio + 1 snic, 1 server + snic. The server is behind the VPC running at the second snic. The server runs YCSB? 
\end{enumerate}

Microbenchmarks (simulation, 3 or 4 Figures):
\begin{enumerate}
    \item One-hop buffer/logic cost. Resource utilization. Just some numbers in text or a Table.
    \item Figure~\ref{fig-5} on-board NF chaining latency compared to PANIC. Couple lines: snic-eager credit allocation, snic-lazy credit allocation, no chaining (PANIC). We can vary the chain length as x-axis. Y-axis should be latency. (Reference to PANIC, come up with a chain model.)
    \item Figure~\ref{fig-6} how our parallesim improve performance. Use an NF chain: [A + B + C] -- [D]. Normal case, PANIC/other systems go through A-B-C-D. Since A,B,C have no dependcies, we can run them at the same time (NF-level parallelism). We should create multiple instances of A/B/C to show Instance-level paralleslim. Show the throughput/latency improvements. X-axis should be pkt size? Y-axis should be latency/throughput. (Find out the Long NF chain paper, use their chain)
    \item Figure~\ref{fig-8} should be a large figure our Auto-scaling/Scheduling policies. The x-axis is Timeline. The y-axis is Space Util, BW Util, Mem Util. At certain timestamps, we make a scheduling decision and show its impact.
    \item a bench without victim cache.
\end{enumerate}

1. micro. latency, throughput, w/ different number of ports.
2. apps. consolidation, fair sharing etc.

List of microbenmarks:
\begin{enumerate}
    \item test PR time, vary bitstream size. not necessary a figure.
    \item system shell performance, vary packet size
    \item packet store, credit store, header store throughput. maybe also a latency breakdown.
    \item scheduler pre-alloc v.s. on-demand allocation.
    \item auto scaling
\end{enumerate}

List of application tests (better to calculate all the capex and opex savings):
\begin{enumerate}
    \item disaggregated memory.
    \item run transport + NF. Baseline is running transport at endhost (e.g., CPU, bluefield, or even FPGA). compared to run that on supernic. This shows our latency performance.
    \item consolidation: disaggregated memory + transports + other NFs.
    \item design a benchmark to trigger auto-scaling.
\end{enumerate}

\fi


\section{Conclusion}
\label{sec:snic:conclude}

We propose network disaggregation and consolidation by building SuperNIC, a new networking device specifically for a disaggregated datacenter.
Our FPGA prototype demonstrates the performance and cost benefits of \snic.
Our experience also reveals many new challenges in a new networking design space that could guide future researchers.

\section{Acknowledgments}
Chapter 5, in part, has been submitted for publication of the material as it may appear in Yizhou Shan, Will Lin, Ryan Kosta, Arvind Krishnamurthy, Yiying Zhang, ``Disaggregating and Consolidating Network Functionalities with SuperNIC'', \textit{arXiv, 2022}. The dissertation author was the primary investigator and author of this paper.

\clearpage

\appendix

\section{Appendix}

\subsection{FPGA Resource Utilization}

The following table shows the FPGA resources used by \snic{} shell.
Most of the resources are left for running \nt{}s.

\begin{center}
\scriptsize
\begin{tabular}{ p{0.6in} | p{0.2in} |p{0.27in} }
 & \textbf{Logic} & \textbf{Memory} \\
\textbf{Module} & \textbf{(LUT)} & \textbf{(BRAM)} \\
\hline
\hline
%Firewall     & 2.8\% & 0.5\% \\
%AES-256       & 0.4\% & 0 \\
%Transport    & 1.3\% & 0.42\% \\
%\hline
%\hline
\snic{} Core & 4.36\%   & 4.74\% \\
Packet Store & 0.91\%   & 9.17\% \\
PHY+MAC      & 0.72\%   & 0.35\% \\
DDR4Controller         & 1.57\%   & 0.29\% \\
MicroBlaze   & 0.25\%   & 1.81\% \\
Misc         & 1.52\%   & 0.75\% \\
%\textbf{Total (w/o \nt{})}        & \textbf{9.33\%}   & \textbf{17.11\%} \\
\hline
\textbf{Total}        & \textbf{9.33\%}   & \textbf{17.11\%} \\
\end{tabular}
\end{center}



\subsection{Cost Calculation}
We explain the different deployment models and the cost calculation formulas behind our CapEx comparisons.
We limit our scope to rack-scale as the higher-level network hierarchies
are orthogonal to the resource pool deployment models.
We calculate that, to deploy a certain number of endpoints, what's the
network cost (i.e., the network interface card, cable, and switch port costs).

We compare the following models:
1) Non-disaggregation model, or the traditional model, termed \texttt{traditional}.
2) Disaggregation model, in which we insert the network pool between endpoints and the ToR switch (Figure~\ref{fig-topology} (a)), termed \texttt{ring}.
3) Disaggregattion model, in which we connect the pool of network devices directly to the ToR switch (Figure~\ref{fig-topology} (b)), termed \texttt{direct}.
For both disaggregation models, we further compare two type of devices: sNIC which has auto-scaling capability and multi-host NIC which can only provision for max resource usage. With runtime dynamic scaling and load balancing features, sNICs can provision for less than the max required resource , the specific ratio is calculated by comparing a particular workload's the sum-of-peak versus the peak-of-sum.

In all, we have the following models under comparison:
\texttt{traditionl, sNIC-direct, sNIC-ring, mhnic-direct, mhnic-ring}.

We now detail the cost calculations.
In the traditional non-disaggregation model,
each endpoint has a full-fledged NIC and a normal high-speed cable for connection to the ToR switch.
In both disaggregation models, since most network tasks are offloaded to the network resource pool, each endpoint can uses a down-scaled NIC.
Furthermore, the last hop link layer between endpoints and the network resource pool is reliable, we can leverage down-scaled, cheaper and less reliable physical cable~\cite{RAIL-NSDI}.

We use the following parameters in our calculation:
\begin{itemize}
\item Deploy \texttt{N} devices.
\item Each switch port has a cost of \texttt{costSwitchPort}
\item A full-fledged NIC's cost is \texttt{costNIC}. A down-scaled NIC cost is \texttt{costDSNIC}.
\item A normal high-speed cable cost is \texttt{costCable}.
A down-scaled less reliable physical cable cost is \texttt{costDSCable}.
\item A consolidation ratio \texttt{consolidRatio} determines how many endpoints are sharing one network resource pool device. We can calculate the number of network pool devices by \texttt{M = N / consolidRatio}.
\item For a network device, only a certain portion is dedicated to running network task, other parts are used as shell. We define the cost ratio used by network task to be \texttt{NTCostRatio}.
\item The peak-of-sum versus the sum-of-peak yields the auto-scaling potentials. A multi-host NIC (mhnic) provisions for the sum-of-peak while an sNIC provisions for the peak-of-sum. We call this ratio \texttt{capExConsolidRatio}.
\item The multi-host NIC's cost can be calculated as \texttt{costMHNIC = costNIC * N}.
\item The sNIC's cost can be calculated as \texttt{costsNIC = costMHNIC * capExRatio}, in which \texttt{capExRatio = (1 - NTCostRatio) + NTCostRatio * capExConsolidRatio}.
\end{itemize}

We now define each model's cost.

The traditional deployment model's cost is straightforward, it includes NIC, cable and switch ports:
\begin{gather}
N * (costNIC + costCable + costSwitchPort)
\end{gather}

The disaggregation models' cost has more moving parts than the traditional. It includes the down-scaled NICs and cables, network pool devices, the cables to the ToR switch, and switch ports.

The first disaggregation model (Figure~\ref{fig-topology} (a)) can be calculated as follows (for both \texttt{sNIC-ring, mhnic-ring}). 
\begin{align}
N * (costDSNIC + costDSCable) + \\
M * (costsNIC + costCable + costSwitchPort)
\end{align}

The second disaggregation model (Figure~\ref{fig-topology} (b)) can be calculated as follows (for both \texttt{sNIC-direct, mhnic-direct}).
\begin{align}
N * (costDSNIC + costCable + costSwitchPort) + \\
M * (costsNIC + costCable + costSwitchPort)
\end{align}

This tables shows the real-world numbers we use.

\begin{center}
\scriptsize
\begin{tabular}{|l|l|l|} 
 \hline
 Parameters & Value & Note \\
 \hline\hline
 costSwitchPort & \$250 & FS 100Gbps switch~\cite{fs-64port-switch} \\
 costNIC & \$500 & Mellanox Connect-X5 \\
 costCable & \$100 & FS DAC 100Gbps cable \\
 costDSNIC & costNIC * 0.2 & Numbers from our prototpe \\
 costDSCable & costCable * 0.6 & ~\cite{RAIL-NSDI} \\
 consolidRatio & 4 & Current model\\
 NTCostRatio & 0.9 & Numbers from our prototype \\
 capExConslidRatio & 0.23 & Facebook Hadoop trace~\cite{facebook-sigcomm15} \\
 \hline
\end{tabular}
\end{center}

%\subsection{Extended Evaluation Results}

{
\begin{figure*}[th]
\begin{minipage}{\figWidthSix}
\begin{center}
\centerline{\includegraphics[width=\columnwidth]{Figures/g_plot_conslid_perf.pdf}}
\vspace{-0.1in}
\mycaption{fig-kv-consolid}{Consolidation Performance w/ FB Key-Value.}
{
}
\end{center}
\end{minipage}
\begin{minipage}{\figWidthSix}
\begin{center}
\centerline{\includegraphics[width=\columnwidth]{Figures/g_plot_conslid_cost.pdf}}
\vspace{-0.1in}
\mycaption{fig-kv-cost}{Consolidation Resource Usage w/ FB KV.}
{
}
\end{center}
\end{minipage}
\vspace{-0.1in}
\end{figure*}
}
\subsection{End-to-End Application Performance and Cost with Consolidation}

To evaluate the benefit and tradeoff of consolidation, we deploy a testbed with four sender and four receiving servers with four setups:
each endhost connects to a ToR switch with 100\Gbps\ or 40\Gbps\ link (baseline, no consolidation), and four endhosts connect to an \snic, each with 100\Gbps\ or 40\Gbps\ link, and the \snic\ connects to the ToR switch with a 100\Gbps\ or 40\Gbps\ link (\snic\ consolidation).
%, and 3) four endhosts connect to an emulated multi-host NIC, each with a 25\Gbps\ link (\S\ref{sec:related}), and the multi-host NIC connects to the ToR switch with a 100\Gbps\ link. %(multi-host NIC, statically partitioned link bandwidth).
For both settings, we execute two \nt{}s, firewall and NAT, in FPGA. 
For the baseline, each endhost has its own set of \nt{}s, while %the multi-host NIC uses one set of \nt{}s in total and 
\snic\ autoscales \nt{}s as described in \S\ref{sec:policy}.
On each server, we generate traffic to follow inter-arrival and size distribution reported in the Facebook 2012 key-value store trace~\cite{Atikoglu12-SIGMETRICS}.
%the Hadoop load distribution reported in the 2015 Facebook workloads~\cite{facebook-sigcomm15}.
%Since there is no reported inter-arrival time for these workloads, we use the inter-arrival time reported by the 2012 Facebook workloads~\cite{Atikoglu12-SIGMETRICS}.
%We measure the application throughput (IOPS) every 10\ms\ time unit to evaluate the throughput changes over time.

Figure~\ref{fig-kv-consolid} reports the throughput comparison of \snic\ and the baseline.
%average IOPS and 95-percentile IOPS across all time units for the three settings. 
\snic\ only adds 1.3\% performance overhead to the baseline under 100\Gbps\ network and 18\% overhead under 40\Gbps\ network. 
We further analyze the workload and found its median and 95-percentile loads to be 24\Gbps\ and 32\Gbps.
With four senders/receivers, the aggregated load is mostly under 100\Gbps\ but often exceeds 40\Gbps.
Note that a multi-host NIC would not be able to achieve \snic's performance, as it subdivides the 100\Gbps\ or 40\Gbps\ into four 25\Gbps\ or 10\Gbps\ sub-links, which would result in each endhost exceeding its sub-link capacity.


We then calculate the amount of FPGA used for running the \nt{}s multiplied by the duration they are used for, to capture the run-time resource consumption with \snic's autoscaling mechanism. The baseline has one set of \nt{}s per endhost for the whole duration.
Figure~\ref{fig-kv-cost} shows this comparison when consolidating two and four endhosts to an \snic\ and using \nt{}s of different performance metrics.
For a slower \nt{} (\eg, one that can only sustain 20\Gbps\ max load), the \snic\ auto-scales more instances of it, resulting in less cost saving.
Our implementation of firewall \nt{} reaches 100\Gbps, while the AES \nt\ is 30\Gbps, resulting in a 64\% cost saving when deploying both of them.



%On the other hand, multi-host NIC incurs higher performance overhead, especially for the tail.
%Whenever any endhost exceeds 25\Gbps\ load, multi-host NIC will have a bottleneck link.
%On the other hand, \snic\ can sustain the peak of aggregated traffic, which is mostly under 100\Gbps, demonstrating the benefit of run-time, dynamic consolidation.

\if 0
\subsubsection{Distributed \snic{}}
To run an \nt\ at a remote \snic,
an \snic{}'s SoftCore first sends a control message to the remote \snic{} to launch the \nt{} and then installs forwarding rules to its parser. This process takes 2.3\mus\ in our testbed.
Afterwards, packets are forwarded to the remote \snic. We observe an addition of 1.3\mus\ latency when packets go through the remote \snic.
\fi
\fi



\chapter{Conclusion}
TODO.

%% APPENDIX
\appendix
\chapter{Final notes}
What to do about things \cite{Martin_1983}.  What did he say \cite{Rilling_Insel_1999}.
  Remove me in case of abdominal pain.



%% END MATTER
% \printindex %% Uncomment to display the index
% \nocite{}  %% Put any references that you want to include in the bib 
%               but haven't cited in the braces.
\bibliographystyle{plain}  %% This is just my personal favorite style. 
%                              There are many others.
%\setlength{\bibleftmargin}{0.25in}  % indent each item
%\setlength{\bibindent}{-\bibleftmargin}  % unindent the first line
%\def\baselinestretch{1.0}  % force single spacing
%\setlength{\bibitemsep}{0.16in}  % add extra space between items
\bibliography{local}
\bibliography{paper}
\singlespace  %to force bibilography environment to use single spacing for each entry 
              %double spacing between entries remains
\end{document}

