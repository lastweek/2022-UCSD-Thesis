\begin{abstract}


Hardware resource disaggregation is a solution that decomposes
general-purpose monolithic servers into segregated, network-attached resource pools,
each of which can be built, managed, and scaled independently.
Despite its management, cost, and fault-tolerance benefits, hardware resource
disaggregation is a drastic departure from the traditional computing paradigm
and it calls for a top-down redesign on system software, hardware, and data center networks.

%This dissertation seeks to address the challenge of building and deploying
%hardware resource disaggregation solutions in real data centers.
This dissertation shows that it is possible to overcome
the challenges of building and deploying hardware resource disaggregation solutions
in real data centers, delivering its promises on better manageability, scalability, and cost.
%

We first explored logical resource disaggregation for emerging persistent memory technologies.
Logical resource disaggregation \textit{logically} breaks
the server boundary by building an indirection layer on top of
monolithic servers to collectively expose a logical resource pool abstraction.
However, we fail to overcome the inherent problems of monolithic servers.
We then explored hardware resource disaggregation to
overcome these limitations by physically separating hardware resources
into network-attached pools.
We emulated disaggregated devices using monolithic servers
and built the first operating system designed for managing disaggregated resources.
It provides backward compatible interfaces while delivering good performance.
However, emulation incurs non-trivial overhead and has limited parallelism in serving highly-concurrent requests.
%
To avoid such overhead,
we then built the first publicly known hardware-based disaggregated memory device, which co-designs networking transport, virtual memory, and hardware.
We soon realized that while an increasing amount of effort goes into
disaggregating compute, memory, and storage, the network has been completely left out. 
%
The final piece of this dissertation proposes the concept of network disaggregation,
which decouples network functionalities from endpoints and then consolidates
them into a centralized network resource pool.
We built a new hardware-based networking device
along with a distributed runtime system to realize such a network resource pool.
%
Together, these four pieces outline a practical path to enable hardware
resource disaggregation solutions in real data centers,
especially how one can navigate the complex trade-offs among performance, cost, and manageability.

%when building operating systems, hardware, and network for it.
\end{abstract}