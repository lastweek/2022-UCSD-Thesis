\begin{abstract}


Hardware resource disaggregation is a solution that decomposes
general-purpose monolithic servers into segregated, network-attached resource pools,
each of which can be built, managed, and scaled independently.
Despite its management, cost, and failure tolerance benefits, hardware resource
disaggregation is a drastic departure from the traditional computing paradigm,
and calls for a top-down redesign on system software, hardware, and data center networks.
This dissertation seeks to address the challenge of building and deploying
hardware resource disaggregation in real data centers.
We first explored logical resource disaggregation for the emerging persistent memory.
We found that the inherent limitations of monolithic servers still persist.
We then took a radical turn to using a complete hardware resource disaggregation model,
and emulate disaggregated devices using monolithic servers.
We built the first operating system capable of managing disaggregated resources,
providing backward compatible interfaces while delivering good performance.
To avoid emulation overhead,
we then built the first practical hardware-based disaggregated memory device, co-designing networking transport, virtual memory, and hardware.
We soon realized that while increasing amount of effort go into
disaggregating compute, memory, and storage, the network has been completely left out. 
The final piece of this dissertation proposed the concept of network disaggregation,
and built a new hardware-based networking device along with a distributed runtime system
to provide network as a service.
Together, these four pieces outline a practical path to enable hardware
resource disaggregation in real data centers,
especially how one can navigate the complex trade-offs among performance, cost, and manageability when building operating system, hardware, and network for it.
\end{abstract}
