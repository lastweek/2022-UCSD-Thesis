\section{RDMA Issues and \sys\ Solutions}

{\em Remote Direct Memory Access} (RDMA), a high-speed network technology that allows direct access to memory on a remote server,
has been adopted by several major datacenter\cite{AZURE-RDMA,ALIBABA-RDMA}.
It is also the communication media adopted by most existing remote and disaggregated memory systems~\cite{XXX}. 
However, as this section will present, RDMA --- its interface, memory system, and network protocol --- 
are all ill fits for memory disaggregation (esp. for virtualized disaggregated memory),
and existing workarounds all have their limitations.
\sys\ solves RDMA's issues by redesigning and co-design with the virtualized disaggregated memory model.
Figure~\ref{fig-arch} illustrates the architectural differences between RDMA and \sys.
%RDMA supports both {\em one-sided} and {\em two-sided} communication to remote memory.
%One-sided RDMA allows one machine to directly read and write another machine's memory without involving the latter's CPU.
%Two-sided RDMA works like traditional messaging, involving both sender and receiver CPU.
%with one-sided RDMA being used by systems where remote memory are managed by local (compute) machines~\cite{DPM}
%and two-sided RDMA used by systems where each machine manages its own memory~\cite{Kalia14-RDMAKV,Mitchell13-ATC}.
%With the help of one-sided RDMA and other one-sided network technologies~\cite{GenZ},
%several software systems have been proposed to treat 
%Albeit its superior performance, RDMA's limited interface is too restrictive for most applications,
%Although several recent proposals of hardware~\cite{StRoM,Aguilera-FarMemory} and software~\cite{Tsai17-SOSP,HyperLoop} extensions to RDMA made it more flexible and performant, 

%fine-grained sharing

\subsection{Memory System} 
\label{sec:memory}

\subsubsection{Memory Abstraction}
\label{sec:abstraction}
%\subsubsection{Tight Coupling to Co-Located CPU}

\boldpara{RDMA issue.}
Addressing of memory regions ({\em MR}s) registered with RNIC is through the virtual memory address space of a process running at the memory node
(the registering process, or {\em RP} in Figure~\ref{fig-arch}. 
Before registration, the RP first needs to allocate virtual memory address for the MR,
and after registration, the RP passes its virtual memory address and an \rkey\ assigned by the RNIC to application processes running at client nodes.
Essentially, RDMA's abstraction is not a {\em client-facing} one, creating two main issue.
First, clients need to communicate with the RP to allocate or use MRs, usually through a non-RDMA, out-of-band communication channel,
causing performance overhead.
Second and more severe for memory disaggregation, RDMA's abstraction implies that each memory node needs to have a co-located CPU that runs an RP.

\boldpara{Current solutions and their limitations.}
The current practice of using RDMA to access remote memory is to add a client-side software indirection layer 
to translate client application's view (\eg, process virtual memory addresses, object IDs) to MRs and RP's virtual addresses.
%to translate client application's view (\eg, process virtual memory addresses, object IDs) to MRs and RP's virtual addresses.
This approach involves two mappings: client address to RP virtual address (performed at client node) and RP virtual address to remote physical address (performed at memory node).
Doing so (unnecessarily) adds performance and memory space overhead.

\boldpara{\sys\ solution and benefits.}
\sys\ provides a {\em client-facing} virtual memory view 
where client application processes can directly access disaggregated memory with their virtual memory addresses.
\sys\ only performs one address mapping: application virtual address to remote physical address (done at memory node),
and \sys\ completely gets rid of the need for registering processes or server boxes at memory nodes.
Client applications can directly perform allocation, read, write, atomic APIs (\eg, \tas, \cas, \fence), 
and extended APIs like array indexing and pointer chasing (\S\ref{sec:extendability}) to disaggregated devices.

Note that the application ``virtual memory addresses'' \sys\ use (we call them \textit{\sys\ virtual memory address}es) may not necessarily come from normal process memory address space.
In fact, there are three ways to use \sys, and they have their own interpretation of what \sys\ virtual memory addresses mean,
as illustrated in Figure~\ref{fig-arch}.
First, an application process can explicitly call \sys\ APIs by linking to a \sys\ library which creates a special \textit{\sys\ virtual memory address space}
under the hood. 
The returned addresses of \alloc\ and input addresses of \Cliosysread/\Cliosyswrite requests will all be virtual addresses in this space ({\em dvma-vaddr}).
Second, applications can sit on top a remote-memory swap system like FastSwap~\cite{FastSwap} and InfiniSwap~\cite{InfiniSwap},
which uses \sys\ virtual memory addresses as swap partition address ({\em swap-vaddr}) and performs \sys\ read/write to swap in/out disaggregated memory.
Third, a disaggregated compute device like pComponent in LegoOS~\cite{Shan18-OSDI} can integrate \sys\ into its hardware cache controller
and send \sys\ read/write with application processes' virtual addresses ({\em AP-vaddr}) for last-level-cache (called ExCache in LegoOS) misses. 

%Using client process virtual memory addresses also has the benefit that disaggregated memory can potentially be 
%integrated into client machine's memory microarchitecture~\cite{Lim09-disaggregate}.


\stepcounter{principle}
\ulinebfpara{Principle \arabic{principle}:}
\textit{The disaggregated memory abstraction should face client processes}.


\subsubsection{Memory Pinning and Page Faults}
\label{sec:pagefault}

\boldpara{RDMA Issue.}
Because RNICs rely on host machine's MMU and OS for virtual memory system support, they do not have any way to deal with page faults.
As a result, RDMA offers only two options, each having their own problem:
1) allocate and ping physical memory when registering MRs with RNIC, 
or 2) no allocation of physical memory and trap to the OS to handle page faults at first accesses.
Option 1) wastes physical memory space before they are used, 
while option 2) has poor performance for first accesses (and thus long tail latency) 
because of the costly communication between RNIC and the host OS (through interrupts)~\cite{ASPLOS17}.
Figure~\ref{fig-} \fixme{XXX}

\boldpara{Current solutions and their limitations.}
To not waste physical memory or causing high page fault performance overhead, 
one option is \textit{fine-grained pinning} where client application dynamically asks 
the memory process to register/unregister a new MR (and pin/unpin physical memory)  
immediately before and after the remote memory usages.
However, it is difficult to infer or ask programmers to specify first memory accesses, 
and RDMA's costly MR register operations significantly degrades application performance
(in our experiments, sending a network request and registering a new MR take \fixme{XXX} in total).
%Another solution proposed recently~\cite{ASPLOS-17} is to 
Thus, the common practice today~\cite{FastSwap,XXX} is to trade memory space for performancy by 
allocating a big MR and pinging its physical memory before an application starts.
%However, such \textit{coarse-grained pinning}'s resulting memory wastes would be a severe problem in a disaggregated setting, 
%as memory at  and should be used by applications as much as possible.

\boldpara{\sys\ solution and benefits.}
\sys\ completely eliminates pinning (and registering) with a virtual memory system that 
performs on-demand physical allocation and page fault handling --- just like what modern OS's provides,
but without the performance cost of RDMA's page fault mechanism.
The key to achieve this is our efficient implementation of the page fault handler in the memory device's main hardware pipeline.
Our main idea here is to move the slower, more complex, and non-deterministic allocation logic (both virtual and physical memory allocation)
to software running at a co-located low-power core, and to use an asynchronous methods to communicate 
between the software allocator and the hardware page fault handler.
With this novel virtual memory system design, our implementation has a \textbf{constant three-cycle latency of page fault handler}.
Figure~\ref{fig-} \fixme{XXX}
We will present the detailed design of our virtual memory system in \S\ref{sec:}.

\stepcounter{principle}
\ulinebfpara{Principle \arabic{principle}:}
\textit{Efficient space usage and performance can both be achieved by integrating virtual memory system in the hardware pipeline}.


\subsubsection{Memory Region}
\label{sec:mr}

\boldpara{RDMA issue.}
RDMA associates a set of metadata with each MR, namely an ID and two keys (\texttt{lkey} and \texttt{rkey}).
RNIC stores these MR metadata in its SRAM cache, and the memory consumption grows linearly with the number of MRs registered with the RNIC. 
When the SRAM is filled up, RNIC will swap the MR metadata to host memory (thrashing), causing significant performance degradation.
Figure~\ref{fig-} \fixme{XXX}

Another issue with MR is that RDMA uses it as the protection scope.
Thus, each client process needs to have at least one MR,
and if two client processes want to share some memory, the RP has to create a new MR.
In addition, metadata of different MRs share the same RNIC cache,
causing (remote) side channels that can be exploited to infer access patterns of victim clients~\cite{Tsai19-Security}.

\boldpara{Current solutions and their limitations.}
One way to reduce the number of MRs is to register a huge MR before an application starts and use a software layer to 
manange this MR (allocation) and map application memory areas to different parts of the MR.
However, as discussed in \S\ref{sec:pagefault}, huge MR will cause either memory waste or high tail latency (due to page faults),
and a new MR needs to be created for each protection scope.
LITE~\cite{Tsai17-SOSP} uses the OS kernel to register the entire server's physical memory as one MR when the system starts up
and to use a kernel module to allocate physical memory to applications.
LITE suffers from performance overhead of user-level-kernel crossings and all the drawbacks of the \pdm\ model (\S\ref{sec:pdm}).
1RMA~\cite{1RMA} uses a fix-sized region table

\boldpara{\sys\ solution and benefits.}
\sys\ gets rid of MR all together. 
With the per-client-process virtual memory space \sys\ offers, applications can allocate (and share) 
as many memory areas as they want without incuring any performance overhead.
%is all  with arbitrary memory areas in the space (through rmalloc instead of registering MR)
Figure~\ref{fig-} \fixme{XXX}

\stepcounter{principle}
\ulinebfpara{Principle \arabic{principle}:}
\textit{}.

\subsubsection{Page Table}
\label{sec:pte}

\boldpara{RDMA issue.}
RDMA uses page tables created and maintained by host CPU and OS to map MRs (RP virtual addresses) to physical memory.
To speed up page table lookup, RNIC caches hot PTEs (page table entries) in its SRAM.
Similar to the scalability problem of MRs, RDMA performance degrades significantly when cached PTEs fill up the SRAM.
Figure~\ref{fig-} \fixme{XXX}

\boldpara{Current solutions and their limitations.}
The common practice to mitigate RDMA's PTE scalability problem is to use huge pages (\eg, 2\GB),
which effectively reduces the number of PTEs~\cite{FaRM,Tsai17-SOSP,XXX}.
However, with large memory, even huge page PTEs could still fill up RNIC SRAM, 
and when it does, it could take up to four DRAM accesses from the RNIC (over PCIe) to walk a x86 page table and fetch a PTE.

\boldpara{\sys\ solution and benefits.}
We propose a new {\em overflow-free} hash-based page table design that bounds the size of PTEs to physical memory size 
and address translation to at most one DRAM access.
Specifically, we store {\em all} PTEs (from all client processes) in a hash table 
whose size is fixed and proportional to the physical memory size of a disaggregated memory device,
\ie, the hash table has enough entries to cover the entire physical memory (\eg, \fixme{XXX \MB\ for 100\GB\ physical memory and XXX page size}).
The hash value of a virtual memory address and its PID is used as the index to determine which hash bucket in the table the corresponding PTE goes to.
To lookup the page table, we always fetch
% compute its hash value (using the {\em lookup3}~\cite{lookup3-wiki} hash function) 
the entire bucket (with all its slots), in a single DRAM read.

Normally, a hash table with fixed slots will have an overflow problem because of hash conflicts.
We use a novel technique to {\em proactively avoid} hash overflows at virtual address allocation time.
To allocate a new virtual memory page, we keep trying available virtual memory addresses until there is one 
whose hash value would go to a bucket that still has an empty slot.
This allocation process is carried out in software running on an ARM core, while the page table lookup is performed by the hardware pipeline of a \sys\ device.
We trade potential retry overhead at allocation time for better run-time performance and simpler hardware design.
% core-memory implementation.
This overhead is manageable because 1) each retry is fast with our implementation~\cite{sec:metadataplane}, 
2) we employ huge pages, which means fewer pages need to be allocated, 
3) we choose a hash function that has very low collision rate~\cite{lookup3-wiki},
and 4) the \sys\ 64-bit virtual address space is huge.

Figure~\ref{fig-} \fixme{XXX}

\stepcounter{principle}
\ulinebfpara{Principle \arabic{principle}:}
\textit{}.

\subsection{Network System}
\label{sec:network}

\subsubsection{Connection, Reliability, and Ordering}

\boldpara{RDMA issue.}
RDMA's RC (reliable connection) mode, the only mode that supports both RDMA one-sided read and write operations,
is connection-oriented, requiring QPs (send\&receive queue pairs) to be established 
between a sender and a receiver before communication can happen.
RDMA implements its transport layer in hardware (RNIC),
and a reliable transport requires per-connection states.
Like MR and PTE, these states and other QP metadata are stored in RNIC SRAM, 
and QPs, too, have scalability issues.
Figure~\ref{fig-} \fixme{XXX}

\boldpara{Current solutions and their limitations}
A common mitigations to the QP scalability issue is QP sharing, 
where multiple threads in a workload or multiple workloads share a QP~\cite{FaSST}.
However, threads sharing a QP content for locks to use the QP,
and the FIFO order enforced by a QP can cause ``false ordering'' among workloads~\cite{1RMA}.
An alternative method is to use RDMA UD (unreliable datagram) mode.
The main issues with UD are its unreliable communication and 
its only support of two-sided RDMA, which requires the receiving node's CPU to be involved in the communication and thus does not fit the disaggregated memory model.

1RMA~\cite{1RMA} proposed a connection-less approach to perform remote read/write
and to use the reply of read/write requests as the ACK.
It relies on 

\boldpara{\sys\ solution and benefits.}
\if 0
Our insight is that we {\em can} achieve reliability without the need for connection or {\em any} states at the memory devices
by exploiting the {\em asymmetric} nature of memory disaggregation.
Our main idea is to shift all reliability tasks at the client side in software
and only include L1 and L2 in hardware (regular Ethernet NIC at client servers and a standard Ethernet MAC module at memory device hardware).
%and perform {\em all} reliable transport operations at the client side in software (\syslib).

Since all \sys\ operations are in the RPC style initiated by client servers, \ie, sending read/write requests and getting data/response back,
we use the reply as the ACK and matches replies to their corresponding requests using a request ID that is embedded in the packet header.
Each request's packet headers also include sender-receiver addresses, which will be used to construct reply packets. 
With these mechanisms, we essentially have a (unreliable) connectionless transport.
Below, we explain how we add reliability.

Network errors can happen when a packet is corrupted, loss, or reordered.
We use link-/physical-layer error correction like Forward Error Correction and retries that are already available in InfiniBand~\cite{MellanoxOFED} \fixme{and Ethernet???}
to deal with packet corruption, similar to prior practices~\cite{RAIL-NSDI17,FaSST}.
As the link layer does not (\fixme{or rarely?}) drop packets, packet loss is mainly due to buffer overflows in the network.
We control (minimize) packet loss with our client-side, software congestion/flow control mechanism (\S\ref{sec:cc}).
In the rare case of packet loss (either in a \sys\ request or its response), the client-side \syslib\ will fail after a request timeout
and retry the entire \sys\ request.
This leaves us with packet reordering, which is rare in non-multi-path network, but can still happen.

We take three approaches in handling packet reordering.
First, we guarantee that each request/reply is always delivered to upper layers in its entirety 
even when it is split into multiple link-layer packets that can be reordered.
We achieve this simply by using intra-request/reply sequence IDs to mark the proper ordering of packets 
and if needed, reorder them before delivering the request/reply.
Second, we offer synchronous APIs, with which we can guarantee that requests in the same session\footnote{Session is a \sys\ notion of }
is executed one after another. 
Thus, even without ordering guarantees from the network layer, synchronous requests will still be executed in order.
Finally, to offer more aggressive performance optimization, we offer asynchronous APIs, 
where multiple requests can be sent out before their replies are received.
Unlike synchronous requests, network reordering can affect asynchrnous requests' ordering.
We handle asynchronous requests' ordering in \sys\ with two methods:
1) the \syslib\ tracks read-write and write-write dependencies (outstanding read and write requests to the same address)
and issue dependent requests in a synchronous way;
2) we· offer a \fence\ operation, which lets a client process to wait until all outstanding asynchrnous requests have finished.
With these mechanisms, applications can achieve {\em release consistency}.

As a result, \sys\ memory devices completely remove the need for network states and thus all network-related scalability bottlenecks,
and client servers only need regular L2 Ethernet NICs with no NIC scalability issues either.
Figure~\ref{fig-} \fixme{XXX}

\stepcounter{principle}
\ulinebfpara{Principle \arabic{principle}:}
\textit{Efficient end-to-end reliability and stateless memory devices can be and is better achieved by shifting reliability handling to client-side software.}
%with unreliable, connectionless network
%Relaxing strict network reliability and handling reliability at system/application level }.  

\subsubsection{Congestion and Flow Control}
\label{sec:cc}

\boldpara{Issue}
RDMA over Converged Ethernet (RoCE) relies on link-layer Priority Flow Control (PFC~\cite{PFC}) to provide a drop-free substrate.
PFC is a mechanism for switches and NICs to avoid buffer overflow by forcing the immediate upstream entity to pause (through PFC pause frames).
However, PFC has a well-known head-of-line blocking issue and a risk of deadlock that make it infeasible to deploy at large scale~\cite{Zhu15-SIGCOMM}.

\boldpara{Current solutions and their limitations}
A common practice adopted by datacenters is to reduce PFC triggering by adding hardware-based congestion control schemes~\cite{Zhu15-SIGCOMM,hpcc-sigcomm19}
or by switching to a backup TCP network~\cite{RDMA-NSDI21}.
These schemes do not fit our needs, as we 
Recently, 1RMA~\cite{1RMA} proposed to 
two RTT writes

\boldpara{\sys\ solution and benefits.}
RDMA's network protocol treats all endpoints the same, \ie, a symmetric protocol.
We make two insights here.
First, with our design and implementation, a \sys\ memory device' internal throughput is always higher than its port bandwidth.
Thus, there will be no buffer overflow in a \sys\ memory device (\ie, it is faster than the network), 
and there is no memory-side in-cast problem and no need for memory-side flow control.
Second, 
client-side controlled congestion control
reactive, delay-based, window based 
solicitation window for client-side incast

\stepcounter{principle}
\ulinebfpara{Principle \arabic{principle}:}
\textit{}.

\subsection{Extendability}
\label{sec:extendability}

\boldpara{Issue}
limited primitives
bundled network stack and memory system, in a closed system
As a result, datacenters cannot easily deploy new transports~\cite{Homa,NDP,pHost} or congestion-control algorithms~\cite{1RMA,XXX}.

\boldpara{Current solutions and their limitations}

\boldpara{\sys\ solution and benefits.}
\sys\ allows \textit{easy extension} to different network stacks, high-level APIs, and to distributed disaggregated memory
modularized approach, each could be replaced by a new one

\stepcounter{principle}
\ulinebfpara{Principle \arabic{principle}:}
\textit{}.

\subsection{Putting It Together}
