

\if 0
For all the non-synchronization APIs, we offer two versions: synchronous and asynchronous.
While synchronous APIs are always strictly ordered within a user (and thus slower),
we relax the ordering of asynchronous APIs to a release consistency where operations can be executed out of order as long as 
1) read/write dependencies (WAR, RAW, WAW) are followed,
2) operations before a \fence\ must all complete before the \fence\ can return,
and 3) \fence{} operations are strictly ordered.
This release consistency is the same as architectures like ARMv8~\cite{ARMv8} and makes it possible for \sys\ to use a connectionless network layer with possible reordering (\S\ref{sec:network}).
Different users can share memory and can use \sys's synchronization primitives to achieve inter-user synchronization (\eg, using \tas\ to define critical section). 
\fi

\if 0
\sys\ is versatile in that there are many ways to use \sys's virtual memory view (we call it {\em \sys\ address space}).
Below and in Figure~\ref{fig-usage}, we list five typical ways to use \sys.
We implemented five applications with U2, U4, and U5;
U1 and U3 require building new hardware and/or OS/low-level systems, which are beyond this paper's scope.
%where client application processes can directly access disaggregated memory with their virtual memory addresses,
%no matter whether they run on \CN{}s or on \MN{}s.
%\sys\ only performs one address mapping: application virtual address to remote physical address (done at memory node),
%ordering

\stepcounter{case}
\boldpara{{\bf U\arabic{case}}: Entire virtual memory controlled by OS/hardware.}
A completely disaggregated solution like LegoOS~\cite{Shan18-OSDI}, where \CN{}s are compute devices with only CPU cache but no memory
can use \sys\ as the memory layer by sending \sys\ read/write requests to fulfill 
last-level-cache misses. % (called ExCache in LegoOS) misses.
With this usage, a \sys\ address space becomes the entire virtual memory address space
of a process, and \sys\ is completely transparent to application processes.

%usages
\stepcounter{case}
\boldpara{{\bf U\arabic{case}}: Extended virtual memory controlled by applications.}
%Applications run purely at \CN{}s and exploit \MN{}s for larger, dynamically allocated memory space.
%Note that the application ``virtual memory addresses'' \sys\ use (we call them \textit{\sys\ virtual memory address}es) 
%may not necessarily come from normal process memory address space.
%In fact, there are three ways to use \sys, and they have their own interpretation of what \sys\ virtual memory addresses mean,
%as illustrated in Figure~\ref{fig-arch}.
Without changing existing server hardware or OS at \CN{}s, 
an application process can explicitly call \sys\ APIs by linking \syslib\ which creates an extra \textit{remote virtual memory address space} (\ie, a \sys\ address space)
that is separate from the process' normal (local) virtual memory address space.
Applications have precise control over what and when to use disaggregated memory and can use or implement high-level APIs like pointer chasing.

%The returned addresses of \alloc\ and input addresses of \sysread/\syswrite requests will all be virtual addresses in this space ({\em dvma-vaddr}).

\stepcounter{case}
\boldpara{{\bf U\arabic{case}}: Remote memory space controlled by a middle layer.}
A system layer like a remote-memory swap system~\cite{InfiniSwap,FastSwap} or a language runtime~\cite{Semeru}
can sit on top of \sys\ and use a \sys\ address space as its own managed space (\eg, a swap partition, a JVM heap).
Applications on top of such a layer can transparently access larger memory backed by \sys.
%Second, applications can sit on top a remote-memory swap system like FastSwap~\cite{FastSwap} and InfiniSwap~\cite{InfiniSwap},
%which uses \sys\ virtual memory addresses as swap partition address ({\em swap-vaddr}) and performs \sys\ read/write to swap in/out disaggregated memory.

%Using client process virtual memory addresses also has the benefit that disaggregated memory can potentially be 
%integrated into client machine's memory micro-architecture~\cite{Lim09-disaggregate}.

\stepcounter{case}
\boldpara{{\bf U\arabic{case}}: Memory services running at \MN{}s.}
Users can deploy {\em memory services} that run entirely at \MN{}s (\eg, 
an in-memory key-value store or object data store) 
and expose their own interface (\eg, key-value get/set) 
to the clients of these services that run at \CN{}s.
These services can be deployed to \sysboard's programmable hardware and/or software platforms.
Each memory service has its own isolated \sys\ address space and can use \sys's APIs, 
which makes hardware implementation easier and execution safer.

\stepcounter{case}
\boldpara{{\bf U\arabic{case}}: Partial computation offloading.}
While U1, U2, and U3 perform computation completely at \CN{}s and U4 performs computation entirely at \MN{}s,
applications can also split their computation across \CN{} and \MN{}.
%An application can offload some of its computation tasks to an \MN{} (run in hardware or software).
\sys\ offers a {\em unified} address space for an application's \CN{} and \MN{} parts.
These parts are treated as different \sys\ threads. 
As explained earlier, if they cache shared data locally, they need to have their own mechanism to make the cache coherent, if desired.
\sys\ offers synchronization primitives to assist them work with their shared data.
%with each being treated as a traditional thread (\eg, \CN{} and \MN{} threads can be synchronized using \tas\ and other \sys\ synchronization primitives).
%\stepcounter{principle}
%\ulinebfpara{Principle \arabic{principle}:}
%\textit{The disaggregated memory abstraction should face client processes}.

\fi
