\section{The Splitkernel OS Architecture}
\label{sec:design}

We propose {\em \splitkernel}, a new OS architecture for resource disaggregation. 
Figure~\ref{fig-architecture} illustrates \splitkernel's overall architecture.
The \splitkernel\ disseminates an OS into pieces of different functionalities,
each running at and managing a hardware component.
All components communicate by message passing over a common network,
and \splitkernel\ globally manages resources and component failures.
Splitkernel is a general OS architecture we propose for hardware resource disaggregation.
There can be many types of implementation of \splitkernel.
Further, we make no assumption on the specific hardware or network type in a disaggregated cluster a \splitkernel\ runs on.
Below, we describe four key concepts of the \splitkernel\ architecture.

\noindent{\textit{\uline{Split OS functionalities.}}}
Splitkernel breaks traditional OS functionalities into {\em \microos{}s}.
Each \microos\ manages a hardware component, virtualizes and protects its physical resources.
Monitors in a \splitkernel\ are loosely-coupled and 
they communicate with other \microos{}s to access remote resources. 
For each \microos\ to operate on its own with minimal dependence on other \microos{}s,
we use a stateless design by sharing no or minimal {\em states}, or metadata,
across \microos{}s.

\noindent{\textit{\uline{Run \microos{}s at hardware components.}}}
We expect each non-processor hardware component in a disaggregated cluster to have a controller that 
can run a \microos.
A hardware controller can be a low-power general-purpose core, an ASIC, or an FPGA.
Each \microos\ in a \splitkernel\ can use its own implementation to manage the hardware component it runs on.
This design makes it easy to integrate heterogeneous hardware in datacenters ---
to deploy a new hardware device, its developers only need to build the device,
implement a \microos\ to manage it, % and communicate with the rest of \splitkernel,
and attach the device to the network. 
Similarly, it is easy to reconfigure, restart, and remove hardware components.

\noindent{\textit{\uline{Message passing across non-coherent components.}}}
Unlike other proposals of disaggregated systems~\cite{HP-TheMachine} that rely on coherent interconnects~\cite{GenZ,ccix,OpenCAPI},
a \splitkernel\ runs on general-purpose network layer like Ethernet and 
neither underlying hardware nor the \splitkernel\ provides cache coherence across components.
We made this design choice mainly because maintaining coherence for our targeted cluster scale 
would cause high network bandwidth consumption.
Instead, all communication across components in a \splitkernel\ is through {\em network messaging}.
A \splitkernel\ still retains the coherence guarantee that hardware already provides within a component (\eg, cache coherence across cores in a CPU),
and applications running on top of a \splitkernel\ can use message passing to implement their desired level of coherence for their data across components.

\noindent{\textit{\uline{Global resource management and failure handling.}}}
One hardware component can host resources for multiple applications and its failure can affect all these applications.
In addition to managing individual components, the \splitkernel\ also needs to 
globally manage resources and failure.
To minimize performance and scalability bottleneck,
the \splitkernel\ only involves global resource management occasionally for coarse-grained decisions, 
while individual \microos{}s make their own fine-grained decisions.
The \splitkernel\ handles component failure by adding redundancy for recovery.
