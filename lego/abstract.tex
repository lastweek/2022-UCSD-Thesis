\section*{Abstract}

The monolithic server model where a server is the unit of deployment, operation, and failure 
is meeting its limits in the face of several recent hardware and application trends. 
To improve resource utilization, elasticity, heterogeneity, and failure handling in datacenters, 
we believe that datacenters should break monolithic servers into {\em disaggregated, network-attached hardware components}. 
Despite the promising benefits of hardware resource disaggregation, 
no existing OSes or software systems can properly manage it.

We propose a new OS model called the {\em \splitkernel} to manage disaggregated systems. 
Splitkernel disseminates traditional OS functionalities into loosely-coupled {\em \microos{}s},
each of which runs on and manages a hardware component.
A \splitkernel\ also performs resource allocation and failure handling of a distributed set of hardware components.
Using the \splitkernel\ model, we built {\em \lego}, 
a new OS designed for hardware resource disaggregation. 
\lego\ appears to users as a set of distributed servers.
Internally, a user application can span multiple processor, memory, and storage hardware components.
%Internally, \lego\ cleanly separates processor, memory, and storage devices 
%both at the hardware level and the OS level.
We implemented \lego\ on x86-64 and evaluated it by emulating hardware components using commodity servers. 
Our evaluation results show that \lego' performance is comparable to monolithic Linux servers,
while largely improving resource packing and reducing failure rate over monolithic clusters.

