\section{Discussion and Conclusion}
\label{sec:lego:conclude}

We presented \lego, the first OS designed for hardware resource disaggregation.
\lego\ demonstrated the feasibility of resource disaggregation and its advantages in 
better resource packing, failure isolation, and elasticity, all without changing Linux ABIs.
\lego\ and resource disaggregation in general can help the adoption of new hardware
and thus encourage more hardware and system software innovations.  

\lego\ is a research prototype and has a lot of room for improvement.
For example, we found that the amount of parallel threads an \mcomponent\ can 
use to process memory requests largely affect application throughput. 
Thus, future developers of real \mcomponent{}s can consider use 
large amount of cheap cores or FPGA to implement memory monitors in hardware.

We also performed an initial investigation in load balancing 
and found that memory allocation policies across \mcomponent{}s can largely affect application performance.
However, since we do not support memory data migration yet, 
the benefit of our load-balancing mechanism is small.
We leave memory migration for future work.
In general, large-scale resource management of a disaggregated cluster is 
an interesting and important topic, but is outside of the scope of this paper.

\section{Acknowledgments}

Chapter 3, in full, is a reprint of Yizhou Shan, Yutong Huang, Yilun Chen, Yiying Zhang, ``LegoOS: A Disseminated, Distributed OS for Hardware Resource Disaggregation'', \textit{OSDI, 2018}. The dissertation author was the primary investigator and author of this paper.